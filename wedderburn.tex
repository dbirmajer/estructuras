\chapter{El teorema de Wedderburn}

Daremos una demostración sencilla de los teoremas de Wedderburn y Artin--Wedderburn 
en el caso de álgebras de dimensión finita. 
Seguiremos la demostración del teorema de Wedderburn de~\cite{MR184969} y 
la demostración del teorema de Artin--Wedderburn de~\cite{MR1244013}.  

\begin{definition}
\index{Álgebra!semiprima}
Diremos que un álgebra $A$ es \textbf{semiprima} si $I^2\ne\{0\}$ para todo ideal no nulo $I$ de $A$.	
\end{definition}

\index{Ideal!minimal}
\index{Ideal a izquierda!minimal}
Un ideal (a izquierda) $I$ de $A$ se dirá \textbf{minimal} si no contiene ideales (a izquierda) 
propios no nulos de $A$, es decir si $J\subseteq I$ es un ideal (a izquierda) no nulo de $A$, entonces
$J=I$. 

\begin{lemma}[Brauer]
\label{lem:Brauer1}
	Sea $A$ un álgebra de dimensión finita. Si $K$ es un ideal a izquierda no nulo minimal tal que
	$K^2\ne\{0\}$, entonces $K=Ae$ para algún idempotente $e\in A$ y $eAe$ es un álgebra de división.
\end{lemma}

\begin{proof}
Como $K^2\ne\{0\}$, existe $u\in K$ tal que $Ku\ne\{0\}$. La minimalidad de $K$ implica que $Ku=K$ y luego
$eu=u$ para algún $e\in K$. 
Sea 
\[
L=\{x\in K:xu=0\}\subsetneq K.
\] 
Si $x\in K$, entonces $xe-x\in L$, pues $(xe-x)u=x(eu)-xu=0$. 
Como $L$ es un ideal a izquierda de $A$, la minimalidad de $K$ implica que
$L=\{0\}$. Luego $e^2-e=0$. 
Nuevamente la minimalidad de $K$ implica que $K=Ae$. 

Dejamos como ejercicio
verificar que $eAe$ es un álgebra con identidad $e$. 

Si $x\in eAe\setminus\{0\}$, entonces $\{0\}\ne Ax\subseteq Ae=K$ y luego
$Ax=Ae$ por la minimalidad de $K$. Sea $y\in A$ tal que $e=yx$. 
Como $x\in eAe$, digamos $e=eae$, entonces $ex=x$. Similarmente $xe=e$.
Luego
\[
(eye)x=(ey)(ex)=eyx=e^2=e.
\]
Como $0\ne eye\in eAe$, existe $z\in A$ tal que $(eze)(eye)=e$. Luego $eze=x$. 
\end{proof}

\begin{lemma}
\label{lem:Brauer2}
Sea $A$ un álgebra semiprima de dimensión finita. 
Todo ideal a izquierda no nulo de $A$ contiene 
un idempotente no nulo.	
\end{lemma}

\begin{proof}
Como $A$ es de dimensión finita, existe un ideal a izquierda 
no nulo $I$ de $A$ de la menor dimensión posible. 
Como $A$ es semiprima, $(IA)^2\ne\{0\}$ 
y luego $I^2\ne\{0\}$, pues  
$(IA)^2=IAIA\subseteq I^2A$. 	
Por el lema de Brauer, existe un idempotente $e\in I$ tal que $I=Ae$.
\end{proof}

Observemos que si $e$ y $g$ son idempotentes, entonces 
\[
eAe\subseteq gAg\Longleftrightarrow eg=g=ge.
\]
Si $eg=e=ge$, entonces $eAe=(ge)A(ge)\subseteq gAg$. Recíprocamente, si $eAe\subseteq gAg$, entonces
$e=e^2\subseteq gAg$, digamos $e=gag$ para algún $a\in A$. Esto implica que 
\[
eg=(gag)g=gag^2=gag=e,\quad
ge=g(gag)=g^2ag=gag=e.
\]  

\begin{theorem}[Wedderburn]
\index{Teorema!de Wedderburn}
	Si $A$ es un álgebra simple 
	de dimensión finita, 
	entonces $A\simeq M_n(D)$ para algún $n\in\N$ y alguna álgebra de división $D$.  
\end{theorem}

\begin{proof}
	Sea $K$ un ideal a izquierda minimal. Como $KA$ es un ideal de $A$, la simplicidad de $A$ implica que
	$KA=A$. Además 
	\[
	A=A^2=(KA)^2=KAKA\subseteq K^2A
	\]
	y luego $K^2\ne\{0\}$. El lema de Brauer implica entonces que existe un idempotente $e\in K$ tal que
	$K=Ae$ y además $D=eAe$ es un álgebra de división. Vemos que $K$ es un $D$-módulo a derecha 
	con la multiplicación a derecha y que 
	para cada $a\in A$ 
	la función $\varphi_a\colon K\to K$, $x\mapsto ax$, es un morfismo de $D$-módulos a derecha, pues
	\[
	\varphi_a(xd)=a(xd)=(ax)d=\varphi_a(x)d.
	\]  
	La función $\varphi\colon A\to\End_D(K)$, $a\mapsto\varphi_a$, es un morfismo de álgebras. 
	
	Veamos que
	$\varphi$ es inyectiva. Si $a\in A$ es tal que $\varphi_a=0$, entonces 
	\[
	0=\varphi_a(K)=aK=aAe
	\]
	Como $A$ es simple, $AeA=A$. Se concluye así que $a=0$, pues 
	$0=aAe=aAeAe=aA$.   
	
	Veamos ahora que $\varphi$ es sobreyectiva. Como $A=AeA$, escribamos
	\[
	1=a_1eb_1+\cdots+a_neb_n
	\]
	para $a_1,\dots,a_n,b_1,\dots,b_n\in A$. 
	Si $\alpha\in\End_D(K)$, veamos que existe $a\in D$ tal que $\alpha=\varphi_a$. 
	Si $a=\sum_{i=1}^n\alpha(a_ie)eb_i$ y $x\in A$, entonces,
	como $\alpha$ es un morfismo de $D$-módulos y cada $e(b_ix)e\in D$, 
	\[
	\alpha(xe)=\alpha\left(\sum_{i=1}^na_ie^2b_ixe\right)=\sum_{i=1}^n\alpha(a_ie)eb_i(xe)=\varphi_a(xe).
	\] 
	Luego $A\simeq\End_D(K)$. Veamos ahora que $K$ es de dimensión finita sobre $D$. Si $\dim_D K=\infty$, entonces
	el conjunto
	\[
	\{\alpha\in\End_D(K):\dim\alpha(K)<\infty\}
	\]
	es un ideal no nulo propio de $\End_D(K)\simeq A$, 
	una contradicción pues $A$ es simple. En conclusión, 
	$A\simeq\End_D(K)\simeq M_n(D)$ para algún $n\in\N$.  
\end{proof}

\begin{theorem}[Artin--Wedderburn]
\index{Teorema!de Artin--Wedderburn}
	Sea $A$ un álgebra de dimensión finita. Si $A$ es semiprima, entonces
	$A$ es isomorfa a una 
	suma directa de finitas álgebras de matrices sobre álgebras de división.
\end{theorem}

\begin{proof}
	Sea $K$ un ideal a izquierda minimal de $A$. 
	
	Afirmamos que $S=KA$ es un ideal de $A$ minimal. Es claro que $S$ es ideal a izquierda de $A$. Por otro lado, 
	si $a\in S$, entonces $aS=a(KA)\subseteq KA=S$ pues $K$ es ideal a izquierda de $A$.  
	Si $I$ es un ideal no nulo de $A$ tal que 
	$I\subseteq S$. entonces $I\cap K\ne\{0\}$, pues si $I\cap K=\{0\}$,  
	\[
	I^2\subseteq I(KA)\subseteq (I\cap K)A=\{0\},
	\]
	una contradicción a la semiprimalidad de $A$. Como entonces $I\cap K\subseteq K$, la minimalidad de $K$ implica que
	$I\cap K=K\subseteq I$. Luego
	$S=KA\subseteq IA\subseteq I$.     	
	
	Por el lema~\ref{lem:Brauer2}, existe un idempotente no nulo $e\in S$ tal que $S=Ae$. 
	Como $A$ es de dimensión finita, podemos elegir 
	$e\in S$ de forma tal que la dimensión de $eAe$ sea lo más grande posible. 

	Afirmamos que $M=\{a\in A:Sa=\{0\}\}$ es un ideal de $A$. Es claro que es un ideal a izquierda de $A$. Si $a\in A$, 
	y $m\in M$, entonces $S(ma)=(Sm)=a\{0\}$. 
	
	Veamos que $A=S+M$. Sabemos que existe un idempotente $e\in K$ tal que $K=Ae$. Veamos que $1-e\in M$. Si $1-e\not\in M$, entonces
	$S(1-e)\ne 0$. Como $S(1-e)$ es ideal a izquierda y $A$ es un álgebra semiprima, 
	existe $f\in S(1-e)$ idempotente no nulo, digamos $f=s(1-e)$ para algún $s\in S$. Luego 
	\[
	fe=s(1-e)e=s(e^2-e)=0.
	\]
	y entonces $g=e+f-ef\in S$ es un idempotente de $S$. Como
	\begin{align*}
	&eg=e(e+f-ef)=e^2+ef-e^2f=e+ef-ef=e,\\
	&ge=(e+f-ef)e=e^2+fe-efe=e,
	\end{align*}
	se tiene que $eAe\subseteq gAg$. La maximalidad de la dimensión de $eAe$ implica entonces
	que $eAe=gAg$, es decir $e=g$. Esto implica que $f=ef$ y luego  
	\[
	f=f^2=(ef)^2=e(fe)f=0,
	\]
	una contradicción. 
	Luego $1-e\in M$ y entonces $A=S+M$. 
	
	Para ver que $A=S\oplus M$ falta ver que $S\cap M=\{0\}$. Como 
	\[
	(S\cap M)^2=(S\cap M)(S\cap M)\subseteq SM=\{0\},
	\]
	la semiprimalidad
	de $A$ implica que $S\cap M=\{0\}$.
	
	Ahora procedemos por inducción en la dimensión de $A$. 
	Si $M=\{0\}$, el resultado se obtiene del teorema de Wedderburn. Si $M\ne\{0\}$, entonces
	$S$ y $M$ son álgebras. Como $S$ es simple, el teorema de Wedderburn implica que $S$ es isomorfa a un álgebra
	de matrices sobre un álgebra de división. Como además $\dim M<\dim A$, la hipótesis inductiva
	implica que $M$ es isomorfa a una suma directa de álgebras de matrices sobre álgebras de división.  
\end{proof}

