\chapter{Ideales}

\begin{definition}
\index{Ideal!a izquierda}
\index{Ideal!a derecha}
Sea $R$ un anillo. Un subconjunto $I$ de $R$ es un \textbf{ideal a izquierda} de $R$ si 
$(I,+)\leq (R,+)$ y además $RI\subseteq I$ (es decir $xu\in I$ para todo $x\in R$ y $u\in I$).  	
\end{definition}

Análogamente pueden definirse ideales a derecha, simplemente hay que reemplazar la condición $RI\subseteq I$ por $IR\subseteq I$.

\begin{example}
Si $R=M_{2}(\mathbb{R})$, entonces  
\[
I=\left(\begin{array}{cc}
\mathbb{R} & \mathbb{R}\\
0 & 0
\end{array}\right)=\left\{ \left(\begin{array}{cc}
x & y\\
0 & 0
\end{array}\right):x,y\in\mathbb{R}\right\} 
\]
es un ideal a derecha de $R$ que no es un ideal a izquierda. 
\end{example}

Dejamos como ejercicio encontrar un ideal a izquierda de $M_{2}(\mathbb{R})$ que no sea un ideal a derecha.

\begin{definition}
\index{Ideal}
\index{Ideal!bilátero}
Sea $R$ un anillo. Un \textbf{ideal} es un subconjunto $I$ de $R$ tal que $I$ es simultáneamente un ideal a derecha y a izquierda. 
\end{definition} 

Los ideales de la definición anterior también se llaman \textbf{ideales biláteros}.

\begin{example}
Si $g(X)\in\mathbb{R}[X]$, entonces el conjunto 
\[
(g(X))=\{f(X)g(X):f(X)\in\mathbb{R}[X]\}
\]
de múltiplos de $g(X)$ es un ideal de $\mathbb{R}[X]$. 
\end{example}

Dejamos como ejercicio verificar los siguientes ejemplos. 
 
\begin{examples}\
\begin{enumerate}
\item $\{0\}$ y $R$ son siempre ideales de $R$.
\item $n\Z$ es un ideal de $\Z$.
\item Si $\{I_\lambda:\lambda\}$ es una colección de ideales de $R$, entonces $\cap_{\lambda}I_\lambda$ es un ideal de $R$.
\item Si $I_1\subseteq I_2\subseteq\cdots$ es una sucesión de ideales de $R$, entonces $\cup_{i\geq1}I_i$ es un ideal de $R$.
\end{enumerate}	
\end{examples}

% \begin{exercise}
% \label{xca:PcapQ}
%     Sean $R$ un anillo conmutativo y $J$, $P$ y $Q$ ideales de $R$. Pruebe
%     que si $J\subseteq P\cup Q$ entonces $J\subseteq P$ o bien $J\subseteq Q$.
% \end{exercise}

Tal como hicimos para grupos, pueden definirse ideales generados por un subconjunto del anillo $R$. Por ejemplo, 
podemos definir el ideal a izquierda generado por el subconjunto $X$ como
\[
(X)_L = \bigcap\{I:X\subseteq I,\,I\text{ ideal a izquierda}\}.
\]
Análogamente se define el ideal a derecha $(X)_R$ generado por $X$.  
Puede demostrarse que
\[
(X)_L=\left\{\sum_{i=1}^nr_ix_i:n\in\N_0,\,r_1,\dots,r_n\in R,\,x_1,\dots,x_n\in X\right\}. 
\]
De la misma forma podemos definir el ideal $(X)$ generado por $X$. En este caso, 
\[
(X) = \left\{\sum_{i=1}^nr_ix_is_i:n\in\N_0,\,r_1,\dots,r_n,s_1,\dots,s_n\in R,\,x_1,\dots,x_n\in X\right\}
\]

\begin{example}
Veamos que los ideales de $\Z/n$ son de la forma $(d)$ con $d$ un divisor de $n$. En efecto, si 
$d$ es un divisor de $n$, entonces $(d)$ es un ideal de $\Z/n$. Demostremos entonces
que todo ideal $I$ de $\Z/n$ es de la forma $(d)$ para algún divisor $d$ de $n$. Como
$I$ es en particular un subgrupo del grupo aditivo de $\Z/n$, el teorema de Lagrange nos dice que 
$m=|I|$ es un divisor de $n$. Como además el grupo
aditivo de $\Z/n$ es cíclico, digamos $\Z/n=\langle 1\rangle$, el grupo aditivo de $I$ también será cíclico, digamos
$I=\langle n/m\rangle$. Luego $I=(n/m)$, donde $d=n/m$ es un divisor de $n$.  
\end{example}

Usaremos ideales para poder definir cocientes. 
Antes de proceder a explicar esta construcción es 
conveniente repasar nociones básicas sobre morfismos de anillos. 

\begin{definition}
\index{Morfismo!de anillos}
Si $R$ y $S$ son anillos, diremos que una función $f\colon R\to S$ es un morfismo de anillos si $f(1)=1$, 
$f(x+y)=f(x)+f(y)$ y $f(xy)=f(x)f(y)$ para todo $x,y\in R$. 	
\end{definition}

Si $f\colon R\to S$ es una función tal que $f(x+y)=f(x)+f(y)$ y $f(xy)=f(x)f(y)$ para todo $x,y\in R$, 
no es necesariament cierto que $f(1)=1$. Se pide esta condición en la definición para evitar patologías. Veremos 
otra posible explicación en el capítulo~\ref{modulos}.

\begin{example}
Sea $f\colon\Z/6\to\Z/6$, $f(x)=3x$. Un cálculo sencillo muestra que $f(x+y)=f(x)+f(y)$ y 
$f(xy)=f(x)f(y)$ para todo $x,y\in\Z/6$, pero $f(1)=3\ne 1$.	
\end{example}

\index{Núcleo!de un morfismo de anillos}
Tal como hicimos en el caso de grupos, podemos definir el núcleo de un morfismo $f\colon R\to S$ como
\[
\ker f=\{x\in R:f(x)=0\}.
\]
Queda como ejercicio demostrar que $\ker f$ es un ideal de $R$. Además $f$ es inyectivo si y sólo si $\ker f=\{0\}$.  

\begin{examples}
Las siguientes funciones son ejemplos de morfismos de anillos:
\begin{enumerate}
	\item La identidad $\id\colon R\to R$.
	\item Las inclusiones $\Q\hookrightarrow\R\hookrightarrow\C$.
	\item $\Z\to R$, $k\mapsto k1_R$.
	\item La evaluación $\operatorname{ev}_{x_0}\colon R[X]\to R$, $f\mapsto f(x_0)$, donde $x_0\in R$ es un elemento fijo. 
\end{enumerate}
\end{examples}

\begin{example}
La función $f\colon\Z[i]\to\Z/5$, $f(a+bi)=a+2b\bmod 5$, es un morfismo de anillos.	El núcleo de $f$
es el conjunto
\[
\ker f=\{a+bi:a+2b\text{ es divisible por 5}\}.
\] 
\end{example}

\begin{exercise}
\label{xca:Zsqrtd}
Sea $R$ un anillo conmutativo y sea $d\in\Z$ libre de cuadrados.  
Demuestre que $\Hom(\Z[\sqrt{d}],R)$ está en biyección con $\{r\in R:r^2=d1_R\}$. 
\end{exercise}

El ejercicio anterior nos dice por ejemplo que $\Hom(\Z[i],\Z)=\emptyset$. 

\begin{example}
La función $\C\to\R^{2\times 2}$, $a+bi\mapsto\begin{pmatrix}a&b\\-b&a\end{pmatrix}$, es un morfismo inyectivo  
de anillos. 
\end{example}

La idea
del ejemplo anterior nos permite también encontrar 
un morfismo de anillos 
$\Z[i]\to\Z^{2\times 2}$.  	

\begin{example}
La función $\Z\to 2\Z$, $x\mapsto 2x$, es un morfismo de grupos abelianos pero no es un morfismo de anillos. 	
\end{example}

\begin{example}
Sean $D$ un anillo de división y $R$ un anillo no trivial. Si $f\colon D\to R$ es un morfismo de anillos, entonces, como $f(1)=1\ne0$,
el núcleo $\ker f$ es un ideal propio de $D$. Luego $f$ resulta ser inyectivo.   	 
\end{example}

\begin{exercise}
\label{xca:sqrt2sqrt3}
Demuestre que los anillos $\Q(\sqrt{2})$ y $\Q(\sqrt{3})$ no son isomorfos.	
\end{exercise}

\begin{exercise}
\label{xca:Z6Z15}
Demuestre que no hay morfismos de anillos $\Z/6\to\Z/15$. 	
\end{exercise}

\begin{exercise}
Demuestre que si $f\colon \Q\to\Q$ es un morfismo de anillos, entonces $f=\id$.  	
\end{exercise}

\begin{exercise}
Demuestre que si $f\colon\R[X_1,\dots,X_n]\to\R$ es un morfismo de anillos tal que la restricción $f|_{\R}$ es la identidad, entonces $f=\operatorname{ev}_{p}$ para algún $p\in\R^n$. 	
\end{exercise}

Para definir cocientes de anillos utilizaremos ideales. Si $I$ es
un ideal de $R$, entonces $(I,+)$ es un subgrupo normal de $(R,+)$, pues el grupo aditivo de $R$ es abeliano. Esto implica que
$R/I$ es un grupo abeliano con la operación
\[
(x+I)+(y+I)=(x+y)+I.
\]
Hasta acá, solamente se necesita que $(I,+)$ sea un subgrupo normal de $(R,+)$. Si queremos que $R/I$ sea un anillo, necesitamos
determinar cómo tiene que ser el producto. La 
estrucuturea de anillo sobre $R/I$ tiene que ser tal que el morfismo canónico   
$\pi\colon R\to R/I$, $\pi(x)=x+I$, sea un morfismo de anillos. Esto nos dice que la multiplicación 
tiene que estar dada por
\[
(x+I)(y+I)=(xy)+I.
\]
Veamos cómo demostrar que esta operación está bien definida. 

Usaremos que $I$ es un ideal. 
Sean $x+I=x_{1}+I$ e $y+I=y_{1}+I$. Queremos demostrar que $xy-x_{1}y_{1}\in I$.
Como $x-x_{1}\in I$ y además $I$ es un ideal a derecha, 
\begin{equation}
xy-x_{1}y=(x-x_{1})y\in I.\label{eq:right_ideal}
\end{equation}
Similarmente, como $y-y_{1}\in I$ y además $I$ es un ideal a izquierda,  
\begin{equation}
x_{1}y-x_{1}y_{1}=x_{1}(y-y_{1})\in I.\label{eq:left_ideal}
\end{equation}
Entonces, al combinar las fórmula~\eqref{eq:left_ideal} con~\eqref{eq:right_ideal} y 
que $I$ es un subgrupo aditivo de $R$, obtenemos que 
la multiplicación de $R/I$ está bien definida pues  
\[
xy-x_{1}y_{1}=xy-x_{1}y+x_{1}y-x_{1}y_{1}\in I.
\]

\begin{theorem}
Sea $R$ un anillo. Si $I$ es un ideal de $R$, entonces existe una 
única estructura de anillo en $R/I$ tal que
$\pi\colon R\to R/I$ es un morfismo de anillos sobreyectivo. 	
\end{theorem}

\begin{proof}
Ya sabemos que las operaciones
\[
(x+I)+(y+I)=(x+y)+I,\qquad(x+I)(y+I)=(xy)+I
\]
están bien definidas gracias a que $I$ es un ideal de $R$. Sabemos también que 
$(R/I,+)$ es un grupo abeliano. Nos queda por demotrar entonces que $R/I$ es un anillo, algo que dejaremos como ejercicio.
\end{proof}

\begin{example}
Sea $R=(\Z/3)[X]$ y sea $I=(2X^2+X+2)$. Sabemos que todo $f\in R$ se escribe como 
\[
f=(2X^2+2X+2)q+r,
\]
donde $q,r\in R$ y $r=0$ o bien $\deg r<2$. Podemos escribir entonces $r=aX+b$ para ciertos $a,b\in\Z/3$. Esto
nos dice que entonces
\[
f+I=( (2X^2+2X+2)q+r)+I=(aX+b)+I.
\]	
Como $a,b\in\Z/3$, tenemos entonces nueve posibilidades. Luego $|R/I|=9$. 

Como ejemplo, calculemos $( (2X+1)+I)( (X+1)+I)$. En efecto, si usamos el algoritmo de división, 
\[
(2X+1)(X+1)=2X^2+3X+1=2X^2+1=(2X^2+X+2)\cdot 1+(2X+2)
\]
y luego $(2X^2+1)+I=(2X+2)+I$. 
\end{example}

Valen además los teoremas de isomorfismos. Dado que no hay mucha diferencia entre lo que se hizo en el caso de grupos y lo que debe hacerse en el caso de anillos, 
enunciaremos los teoremas más importantes y dejaremos las demostraciones como ejercicio. 

\begin{theorem}
Sea $f\colon R\to S$ un morfismo de anillos y $I$ un ideal de $R$ tal que $I\subseteq\ker f$. Existe entonces
un único morfismo $\varphi\colon R/I\to S$ tal que el diagrama
\[
        \xymatrix{
        R
        \ar[d]_\pi
        \ar[r]^f
        & S
        \\
        R/I\ar@{-->}[ur]_{\varphi}
        }
\]
es conmutativo, lo que significa que $\varphi\circ\pi=f$, donde $\pi\colon R\to R/I$ es el morfismo canónico. 
Más aún, $\ker\varphi=\ker f/I$ 
y $\varphi(R/I)=f(R)$. En particular, $\varphi$ es inyectiva si y sólo si $\ker f=I$ y $\varphi$ es sobreyectiva si y sólo si $f$ es sobreyectiva. 
\end{theorem}

\begin{proof}
Queda como ejercicio, ya que es muy similar a la demostración hecha en el caso de grupos.
\end{proof}

Como corolario obtenemos:

\begin{corollary}[primer teorema de isomorfismos]
Si $f\colon R\to S$ es un morfismo de anillos, entonces $R/\ker f\simeq f(R)$. 
\end{corollary}

% \begin{proof}
% Es consecuencia inmediata del teorema anterior.
% \end{proof}

Veamos algunas aplicaciones del primer teorema de isomorfismos. 

% todo: hay que tener las definiciones de idales generados!

\begin{example}
Con el morfismo $\R[X]\to\C$, $f\mapsto f(i)$, se demuestra que 
\[
\R[X]/(X^2+1)\simeq\C.
\]  	
\end{example}

\begin{example}
Con el morfismo $\Z[X]\to(\Z/7)[X]$, $\sum_{i=0}^na_iX^i\mapsto \sum_{i=0}^n\overline{a_i}X^i$, donde $\overline{a}$ es $a$ módulo 7, se demuestra que
\[
\Z[X]/(7)\simeq (\Z/7)[X].
\]	
\end{example}

\begin{example}
Sea $R$ el anillo de funciones continuas $[0,2]\to\R$. Veamos que el conjunto $I=\{f\in R:f(1)=0\}$ es un ideal de $R$ y calculemos
el cociente $R/I$. Para ver que $I$ es un ideal
consideramos la evaluación $\varphi\colon R\to\R$, $\varphi(f)=f(1)$. Sabemos que $\varphi$ es un morfismo de anillos y además 
podemos verificar que
\[
\ker\varphi=\{f\in R:\varphi(f)=0\}=\{f\in R:f(1)=0\}=I.
\]
Como $\varphi$ es sobreyectiva (basta tomar funciones constantes), el teorema de isomorfismos implica que $R/I\simeq\R$.  	
\end{example}

\begin{example}
Sea $R$ el anillo de matrices de la forma $\begin{pmatrix}a&b\\0&a\end{pmatrix}$, donde $a,b\in\Q$. La función 
\[
f\colon R\to\Q,
\quad
\begin{pmatrix}a&b\\0&a\end{pmatrix}\mapsto a,
\]
es un morfismo sobreyectivo de anillos tal que $I=\ker f$ es el ideal formado por las matrices de la forma
$\begin{pmatrix}0&b\\0&0\end{pmatrix}$, donde $b\in\Q$. Entonces $R/I\simeq\Q$.  
\end{example}

\begin{example}
Sea $R=\Z[\sqrt{10}]$ y sea 
\[
I=(2,\sqrt{10})=\{a+b\sqrt{10}:a\equiv 0\bmod 2\}.
\]
La función
\[
f\colon R\to\Z/2,\quad
a+b\sqrt{10}\mapsto a\bmod 2,
\]
es un morfismo sobreyectivo tal que $\ker f=I$. Luego $R/I\simeq\Z/2$. 
\end{example}

\begin{example}
Si $I$ es un ideal de $R$, entonces $M_n(I)$ es un ideal de $M_n(R)$ y además 
$M_n(R)/M_n(I)\simeq M_n(R/I)$. Un cálculo sencillo muestra que
$M_n(I)$ es un subgrupo de $M_n(R)$. 
Además si $a=(a_{ij})\in M_n(R)$ y $y\in M_n(I)$, entonces
\[
(ay)_{ij}=\sum_{k=1}^n a_{ik}y_{kj}\in I
\]
para todo $i,j\in\{1,\dots,n\}$. Similarmente vemos que $ya\in M_n(I)$. Sea $\pi\colon R\to R/I$ el morfismo
canónico y sea $\varphi\colon M_n(R)\to M_n(I)$, $(a_{ij})\mapsto (\pi(a_{ij}))$. Entonces
$\varphi$ es un morfismo sobreyectivo de anillos tal que 
\[
\ker\varphi=\{(a_{ij})\in M_n(R):a_{ij}\in I\text{ para todo $i,j\in\{1,\dots,n\}$}\}. 
\]
Por el primer teorema de isomorfismos, $M_n(R)/M_n(I)\simeq M_n(R/I)$. 
\end{example}

\begin{example}
Vamos a demostrar que $\Z[i]/(1+3i)\simeq\Z/10$. Sea $f$ la composición
\[
\Z\hookrightarrow\Z[i]\xrightarrow{\pi} \Z[i]/(1+3i),
\]
donde $\pi$ es el morfimso canónico. Claramente $f$ es morfismo de anillos por ser composición de morfismos. 

Veamos que $f$ es sobreyectiva. Para eso, alcanza con encontrar un entero $b\in\Z$ tal que $f(b)=i$, es decir: queremos $b\in\Z$ 
tal que $b-i\in (1+3i)$. Observemos que
\[
b-i\in(1+3i)\Longleftrightarrow b-i=(1+3i)(x+yi)=(x-3y)+i(3x+y)
\]
para $x,y\in\Z$. Si tomamos $x=1$, $y=-4$ y $b=13$, entonces vemos que
$f$ es sobreyectiva pues $f(a+13b)=f(a)+f(b)f(13)=a+bi$. 

Calculemos ahora el núcleo de $f$. Afirmamos que $\ker f=(10)$. Primero observamos que, como $10=(1+3i)(1-3i)$, entonces
$f(10)=\pi(1+3i)\pi(1-3i)=0$ y luego $(10)\subseteq\ker f$. Recíprocamente, si $m\in\ker f$, entonces $m\in (1+3i)$, es decir
\[
m=(1+3i)(x+iy)=(x-3y)+i(3x+y)
\]
para ciertos $x,y\in\Z$. Pero entonces $3x+y=0$, lo que implica que $y=-3x$ y que entonces $m=x-3(-x)=10x$, es decir $m\in(10)$. 
En conclusión, por el primer teorema de isomorfismos de anillos, $\Z/(1+3i)\simeq\Z/10$. 
\end{example}

\begin{exercise}
Demuestre que $\Z[i]/(2+3i)\simeq\Z/13$. 
\end{exercise}

\begin{exercise}
Demuestre que no existe un ideal $I$ de $\Z[i]$ tal que $\Z[i]/I\simeq\Z/15$.
\end{exercise}

% pues si $f$ es ese isomorfismo, $f(1)=1$, $f(-1)=-1=14$ y $f(-1)=f(i^2)=f(i)^2$, luego
% $m^2=14$ para algún $m\in\Z/15$, una contradicción.  


%\begin{example}
%Vamos a demostrar que $\Z[i]/(2+3i)\simeq\Z/13$. Sea $f$ la composición
%\[
%\Z\hookrightarrow\Z[i]\xrightarrow{\pi} \Z[i]/(2+3i),
%\]
%donde $\pi$ es el morfimso canónico. Claramente $f$ es morfismo de anillos por ser composición de morfismos. 
%Veamos que $f$ es sobreyectiva...	
%\end{example}
%

\begin{theorem}
\index{Teorema!de la correspondencia}
Si $f\colon R\to S$ es un morfismo sobreyectivo de anillos con $K=\ker f$, existe una correspondencia biyectiva entre los ideales de $R$ que contienen a $K$ y los ideales de $S$. 
La correspondencia está dada por $I\mapsto f(I)$ y $f^{-1}(J)\mapsfrom J$. Más aún, si 
$f(I)=J$, entonces $R/I\simeq S/J$.  
\end{theorem}

\begin{proof}[Bosquejo de la demostración]
Hay que demostrar las siguientes afirmaciones:
\begin{enumerate}
\item $f(I)\subseteq S$ es un ideal.
\item $f^{-1}(J)\subseteq R$ es un ideal que contiene a $K$.
\item $f(f^{-1}(J))=J$ y además $f^{-1}(f(I))=I$.
\item Si $f(I)=J$, entonces $R/I\simeq S/J$.	
\end{enumerate}
Las primeras tres afirmaciones quedarán como ejercicio. 
Demostremos (4). Primero obervamos que la tercera afirmación implica que
$f(I)=J$ si y sólo si $I=f^{-1}(J)$. Sea $\pi\colon S\to S/J$ 
el morfismo canónico y sea 
$g=\pi\circ f$. Como
\[
\ker g=\{x\in R:g(x)=0\}=\{x\in R:f(x)\in J\}=\{x\in R:x\in f^{-1}(J)=I\}=I,
\]
existe un morfismo de anilos $h\colon R/I\to S/J$ tal que $h\circ p=g$, donde $p\colon R\to R/I$ es el morfismo canónico. Dejamos como ejercicio verificar que $h$ es biyectivo.
\end{proof}


\begin{definition}
\index{Ideal!principal}
Un ideal $I$ de $R$ se dice \textbf{principal} si existe $x\in R$ tal que $I=(x)$. 
\end{definition}

Análogamente pueden definirse ideales a izquierda principales e ideales a derecha principales. 

\begin{example}
Todo ideal de $\Z$ es principal.	
\end{example}
 
\begin{exercise}
Sea $I$ un ideal (a izquierda) de $R$. Demuestre que $I=R$ si y sólo si existe $x\in I\cap\mathcal{U}(R)$.	
\end{exercise}

El ejercicio anterior nos permite demostrar que un anillo $R$ es de 
división si y sólo si $R$ admite únicamente dos ideales a izquierda.

\begin{remark}
$u\in\mathcal{U}(R)\Longleftrightarrow (u)=R$.
\end{remark}
