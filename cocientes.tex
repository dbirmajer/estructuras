\chapter{Cocientes}

\begin{definition}
	\index{Subgrupo!normal}
	Sea $G$ un grupo. 
	Un subgrupo $N$ de $G$ se dice \textbf{normal} si $gNg^{-1}=N$ para todo
	$g\in G$.
\end{definition}

\begin{example}
    Si $G$ es un grupo abeliano, todo subgrupo de $G$ es normal en $G$. 
\end{example}

Observemos que un subgrupo $N$ es normal si y sólo si $gN=Ng$ para todo $g\in
G$.  Si $N$ es un subgrupo finito de $G$, entonces $N$ es normal en $G$ si y
sólo si $gNg^{-1}\subseteq N$ para todo $g\in G$. 

\begin{proposition}
	Sea $N$ un subgrupo de $G$. Las siguientes propiedades son equivalentes:
	\begin{enumerate}
		\item $N$ es normal en $G$.
		\item $(gN)(hN)=(gh)N$ para todo $g,h\in G$.
	\end{enumerate}
\end{proposition}

\begin{proof}
	Vamos a demostrar que $(1)\implies(2)$. Sea $g\in G$. Como $gNg^{-1}=N$,
	entonces $(gN)(hN)=g(Nh)N=g(hN)N=(gh)N$. Veamos ahora que $(2)\implies(1)$. Si $g\in G$, entonces
	$gNg^{-1}\subseteq (gN)(g^{-1}N)=(gg^{-1})N=N$. 
\end{proof}

