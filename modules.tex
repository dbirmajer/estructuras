\chapter{Módulos}
\label{modulos}

Un módulo sobre un anillo $R$ será un grupo aditivo junto con un morfismo
de anillos $R\to\End(M)$, que será la acción de $R$ en $M$. Al traducir
qué significa tener tal morfismo de anillos, obtenemos la siguiente definición:

\begin{definition}
\index{Módulo!sobre un anillo}
Sea $R$ un anillo. Un grupo abeliano aditivo $M$ junto con una operación
$R\times M\to M$, $(x,m)\mapsto x\cdot m$, 
será un \textbf{módulo} (a izquierda) sobre $R$ (o también $R$-módulo a izquierda) si
se cumplen las siguientes propiedades:
\begin{enumerate}
\item $(r_1+r_2)\cdot m=x_1\cdot m+x_2\cdot m$ para todo $r_1,r_2\in R$ y $m\in M$.
\item $r\cdot (m_1+m_2)=r\cdot m_1+x\cdot m_2$ para todo $r\in R$ y $m_1,m_2\in M$.
\item $r_1\cdot (r_2\cdot m)=(r_1r_2)\cdot m$ para todo $r_1,r_2\in R$ y $m\in M$.
\item $1\cdot m=m$ para todo $m\in M$.	
\end{enumerate}
\end{definition}

Similarmente uno puede definir módulos a derecha. 

Tabajaremos con módulos a izquierda, por lo tanto los llamaremos simplemente módulos y no habrá peligro de confusión.  Muchas veces no haremos referencia al anillo
sobre el que se define el módulo.

\begin{example}
Si $R$ es un cuerpo, entonces un $R$-módulo 	es un espacio vectorial. 
\end{example}

\begin{example}
Todo grupo abeliano es un $\Z$-módulo.	
\end{example}

\begin{example}
\index{Representación!regular de un anillo}
Si $R$ es un anillo, entonces $R$ es un $R$-módulo con $x\cdot m=xm$. 
Este módulo es la \textbf{representación regular (a izquierda)} de $R$.  
La notación que utilizaremos para este módulo será $M=\prescript{}{R}R$. 
\end{example}

\begin{example}
Si $R$ es un anillo, $R^n=\{(x_1,\dots,x_n):x_1,\dots,x_n\in R\}$ 
es un $R$-módulo con 
$r\cdot (x_1,\dots,x_n)=(rx_1,\dots,rx_n)$. 
\end{example}

\begin{example}
Si $R$ es un anillo, $M_{m,n}(R)$ es un $R$-módulo.
\end{example}

El siguiente ejemplo explica por qué es útil 
pedir que un morfismo $f$ 
de anillos cumpla con la condición $f(1)=1$. 

\begin{example}
\label{exa:f(1)=1}
Si $f\colon R\to S$ es un morfismo de anillos y $M$ es un $S$-módulo con la acción $(s,m)\mapsto sm$, entonces
$M$ es un $R$-módulo con $r\cdot m=f(r)m$ para $r\in R$ y $m\in M$. En efecto,
\begin{align*}
&1\cdot m=f(1)m=1m=m,\\
&r_1\cdot (r_2\cdot m)=f(r_1)(r_2\cdot m)=f(r_1)(f(r_2)m)=(f(r_1)f(r_2))m=f(r_1r_2)m
\end{align*}
para todo $r_1,r_2\in R$ y $m\in M$.	  	
\end{example}

El ejemplo siguiente es particularmente importante.

\begin{example}
Sean $R=\R[X]$, $T\colon\R^n\to\R^n$ una transformación lineal y $M=\R^n$, entonces $M$ es un $R$-módulo con 
\[
\left(\sum_{i=0}^na_iX^i\right)\cdot v=\sum_{i=0}^na_iT^i(v).
\]	
\end{example}

\begin{example}
\index{Producto directo!de módulos}
Si $\{M_i|i\in I\}$ es una familia de módulos, entonces  	
\[
\prod_{i\in I}M_i=\{(m_i)_{i\in I}:m_i\in M_i\text{ para todo $i\in I$}\}
\]
es un módulo con
la operación 
$x\cdot (m_i)_{i\in I}=(x\cdot m_i)_{i\in I}$, 
donde el símbolo $(m_i)_{i\in I}$ denota a la función $I\to M_i$, $i\mapsto m_i$.
Este módulo se conoce como el \textbf{producto directo} de los $M_i$.
\end{example}

\begin{example}
\index{Suma directa!de módulos}
Si $\{M_i|i\in I\}$ es una familia de módulos, entonces  	
\[
\bigoplus_{i\in I}M_i=\{(m_i)_{i\in I}:m_i\in M_i\text{ para todo $i\in I$ y $m_i=0$ salvo finitos $i\in I$}\}
\]
es un módulo con la operación
$x\cdot (m_i)_{i\in I}=(x\cdot m_i)_{i\in I}$. 
Este módulo se conoce como la \textbf{suma directa} de los $M_i$. 
\end{example}

\begin{exercise}
Si $M$ es un módulo, entonces
\begin{enumerate}
\item $0\cdot m=0$ para todo $m\in M$,
\item $x\cdot 0=0$ para todo $x\in R$ y además
\item $-m=(-1)\cdot m$ para todo $m\in M$. 	
\end{enumerate}
\end{exercise}

\begin{example}
$M=\Z/6$ es un $\Z$-módulo tal que $3\cdot 2=0$ pero $3\ne 0$ (en $\Z$) y $2\ne 0$ (en $\Z/6$).  
\end{example}

\begin{definition}
\index{Submódulo}
	Sea $M$ un $R$-módulo. Un subconjunto $S$ de $M$ será un \textbf{submódulo} de $M$ si $(S,+)$ es un subgrupo de $(M,+)$ y además
	$x\cdot s\in S$ para todo $x\in R$ y $s\in S$. 
\end{definition}

\begin{examples}
Sea $M$ un $R$-módulo. 
\begin{enumerate}
\item $\{0\}$ y $M$ son submódulos de $M$.
\item Si $R$ es un cuerpo, $S$ es un submódulo de $M$ si y sólo si $S$ es un subespacio de $M$.
\item Si $R=\Z$, entonces $S$ es un submódulo si y sólo si $S$ es un subgrupo de $M$. 	
\end{enumerate}
\end{examples}

\begin{example}
Si $M=\prescript{}{R}R$, entonces $S\subseteq M$ es un submódulo si y sólo si $S$ es un ideal a izquierda de $R$
\end{example}

\begin{example}
Si $V$ es un espacio vectorial y $T\colon V\to V$ es una transformación lineal, entonces
$V$ es un $K[X]$-módulo con 
\[
\left(\sum_{i=0}^na_iX^i\right)\cdot v=\sum_{i=0}^na_iT^i(v).
\]
Un submódulo $W$ será entonces un subespacio $T$-invariante de $V$, es decir un subspacio vectorial de $V$ 
tal que $T(W)\subseteq W$. 
\end{example}

\begin{exercise}
Demuestre que un subconjunto $S$ de $M$ es un submódulo si y sólo si $r_1s_1+r_2s_2\in S$ para
todo $r_1,r_2\in R$ y $_1,s_2\in S$. 	
\end{exercise}

\begin{exercise}
Si $S$ y $T$ son submódulos de $M$, entonces 
\[
S+T=\{s+t:s\in S,\,t\in T\}
\]
es un submódulo de $M$.
\end{exercise}

\begin{definition}
\index{Morfismo!de módulos}
Sean $M$ y $N$ módulos sobre $R$. 
Un \textbf{morfismo} de módulos es una función $f\colon M\to N$ tal que $f(x+y)=f(x)+f(y)$ y 
$f(r\cdot x)=r\cdot f(x)$ para todo $x,y\in M$ y $r\in R$. 
\end{definition}

\begin{exercise}
Sea $f\colon M\to N$ un morfismo de módulos. 
\begin{enumerate}
\item Si $S$ es un submódulo de $M$, entonces $f(S)$ es un submódulo de $N$.
\item Si $T$ es un submódulo de $N$, entonces $f^{-1}(T)$ es un submódulo de $M$.
\end{enumerate}
\end{exercise}

\index{Núcleo!de un morfismo de módulos}
\index{Monomorfismo!de módulos}
\index{Epimorfismo!de módulos}
\index{Isomorfismo!de módulos}
Si $f\in\Hom_R(M,N)$, se define el \textbf{núcleo} de $f$ como el submódulo 
\[
\ker f=f^{-1}(\{0\})=\{m\in M:f(m)=0\}
\]
de $M$. Diremos que $f$ es un \textbf{monomorfismo} si $f$ es inyectiva, que
es un \textbf{epimorfismo} si $f$ es sobreyectiva y que es un \textbf{isomorfismo} 
si $f$ es biyectiva. 

\begin{exercise}
Sea $f\in\Hom_R(M,N)$. Son equivalentes:
\begin{enumerate}
\item $f$ es monomorfismo.
\item $\ker f=\{0\}$.
\item Para todo módulo $T$ y todo $g,h\in\Hom_R(T,M)	$, $f\circ g=f\circ h\implies g=h$.
\item Para todo módulo $T$ y todo $g\in\Hom(T,M)$, $f\circ g=0\implies g=0$.
\end{enumerate}
\end{exercise}


\begin{example}
	Sea $R=
		\begin{pmatrix}
			\R & 0\\
			0 & \R
		\end{pmatrix}$. 
	Veamos que
	$\begin{pmatrix}
			\R\\
			0
		\end{pmatrix}
		\not\simeq\begin{pmatrix}
			0\\
			\R
		\end{pmatrix}$
	como $R$-módulos, donde la estructura de módulos está dada por la multiplicación usual de matrices. 
	Sea 
	$f\colon\begin{pmatrix}
			0\\
			\R
		\end{pmatrix}
		\to\begin{pmatrix}
			\R\\
			0
		\end{pmatrix}$  
	un isomorfismo de módulos y sea 
	$x_0\in\R\setminus\{0\}$ 
	tal que $f\begin{pmatrix}0\\1\end{pmatrix}=\begin{pmatrix}x_0\\0\end{pmatrix}$. Entonces
	\[
	f\begin{pmatrix}
	0\\
	1\end{pmatrix}
	=f\left(\begin{pmatrix}
	0&0\\
	0&1\end{pmatrix}
	\cdot 
	\begin{pmatrix}
	0\\
	1
	\end{pmatrix}\right)
	=\begin{pmatrix}
	0&0\\
	0&1\end{pmatrix}\cdot f\begin{pmatrix}0\\1\end{pmatrix}
	=\begin{pmatrix}
	0&0\\
	0&1
	\end{pmatrix}
	\cdot 
	\begin{pmatrix}		
	x_0\\
	0
	\end{pmatrix}
	=\begin{pmatrix}
	0\\
	0
	\end{pmatrix},
	\]	
	una contradicción pues $f$ es inyectiva.   
\end{example}

\index{Suma directa!de módulos}
\index{Complemento!de un submódulo}
\index{Sumando directo!de un módulo}
Si $S$ y $T$ son submódulos de $M$, diremos que $M$ es \textbf{suma directa} de $S$ y $T$ si 
$M=S+T$ y además $S\cap T=\{0\}$. La notación que utilizaremos en este caso será $M=S\oplus T$. Observemos que
si $M=S\oplus T$, entonces 
todo $m\in M$ puede escribirse unívocamente como $m=s+t$ para ciertos $s\in S$ y $t\in T$. En efecto,
la descomposición existe gracias a que $M=S+T$. Si $m\in M$ se descompone como
$m=s+t=s_1+t_1$, donde $s,s_1\in S$ y $t,t_1\in T$, entonces
$-s_1+s=t_1-t\in S\cap T=\{0\}$ y luego $s=s_1$ y $t=t_1$.  Si $M=S\oplus T$, el submódulo $S$ es un \textbf{sumando directo} de $M$ y el submódulo $T$ es un \textbf{complemento} para $S$ en $M$.   	
\begin{examples}\
	\begin{enumerate}
	\item Para todo módulo $M$, los submódulos $\{0\}$ y $M$ son sumandos directos de $M$.
	\item Si $M=\R^2$ con la estructura usual de espacio vectorial, entonces todo subespacio de $M$ es un sumando directo. 	
	\end{enumerate}
\end{examples}

\begin{example}
Si $M=\Z$ como $\Z$-módulo, $m\Z$ es sumando directo de $M$ si y sólo si $m\in\{0,1\}$, pues $n\Z\cap m\Z=\{0\}$ si y sólo si $nm=0$.
\end{example}

\begin{proposition}
\index{Proyector}
Un módulo $N$ es isomorfo a un sumando directo del módulo $M$ si y sólo si 
existen morfismos $i\colon N\to M$ y $p\colon M\to N$ 
tales que $p\circ i=\id_N$. En este caso, $M=\ker p\oplus i(N)$.  
\end{proposition}

\begin{proof}
	Supongamos que $N$ es isomorfo a un sumando directo de $M$, es decir $M=S\oplus T$ y sea $s\colon N\to S$ un isomorfismo. Para
	cada $m\in M$ existen únicos $s\in S$ y $t\in T$ tales que $m=s+t$. Definimos entonces el epimorfismo 
	$q\colon M\to S$, $m\mapsto s$. Observemos que $q(m)=m$ si y sólo si $m\in S$, 
	es decir que $q$ es un \textbf{proyector} de $M$ sobre $S$ con respecto a $T$. 
	Definimos además el morfismo 
	$i\colon N\to M$, $n\mapsto s(n)$, y el morfismo $p\colon M\to N$, $m\mapsto s^{-1}(q(m))$. Como $s(n)\in S$, 
	\[
	p(i(n))=p(s(n))=s^{-1}(q(s(n)))=s^{-1}(s(n))=n
	\]
	para todo $n\in N$. 
	
	Demostremos ahora la recíproca. Afirmamos que $i$ es monomorfismo: si $i(n)=0$, entonces
	$n=p(i(n))=p(0)=0$. Luego $i\colon N\to i(N)$ es un isomorfismo. 
	Veamos ahora que $M=\ker p\oplus i(N)$. Si $m\in M$, entonces
	\[
	m=m-i(p(m))+i(p(m))\in\ker p+i(N),
	\] 
	pues
	$p(m-i(p(m)))=p(m)-p(m)=0$. Si $m\in\ker p\cap i(N)$, entonces $0=p(m)$ y además $m=i(n)$ para algún $n\in N$. Entonces
	$0=p(m)=p(i(n))=n$ y luego $m=0$.      
\end{proof}

La \textbf{suma directa} de submódulos puede extenderse a un número finito de sumandos. 
Si $S_1,\dots,S_n$ son submódulos de $M$, diremos que
$M=S_1\oplus\cdots\oplus S_n$ si todo $m\in M$ puede escribirse unívocamente
como $m=s_1+\cdots+s_n$ para ciertos $s_1\in S_1,\dots,s_n\in S_n$. 

\begin{exercise}
Demuestre que $M=S_1\oplus\cdots\oplus S_n$ si y sólo si
$M=S_1+\cdots+S_n$ y además 
\[
S_i\cap\left(\sum_{j\ne i}S_j\right)=\{0\}
\]	
para todo $i\in\{1,\dots,n\}$.
\end{exercise}

\begin{exercise}
Si $\{S_i:i\in I\}$ es una familia de submódulos de $M$, entonces $\cap_{i\in I}S_i$ es un submódulo de $M$.
\end{exercise}

\begin{example}
Sea $T\colon\R^2\to\R^2$, $T(x,y)=(0,y)$ y sea $M=\R^2$ con la estructura de $\R[X]$-módulo dada por
\[
\left(\sum_{i=0}^n a_iX^i\right)\cdot (x,y)=\sum_{i=0}^n a_iT(x,y).
\]
Vamos a demostrar que
$\{0\}$, $M$, $\R\times\{0\}$ y $\{0\}\times\R$ son los únicos submódulos de $M$. Si $N$ es un submódulo no nulo de $M$, sea
$(x_0,y_0)\in N\setminus\{(0,0)\}$. Si $(x,y)\in M$ es tal que $xy\ne 0$, entonces
\[
\left(\frac{x}{x_0}+\left(\frac{y}{y_0}-\frac{x}{x_0}\right)X\right)\cdot (x_0,y_0)=(x,y) 
\]
y luego $N=M$. Si $y_0=0$, entonces $N=\R\times\{0\}$, pues $\frac{x}{x_0}\cdot (x_0,0)=(x,0)$. Si $x_0=0$, entonces
$N=\{0\}\times\R$, pues 
$\frac{y}{y_0}\cdot (0,y_0)=(0,y)$ 
\end{example}


\begin{example}
Sea $M=\R^2$ como $\R[X]$-módulo con la acción
\[
\left(\sum_{i=0}^n a_iX^i\right)\cdot (x,y)=\sum_{i=0}^n a_iT^i(x,y),
\]
donde $T\colon M\to M$, $T(x,y)=(y,x)$. 
Vamos a calcular todos los submódulos de $M$. 
Si $N\subseteq M$ es un submódulo entonces $N$ es un espacio vectorial real. Supongamos que $N\ne\{(0,0)\}$ y que $N\ne\R^2$. Como entonces $\dim N=1$, 
sea $\{(a_0,b_0)\}$ una base de $N$. Como $N$ es un submódulo,
$(b_0,a_0)=X\cdot (a_0,b_0)\in N$. En particular, existe $\lambda\in\R$ tal que $(b_0,a_0)=\lambda (a_0,b_0)$. Como $(a_0,b_0)\ne(0,0)$, sin perder generalidad
podemos suponer que $a_0\ne 0$. Esto implica que
$\lambda^2 a_0=\lambda (\lambda a_0)=\lambda b_0=a_0$ y entonces $\lambda^2=1$. Si $\lambda=1$, entonces
$a_0=b_0$. Si $\lambda=-1$, entonces $a_0=-b_0$. En conclusión, $N$ está generado por $(1,1)$ o por $(1,-1)$.  
\end{example}

\begin{example}
Si $V$ es un espacio vectorial y $T\colon V\to V$ es una transformación lineal, entonces
$V$ es un $K[X]$-módulo con 
\[
\left(\sum_{i=0}^n a_iX^i\right)\cdot v=\sum_{i=0}^n a_iT^i(v).
\]
Si $g\colon V\to V$ es un morfismo de $K[X]$-módulos, entonces $g$ conmuta con $T$, pues
\[
(g\circ T)(v)=g(T(v))=g(X\cdot v)=X\cdot g(v)=T(g(v))=(T\circ g)(v)
\]
para todo $v\in V$.
%Sea $K$ un cuerpo y sea $M$ un $K[X]$-módulo. Entonces $M$ es un $K$-espacio vectorial con la restricción 
%de la acción de $K[X]$ en $M$. 
%
%Si $f\colon M\to M$, $f(m)=X\cdot m$, entonces $f$ es una transformación lineal. 
%
%Si $V$ es un $K$-espacio vectorial y $f\colon V\to V$ es una transformación lineal, definimos
%$p\cdot v=p(f)(v)$ para $p\in K[X]$ y $v\in V$. Entonces $M$ es un $K[X]$-módulo si y sólo si $M$ es un $K$-espacio vectorial
%y $x\cdot v=f(v)$ para alguna transformación lineal $f\colon V\to V$. 
%
%Además, si $g\colon M\to M$ es un morfismo de $K[X]$-módulos, entonces $g\colon M\to M$ es una transformación lineal que conmuta con $f$
%pues 
%\[
%g(f(v))=g(X\cdot v)=X\cdot g(v)=f(g(v)).
%\]
\end{example}

Sea $\Hom_R(M,N)$ el conjunto de morfismos de módulos $M\to N$. 

\begin{example}
\label{exa:Hom}
Veamos que $\Hom_R(M,N)$ es un $Z(R)$-módulo. Si $f\in\Hom_R(M,N)$ y $r\in R$, definimos la función 
$r\cdot f\colon M\to N$, $m\mapsto f(r\cdot m)$, que es un morfismo de grupos abelianos. 
Si $r,s\in Z(R)$, entonces $f$ es un morfismo pues
\begin{align*}
(r\cdot (s\cdot f))(m)&=
(s\cdot f)(r\cdot m)\\
&=f(s\cdot (r\cdot m))=f((sr)\cdot m)=f( (rs)\cdot m)=((rs)\cdot f)(m). 
\end{align*} 	
\end{example}

\index{Módulo!cociente}
\index{Epimorfismo!canónico de módulos}
Si $M$ es un módulo y $N$ es un submódulo, entonces $M/N$ es un grupo abeliano y el morfismo
canónico $\pi\colon M\to M/N$, $x\mapsto x+N$, es un morfismo sobreyectivo de grupos. Veamos que 
el \textbf{cociente} $M/N$ es un módulo con 
\[
r\cdot (x+N)=(r\cdot x)+N,
\]
donde $r\in R$ y $x\in M$. Para esto, tenemos que ver la buena definición de la acción en $M/N$. Si $x+N=y+N$, entonces, como
$x-y\in N$, se tiene que 
\[
r\cdot x-r\cdot y=r\cdot (x-y)\in N,
\]
es decir $r\cdot (x+N)=r\cdot (y+N)$. Dejamos como ejercicio demostrar que la función $\pi\colon M\to M/N$, $x\mapsto x+N$, es
un morfismo sobreyectivo de módulos. 

\begin{example}
Si $R=M=\Z$ y $N=2\Z$, entonces $M/N\simeq\Z/2$. 
\end{example}

\begin{example}
Sea $R$ un anillo conmutativo. Veamos que $M\simeq\Hom_R(\prescript{}{R}R,M)$. 
Como $R$ es un anillo conmutativo, $\Hom_R(\prescript{}{R}R,M)$ es un módulo, ver ejemplo~\ref{exa:Hom}.
Sea $\varphi\colon M\to\Hom_R(\prescript{}{R}R,M)$, $m\mapsto f_m$, donde $f_m\colon R\to M$, $r\mapsto r\cdot m$. Para ver que $\varphi$ está bien definida
hay que observar que $\varphi(m)\in\Hom_R(\prescript{}{R}R,M)$, es decir 
\[
f_m(r+s)=(r+s)\cdot m=r\cdot m+s\cdot m,\quad
f_m(rs)=(rs)\cdot m=r\cdot (s\cdot m)=r\cdot f_m(s).
\]

Para ver que $\varphi$ es morfismo primero vemos que $\varphi(m+n)=\varphi(m)+\varphi(n)$ para todo $m,n\in M$, pues 
\begin{align*}
\varphi(m+n)(r)&=f_{m+n}(r)=r\cdot (m+n)\\
&=r\cdot m+r\cdot n=f_m(r)+f_n(r)=\varphi(m)(r)+\varphi(n)(r).
\end{align*}
Además $\varphi(r\cdot m)=r\cdot\varphi(m)$ para 
todo $r\in R$ y $m\in M$, pues 
\begin{align*}
%\varphi(r\cdor m)(s)=f_{r\cdot m}(s)=(r\cdot \varphi(m))(s).
%
\varphi(r\cdot m)(s)&=f_{r\cdot m}(s)
%(r\cdot f_m)(s)=f_m(rs)\\
=s\cdot (r\cdot m)
=(sr)\cdot m\\
&=(rs)\cdot m=f_m(rs)=\varphi(m)(rs)=(r\cdot\varphi(m))(s).
%&=(rs)\cdot m=(sr)\cdot m=s\cdot (r\cdot m)\\
%&=f_{r\cdot m}(s)=\varphi(r\cdot m)(s)=(r\cdot \varphi)(m)(s). 
\end{align*}

Falta ver que $\varphi$ es un isomorfismo. 
Veamos primero que $\varphi$ es monomorfismo. Si $\varphi(m)=0$, entonces $r\cdot m=\varphi(m)(r)=0$ para todo $r\in R$. En particular, 
$m=1\cdot m=0$. 	Veamos ahora que $\varphi$ es epimorfismo. Si $f\in\Hom_R(\prescript{}{R}R,M)$, sea $m=f(1)$. Entonces $\varphi(m)=f$ pues
$\varphi(m)(r)=r\cdot m=r\cdot f(1)=f(r)$.
\end{example}

Tal como hicimos con grupos, puede demostrarse que si $M$ es un módulo y $N$ es un submódulo de $M$, el par
$(M/N,\pi\colon M\to M/N)$ tiene las siguientes propiedades:
\begin{enumerate}
\item $N\subseteq \ker \pi$.
\item Si $f\colon M\to T$ es un morfismo tal que $N\subseteq \ker f$, entonces existe un único morfismo $\varphi\colon M/N\to T$ tal que $\varphi\circ\pi =f$.  
\end{enumerate}

\index{Teoremas de isomorfismos!para módulos}
Recordemos que si $S$ y $T$ son submódulos de un módulo $M$, entonces
$S\cap T$ y $S+T=\{s+t:s\in S,\,t\in T\}$ son ambos submódulos de $M$. 
Se tienen entonces los teoremas de isomorfismos. 
\begin{enumerate}
	\item Si $f\in\Hom_R(M,N)$, entonces $M/\ker f\simeq f(M)$.
	\item Si $T\subseteq N\subseteq M$ son submódulos, entonces 
	\[
	\frac{M/T}{N/T}\simeq M/N
	\]
	\item Si $S$ y $T$ son submódulos de $M$, entonces $(S+T)/S\simeq T/(S\cap T)$. 
\end{enumerate}

\begin{example}
Si $R$ es un cuerpo y $V$ es un $R$-módulo, entonces $V$ es un espacio vectorial. 
Si $S$ y $T$ son subespacios de $V$, entonces son submódulos de $V$. 
El segundo teorema de isomorfismos nos dice que $(S+T)/T\simeq S/(S\cap T)$, 
un isomorfismo de espacios vectoriales. Al aplicar dimensión, 
\[
\dim(S+T)-\dim T=\dim(S)-\dim(S\cap T).
\]
\end{example}

\begin{example}
Si $S$ es un sumando directo de $M$ y $T$ es un complemento para $S$, entonces $T\simeq M/S$, pues
\[
M/S=(S\oplus T)/S\simeq T/(S\cap T)=T/\{0\}\simeq T
\]
por el segundo teorema de isomorfismos. Luego todos los complementos
de $S$ en $M$ serán isomorfos.  	
\end{example}

Puede demostrarse además el teorema de la correspondencia, que afirma que existe una correspondencia biyectiva 
entre los submódulos de $M/N$ y los submódulos de $M$ que contienen a $N$. La correspondencia está dada
por $S\mapsto \pi^{-1}(S)$ y $\pi(T)\mapsfrom T$. 

\begin{exercise}
Sea $f\in\Hom_R(M,N)$. Son equivalentes: 
\begin{enumerate}
\item $f$ es epimorfismo.
\item $N/f(M)\simeq\{0\}$. 
\item Para todo módulo $T$ y todo $g,h\in\Hom_R(N,T)$, $g\circ f=h\circ f\implies g=h$.
\item Para todo módulo $T$ y todo $g\in\Hom_R(N,T)$, $g\circ f=0\implies g=0$. 
\end{enumerate}
\end{exercise}

\begin{exercise}
\label{xca:mod_iso_max}
    Sea $R$ un anillo conmutativo y sean $M_1$ y $M_2$ ideales
    maximales de $R$. Pruebe que $R/M_1\simeq R/M_2$ como $R$-módulos 
    si y sólo si existe $r\in R\setminus M_2$ tal que $rM_1\subseteq M_2$. 
\end{exercise}


