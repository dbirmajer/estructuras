\chapter{El teorema de Maschke}

\index{Representación!de un grupo}
Si $G$ es un grupo finito, un morfismo de grupos $G\to\GL(V)$, donde $V$ es un espacio
vectorial complejo de dimensión finita, se dice una 
\textbf{representación} de $G$. Si el espacio vectorial $V$ tiene dimensión $n$, al fijar una base 
para $V$ podemos considerar $G\to\GL(V)\simeq\GL_n(\C)$.
 
\begin{example}
Como $\Sym_3=\langle (12),(123)\rangle$, la función $\rho\colon \Sym_3\to\GL_3(\C)$,
\[
(12)\mapsto\begin{pmatrix}
0 & 1 & 0\\
1 & 0 & 0\\
0 & 0 & 1
\end{pmatrix},\quad
(123)\mapsto\begin{pmatrix}
0 & 0 & 1\\
1 & 0 & 0\\
0 & 1 & 0
\end{pmatrix}
\] 
es una representación de $\Sym_3$. 
\end{example}

\begin{example}
Como el grupo de cuaterniones $Q_8=\{1,-1,i,-i,j,-j,k,-k\}$ está generado por $\{i,j\}$, 
la función $\rho\colon G\to\GL_2(\C)$, 
\[
i\mapsto\begin{pmatrix}
i & 0\\
0 & i
\end{pmatrix},
\quad
j\mapsto\begin{pmatrix}
0 & -1\\
1 & 0	
\end{pmatrix}
\]
es una representación de $Q_8$.
\end{example}

\begin{example}
Sea $G=\langle g\rangle$ cíclico de orden seis. 
La función $\rho\colon G\to\GL_2(\C)$, 
\[
g\mapsto
\begin{pmatrix}
1&1\\
-1&0
\end{pmatrix}
\] 
es una representación del grupo $G$ cíclico de orden seis. 
\end{example}

\begin{example}
Sea $G=\langle g\rangle$ cíclico de orden cuatro. 
La función $\rho\colon G\to\GL_2(\C)$, 
\[
g\mapsto
\begin{pmatrix}
0&-1\\
1&0
\end{pmatrix}
\] 
es una representación del grupo $G$ cíclico de orden cuatro. 
\end{example}

Observemos que existe una correspondencia biyectiva 
\[
\{\text{representaciones de $G$}\}\leftrightarrow\{\text{$\C[G]$-módulos de dimensión finita}\}.
\]
Si $\rho\colon G\to\GL(V)$ es una representación, entonces 
$V$ es un $\C[G]$-módulo con
\[
\left(\sum_{g\in G}\lambda_gg\right)\cdot v=\sum_{g\in G}\lambda_g\rho(g)(v).
\]
Recíprocamente, si $V$ es un $\C[G]$-módulo, entonces $\rho\colon G\to\GL(V)$, 
$\rho(g)(v)=g\cdot v$, es una representación de $G$ en $V$. Puede verificarse que estas
construcciones son una la inversa de la otra.  

\begin{definition}
\index{Módulo!simple}
\index{Módulo!irreducible}
Un módulo $M$ se dice \textbf{simple} (o irreducible) si $M\ne\{0\}$ y $M$ no tiene
submódulos propios no triviales.  
\end{definition}

\begin{example}
Si $A$ es un álgebra, vimos que todo $A$-módulo es un espacio vectorial. Los módulos
de dimensión uno serán entonces módulos simples.
\end{example}

\begin{example}
Sea $G=\langle g\rangle$ cíclico de orden tres y sea $M=\R^3$ con la estructura de $\R[G]$-módulo 
dada por $g\cdot (x,y,z)=(y,z,x)$. El conjunto
\[
N=\{(x,y,z)\in\R^3:x+y+z=0\}
\]
es un submódulo de $M$. Veamos que $N$ es simple. Si $N$ contiene un submódulo no trivial $S$, 
sea $(x_0,y_0,z_0)\in S\setminus\{(0,0,0)\}$. Como $S$ es un submódulo, 
\[
(y_0,z_0,x_0)=g\cdot (x_0,y_0,z_0)\in S.
\]
Afirmamos
que $\{(x_0,y_0,z_0),(y_0,z_0,x_0)\}$ es un conjunto linealmente independiente. 	Si existe $\lambda\in\R$ 
tal que $\lambda(x_0,y_0,z_0)=(y_0,z_0,x_0)$, entonces $x_0=\lambda^3 x_0$. Como $x_0=0$ implica que 
$y_0=z_0=0$, entonces $\lambda=1$. En particular, $x_0=y_0=z_0$, una contradicción, pues $x_0+y_0+z_0=0$. 
Luego $\dim S=2$ y entonces
$S=N$. 
\end{example}

\begin{example}
Sea $M=\R^2$ con la estructura de $\R[X]$-módulo dada por
\[
\left(\sum_{i=0}^n a_iX^i\right)\cdot m=\sum_{i=0}^n a_iT^i(m).
\]
donde $T\colon\R^2\to\R^2$, $T(x,y)=(y,-x)$. 
Veamos que $M$ es simple. Si $N$ es un submódulo no nulo, sea $(x_0,y_0)\in N\setminus\{(0,0)\}$. Si $(x,y)\in M$, veamos que 
existen
$\alpha,\beta\in\R$ tales que
\[
(\alpha+\beta X)\cdot (x_0,y_0)=(x,y).
\]
En efecto, basta tomar 
\[
\alpha=\frac{x_0x+y_0y}{x_0^2+y_0^2},\quad
\beta=\frac{y_0x-x_0y}{x_0^2+y_0^2},
\]
pues
\begin{align*}
(\alpha+\beta X)\cdot (x_0,y_0)&=\alpha(x_0,y_0)+\beta\cdot (X\cdot (x_0,y_0))\\
&=(\alpha x_0,\alpha y_0)+(\beta y_0,-\beta x_0)\\
&=(\alpha x_0+\beta y_0,\alpha y_0-\beta x_0)\\
&=(x,y).	
\end{align*}
\end{example}

\begin{definition}
\index{Módulo!semisimple}
\index{Módulo!completamente reducible}
Un módulo $M$ se dice \textbf{semisimple} (o completamente reducible) 
si es suma directa de módulos simples.
\end{definition}

Vimos en el capítulo anterior que si $M$ es un módulo, se dice que un submódulo $S$ de $M$ se complementa en $M$ si existe un submódulo $T$ de $M$ tal que $M=S\oplus T$.

\begin{lemma}
Si $p\colon M\to M$ es un morfismo tal que $p^2=p$, entonces 
\[
M=\ker p\oplus p(M).
\]
\end{lemma}

\begin{proof}
	Como $p$ es un morfismo, $\ker p$ y $p(M)$ son submódulos de $M$. 
	Para ver que $M=\ker p+p(M)$ alcanza con observar que todo $m\in M$ puede escribirse como
	$m=(m-p(m))+p(m)$ 
	y que $m-p(m)\in\ker p$ pues 
	\[
	p(m-p(m))=p(m)-p^2(m)=p(m)-p(m)=0. 
	\]
	Veamos ahora que $\ker p\cap p(M)=\{0\}$. Si $m\in\ker p\cap p(M)$, escribimos $m=p(m_1)$ para algún $m_1\in M$. Como entonces 
	$0=p(m)=p^2(m_1)=m_1$, se concluye que $m=0$.   
\end{proof}

\index{Proyección}   
Recordemos que 
una \textbf{proyección} (o proyector) de un módulo $M$ es un   
morfismo $p\colon M\to M$ tal que $p^2=p$. 
 
\begin{lemma}
Si $A$ es un álgebra y $M$ es un $A$-módulo de dimensión finita tal que
todo submódulo de $M$ se complementa, entonces $M$ es semisimple.
\end{lemma}

\begin{proof}
Procederemos por inducción en $\dim M$. Si $M=\{0\}$ el resultado es trivial. Si $M\ne\{0\}$, 
sea $S$ un submódulo no nulo de $M$ de dimensión minimal. En particular, $S$ es simple. Por hipótesis sabemos que existe un submódulo $T$ de $M$ tal que $M=S\oplus T$. Como $\dim T<\dim M$, la hipótesis inductiva implica que $T$ es suma directa de módulos simples. Luego $M$ también lo es. 
\end{proof}

El lema anterior vale también para módulos arbitrarios sobre anillos. Sin embargo, la demostración 
requiere el uso del lema de Zorn.

\begin{example}
Sea $R=M_2(\C)$ y sea $M=\prescript{}{R}R$. Los subconjuntos
\[
I=\begin{pmatrix}
\C&0\\
\C&0
\end{pmatrix},\quad
J=\begin{pmatrix}
0&\C\\
0&\C
\end{pmatrix}
\]
son submódulos de $M$ tales que $M\simeq I\oplus J$. Veamos que $M$ es semisimple, es decir que
$I$ y $J$ son simples. 

Si $S$ es un submódulo no nulo de $I$, sea 
$\begin{pmatrix}
a&0\\
c&0
\end{pmatrix}\in S$ no nulo. Supongamos que 
$a\ne 0$, el caso $c\ne 0$ es similar. Entonces
\[
\begin{pmatrix}
a^{-1} & 0\\
0 & 0\end{pmatrix}
\begin{pmatrix}
a&0\\
c&0
\end{pmatrix}
=\begin{pmatrix}
1&0\\
0&0
\end{pmatrix}\in S.
\]
Análogamente se demuestra que $\begin{pmatrix}0&0\\1&0\end{pmatrix}\in S$. Luego $S=I$ y entonces $I$ es simple.  
La misma técnica nos permite demostrar que $J$ es simple.
\end{example}

Es importante observar que si $A$ es un álgebra y $M$ es un módulo
semisimple de dimensión finita, entones $M$ es suma directa de finitos simples. 

\begin{theorem}[Maschke]
\index{Teorema!de Maschke}
Sea $G$ un grupo finito y sea $M$ un $\C[G]$-módulo de dimensión finita. Entonces
$M$ es semisimple.
\end{theorem}

\begin{proof}
Gracias al lema anterior, alcanza con demostrar que todo submódulo $S$ de $M$ se complementa. 
Como, en particular, $S$ es un subespacio de $M$, existe un subespacio $T_0$ de $M$ 
tal que $M=S\oplus T_0$ (como espacios vectoriales). Vamos a usar el espacio vectorial
$T_0$ para construir un submódulo $T$ de $M$ que complementa a $S$. Como $M=S\oplus T_0$, 
cada $m\in M$ puede escribirse unívocamente como $m=s+t_0$ para ciertos $s\in S$ y $t_0\in T$. 
Podemos definir entonces la transformación lineal 
\[
p_0\colon M\to S,\quad
p_0(m)=s,
\]
donde $m=s+t_0$ con $s\in S$ y $t_0\in T$. 
Observemos que si $s\in S$, entonces $p_0(s)=s$. En particular, $p_0^2=p_0$ pues
$p_0(m)\in S$. 

El problema 
es que $p_0$ no es, en general, un morfismo de $\C[G]$-módulos. Promediamos
sobre el grupo $G$ para conseguir un morfismo de grupos: Sea 
\[
p\colon M\to S,\quad
p(m)=\frac{1}{|G|}\sum_{g\in G}g^{-1}\cdot p_0(g\cdot m).
\]

Primero demostramos que $p$ es un morfismo de $\C[G]$-módulos. Alcanza con ver que
$p(g\cdot m)=g\cdot p(m)$ para todo $g\in G$ y $m\in M$. En efecto,
\[
p(g\cdot m)=\frac{1}{|G|}\sum_{h\in G}h^{-1}\cdot p_0(h\cdot (g\cdot m))
=\frac{1}{|G|}\sum_{h\in G}(gh^{-1})\cdot p_0(h\cdot m)=g\cdot p(m).
\]

Veamos ahora que $p(M)=S$. La inclusión $\subseteq$ es trivial, pues $S$ es un submódulo de $M$ 
y además $p_0(M)\subseteq S$. Recíprocamente, si $s\in S$, entonces $g\cdot s\in S$, pues
$S$ es un submódulo. Luego 
$s=g^{-1}\cdot (g\cdot s)=g^{-1}\cdot p_0(g\cdot s)$ y en consecuencia
\[
s=\frac{1}{|G|}\sum_{g\in G}g^{-1}\cdot (g\cdot s)=\frac{1}{|G|}\sum_{g\in G}g^{-1}\cdot (p_0(g\cdot s))=p(s).
\]
Como $p(m)\in S$ para todo $m\in M$, entonces $p^2(m)=p(m)$, es decir que $p$ es un proyector en $S$. Luego $S$ se complementa en $M$, es decir $M=S\oplus\ker(p)$.
\end{proof} 

La misma demostración del teorema de Maschke vale para el álgebra de grupo real o racional. 
La descomposición de un módulo sobre el álgebra de grupo dependerá
fuertemente del cuerpo sobre el que se trabaje. 

\begin{example}
Sea $G=\langle g\rangle$ el grupo cíclico de orden cuatro y sea $\rho_g=\begin{pmatrix}
0&-1\\
1&0\end{pmatrix}$. 
Sea $M=\C^{2\times 1}$ con la estructura de $\C[G]$-módulo dada por 
\[
g\cdot\begin{pmatrix}u\\v\end{pmatrix}
%\begin{pmatrix}0&-1\\1&0\end{pmatrix}\begin{pmatrix}u\\v\end{pmatrix}
=\begin{pmatrix}-v\\u\end{pmatrix},
\]
es decir, si $a,b,c,d\in\C$, entonces 
\[
(a1+bg+cg^2+dg^3)\cdot\begin{pmatrix}u\\v\end{pmatrix}
=\begin{pmatrix}
(a-d)u+(c-b)v\\
(1-b)u+(a-d)v
\end{pmatrix}.
\]
Sabemos por el teorema de Maschke que $M$ es semisimple. Veamos cómo descomponer el módulo $M$ como suma directa de simples. 
Como $\dim M=2$, tendremos que $M$ es suma directa de dos submódulos de dimensión uno. 
Observemos que si $S$ es un submódulo tal que $\{0\}\subsetneq S\subsetneq M$, 
entonces $\dim S=1$. Además 
\[
S=\left\{\lambda\begin{pmatrix}
u_0\\
v_0
\end{pmatrix}:\lambda\in\C\right\}
\text{ es un submódulo de $M$}
\Longleftrightarrow
\begin{pmatrix}
u_0\\
v_0
\end{pmatrix}
\text{ es autovector de $\rho_g$}.
\]
Como la matriz $\rho_g$ tiene polinomio característico $X^2+1$, se sigue 
que  
$\begin{pmatrix}
i\\
1\end{pmatrix}$ es autovector de $\rho_g$ de autovalor $-i$ y que
$\begin{pmatrix}
-i\\
1\end{pmatrix}$ es autovector de autovalor $i$. 
Luego $M$ se descompone en suma directa de simples como 
\[
M=\C\begin{pmatrix}
i\\
1\end{pmatrix}
\oplus
\C
\begin{pmatrix}
-i\\
1\end{pmatrix}
\]
\end{example}

Observar que en ejemplo anterior pudimos descomponer a la matriz $\rho_g$ 
gracias a la existencia de autovectores, algo 
que no pasaría si consideramos módulos sobre el álgebra de grupo real.   

\begin{example}
Sea $G=\langle g\rangle$ el grupo cíclico de orden cuatro y sea $\rho_g=\begin{pmatrix}
0&-1\\
1&0\end{pmatrix}$. 
Sea $M=\R^{2\times 1}$ con la estructura de $\R[G]$-módulo dada por 
\[
g\cdot\begin{pmatrix}u\\v\end{pmatrix}
=\begin{pmatrix}-v\\u\end{pmatrix}.
\]
Tal como hicimos en el ejemplo anterior, 
como $\dim M=2$, si $S$ es un submódulo de $M$ tal que $\{0\}\subsetneq S\subsetneq M$, entonces $\dim S=1$. 
Pero como $\rho_g$ no tiene autovectores reales, $M$ no tendrá submódulos de dimensión uno.  
En consecuencia, $M$ es simple como $\R[G]$-módulo. 
\end{example}