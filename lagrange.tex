\chapter{El teorema de Lagrange}

Sean $G$ un grupo y $H$ un subgrupo de $G$. Diremos que dos elementos $x,y\in
G$ son equivalentes a izquierda módulo $H$ si $x^{-1}y\in H$.  
Usaremos la siguiente notación:
\[
x\equiv y\bmod
H\Longleftrightarrow x^{-1}y\in H.
\]  

\begin{exercise}
	Demuestre que hemos definido una relación de equivalencia. 
	Esto significa que se tienen las siguientes propiedades:
	\begin{enumerate}
	\item $x\equiv x\bmod H$ para todo $x$.
	\item Si $x\equiv y\bmod H$, entonces $y\equiv x\bmod H$.
	\item Si $x\equiv y\bmod H$ y además $y\equiv z\bmod H$, entonces $x\equiv z\bmod H$.  	
	\end{enumerate}
\end{exercise}

Las clases de equivalencia de esta relación módulo $H$ 
son los conjuntos de la forma $xH=\{xh:h\in H\}$
pues la clase de un cierto elemento $x\in G$ es el conjunto
\[
	\{y\in G:x\equiv y\bmod H\}=\{y\in G:x^{-1}y\in H\}=\{y\in G:y\in xH\}=xH.
\]
El conjunto $xH$ se 
llama \textbf{coclase a izquierda} de $H$ en $G$. 

\begin{proposition}
Sean $G$ un grupo y $H$ un subgrupo de $G$.  
\begin{enumerate}
\item Si $xH\cap yH\ne\emptyset$, entonces $xH=yH$. 	
\item El grupo $G$ puede descomponerse como unión disjunta
de distintas coclases a izquierda de $H$. 
\end{enumerate}
\end{proposition}

\begin{proof}
Demostremos la primera afirmación. 
Si $g\in xH\cap yH$, escribimos
$g=xh$ para algún $h\in H$ y entonces
\[
gH=(xh)H=x(hH)=xH.
\]
Similarmente, $gH=yH$. En consecuencia, $xH=yH$.

La segunda afirmación se obtiene inmediatamente de la primera.  
\end{proof}

Podríamos haber definido coclases a derecha mediante la relación $x\equiv
y\bmod H$ si y sólo si $xy^{-1}\in H$. En este caso, las clases de equivalencia
serían los conjuntos $Hx$ con $x\in X$. $Hx$ se llama \textbf{coclase a derecha}
de $H$ en $G$. 

\begin{proposition}
	Si $H$ es un subgrupo de $G$, entonces $|Hx|=|H|=|xH|$ para todo $x\in G$. 
\end{proposition}

\begin{proof}
	Sea $x\in G$. La función $H\to Hx$, $h\mapsto hx$, es una biyección con
	inversa $hx\mapsto h$. Análogamente se demuestra que la función $H\to xH$,
	$h\mapsto xh$, es una biyección. 
\end{proof}

La función
\[
	\{\text{coclases a derecha de $H$ en $G$}\}\to\{\text{coclases a izquierda de $H$ en $G$}\}
\]
dada por $Hx\mapsto x^{-1}H$ es una biyección pues 
\[
	Hx=Hy
	\Longleftrightarrow xy^{-1}\in H
	\Longleftrightarrow (x^{-1})^{-1}y^{-1}\in H
	\Longleftrightarrow x^{-1}H=y^{-1}H.
\]
En particular, la cantidad de coclases a derecha de $H$ en $G$ coincide con la
cantidad de coclases a izquierda de $H$ en $G$.

\begin{example}
Si $G=\Z$ y $S=n\Z$, entonces
\[
a+S=\{a+nq:q\in\Z\}=\{k\in\Z:k\equiv a\bmod n\}.
\]	
\end{example}

\begin{example}
Los subgrupos de $\Sym_3$ son $\{\id\}$, $\Sym_3$, los subgrupos $\langle(12)\rangle$, $\langle(13)\rangle$ y $\langle(23)\rangle$ de orden dos 
y el subgrupo $\langle(123)\rangle=\{\id,(123),(132)\}$ de orden tres.  	Si $H=\langle(12)\rangle=\{\id,(12)\}$, entonces
\begin{align*}
&H=(12)H=\{\id,(12)\},\\
&(123)H=(13)H=\{(13),(123)\},\\
&(132)H=(23)H=\{(23),(132)\}.
\end{align*}
Observemos que en este caso se tiene la descomposición
\[
\Sym_3=H\cup (123)H\cup (132)H\quad\text{(unión disjunta)}.
\]
\end{example}

\begin{example}
Sea $G=\R^2$ con la suma usual y sea $v\in\R^2$. La recta $L=\{\lambda v:\lambda\in\R\}$ es un subgrupo de $G$ y 
para cada $p\in R^2$, la coclase $p+L$ es la recta paralela a $L$ que pasa por el punto $p$.  	
\end{example}


\begin{definition}
	Si $H$ es un subgrupo de $G$, se define el \textbf{índice} de $H$ en $G$
	como la cantidad $(G:H)$ de coclases a izquierda (o a derecha) de $H$ en $G$. 
\end{definition}

Tener una relación de equivalencia módulo $H$ nos permite escribir a $G$ como
unión disjunta de coclases a izquierda (o a derecha) de $H$ en $G$. Además dos
coclases cualesquiera son iguales o disjuntas. 

\begin{theorem}[Lagrange]
\index{Teorema!de Lagrange}
	Si $G$ es un grupo finito y $H$ es un subgrupo de $G$, entonces
	$|G|=|H|(G:H)$. En particular, $|H|$ divide a $|G|$. 
\end{theorem}

\begin{proof}
	Tenemos una relación de equivalencia módulo $H$ que nos permite descomponer
	en $G$ en clases de equivalencia, digamos
	\[
	G=\bigcup_{i=1}^n x_iH\quad\text{(unión disjunta)}
	\]
	para ciertos $x_1,\dots,x_n\in G$, donde $n=(G:H)$. Como cada una de esas clases tiene exáctamente
	$|H|$ elementos,  
	\[
		|G|=\sum_{i=1}^n|x_iH|=\sum_{i=1}^n|H|=|H|(G:H).\qedhere
	\]
\end{proof}

Veamos algunos corolarios. 

\begin{corollary}
	Si $G$ es un grupo finito y $g\in G$, entonces $g^{|G|}=1$. 	
\end{corollary}

\begin{proof}
	Por definición $|g|=|\langle g\rangle|$. El teorema de Lagrange aplicado al
	subgrupo $H=\langle g\rangle$ nos dice que 
	\[
		g^{|G|}=g^{|H|(G:H)}=(g^{|H|})^{(G:H)}=1.\qedhere
	\]
\end{proof}

\begin{corollary}
	Si $G$ es un grupo de orden primo, entonces $G$ es cíclico.
\end{corollary}

\begin{proof}
	Sea $g\in G\setminus\{1\}$ y sea $H=\langle g\rangle$. Por el teorema de
	Lagrange, $|H|$ divide a $|G|$ y luego $|H|=|G|$ pues $|G|$ es un número
	primo. En consecuencia, $G=H=\langle g\rangle$. 
\end{proof}

\begin{corollary}
\label{cor:ordenes_coprimos}
	Si $G$ es un grupo abeliano y $g,h\in G$ son elementos de órdenes finitos y coprimos, entonces
	$|gh|=|g||h|$.
\end{corollary}

\begin{proof}
Sean $n=|g|$, $m=|h|$ y $l=|gh|$. Como $G$ es abeliano, 
\[
(gh)^{nm}=(g^n)^m(h^m)^n=1
\]
y luego $l$ divide a $nm$. Por otro lado, como $(gh)^l=1$, 
$g^l=h^{-l}\in \langle g\rangle\cap\langle h\rangle=\{1\}$ (pues como $|\langle g\rangle|=n$ y $|\langle h\rangle|=m$ son coprimos, 
entonces $nm$ divide a $l$ gracias al teorema de Lagrange). 
\end{proof}

%\begin{example}
%Gracias al teorema de Lagrange podemos demostrar fácilmente que $(n+m)!$ divide a $n!m!$, 
%basta con observar que $\Sym_n\times\Sym_m\leq\Sym_{n+m}$
%\end{example}

El pequeño teorema de Fermat es un caso particular del teorema de Lagrange.

\begin{exercise}[pequeño teorema de Fermat]
	\index{Teorema!de Fermat}
	Sea $p$ un número primo. Demuestre que 
	$a^{p-1}\equiv1\bmod p$  
	para todo $a\in\{1,2,\dots,p-1\}$. 
\end{exercise}

El siguiente corolario utiliza la función $\varphi$ de Euler. Recordemos que
$\varphi(n)$ es la cantidad de enteros positivos $k\in\{1,\dots,n\}$ 
coprimos con $n$. El grupo de
unidades de $\Z/n$ tiene $\varphi(n)$ elementos (pues $x\in\Z/n$ es inversible
si y sólo si $x$ es coprimo con $n$). 

\begin{exercise}[teorema de Euler]
	\index{Teorema!de Euler}
	Sean $a$ y $n$ enteros coprimos. Demuestre que 
	$a^{\varphi(n)}\equiv1\bmod n$.
\end{exercise}

No vale la recíproca del teorema de Lagrange.

\begin{example}
Consideremos el grupo alternado 
\begin{multline*}
\Alt_4=\{\id,(234),(243),(12)(34),(123),(124),\\(132),(134),(13)(24),(142),(143),(14)(23)\}\leq\Sym_4.	
\end{multline*}
Vamos a demostrar que $\Alt_4$ no tiene subgrupos de orden seis. Si $H\leq\Alt_4$ es tal que 
$|H|=6$, entonces, como $(\Alt_4:H)=2$, para todo $x\not\in H$ podríamos descomponer a $\Alt_4$ como $\Alt_4=H\cup xH$ (unión disjunta). 

Afirmamos que
para todo $g\in\Alt_4$ vale que $g^2\in H$ (pues si $g\not\in H$, entonces, como $g^2\in\Alt_4=H\cup gH$, se concluye que $g^2\in H$). En particular, como
$(ijk)=(ikj)^2$, 
todos los elementos de orden tres de $\Alt_4$ están en el subgrupo $H$, una contradicción pues hay ocho elementos de orden tres.   
\end{example}

Todos deberíamos tener un grupo favorito. El mío es $\SL_2(3)$,
el grupo formado por las matrices de $2\times2$ con coeficientes en $\Z/3$ 
con determinante uno.

\begin{exercise}
Demuestre que 
\[
\SL_2(3)=\left\{\begin{pmatrix}a&b\\c&d\end{pmatrix}:ad-bc=1,\,a,b,c,d\in\Z/3\right\}
\]
es un grupo de orden 24 que no posee subgrupos de orden 12.	
\end{exercise}


%\begin{proof}
%	Si $a$ y $n$ son coprimos, entonces $a$ es una unidad de $\Z/n$. Como el
%	grupo de unidades de $\Z/n$ tiene orden $\varphi(n)$, entonces
%	$a^{\varphi(n)}\equiv1\bmod n$. 
%\end{proof}
%


%
%\section{Notas}
%
%Un teorema de Hall de 1935: Si $G$ es un grupo finito y $S$ es un subgrupo de $G$ de índice $n$, entonces
%existen $t_1,\dots,t_n\in G$ tales que los $t_1S,\dots,t_nG$ forman un conjunto de representantes de las las coclases a izquierda y los 
%$St_1,\dots,St_n$ forman un conjunto de representantes de coclases a derecha. 


