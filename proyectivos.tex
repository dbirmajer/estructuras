\chapter{Módulos proyectivos}


\begin{definition}
\index{Módulo!proyectivo}
Un módulo $P$ se dice \textbf{proyectivo} si dados $f\in\Hom_R(P,N)$ y un epimorfismo $g\in\Hom_R(M,N)$ 
existe $h\in\Hom_R(P,M)$ tal que $g\circ h=f$, es decir que existe un morfismo $h$ que
hace conmutativo al diagrama
\[
\xymatrix{ & P\ar[d]^f\ar@{-->}[ld]_h\\ M\ar[r]^g & N\ar[r] & 0	}
\]
\end{definition}

\begin{example}
Todo módulo libre es proyectivo. En efecto, si $S$ es base de módulo libre $P$, dado $s\in S$ existe
$m\in M$ tal que $g(m)=f(s)$. Definimos entonces
$h\colon P\to M$, $h(s)=m$ y extendemos por linealidad.  
\end{example}

En particular, $\Z$ es libre como $\Z$-módulo. 

\begin{theorem}
\label{thm:proyectivo}
	Sea $P$ un módulo. Las siguientes afirmaciones son equivalentes:
	\begin{enumerate}
		\item $P$ es proyectivo.
		\item $\Hom_R(P,-)$ es exacto.
		\item Toda sucesión exacta
			\[
			\xymatrix{
			0\ar[r] 
			& N
			\ar[r]^-{f}
			& M
			\ar[r]^-{g}
			& P\ar[r]
			& 0
			}
			\]
 			se parte, es decir que existe $h\in\Hom_R(P,M)$ tal que $g\circ h=\id_P$. En particular, 
 			$M\simeq N\oplus P$. 
 		\item $P$ es sumando directo de un libre.
	\end{enumerate}
\end{theorem}

\begin{proof}
	Probemos primero que $(1)\Leftrightarrow(2)$. Sea 
	\[
	\xymatrix{
	 0\ar[r] 
	 & A
	 \ar[r]^-{f}
	 & B
	 \ar[r]^-{g}
	 & C\ar[r]
	 & 0
	 }
	\]
	una sucesión exacta. Tenemos que probar que si $P$ es proyectivo, entonces la
	sucesión
	\[
	\xymatrix{
	 0\ar[r] 
	 & \hom_A(P,A)
	 \ar[r]^-{f_*}
	 & \hom_A(P,B)
	 \ar[r]^-{g_*}
	 & \hom_A(P,C)\ar[r]
	 & 0
	 }
	\]
	es exacta. Dado un morfismo $\beta\colon P\to C$, como $P$ es proyectivo y
	$g$ es epimorfismo, existe $\alpha\in\Hom_R(P,B)$ tal que
	$g_*(\alpha)=g\circ \alpha=\beta$. Luego $g_*$ es epimorfismo. Recíprocamente, es
	claro que si $g_*$ es un epimorfismo entonces $P$ es proyectivo. 

Veamos ahora que $(1)\implies(3)$. Si $0\to N\to M\to P\to 0$ es una sucesión exacta, entonces se parte, pues como
$P$ es proyectivo 
existe $h\in\Hom_P(P,N)$ tal que el diagrama
	\[
	\xymatrix{ & P\ar@{=}[d]^{\id_P}\ar@{-->}[ld]_h\\ M\ar[r]^g & P\ar[r] & 0	}
	\]
es conmutativo, lo que significa que $g\circ h=\id_P$. Para ver
que $M\simeq N\oplus P$ basta usar la proposición~\ref{pro:split}.

Veamos ahora que $(3)\implies(4)$. Sabemos que $P$ es cociente de un módulo libre, es decir
que existe un conjunto $I$ y un epimorfismo $\varphi\colon R^{(I)}\to P$. Como 
la sucesión
	\[
	\xymatrix{
	 0\ar[r] 
	 & \ker\varphi 
	 \ar[r]
	 & R^{(I)}
	 \ar[r]^-{\varphi}
	 & P\ar[r]
	 & 0
	 }
	\]
de módulos y morfismos 
es exacta, se parte es decir que existe $h\in\Hom_R(P,R^{(I)})$ tal que
$\varphi\circ h=\id_P$. Luego $P$ es sumando directo de un libre, pues
$R^{(I)}\simeq\ker\varphi\oplus P$. 

	Finalmente probemos $(4)\Rightarrow(1)$. Si $f\colon M\to N$ es un epimorfismo y 
	$g\colon P\to N$, como $P$ es sumando directo del libre $R^{(I)}$, definimos
	$\beta\colon P\to M$, $\beta=\alpha\circ i$, donde
	$i\colon P\to R^{(I)}$ es la inclusión. El diagrama 
	\[
	\xymatrix{ & R^{(I)}\ar[d]^p\ar@{-->}[ddl]_{\alpha} \\
	& P\ar[d]^f\ar@{-->}[dl]^{\beta}\\ M\ar[r]_g & N\ar[r] & 0 }
	\]
	donde $p\colon R^{(I)}\to P$ es tal que $p\circ i=\id_P$, es entonces conmutativo, pues 
	\[
		g\circ \beta=(g\circ \alpha)\circ i=f\circ (p\circ i)=f\circ \id_P=f.\qedhere
	\]	
\end{proof}

No todo proyectivo es libre.

\begin{example}
Sea $R=\Z\times\Z$. Como $R$ es libre como $R$-módulo, 
$\Z\times\{0\}$ es proyectivo, pues es sumando directo de un módulo libre. Sin embargo, $\Z\times\{0\}$ no es libre como $R$-módulo. En efecto,
si lo fuera, sea $\{(x_i,0):i\in I\}$ una base de $\Z\times\{0\}$. Entonces
$(1,1)\cdot (x_1,0)=(x_1,0)=(1,0)\cdot (x_1,0)$, una contradicción. 
\end{example}

Cocientes de proyectivos pueden no ser proyectivos. 

\begin{example}
Veamos que el $\Z$-módulo $\Z/n\Z$ no es proyectivo. 
Si $\Z/n\Z$ es proyectivo, entonces, gracias al teorema anterior, 
la sucesión exacta
\[
	\xymatrix{
	 0\ar[r] 
	 & n\Z 
	 \ar[r]^i
	 & \Z
	 \ar[r]^-{\pi}
	 & \Z/n\Z\ar[r]
	 & 0,
	 }
\]
donde $i$ es la inclusión y $\pi$ es el epimorfismo canónico, 
se parte, es decir existe $h\in\Hom(\Z/n,\Z)$ tal que $\pi\circ h=\id_{\Z/n}$, una contradicción
pues sabemos que $\Hom(\Z/n,\Z)=\{0\}$.
% Si $f\in \Hom(\Z/n,\Z)$, entonces 
%\[
%0=f(0)=f(n)=nf(1)\in\Z
%\]
%y luego $f(1)=0$.  	
\end{example}

Submódulos de proyectivos pueden no ser proyectivos. 

\begin{example}
Sea $M=\Z/4$ como $\Z/4$-módulo. Como $M$ es libre, es proyectivo. 
Veamos que el submódulo $N=\{0,2\}$ no es proyectivo. Si $N$ fuera proyectivo, entonces
la sucesión exacta
\[
	\xymatrix{
	 0\ar[r] 
	 & \Z/2
	 \ar[r]^i
	 & \Z/4
	 \ar[r]^-{\pi}
	 & \Z/2
	 \ar[r]
	 & 0,
	 }
\]
debería partirse, lo que implicaría que tendríamos en particular un isomorfismo 
$\Z/4\simeq\Z/2\oplus\Z/2$ de grupos abelianos, una contradicción.
\end{example}

\begin{exercise}
\label{xca:cociente_libre}
Si $N$ es un submódulo de $M$ tal que $M/N$ es libre, entonces $N$ es sumando directo de $M$. 	
\end{exercise}

\index{Idempotente}
Si $R$ es un anillo, $e\in R$ es \textbf{idempotente} si $e^2=e$. Ejemplos triviales de idempotentes son 0 y 1.
Observemos que si $e$ es un idempotente no trivial, entonces 
$I_1=Re$ y $I_2=R(1-e)$ son ideales a izquierda tales que $I_1\cap I_2=\{0\}$ y además 
$R=I_1+I_2$. En efecto, para ver que $R=I_1+I_2$ basta observar que, como 
$1=e+(1-e)$, entonces $r=r1=r(e+(1-e))=re+r(1-e)$ para todo $r\in R$. Para ver que $I_1\cap I_2=\{0\}$ 
sea $x\in I_1\cap I_2$. Entonces $x=re=s(1-e)$ para ciertos $r,s\in R$ y luego
\[
0=s(1-e)e=xe=(re)e=re^2=re=x.   
\]
En particular, por el primer teorema de isomorfismos, 
\[
R\simeq R/(I_1\cap I_2)\simeq R/I_1\times R/I_2.
\]

\begin{example}
Si $e\in \Z/24$ es idempotente, entonces $e\in\{0,1,9,16\}$.  	
\end{example}

\begin{example}
Los idempotentes no triviales de $M_2(\R)$ son las matrices 
de la forma $\begin{pmatrix}a&b\\c&d\end{pmatrix}$ con $a+d=1$ y $ad=bc$.  	
\end{example}

Los idempotentes del anillo están en correspondencia biyectiva con los proyectores
de la representación regular $M=\prescript{}{R}R$. En efecto, si $p$ es un proyector, entonces $e=p(1)$ es un idempotente. Recíprocamente, 
si $e$ es un idempotente del anillo, entonces el morfismo $p\colon M\to M$, $m\mapsto e\cdot m$, es un proyector, pues
\[
p(p(m))=e\cdot (e\cdot m)=e^2\cdot m=e\cdot m=p(m)
\]
para todo $m\in M$.

\begin{example}
Como $0$ y $1$ son los únicos idempotentes del anillo $\Z$, la correspondencia biyectiva entre
idempotentes del anillo y proyectores del módulo 
nos da otra demostración de que los únicos sumandos directos de $\Z$ son $\{0\}$ y $\Z$.
\end{example}

\begin{example}
Si $e\in R$ es idempotente, el ideal a izquierda $Re$ es proyectivo como $R$-módulo. En efecto, la función
$\varphi\colon R\to Re$, $r\mapsto re$, es un epimorfismo de módulos. La sucesión
exacta
\[
	\label{eq:exacta}	
		\xymatrix{
        0\ar[r]
        & \ker\varphi
        \ar[r]
        & R
        \ar[r]^\varphi
        & Re\ar[r]
        & 0,
        }
\]
se parte, pues la inclusión $i\colon Re\to R$ es una sección: $\varphi(i(re))=\varphi(re)=re^2=re$. Luego
$R\simeq\ker\varphi\oplus Re$ y entonces $Re$ es proyectivo por ser sumando directo de un módulo libre. 
\end{example}

\begin{exercise}
\label{xca:I^2}
Sea $R$ un anillo conmutativo y sea $I$ un ideal de $R$. Demuestre que $I$ es un sumando directo de $R$ si y sólo si
existe $u\in R$ tal que $I=(u)=(u^2)$. 	
\end{exercise}

\begin{exercise}
\label{xca:ss_idempotente}
Demuestre que $\prescript{}{R}R$ es semisimple si y sólo si todo ideal a izquierda de $R$ está generado por un idempotente.
\end{exercise}

\begin{proposition}
Sea $M$ un módulo finitamente generado. Entonces $M$ es proyectivo si y sólo si existe $n\in\N$ y $p\in\Hom_R(R^n,R^n)$ tal que
$M\simeq p(R^n)$ y $p^2=p$. 	
\end{proposition}

\begin{proof}
	Supongamos primero que $M$ es proyectivo. Entonces 
	$M$ es sumando directo de $R^n$, digamos $R^n=M\oplus N$ para algún modulo $N$. 
	La función $p\colon R^n\to R^n$, $p(m,n)=m$, es un morfismo 
	de módulos
	tal que $p^2=p$ y $p(R^n)=M$. 
	
% 	todo módulo $M$ es cociente de un libre, 
% 	existen $n\in\N$ y un epimorfismo $\varphi\colon R^n\to M$. Como $M$ es proyectivo, existe una sección 
% 	$s\colon M\to R^n$ y luego $R^n=s(M)\oplus N$ para algún submódulo $N$ de $M$. Sea $p=s\circ\varphi\in\Hom_R(R^n,R^n)$. Entonces
% 	\[
% 	p(R^n)=s(\varphi(R^n))=s(M)\simeq M,
% 	\]
% 	pues $s$ es inyectiva. Además $p^2=s\circ(\varphi\circ s)\circ\varphi=s\circ\varphi=p$.   
	
	Recíprocamente, si existe $p\in\Hom_R(R^n,R^n)$ tal que $p^2=p$ y $p(R^n)\simeq M$, entonces $p\circ (\id-p)=0$. Esto nos
	permite descomponer al anillo $R^n$ en idempotentes ortogonales, es decir
	$R^n\simeq p(R^n)\oplus (\id-p)(R^n)$. Luego $M\simeq p(R^n)$ es proyectivo por ser sumando directo de un libre.  
\end{proof}

\begin{example}
Sea $R=\Z/6=\{0,1,\dots,5\}$ y sea $M=\prescript{}{R}R$. Sabemos que $M$ es libre con base $\{1\}$. Si $I=\{0,2,4\}$ y $J=\{0,3\}$, entonces
$I$ y $J$ son ideales de $R$, es decir que son submódulos de $M$. Si $m\in M$, entonces
$m=-2m+3m\in I+J$. Como además $m\in I\cap J=\{0\}$, se concluye que $M\simeq I\oplus J$. En particular, $I$ y $J$ son proyectivos
como $R$-módulos por ser sumandos directos del módulo libre $M$. Observemos que $I$ está generado por el idempotente
$e=4$ y que $J$ está generado por el idempotente $1-e=3$.
\end{example}


\begin{proposition}
El módulo $S\oplus T$ es proyectivo si y sólo si $S$ y $T$ son proyectivos. 	
\end{proposition}

\begin{proof}
Consideremos el diagrama
	\[
	\xymatrix{ & S\ar[d]^{i_1}\ar@{-->}[ddl]_{h_1} \\
	& S\oplus T\ar[d]^f\ar@{-->}[dl]^{h}\\ M\ar[r]_g & N\ar[r] & 0 }
	\quad
	\]

Veamos primero que si $S$ y $T$ son proyectivos, entonces $S\oplus T$ es proyectivo. Como $S$ y $T$ 
son proyectivos, existen $h_1\colon S\to M$ y $h_2\colon T\to M$ tales que
$g\circ h_1=f\circ i_1$ y $g\circ h_2=f\circ i_2$, donde
$i_1\colon S\to S\oplus T$, $i_1(s)=(s,0)$, $i_2\colon T\to S\oplus T$, $i_2(t)=(0,t)$.  
Si definimos el morfismo 
$h\colon S\oplus T\to M$,
\[
h(s,t)=h_1(s)+h_2(t),
\] 	
entonces tenemos que 
\begin{align*}
g(h(s,t))&=g(h_1(s)+h_2(t))=g(h_1(s))+g(h_2(t))\\
&=f(i_1(s))+f(i_2(t))=f(s,0)+f(0,t)=f(s,t).
\end{align*}
	
Demostremos ahora que si $S\oplus T$ es proyectivo, entonces $S$ es proyectivo. 
Consideramos el diagrama
	\[
	\xymatrix{ & S\oplus T\ar[d]^{p_1}\ar@{-->}[ddl]_{h} \\
	& S\ar[d]^f\ar@{-->}[dl]^{h_1}\\ M\ar[r]_g & N\ar[r] & 0 }
	\quad
	\]

Sea $i_1\colon S\to S\oplus T$, $i_1(s)=(s,0)$.  
Como por hipótesis $S\oplus T$ es proyectivo, existe un morfismo 
$h\colon S\oplus T\to M$ tal que $g\circ h=f\circ p_1$. 
Definimos entonces $h_1\colon S\to M$, $h_1(s)=h(s,0)$ 
y vemos que
\[
g(h_1(s))=g(h(s,0))=f(s).
\]
Similarmente se demuestra que $T$ es también proyectivo.
\end{proof}

Dejamos como ejercicio demostrar 
que $\oplus_{i\in I}M_i$ es proyectivo si y sólo si $M_i$ es proyectivo para todo $i\in\I$. 	

\begin{example}
	Sea $R=\Z/6$. 
	Veamos que todo $R$-módulo es proyectivo. Si $M$ es un $R$-módulo, afirmamos que $M=2M\oplus 3M$, donde
	$2M=\{2m:m\in M\}\subseteq M$ y $3M=\{3m:m\in M\}\subseteq M$, ambos submódulos de $M$. En efecto, $M=2M+3M$, pues $m=-2m+3m\in 2M+3M$. Además 
	$2M\cap 3M=\{0\}$ pues si $m\in 2M\cap 3M$, entonces, como $m=2x=3y$ para ciertos $x,y\in M$, se tiene que
	\[
	0=6x=3(2x)=3m=3(3y)=9y=3y=m.
	\]
	
	Veamos que $2M$ es proyectivo como $R$-módulo. 
	Como $2M$ es un $R$-módulo, tenemos un morfismo de anillos
	$\rho\colon R\to\End(2M)$ con núcleo $\ker\rho=\{0,3\}$. 
	Como $(\Z/6)/\ker\rho\simeq\Z/3$, la propiedad universal del cociente 
	implica  
	que existe un único morfismo $\varphi$ 
	tal que el diagrama 
	\[
        \xymatrix{
        \Z/6
        \ar[d]_g
        \ar[r]^\rho
        & \End(2M)
        \\
        \Z/3
        \ar[d]
        \ar@{-->}[ur]_{\varphi}
        \\
        0
        }
	\]
	donde $g$ es un epimorfismo, 
	conmuta, es decir
	\[
	\rho(k)(2m)=\varphi(g(k))(2m).
	\] 
	Luego $2M$ es un $R$-módulo si y sólo si $2M$ es un $\Z/3$-módulo. 
	Como $\Z/3$ es un cuerpo, $2M$ es
	libre como $\Z/3$-módulo, 
	por ser un espacio vectorial sobre el cuerpo $\Z/3$. En particular, 
	$2M$ es proyectivo como $\Z/3$-módulo. En conclusión, $2M$ es proyectivo
	como $R$-módulo. Análogamente se demuestra que $3M$ es proyectivo. Luego
	$M$ es proyectivo por ser suma directa de proyectivos.  
\end{example}

\begin{example}
Sea $R=\begin{pmatrix}
\R & \R\\
0 & \R
\end{pmatrix}$. Como $R$ es libre como $R$-módulo y además 
\[
R=\begin{pmatrix}
\R & 0\\
0 & 0
\end{pmatrix}
\oplus\begin{pmatrix}
0 & \R\\
0 & \R
\end{pmatrix}
\]
los submódulos $\begin{pmatrix}
\R & 0\\
0 & 0
\end{pmatrix}$ y 
$\begin{pmatrix}
0 & \R\\
0 & \R
\end{pmatrix}$ son proyectivos. 
%El submódulo $\begin{pmatrix}
%\R & \R\\
%0 & 0
%\end{pmatrix}$ no es proyectivo porque
%no admite un complemento en $R$. 
\end{example}

Veamos otro ejemplo de un módulo proyectivo que no es libre. Recordemos
que si $R$ es un anillo conmutativo e $I$ y $J$ son ideales de $R$, entonces
	\[
		I+J=\{x+y\mid x\in I,\;y\in J\}
	\]
es el menor ideal que contiene a $I\cup J$. Además $IJ$ es el ideal formado
por las sumas finitas $\sum_i x_iy_i$, donde $x_i\in I$, $y_i\in J$. 

	\begin{lemma}
		Sea $R$ un dominio íntegro y sean $I$ y $J$ ideales tales que
		$I+J=R$. Consideremos el morfismo de $R$-módulos 
		$g\colon I\times J\to R$, $(x,y)\mapsto x+y$. 
		Entonces:
		\begin{enumerate}
			\item $g$ es sobreyectiva.
			\item $\ker(g)=\{(x,-x)\mid x\in I\cap J\}$.
			\item $I\times J\simeq (I\cap J)\oplus R$ como $R$-módulos.
			\item Si $I\cap J$ es principal entonces $I$ y $J$ son $R$-módulos
				proyectivos.
		\end{enumerate}
	\end{lemma}

	\begin{proof}
		Las primeras dos afirmaciones son evidentes. 
		
		Para demostrar (3) consideremos la sucesión exacta
		\[
		\xymatrix{
		0\ar[r] 
		& I\cap J
		\ar[r]^-{f}
		& I\times J
		\ar[r]^-{g}
		& R\ar[r]
		& 0
		}
		\]
		donde $f(x)=(x,-x)$. Como $R$ es libre como $R$-módulo, $R$ es proyectivo 
		y entonces la sucesión se parte, es decir $I\oplus J\simeq 
		I\times J\simeq (I\cap J)\oplus R$
		como $R$-módulos. 
		
		Para demostrar (4), supongamos que $I\cap
		J=(x)$. Si $x=0$, como $I\cap J=\{0\}$, el 
		ítem nos dice que $I\oplus J\simeq I\times J\simeq R$ y 
		luego $I$ y $J$ son proyectivos por ser sumandos directos de un libre.  
		Si $x\ne 0$, entonces, como $R$ es un dominio íntegro, $R\to I\cap J$, $r\mapsto rx$, es un isomorfismo de
		$R$-módulos. Luego 
		$I\oplus J\simeq I\times J\simeq (I\cap J)\oplus R\simeq R\oplus R$ y entonces 
		$I$ y $J$ son proyectivos por ser sumandos directos de
		un libre.
	\end{proof}

Veamos una aplicación concreta del lema anterior.  

\begin{example}
	Sea $R=\Z[\sqrt{-5}]$ y consideremos los ideales 
	\[
		I=(3,1+\sqrt{-5}),\quad 
		J=(3,1-\sqrt{-5}).
	\]
	Vamos a demostrar lo siguiente:
	\begin{enumerate}
		\item $I$ no es principal.
		\item $I$ no es libre como $R$-módulo.
		\item $I$ es proyectivo.
		\item $R/I\simeq\Z_3$ y luego $I$ es maximal.
	\end{enumerate}
	Valen además las afirmaciones análogas para $J$.

	Demostremos la primera afirmación. Si $I$ es principal, digamos $I=(a+b\sqrt{-5})$ entonces,
	como $3\in I$, existen $c,d\in\Z$ tales que
	$3=(a+b\sqrt{-5})(c+\sqrt{-5}d)$.  Si aplicamos la función multiplicativa 
	$N(a+b\sqrt{-5})=a^2+5b^2$,
	\[
	9=(a^2+5b^2)(c^2+5d^2)
	\]
	y entonces $a^2+5b^2\in\{1,3,9\}$. Como $I\ne R$,  
	$a^2+5b^2\geq5$ y luego $a^2+5b^2=9$. En particular,
	\[
	(a,b)\in\{(-3,0),(3,0),(2,1),(2,-1),(-2,-1),(-2,1)\}
	\]
	Vimos que los casos $(a,b)\in\{(-3,0),(3,0)\}$ no son posibles. Para el resto de los posibles valores de $(a,b)$ 
	se tiene que 
	$9=9(c^2+5d^2)$ y luego $c^2+5d^2=1$, lo que implica que $c\in\{-1,1\}$ y entonces 
	$3=a+b\sqrt{-5}$ o bien $3=-a-b\sqrt{-5}$, una contradicción. 

	Demostremos la segunda afirmación. Supongamos que $I$ es libre como $R$-módulo. Como 
	$R$ es numberable, sea $\{x_i\mid i\in\N\}$ es una base de $I$. Tenemos entonces
	que $|I|=1$, pues de lo contrario, si $|I|\geq2$, como $x_1(-x_2)+x_2x_1=0$, se
	tendría que el conjunto $\{x_i\mid i\in I\}$ es linealmente
	dependiente. Tenemos entonces que $|I|=1$, es decir $I$ es principal, una contradicción. 
	
	Demostremos ahora la tercera afirmación. Primero observemos que $I+J=R$ pues $I+J$ es un ideal
	y además $1 = (3-(1+\sqrt{-5})) - (1-\sqrt{-5})\in I+J$. Probemos ahora $I\cap J$ es
	principal, más precisamente $I\cap J=(3)$. Para esto primero observemos que,
	en general, vale la fórmula
	\[
		(I+J)(I\cap J)\subseteq IJ
	\]
	y entonces, como $I+J=R$, se tiene que $I\cap J=IJ$. Todo elemento del ideal $IJ$ 
	es una suma finita $\sum_i x_iy_i$, donde $x_i\in I$, $y_i\in J$. Como
	\begin{align*}
		\sum (3u+(1&+\sqrt{-5})v(3u'+(1-\sqrt{-5})v'\\
				&=3\sum (3uu'+uv'(1-\sqrt{-5})+u'v(1+\sqrt{-5})+2vv'),
	\end{align*}
	es claro que $IJ\subseteq (3)$. La otra inclusión es evidente. El lema
	anterior implica que entonces $I$ y $J$ son ambos proyectivos.

	Probemos ahora la última afirmación. Consideremos la inclusión $i\colon
	\Z\to\Z[\sqrt{-5}]$ y el epimorfismo canónico $p\colon
	\Z[\sqrt{-5}]\to\Z[\sqrt{-5}]/I$. Entonces $f=p\circ i$ es un morfismo sobreyectivo 
	tal que $\ker(p\circ i)=(3)$ pues si $m\in\ker(p\circ i)$ entonces 
	\begin{align*}
		m&=3(a+b\sqrt{-5})+(1+\sqrt{-5})(c+d\sqrt{-5})\\
		&=(3a+c-5d)+(3b+d+c)\sqrt{-5}.
	\end{align*}
	Luego $m=3a-3b-6d\in3\Z$. La inclusión $3\Z\subseteq\ker(p\circ i)$ es trivial.
	Para ver que $f$ es sobreyectivo basta observar que $f(a-b)=a+b\sqrt{-5}$, pues
    \[
    f(a-b)=p(a-b)=(a-b)+I=(a+b\sqrt{-5})+I
    \]
    ya que $(a-b)-(a+b\sqrt{-5})=-b(1+\sqrt{-5})\in I$. 
	Por el teorema de isomorfismos, $\Z/3\Z\simeq p(i(\Z))$ y
	luego $I$ es un ideal maximal. 
	\end{example}

\begin{exercise}
\label{xca:proyectivo1}
	Sean $P$ y $P_1$ dos módulos proyectivos. Consideremos el diagrama
	\begin{align*}
		&\xymatrix{
		0\ar[r] 
		& K
		\ar[r]^-{f}
		& P
		\ar[r]^-{g}
		& M\ar[r]\ar@{=}[d]
		& 0\\
		0\ar[r] 
		& K_1
		\ar[r]^-{f_1}
		& P_1
		\ar[r]^-{g_1}
		& M\ar[r]
		& 0
		}
	\end{align*}
	con filas exactas. Demuestre que $P\oplus K_1\simeq P_1\oplus K$.
\end{exercise}

