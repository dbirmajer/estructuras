\chapter{Módulos libres}

\begin{definition}
\index{Conjunto!linealmente independiente}
Sea $X$ un subconjunto de un módulo $M$. Diremos que $X$ es \textbf{linealmente independiente}
si para cada $k\in\N$, $r_1,\dots,r_k\in R$ y $m_1,\dots,m_k\in X$ tales que 
$r_1\cdot m_1+\cdots+r_k\cdot m_k=0$ se tiene que $r_1=\cdots=r_k=0$. 
Un conjunto que no es linealmente independiente se dirá \textbf{linealmente dependiente}.\end{definition}

La independencia lineal en módulos es levemente distinta a la que conocemos para
espacios vectoriales. En módulos la dependencia lineal no garantiza que uno de los elementos del conjunto
pueda escribirse como combinación lineal de los otros, ya que en el anillo $R$ no siempre podremos dividir.

\begin{example}
Consideramos $\Z$ como $\Z$-módulo. 
El conjunto $\{2,3\}\subseteq\Z$ es linealmente dependiente pues $(-3)\cdot 2+2\cdot 3=0$. Observemos
que $2$ no es un múltiplo entero de $3$, tampoco $3$ es un múltiplo entero de $2$.   	
\end{example}

\begin{example}
El conjunto $\{1\}$ del módulo $\prescript{}{R}R$ es linealmente independiente. Si $r$ es un 
divisor de cero de $R$, entonces $\{r\}$ es linealmente dependiente.	
\end{example}

En módulos, un conjunto minimal de generadores puede ser linealmente dependiente. 

\begin{example}
Sea $R=M_2(\R)$ y sea $M=\begin{pmatrix} 0 & \R\\ 0 & \R\end{pmatrix}$. Dejamos como ejercicio demostrar
que $M$ es un módulo sobre el anillo $R$ con la multiplicación usual de matrices. 
El conjunto $\left\{\begin{pmatrix} 0&1\\0&0\end{pmatrix}\right\}$ es un 
conjunto minimal de generadores y es linealmente dependiente. 	
\end{example}

\begin{example}
Sea $M=\Q$ como $\Z$-módulo. Si $x\in\Q\setminus\{0\}$, entonces $\{x\}$ es linealmente independiente. Si $x,y\in\Q$ son tales
que $x\ne y$, entonces $\{x,y\}$ es linealmente dependiente. 	
\end{example}

Si $V$ es un espacio vectorial y $v\in V\setminus\{0\}$, entonces $\{v\}$ es linealmente independiente, pues
si $\lambda\ne0$ y $v\ne 0$, entonces $\lambda v\ne 0$. 
En la teoría de módulos, las cosas pueden ser distintas. 

\begin{example}
Todo subconjunto del $\Z$-módulo $\Z/n$ es linealmente dependiente. 
\end{example}

\begin{exercise}
Sea $f\in\Hom_R(M,N)$ y sea $X$ un subconjunto de $M$. 
\begin{enumerate}
\item Si $X$ es linealmente dependiente, entonces $f(X)$ también.
\item Si $X$ es linealmente independiente y $f$ es monomorfismo, entonces $f(X)$ es linealmente independiente.
\item Si $M=(X)$ y $f$ es epimorfismo, entonces $N=(f(X))$. 	
\end{enumerate}
\end{exercise}

\begin{definition}
	\index{Base}
	Sea $M$ un módulo. Un subconjunto $B$ de $M$ es una \textbf{base} de $M$ si 
	es linealmente independiente y además $(B)=M$. Un módulo $M$ se dice \textbf{libre} si 
	admite una base.  
\end{definition}

\begin{examples}\
\begin{enumerate}
	\item Todo espacio vectorial es un módulo libre.
	\item $\prescript{}{R}R$ es libre con base $\{1\}$. 
	\item El $\Z$-módulo $\Q$ no es libre.  
	\item $R^n$ es libre como $R$-módulo.
\end{enumerate}	
\end{examples}

\begin{example}
Las únicas bases de $\Z$ como $\Z$-módulo son $\{1\}$ y $\{-1\}$.	
\end{example}

El ejemplo siguiente es bien conocido en el caso de espacios vectoriales. Los módulos
sobre anillos de división son muy similares a los espacios vectoriales 
y por esa razón se los llama 
\textbf{espacios vectoriales sobre anillos de división}.

\begin{example}
\index{Espacios vectoriales}
\index{Módulos!sobre anillos de división}
Sea $R$ un anillo de división y sea $M$ un $R$-módulo no nulo finitamente generado. 
Vamos a demostrar las siguientes propiedades: 
\begin{enumerate}
\item Todo conjunto finito de generadores contiene una base. En particular, $M$ es libre. 
\item Todo conjunto linealmente independiente puede extenderse a una base.
\item Dos bases cualesquiera tienen la misma cantidad de elementos. 
\end{enumerate}

Para demostrar la primera afirmación procederemos por inducción en la cantidad de generadores de $M$. Si $M=(m)$, entonces
$\{m\}$ es base pues $\{m\}$ es linealmente independiente: si $r\cdot m=0$ y $r\ne 0$, entonces
\[
m=1\cdot m=(r^{-1}r)\cdot m=r^{-1}\cdot (r\cdot m)=0.
\]
Si vale para $k-1$ generadores, sea $M=(m_1,\dots,m_k)$. Si $\{m_1,\dots,m_k\}$ no es linealmente
independiente, entonces existen $r_1,\dots,r_k\in R$ no todos cero tales
que
\[
r_1\cdot m_1+\cdots+r_k\cdot m_k=0.
\]
Sin perder generalidad podemos suponer que $r_k\ne 0$. Entonces
\[
v_k=\sum_{i=1}^{k-1} (r_k^{-1}r_i)\cdot m_i\in (m_1,\dots,m_{k-1}).
\]
Como entonces $M=(m_1,\dots,m_k)=(m_1,\dots,m_{k-1})$, la hipótesis inductiva implica que
$M$ es libre. 

Vamos a demostrar ahora que todo
conjunto $X$ linealmente independiente puede extenderse a una base. 	Sea $X=\{x_1,\dots,x_k\}$ tal que $M=(X)$. 
Como $M\ne\{0\}$, sin perder generalidad podemos suponer que $x_1\ne 0$. Como $R$ es de división,
el conjunto $\{x_1\}$ es linealmente independiente, pues si $r\ne 0$ y $r\cdot x_1=0$, entonces 
\[
x_1=1\cdot r=(r^{-1}r)\cdot x_1=r^{-1}\cdot (r\cdot x)=r^{-1}\cdot 0=0.
\]
Sea $Y=\{y_1,\dots,y_l\}$ un subconjunto de $X$ maximal tal que $Y$ es linealmente independiente. Veamos que $X\subseteq (Y)$. Sea $x\in X$. Si $x\not\in Y$,  
entonces, como $Y\subseteq Y\cup \{x\}$, la maximalidad de $Y$ implica que $\{x\}\cup Y$ es linealmente dependiente, es decir
que existen $r,r_1,\dots,r_k\in R$ no todos cero tales que
\[
r\cdot x+\sum_{i=1}^l r_i\cdot y_i=0.
\]
Si $r=0$, entonces $r_1=\cdots=r_l=0$ porque los $y_j$ son linealmente independientes, una contradicción. Luego $r\ne 0$ y entonces
\[
x=-\sum_{i=1}^l (-r^{-1}r_i)\cdot y_i\in (Y).
\]
Luego $X\subseteq (Y)$. En conclusión $Y$ es una base de $M$ pues $M=(Y)$ 
y además $Y$ es linealmente independiente. 

Demostremos que dos bases finitas cualesquiera tienen la misma cantidad de elementos. 
Para eso es suficiente demostrar
que si $X$ e $Y$ son conjuntos finitos linealmente independientes
tales que $(X)\subseteq (Y)$, entonces $|X|\leq |Y|$. Supongamos que $|X|=k$ e $|Y|=l$. 
Procederemos por inducción en $l$. Si $l=1$ y $k>1$, entonces
exiten $r_1,r_2\in R$ tales que $x_1=r_1\cdot y_1$ y $x_2=r_2\cdot y_1$. Luego
\[
x_2=r_2\cdot y_1=r_2\cdot (r_1^{-1}\cdot x_1)=(r_2r_1^{-1})\cdot x_1,
\]
una contradicción pues $\{x_1,x_2\}$ es linealmente independiente. 
Supongamos ahora que el resultado 
es verdadero para $l-1$ y sea $l=|Y|$. Para cada $j$ escribimos
\[
x_j = \sum_{i=1}^l r_{ji}\cdot y_i,
\]
donde $r_{ji}\in R$. Si $r_{j1}\ne 0$ para todo $j$, entonces 
$x_j=\sum_{i=2}^l r_{ji}\cdot y_i$ para todo $j$ y luego $(X)\subseteq (y_2,\dots,y_l)$, que implica que
$|X|\leq l-1<l=|Y|$. Si existe $j$ tal que $r_{j1}\ne 0$, sin perder generalidad podemos
suponer que $r_{11}\ne 0$. Para cada $j\in\{2,\dots,k\}$ sea
\[
z_j = x_j-(r_{j1}r_{11}^{-1})\cdot x_1.
\]
Como $z_j\in (y_2,\dots,y_l)$ para todo $j$ y los $z_j$ son linealmente independientes, 
la hipótesis inductiva impica que $k-1\leq l-1$, es decir $|X|\leq |Y|$.  
\end{example}

\index{Dimensión!de un módulo sobre un anillo de división}
\index{Dimensión!de un espacio vectorial}
El ejemplo anterior nos permite hablar de 
la \textbf{dimensión} de un módulo sobre un anillo de división. 

\begin{example}
El anillo $\R[X]$ es un $\R$-módulo libre con base $\{1,X,X^2,\dots\}$. También es un $\R[X]$-módulo libre con base $\{1\}$.  	
\end{example}

\begin{exercise}
Demuestre que el conjunto $\{(a,b),(c,d)\}$ es base del $\Z$-módulo $\Z\times\Z$ si y sólo si $ad-bc\in\{-1,1\}$. 
\end{exercise}

En particular, $\{(1,0),(0,1)\}$ es una base de $\Z\times\Z$ como $\Z$-módulo. 

\begin{example}
Si $u\in\mathcal{U}(R)$, entonces $\{u\}$ es una base de $\prescript{}{R}R$. Recíprocamente, si $R$ es un dominio íntegro y  
$\{z\}$ es una base de $\prescript{}{R}R$, entonces $z\in\mathcal{U}(R)$, pues, como
$1=yz$ para algún $y\in R$, también se tiene que $zy=1$ pues  
\[
(zy-1)z=z(yz)-z=z1-z=z-z=0.
\]	
\end{example}

\begin{example}
Sea $I$ un conjunto no vacío. El $R$-módulo $R^{(I)}$ es libre con base $\{e_i:i\in I\}$, donde 
\[
(e_i)_j=\begin{cases}
	1 & \text{si $i=j$},\\
	0 & \text{si $i\ne j$.}
	\end{cases}	
\]
\end{example}

\begin{example}
Si $R=M_2(\Z)$, entonces $M=\prescript{}{R}R$ es libre con base $\left\{\begin{pmatrix}	1&0\\0&1\end{pmatrix}\right\}$. El submódulo
$N=\begin{pmatrix}\Z&0\\\Z&0\end{pmatrix}$ no admite una base como $R$-módulo.	
\end{example}

A diferencia de lo que pasa con espacios vectoriales, el tamaño de una base
no es un invariante. 

\begin{example}
Sea $V$ el espacio vectorial (complejo) con base infinita $e_0,e_1,e_2,\dots$ y sea $R=\End(V)$ con la estructura
de anillo dada por 
\[
(f+g)(v)=f(v)+g(v),\quad
(fg)(v)=f(g(v))
\]
para $f,g\in R$ y $v\in V$. 

Sea $M=\prescript{}{R}R$. 
Sabemos que $\{\id\}$ es una base para $R$.  
Mostraremos que $M$ admite también una base que tiene dos elementos.  
Si $r,s\in R$ son tales que
\begin{align*}
&r(e_{2n})=e_n, && r(e_{2n+1})=0,\\
&s(e_{2n})=0,&& s(e_{2n+1})=e_{2n},
\end{align*}
entonces $\{r,s\}$ es base de $M$. 

Si $f\in R$, entonces
$f=\alpha r+\beta s$, donde $\alpha\colon V\to V$, $e_n\mapsto f(e_{2n})$ para todo $n\in\N$, y $\beta\in V\to V$, $e_n\mapsto f(e_{2n+1})$ para todo $n\in\N$. En efecto,
\begin{align*}
&(\alpha r+\beta s)(e_{2n})=\alpha(r(e_{2n}))+\beta(s(e_{2n}))=f(e_{2n}),\\
&(\alpha r+\beta s)(e_{2n+1})=\alpha(r(e_{2n+1}))+\beta(s(e_{2n+1}))=f(e_{2n+1}).
\end{align*}
Además $\{r,s\}$ es linealmente independiente, pues si $\alpha r+\beta s=0$ para $\alpha,\beta\in R$, entonces al evaluar 
en los $e_{2n}$ se obtiene que $\alpha=0$ y al evaluar en los $e_{2n+1}$ se obtiene que $\beta=0$.   
\end{example} 

\begin{example}
Si $M$ es un módulo libre con base $X$ y $N$ es un módulo libre con base $Y$, entonces
$M\oplus N$ es un módulo libre con base 
\[
\{(x,0):x\in X\}\cup \{(0,y):y\in Y\}.
\]	
\end{example}

\begin{exercise}
Sea $R$ es un anillo conmutativo. Si $M$ y $N$ son libres y finitamente generados, entonces
$\Hom_R(M,N)$ es libre y finitamente generado.	
\end{exercise}

% Como $R$ es conmutativo, $\Hom_R(M,N)$ es un $R$-módulo. Si $\{m_1,\dots,m_k\}$ es base de $M$ 
% y $\{n_1,\dots,n_l\}$ es base de $N$, entonces definimos para cada $i\in\{1,\dots,k\}$ y $j\in\{1,\dots,l\}$ 
% definimos $f_{ij}$ 
% \[
% f_{ij}(m_k)=\begin{cases}
% n_j & \text{si $k=i$},\\
% 0 & \text{si $k\ne i$}.
% \end{cases}
% \]
% Entonces $\{i_ij}$ es base de $\Hom_R(M,N)$. 
Veamos algunos resultados básicos que nos permiten entender qué significa tener un módulo libre. 

\begin{proposition}
Si $M$ es libre, entonces existe un subconjunto $\{m_i:i\in I\}$ de $M$ tal que 
para cada $m\in M$ existen únicos $r_i\in R$, $i\in I$, 
donde $r_i=0$ salvo finitos $i\in I$ 
tales que $m=\sum r_i\cdot m_i$. 
\end{proposition}

\begin{proof}
Como $M$ es libre, existe una base $\{m_i:i\in I\}$ de $M$. Si $m\in M$, entonces
podemos escribir $m=\sum r_i\cdot m_i$ (suma finita) para ciertos $r_i\in I$. Veamos que los $r_i$ son únicos. Si $m=\sum s_i\cdot m_i$, entonces
$\sum (r_i-s_i)\cdot m_i=0$. La independencia lineal del conjunto $\{m_i:i\in I\}$ implica entonces que $r_i=s_i$ para todo $i\in I$.  	
\end{proof}

\begin{proposition}
\label{pro:libre}
Sea $M$ libre con base $\{m_i:i\in I\}$ y sea $N$ un submódulo de $M$. Si $\{n_i:i\in I\}$ es un módulo, existe
un único $f\in\Hom_R(M,N)$ tal que $f(m_i)=n_i$ para todo $i\in I$.  
\end{proposition}

\begin{proof}[Bosquejo de la demostración]
Basta con observar que el único morfismo $f\colon M\to N$ debe definirse como $f(\sum r_im_i)=\sum r_i\cdot n_i$.  	
\end{proof}

Una aplicación sencilla de la proposición anterior:

\begin{example}
Veamos que no existe un epimorfismo $\Z\to\Z\times\Z$ (de $\Z$-módulos). En efecto, si $f\colon\Z\to\Z\times\Z$ es un epimorfismo, 
sea $\{u,v\}$ una base de $\Z\times\Z$. Entonces $f(k)=u$ y $f(l)=v$ para ciertos $k,l\in\Z$. La proposición~\ref{pro:libre} implica
que existe un morfismo $g\colon\Z\times\Z\to\Z$ tal que $g(u)=k$ y $g(v)=l$. En particular, $f\circ g=\id_{\Z\times\Z}$ y luego 
$g$ es un monomorfismo. 	Como 
\[
g(lu-kv)=lg(u)-kg(v)=lk-kl=0,
\]
se tiene entonces que $lu-kv=0$ y luego $k=l=0$, pues $\{u,v\}$ es linealmente independiente, una contradicción.  
\end{example}

Otra propiedad importante de los módulos libres:

\begin{proposition}
Si $M$ es libre, entonces $M\simeq R^{(I)}$ para algún conjunto $I$.
\end{proposition}

\begin{proof}
Supongamos que $M$ es libre con base $\{m_i:i\in I\}$. Vimos una proposición que nos dice que 
existe un único morfismo $f\in\Hom_R(M,R^{(I)})$ tal que 
$f(m_i)=e_i$ para todo $i\in I$, donde 
\[
(e_i)_j=\begin{cases}
	1 & \text{si $i=j$},\\
	0 & \text{si $i\ne j$.}
	\end{cases}	
\]	
Veamos que $f$ es un isomorfismo. 
Primero veamos que $f$ es epimorfismo: dado $(r_i)_{i\in I}\in R^{(I)}$, entonces
$f(\sum r_i\cdot m_i)=(r_i)_{i\in I}$. Para ver que es monomorfismo:
\[
0=f(\sum r_i\cdot m_i)=\sum r_i\cdot f(m_i)=\sum r_i\cdot e_i\implies r_i=0\text{ para todo $i\in I$}.\qedhere
\]
\end{proof}

\begin{corollary}
Todo módulo es (isomorfo a un) cociente de un módulo libre.
\end{corollary}

\begin{proof}
Si $M$ es un módulo, vamos a demostrar que existe un módulo libre $L$ y un epimorfismo $f\in\Hom_R(L,M)$, ya que
entonces $L/\ker f\simeq M$ por el primer teorema de isomorfismos. Sea 
$\{m_i:i\in I\}$ un conjunto de generadores de $M$, que siempre existe, ya que podríamos tomar por ejemplo el conjunto
$\{m:m\in M\}$ y sea $L=R^{(I)}$. Entonces $L$ es libre y $f\colon R^{(I)}\to M$, $e_i\mapsto m_i$, es un epimorfismo. 
\end{proof}

Otra propiedad importante de los módulos libres:

\begin{proposition}
	Si $M$ es un módulo libre, $f\in\Hom_r(N,T)$ es un epimorfismo y $h\in\Hom_R(M,T)$, entonces existe
	$\varphi\in\Hom_R(M,N)$ tal que $f\circ \varphi=h$. 
	% todo: diagrama
\end{proposition}

\begin{proof}
Si $\{m_i:i\in I\}$ es base de $M$, como $f$ es un epimorfismo, para cada $i\in I$ existe $n_i\in N$ tal que $f(n_i)=h(m_i)$. Como 
$M$ es libre, existe un único morfismo $\varphi\colon M\to N$ tal que $\varphi(m_i)=n_i$ para todo $i\in I$. Este morfismo
cumple $f\circ \varphi=h$. 
\end{proof}

%\index{Sumando directo!de un submódulo}
%Recordemos que un submódulo $N$ de $M$ es un \textbf{sumando directo} de $M$ si existe un submódulo $T$ de $M$ tal que
%$M=N\oplus T$. 

\index{Cuerpo!de fracciones}
Sea $R$ un dominio íntegro y sea $S=R\setminus\{0\}$. 

En el conjunto $R\times S$ definimos
la siguiente relación:
\[
(r,s)\sim (r_1,s_1)\Longleftrightarrow rs_1-r_1s=0.
\]

Dejamos como ejercicio verificar que $\sim$ es una relación de equivalencia. 

La clase
de equivalencia del par $(r,s)$ será denotada por $r/s$ o bien $\frac{r}{s}$. 

Puede demostrarse que 
el conjunto
$K(R)=(R\times S)/{\sim}$ de clases de equivalencia 
es un cuerpo con las operaciones
\begin{equation}
\label{eq:K(R)}
\frac{r}{s}+\frac{r_1}{s_1}=\frac{rs_1+r_1s}{ss_1},
\quad
\frac{r}{s}\frac{r_1}{s_1}=\frac{rr_1}{ss_1}.
\end{equation}
$K(R)$ se llama el \textbf{cuerpo de fracciones} de $R$. 
Por ejemplo, 
$K(\Z)=\Q$.

Para demostrar que $K(R)$ es un cuerpo primero
debemos ver que las operaciones~\ref{eq:K(R)} están bien definidas. Por ejemplo, si $r/s\sim r'/s'$ y $r_1/s_1\sim r_1'/s_1'$, entonces
$r/s+r_1/s_1\sim r'/s'+r_1'/s_1'$. En efecto,
como $r/s\sim r'/s'$, entonces $rs'-r's=0$. Similarmente, $r_1s_1'-r_1's_1=0$, pues $r_1/s_1\sim r_1'/s_1'$. Luego 
\begin{gather*}
\frac{rs_1+r_1s}{ss_1}=\frac{r's_1'+r_1's'}{s's_1'}
\shortintertext{pues}
(rs_1+r_1s)s's_1'=rs_1s's_1'+r_1ss's_1'
=r'ss_1s_1'+r_1's_1ss'
=(r's_1'+r_1's)ss_1.
\end{gather*}
De la misma forma se demuestra que el producto también está bien definido. Dejamos como ejercicio
verificar que con estas operaciones $K(R)$ es un cuerpo. 

\begin{theorem}
Sea $R$ un dominio íntegro. 
Si $M$ es un módulo libre y finitamente generado, entonces dos bases cualesquiera de $M$ 
tienen la misma cantidad de elementos.
\end{theorem}

\begin{proof}
Sea $K=K(R)$ el cuerpo de fracciones de $R$. Es sencillo verificar que el grupo abeliano  
$V=\Hom_R(M,K)$ es un $K$-espacio vectorial con la acción 
\[
(\lambda f)(m)=\lambda f(m),
\]
donde $\lambda\in K$, $f\in V$ y $m\in M$.

El espacio vectorial $V$ tiene bien definida su dimensión. 
Calculemos entonces $\dim V$. Para eso, sea $\{e_1,\dots,e_n\}$ una base de $M$. 
Para cada $i\in\{1,\dots,n\}$ sea
\[
f_i\colon M\to K,\quad
e_j\mapsto\begin{cases}
1 & \text{si $i=j$},\\
0 & \text{si $i\ne j$}.
\end{cases}
\]
Veamos que $\{f_1,\dots,f_n\}$ es base de $V$. Es un conjunto de generadores de $V$ pues 
si $f\in V$, entonces
\[
f=\sum_{i=1}^n f(e_i)f_i,
\]
pues estos morfismos coinciden en los elementos de una base de $M$, es decir 
$f(e_j)=(\sum_{i=1}^n f(e_i)f_i)(e_j)$ para todo $j\in\{1,\dots,n\}$.    
Además $\{f_1,\dots,f_n\}$ es linealmente independiente, pues si $0=\sum_{i=1}^n\lambda_if_i$, 
entonces, al evaluar en cada $e_j$, se tiene que 
\begin{align*}
0=\left(\sum_{i=1}^n \lambda_if_i\right)(e_j)=\lambda_j
\end{align*}
para todo $j\in\{1,\dots,n\}$. Luego $n=\dim V$. 
\end{proof}

\begin{definition}
\index{Rango!de un módulo}
Sea $R$ un dominio íntegro. Si $M$ es un módulo finitamente generado, se 
define el \textbf{rango} de $M$ como el cardinal de una base de $M$. Si $M$ no es finitamente
generado diremos que el rango de $M$ es infinito. El rango del módulo $M$ será denotado por $\rank(M)$.   
\end{definition}


El teorema anterior puede extenderse para módulos sobre anillos conmutativos. En consecuencia,
la noción de rango de un módulo queda bien definida para módulos sobre anillos conmutativos.
En efecto, sea $F$ un módulo libre con base en un conjunto $X$. 
Como $R$ es conmutativo, sabemos que $R$ admite un ideal 
maximal $I$. Como $I$ es maximal, el cociente $K=R/I$ es entonces un cuerpo. Puede verificarse
que el subconjunto
\[
I\cdot F=\left\{\sum_{i=1}^n r_i\cdot x_i:n\in\N,\,r_1,\dots,r_n\in I\right\}
\]
es un submódulo de $F$. El módulo cociente $V=F/(I\cdot F)$ es un $K$-espacio
vectorial con base $\{x+I\cdot F:x\in X\}$. Luego $|X|=\dim V$. 