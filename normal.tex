\chapter{Cocientes}
\label{cocientes}

Si $G$ es un grupo y $N$ es un subgrupo de $G$, nos interesa saber cuándo 
el conjunto $G/N$ de coclases es un grupo con 
la operación $G/N\times G/N\to G/N$,  
$(xN,yN)\mapsto xyN$, es decir, cuándo esta operación está 
bien definida. ¿Qué significa eso? Queremos que esa operación sea una función. 
Para eso, se necesita que si $xN=x_1N$ y además $yN=y_1N$, entonces
$xyN=x_1y_1N$. Veamos cómo puede interpretarse esa condición. Si $x^{-1}x_1\in N$ y 
$y^{-1}y_1\in N$, entonces $x_1=xn$ y además $y_1=ym$ para ciertos $m,n\in N$. Entonces
\[
(xy)^{-1}(x_1y_1)=y^{-1}x^{-1}x_1y_1=y^{-1}nym\in N
\]
si y sólo si $y^{-1}ny\in N$. 

\begin{example}
Si $G=\Sym_3$ y $H=\langle (12)\rangle$, entonces $(xN,yN)\mapsto xyN$ no es una función. Para verlo, primero recordemos que 
$G/H=\{H,(123)H,(132)H\}$, donde 
$H=(12)H$, $(123)H=(13)H$ y $(132)H=(23)H$. Tenemos
\[
(132)N=(13)(23)N=(13)N(23)N=(123)N(132)N=N,
\]
una contradiccion. 
\end{example}

\begin{definition}
	\index{Subgrupo!normal}
	Sea $G$ un grupo. 
	Un subgrupo $N$ de $G$ se dice \textbf{normal} si $gNg^{-1}\subseteq N$ para todo
	$g\in G$. Notación: si $N$ es normal en $G$, entonces $N\unlhd G$.
\end{definition}

Si $G$ es un grupo abeliano, todo subgrupo de $G$ es normal en $G$. 

\begin{proposition}
\label{pro:normalidad}
Sea $N$ un subgrupo de $G$. Las siguientes afirmaciones son equivalentes:
\begin{enumerate}
	\item $gNg^{-1}\subseteq N$ para todo $g\in G$.
	\item $gNg^{-1}=N$ para todo $g\in G$.
	\item $gN=Ng$ para todo $g\in G$.
\end{enumerate}	
\end{proposition}

\begin{proof}
Demostremos que $(1)\implies (2)$, que es la única implicación no trivial. Si $n\in N$ y $g\in G$, entonces
$n=g(g^{-1}ng)g^{-1}\in gNg^{-1}$. 	
\end{proof}

\begin{proposition}
	Sea $N$ un subgrupo de $G$. Las siguientes propiedades son equivalentes:
	\begin{enumerate}
		\item $N$ es normal en $G$.
		\item $(gN)(hN)=(gh)N$ para todo $g,h\in G$.
	\end{enumerate}
\end{proposition}

\begin{proof}
	Vamos a demostrar que $(1)\implies(2)$. Sea $g\in G$. Como $gNg^{-1}=N$,
	entonces $(gN)(hN)=g(Nh)N=g(hN)N=(gh)N$. Veamos ahora que $(2)\implies(1)$. Si $g\in G$, entonces
	$gNg^{-1}\subseteq (gN)(g^{-1}N)=(gg^{-1})N=N$. 
\end{proof}

\begin{examples}
Si $G$ es un grupo, entonces 
$\{1\}$ y $G$ son subgrupos normales de $G$.
\end{examples}

\begin{example}
\index{Centro!de un grupo}
Si $G$ es un grupo, $Z(G)$ es un subgrupo normal de $G$. Más aún, si $N\leq Z(G)$, entonces $N\unlhd G$. 	
\end{example}

\begin{example}
Si $G$ es un grupo, entonces $[G,G]$ es un subgrupo normal de $G$ pues si $x\in [G,G]$ y $g\in G$, entonces 
$gxg^{-1}=(gxg^{-1}x^{-1})x=[g,x]x\in [G,G]$. Alternativamente, 
\[
g\left(\prod_{i=1}^k[x_i,y_i]\right)g^{-1}=\prod_{i=1}^k [gx_ig^{-1},gy_ig^{-1}]
\]
para todo $g,x_1,\dots,x_k,y_1,\dots,y_k\in G$. 	
\end{example}

\begin{example}
Para todo $n\in\N$, $\Alt_n$ es un subgrupo normal de $\Sym_n$. 
De hecho, si $\sigma\in\Alt_n$ y $\tau\in\Sym_n$, entonces $\tau\sigma\tau^{-1}\in\Alt_n$ pues  
\[
\sgn(\tau\sigma\tau^{-1})=\sgn(\sigma)=1. 
\]
\end{example}

\begin{example}
Si $N$ es un subgrupo de $G$ tal que $(G:N)=2$, entonces $N$ es normal en $G$. Queremos demostrar que $gN=Ng$ para todo $g\in G$. Sea $g\in G$. Si $g\in N$, entonces $gN=Ng$. Si $g\not\in N$, entonces
$gN\ne N$. Como $(G:N)=2$, podemos escribir a $G$ como $G=N\cup gN$ (unión disjunta). En consecuencia, $gN=G\setminus N$. Similarmente se demuestra que
$Ng=G\setminus N$ y luego $gN=Ng$. 	
\end{example}

\begin{example}
El ejemplo anterior nos permite demostrar que $\langle (123)\rangle\unlhd\Sym_3$. Por otro lado, $\langle (12)\rangle$ no es normal en $\Sym_3$ pues
por ejemplo $(13)(12)(13)=(23)\not\in\langle(12)\rangle$. 
\end{example}

\begin{example}
$\SL_n(\R)$ es normal en $\GL_n(\R)$ pues si $g\in\GL_n(\R)$ y $x\in\SL_n(\R)$, entonces $\det(gxg^{-1})=(\det g)(\det x)(\det g)^{-1}=1$. 
\end{example}

\begin{example}
\index{Grupo!de Klein}
El grupo de Klein $K=\{\id,(12)(34),(13)(24),(14)(23)\}$ es normal en $\Sym_4$. 
\end{example}

%\begin{example}
%\end{example}

\index{Producto!semidirecto}
En el ejercicio siguiente nos encontramos con un caso particular del producto semidirecto de dos grupos, una construcción general que resulta de mucha utilidad. 

\begin{exercise}
\index{Producto!semidirecto}
Sea $G=\Z/p\times(\Z/p)^\times$ el grupo dado por la operación 
\[
(x,y)(u,v)=(x+yu,yv).
\]	
Demuestre que $\{(x,1):x\in\Z/p\}$ es normal en $G$ y que $\{(0,y):y\in(\Z/p)^\times\}$ no es normal en $G$. 
\end{exercise}

El siguiente ejercicio es útil:

\begin{exercise}
\index{Normalizador!de un subgrupo}
Si $S$ es un subgrupo de $G$, se define el \textbf{normalizador} de $S$ en $G$ al subgrupo
\[
N_G(S)=\{g\in G:gSg^{-1}=S\}.
\]	
Demuestre que valen las siguientes afirmaciones:
\begin{enumerate}
\item $S\unlhd N_G(S)$.
\item Si $S\leq T\leq G$ y $S\unlhd T$, entonces $T\leq N_G(S)$.
\end{enumerate}
\end{exercise}

El ejercicio anterior nos dice que el normalizador de un subgrupo $S$ en $G$ es el mayor subgrupo de $G$ que contiene a $S$ como subgrupo normal. 
 
Veamos algunos ejemplos de subgrupos normales un poco más difíciles. Primero calcularemos los subgrupos normales de $\Alt_4$. 

\begin{example}
\index{Subgrupos!normales de $\Alt_4$}
Vamos a demostrar que 
$\{\id\}$, $K=\{\id,(12)(34),(13)(24),(14)(23)\}$ y $\Alt_4$ son los únicos subgrupos normales de $\Alt_4$. 
 	
Como $\Alt_4=\{\text{3-ciclos}\}\cup K$, $K$ es el único subgrupo de $\Alt_4$ con cuatro elementos, y esto implica que $K$ es normal en $\Alt_4$ (pues cada conjugado $gKg^{-1}$ también será un subgrupo de $\Alt_4$ de cuatro elementos). Sea $N\ne\{\id\}$ un subgrupo normal de $\Alt_4$. Si $N$ contiene un 3-ciclo, digamos
$(abc)\in N$. entonces
\[
(acd)=(bcd)(abc)(bcd)^{-1}\in N
\]
y luego $N=\Alt_4$ (pues todos los 3-ciclos están en $N$). Supongamos entonces que $N$ no contiene 3-ciclos. Entonces algún elemento no trivial de $K$ 
pertenece a $N$, digamos $(ab)(cd)\in N$. En consecuencia, 
\[
(ac)(bd)=(bcd)(ab)(cd)(bcd)^{-1}\in N,\quad
(ad)(bc)=(ab)(cd)(ac)(bd)\in N
\]
y luego $N=K$. 
\end{example}
 
Es importante remarcar que la normalidad no es transitiva.

\begin{exercise}
\index{Grupo!diedral}
Sea $G=\D_4$ el grupo diedral de tamaño ocho y sean $N=\langle s,r^2\rangle$ y $H=\langle s\rangle$. Demuestre que 
$H$ es normal en $N$, $N$ es normal en $G$ pero $H$ no es normal en $G$.    
\end{exercise} 
 
Vamos a calcular ahora los subgrupos normales de $\Sym_4$. 
  
\begin{example}
\index{Subgrupos!normales de $\Sym_4$}
Vamos a demostrar que $\{\id\}$, $K$, $\Alt_4$ y $\Sym_4$ son los únicos subgrupos normales de $\Sym_4$.

Sea $N$ un subgrupo normal de $\Sym_4$. Si $N\subseteq\Alt_4$, entonces $N$ es normal en $\Alt_4$ y luego, por lo visto en el ejemplo anterior, $N=\{\id\}$, 
$N=K$ o bien $N=\Alt_4$. Supongamos entonces que $N\not\subseteq\Alt_4$, es decir $N$ contiene una permutación impar. Si $\sigma\in\Sym_4$ es una permutación impar, entonces $\sigma$ es una trasposición o $\sigma$ es un 4-ciclo. 

Si $N$ contiene una trasposición, entonces todas las trasposiciones
también pertenecen a $N$ pues
\[
\tau(ij)\tau^{-1}=(\tau(i)\,\tau(j))
\]
para todo $\tau\in\Sym_4$. En este caso, $N=\Sym_4$ pues $\Sym_4$ está generado por trasposiciones.   

Si $N$ contiene un 4-ciclo, todos los 4-ciclos también están en $N$ pues
\[
\tau(ijkl)\tau^{-1}=(\tau(i)\,\tau(j)\,\tau(k)\,\tau(l))
\]
para todo $\tau\in\Sym_4$ y además $K\subseteq N$ pues
\[
(ac)(bd)=(abcd)^2.
\]
Esto nos dice que $|N|\geq10$. Como además $K\subseteq N$, se tiene que $|N\cap\Alt_4|\geq 5$. Por otro lado, $N\cap\Alt_4$ es un subgrupo normal de $\Alt_4$. 
Por lo visto en el ejemplo anterior, $N\cap\Alt_4=\Alt_4\subseteq N$. En conclusión, $N=\Sym_4$.   
\end{example}

\begin{theorem}
\label{Grupo!cociente}
Si $N$ es un subgrupo normal de $G$, entonces $G/N$ es un grupo con la operación $(xN)(yN)=(xy)N$.  
\end{theorem}

\begin{proof}
Sabemos que la normalidad de $N$ en $G$ garantiza la buena definición de la operación. Calculos rutinarios, que dejamos como ejercicio, demuestran que 
esta operación transforma al conjunto $G/N$ en un grupo. 
\end{proof}

% todo: Mover cocientes de S4 al capítulo de isomorfismos
No estamos en condiciones de poder entender qué tipo de grupo obtenemos
como grupo cociente, ya que para eso es necesario poder entender qué significa 
que dos grupos sean ``iguales'' aunque parezcan distintos. 
 
%Veamos cómo son los posibles cocientes de $\Sym_4$. 
% todo: Listar las coclases de S4 por K, i.e. los elementos del grupo cociente#
\begin{example}
\index{Cocientes!de $\Sym_4$}
	Sabemos que $\{\id\}$, $K$, $\Alt_4$ y $\Sym_4$ son los únicos subgrupos normales de $\Sym_4$. Trivialmente 
	obtenemos que
	\[
	\Sym_4/\{\id\}\simeq\Sym_4,\quad
	\Sym_4/\Alt_4\simeq\Z/2,\quad
	\Sym_4/\Sym_4\simeq\{\id\}.
	\]
	Veamos qué podemos decir del cociente $Q=\Sym_4/K$. Sabemos que $Q$ tiene orden seis y que $Q$ es no abeliano pues
	\[
	(12)K(13)K=(12)(13)K=(132)K\ne (123)K=(13)(12)K=(13)K(12)K.
	\]
	Vimos que existe un único grupo no abeliano de orden seis. Luego $Q\simeq\Sym_3$. 
\end{example}

%Para terminar el capítulo mencionamos dos ejercicios de mucha utilidad. 

\begin{proposition}
Si $H$ es un subgrupo normal de $G$, entonces $G/H$ es abeliano si y sólo si $[G,G]\subseteq H$. 
\end{proposition}

\begin{proof}
Sean $x,y\in G$. Entonces 
\begin{align*}
    (xH)(yH)=(yH)(xH) \Longleftrightarrow (xy)H=(yx)H \Longleftrightarrow x^{-1}y^{-1}xy\in H.
\end{align*}
Luego $G/H$ es conmutativo si y sólo si $[x,y]=xyx^{-1}y^{-1}\in H$ para todo $x,y\in G$. 
\end{proof}

Veamos una pequeña aplicación:

\begin{example}
\index{Conmutador!de $\Alt_4$}
$[\Alt_4,\Alt_4]=K=\{\id,(12)(34),(13)(24),(14)(23)\}$. 
Sabemos que $K$ es normal en $\Alt_4$. Como $\Alt_4/K$ tiene tres elementos, es abeliano. El ejercicio anterior, entonces, 
nos dice que $[\Alt_4,\Alt_4]\subseteq K$. Por otro lado, como 
\[
(ab)(cd)=[(abc),(cda)],
\]   	
se concluye que $K\subseteq[\Alt_4,\Alt_4]$. 
\end{example}

Otra propiedad importante:

\begin{proposition}
Si $G/Z(G)$ es cíclico, entonces $G$ es abeliano.
\end{proposition}

\begin{proof}
Supongamos que $G/Z(G)=\langle gZ(G)\rangle$. Sean $x,y\in G$. Escribamos $xZ(G)=g^kZ(G)$ y también $yZ(G)=g^lZ(G)$, es decir
$x=g^kz_1$, $y=g^lz_2$ para ciertos $k,l\in\Z$ y $z_1,z_2\in Z(G)$. Luego $xy=yx$. 
\end{proof}

\begin{theorem}
Sea $p$ un número primo y sea $H$ un subgrupo de $G$. Si $(G:H)=p$, las siguientes afirmaciones son equivalentes:
\begin{enumerate}
\item $H$ es normal en $G$.
\item Si $g\in G\setminus H$, entonces $g^p\in H$.
\item Si $g\in G\setminus H$, entonces $g^n\in H$ para algún $n\in\N$ sin divisores primos $<p$.
\item Si $g\in G\setminus H$, entonces $g^k\not\in H$ para todo $k\in\{2,\dots,p-1\}$. 
\end{enumerate}
\end{theorem}

\begin{proof}
	La implicación $(1)\implies(2)$ es consecuencia inmediata del teorema de Lagrange, pues $|G/H|=p$. 
	
	La implicación $(2)\implies(3)$ es trivial pues $p$ es un número primo. 
	
	Demostremos que $(3)\implies(4)$. Si $g^k\in H$ para algún $k\in\{2,\dots,p-1\}$, como $\gcd(k,n)=1$, existen $r,s\in\Z$ tales que
	$rk+sn=1$. Luego
	\[
	g=g^1=g^{rk+sn}=(g^k)^r(g^n)^s\in H,
	\]
	una contradicción. 
	
	Para finalizar demostremos que $(4)\implies(1)$. Sea $x\in G\setminus H$ y sea $h\in H$. Queremos demostrar que entonces $xhx^{-1}\in H$. Si $y=xhx^{-1}\not\in H$, entonces
	$y^k\not\in H$ para todo $k\in\{2,\dots,p-1\}$. Esto implica que las coclases 
	\[
	H,yH,y^2H,\dots,y^{p-1}H
	\]
	son todas distintas (pues si $y^iH=y^jH$ para $i,j$ tales que $i<j$, entonces $y^{j-i}\in H$ con $j-i\leq p-2$). Como $y=xhx^{-1}$, 
	entonces
	\[
	(yx)H=(xh)H=xH=y^iH
	\]
	para algún $i\in\{0,1,\dots,p-1\}$. Si $i=0$, entonces $yx=xh\in H$ y luego $x\in H$, una contradicción. Luego $(yx)H=y^iH$ para algún $i\in\{1,\dots,p-1\}$ y entonces
	\[
	y^iH=xH=y^{i-1}H
	\]
	para algún $i\in\{0,\dots,p-2\}$, una contradicción.  
\end{proof}

Veamos algunas consecuencias. La primera se hará en el caso en que el grupo sea finito. 

\begin{corollary}
\label{cor:p_menor}
	Sea $p$ el menor número primo que divide al orden de un grupo finito  
	$G$ y sea $H$ es un subgrupo de $G$ índice $p$. Entonces $H$ es normal en $G$. 
\end{corollary}

\begin{proof}
	Si $g\in G\setminus H$, entonces $g^n=1\in H$, donde $n=|G|$. Como $p$ es primo, $n$ no tiene divisores primos $<p$. El teorema anterior impica entonces que $H$ es normal en $G$. 
\end{proof}

En el teorema no pedimos que $G$ sea un grupo finito. Podemos entonces obtener el siguiente resultado.

\begin{corollary}
Sea $p$ un número primo y sea $G$ un grupo tal que todo elemento tiene orden una potencia de $p$. Si $H$ es un subgrupo de $G$ de índice $p$, entonces $H$ es normal en $G$.  
\end{corollary}

\begin{proof}
Sea $g\in G\setminus H$ y sea $n=|g|$. Como todo elemento de $G$ tiene orden una potencia de $p$, 
$n$ es en particular una potencia de $p$ y, en consecuencia, $n$ no posee divisores primos $<p$. Como además $g^n=1\in H$, el teorema anterior implica que $H$ es normal en $G$. 
en particular $g^n\in H$.  	
\end{proof}

Terminamos el capítulo con una definición importante. 

\begin{definition}
\index{Grupo!simple}
Diremos que un grupo $G$ es \textbf{simple} si $G\ne\{1\}$ y sus únicos subgrupos normales
son $G$ y $\{1\}$.
\end{definition}

Por ahora, nos quedaremos conformes al observar que si $p$ es un número primo, entonces $\Z/p$ es un grupo simple. Veremos otros ejemplos más adelante.  


