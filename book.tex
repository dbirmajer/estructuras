\documentclass[graybox,envcountsect]{svmono}

\usepackage[T1]{fontenc}
\usepackage[utf8]{inputenc}

%\usepackage[notref,notcite]{showkeys}
\usepackage{float}
\usepackage[small,bf]{caption}
\usepackage{amssymb}
\usepackage{amstext}
\usepackage{amsmath}
\usepackage{mathtools}
\usepackage{xcolor} 
\usepackage{centernot}
\usepackage{listings}
\usepackage{multicol}
\usepackage{mathptmx}
\usepackage{amsmath}
\usepackage{helvet}
\usepackage{courier}
\usepackage{type1cm}         
\usepackage{makeidx}        
\usepackage{graphicx}        
\usepackage{multicol}        
\usepackage[spanish]{babel}
\usepackage[all]{xy}
\usepackage{hyperref} 
\usepackage{stmaryrd}

% https://q.uiver.app/
\usepackage{tikz-cd}
%\usepackage{quiver}
% I need that line to be able to write commutative diagrams
\usetikzlibrary{babel}
	
\usepackage[bottom]{footmisc}

% for QED
\let\proof\relax\let\endproof\relax
\usepackage{amsthm}

\overfullrule=1mm

%%% for Spanish
\def\abstractname{Resumen}%
\def\ackname{Agradecimientos}%
\def\andname{y}%
\def\bibname{Referencias}%
\def\lastandname{, y}%
\def\appendixname{Apéndice}%
\def\chaptername{Capítulo}%
\def\claimname{Afirmación}%
\def\conjecturename{Conjetura}%
\def\contentsname{Contenidos}%
\def\corollaryname{Corolario}%
\def\definitionname{Definici\'on}%
\def\emailname{e-mail}%
\def\examplename{Ejemplo}%
\def\examplesname{Ejemplos}%
\def\exercisename{Ejercicio}%
\def\figurename{Fig.}%
\def\forewordname{Foreword}%
\def\keywordname{{\bf Palabras clave:}}%
\def\indexname{Índice}%
\def\lemmaname{Lema}%
\def\listfigurename{Figuras}%
\def\listtablename{Tablas}%
\def\notename{Nota}%
\def\partname{Parte}%
\def\prefacename{Prefacio}%
\def\problemname{Problema}%
\def\proofname{Demostración}%
\def\propertyname{Propiedad}%
\def\propositionname{Proposici\'on}%
\def\questionname{Pregunta}%
\def\refname{Referencias}%
\def\remarkname{Observación}%
\def\seename{see}%
\def\solutionname{Solución}%
\def\tablename{Tabla}%
\def\theoremname{Teorema}
\def\notationname{Notación}
\def\conventionname{Convención}

% change numbers 
\let\remark\relax
\let\theorem\relax
\let\lemma\relax
\let\definition\relax
\let\proposition\relax
\let\corollary\relax
\let\exercise\relax
\let\example\relax
\let\conjecture\relax
\spnewtheorem{theorem}{\theoremname}[section]{\bfseries}{\itshape}
\renewcommand\thetheorem{\thesection.\arabic{theorem}}
\spnewtheorem{lemma}[theorem]{\lemmaname}{\bfseries}{\itshape}
\spnewtheorem{definition}[theorem]{\definitionname}{\bfseries}{\upshape}
\spnewtheorem{proposition}[theorem]{\propositionname}{\bfseries}{\itshape}
\spnewtheorem{corollary}[theorem]{\corollaryname}{\bfseries}{\itshape}
\spnewtheorem{exercise}[theorem]{\exercisename}{\bfseries}{\upshape}
\spnewtheorem{example}[theorem]{\examplename}{\bfseries}{\upshape}
\spnewtheorem{examples}[theorem]{\examplesname}{\bfseries}{\upshape}
\spnewtheorem{remark}[theorem]{\remarkname}{}{\upshape}
\spnewtheorem{conjecture}[theorem]{\conjecturename}{\bfseries}{\upshape}
\spnewtheorem{notation}[theorem]{\notationname}{\bfseries}{\upshape}
\spnewtheorem{convention}[theorem]{\conventionname}{\bfseries}{\upshape}

\makeindex             

\renewcommand{\I}{\operatorname{I}}
\newcommand{\II}{\operatorname{II}}

\newcommand{\GAP}{\textsf{GAP}}
\newcommand{\FK}{\mathcal{E}}
\newcommand{\ad}[1]{\operatorname{ad}\,#1}

\newcommand{\N}{\mathbb{N}}
\newcommand{\Q}{\mathbb{Q}}
\newcommand{\Z}{\mathbb{Z}}
\newcommand{\F}{\mathbb{F}}
\newcommand{\R}{\mathbb{R}}
\newcommand{\C}{\mathbb{C}}
\renewcommand{\H}{\mathbb{H}}
\newcommand{\A}{\mathbb{A}}
\newcommand{\K}{\mathbb{K}}
\newcommand{\T}{\mathbb{T}}
\renewcommand{\D}{\mathbb{D}}
\newcommand{\B}{\mathbb{B}}

\newcommand{\cL}{\mathcal{L}}
\newcommand{\cE}{\mathcal{E}}
\newcommand{\cH}{\mathcal{H}}

\newcommand{\GF}{\mathsf{GF}}
\newcommand{\MAX}{\operatorname{MAX}}
\newcommand{\MIN}{\operatorname{MIN}}
\newcommand{\cf}{\operatorname{cf}}
\newcommand{\cl}{\operatorname{cl}}
\newcommand{\cd}{\operatorname{cd}}
\newcommand{\bL}{\mathbf{L}}
\newcommand{\bP}{\mathbf{P}}

\newcommand{\Nil}{\operatorname{Nil}}
\newcommand{\rad}{\operatorname{rad}}
\newcommand{\rank}{\operatorname{rank}}

\newcommand{\Aff}{\mathrm{Aff}}
\newcommand{\Ann}{\operatorname{Ann}}
\newcommand{\Der}{\operatorname{Der}}
\newcommand{\Core}{\operatorname{Core}}
\newcommand{\Soc}{\operatorname{Soc}}
\newcommand{\Rad}{\mathrm{rad}}
\newcommand{\Inn}{\mathrm{Inn}}
\newcommand{\dist}{\mathrm{dist}}
\newcommand{\Out}{\mathrm{Out}}
\newcommand{\Ext}{\mathrm{Ext}}
\newcommand{\Img}{\mathrm{im}}
\newcommand{\Hol}{\operatorname{Hol}}
\newcommand{\Hom}{\operatorname{Hom}}
\newcommand{\Alg}{\operatorname{Alg}}
\newcommand{\Bil}{\operatorname{Bil}}
\newcommand{\op}{\operatorname{op}}
\newcommand{\gr}{\operatorname{gr}}
\newcommand{\Syl}{\mathrm{Syl}}
\newcommand{\id}{\operatorname{id}}
\newcommand{\Aut}{\operatorname{Aut}}
\newcommand{\End}{\operatorname{End}}
\newcommand{\Irr}{\operatorname{Irr}}
\newcommand{\Alt}{\mathbb{A}}
\newcommand{\Sym}{\mathbb{S}}
\newcommand{\lcm}{\operatorname{mcm}}
\renewcommand{\gcd}{\operatorname{mcd}}
\newcommand{\diag}{\operatorname{diag}}
\newcommand{\spec}{\operatorname{Spec}}
\newcommand{\supp}{\operatorname{supp}}
\newcommand{\trace}{\operatorname{traza}}
\newcommand{\sgn}{\operatorname{signo}}


\newcommand{\inner}{\operatorname{inn}}
\newcommand{\ext}{\operatorname{ext}}
\newcommand{\im}{\operatorname{im}}
\newcommand{\Fix}{\operatorname{Fix}}
\newcommand{\GL}{\mathbf{GL}}
\newcommand{\SL}{\mathbf{SL}}
\newcommand{\PSL}{\mathbf{PSL}}
\newcommand{\PGL}{\mathbf{PGL}}
\newcommand{\G}{\mathbb{G}}

\newcommand{\legendre}[2]{\left(\frac{#1}{#2}\right)}

% para enumerar
\renewcommand{\labelenumi}{\arabic{enumi})}

% multiset
\def\multiset#1#2{\ensuremath{\left(\kern-.3em\left(\genfrac{}{}{0pt}{}{#1}{#2}\right)\kern-.3em\right)}}

% column vector
\newcount\colveccount
\newcommand*\colvec[1]{
\global\colveccount#1
\begin{pmatrix}
	\colvecnext
	}
	\def\colvecnext#1{
	#1
	\global\advance\colveccount-1
	\ifnum\colveccount>
	\\
	\expandafter\colvecnext
	\else
\end{pmatrix}
\fi
}

% To remove Springer from the title page
\usepackage{etoolbox}
\makeatletter
\patchcmd{\@maketitle}{{\Large Springer\par}}{}{}{}
\makeatother

\begin{document}

\lstset{language=GAP,
  showstringspaces=false,
  xleftmargin=0.0cm,
  xrightmargin=0.0cm,
  basicstyle=\small\ttfamily,
  frame=single,
  framerule=0pt,
}

% no quiero capítulos, solamente secciones
\renewcommand{\thesection}{\arabic{chapter}} 


\author{Leandro Vendramin}
\title{Álgebra II}
\subtitle{-- Notas --}
\maketitle

\frontmatter

%\include{dedic}
%\include{foreword}
\preface

Las notas corresponden al curso Álgebra II. 

Muchas personas leyeron las notas y reportaron errores: 
Cristian Meza Alarcon, Jessica Singer, Matías Conde, Santiago Varela. 

\vspace{\baselineskip}
\begin{flushright}\noindent
Buenos Aires, XX de XX de 2021\hfill {\it Leandro Vendramin}\\
\end{flushright}

%\include{}

\tableofcontents

%\include{acronym}

\mainmatter

\part{Grupos}

\chapter{Grupos y subgrupos}
\label{grupos}

Antes de dar la definición de grupo recordemos que una operación binaria en un
cierto conjunto $X$ es una función $X\times X\to X$, $(x,y)\mapsto xy$. Observemos que la notación que utilizamos para esta operación binaria genérica es la misma que usualmente se usa para la multiplicación de números, aunque nuestra operación sea algo mucho más general. Por ejemplo, $(x,y)\mapsto x-y$ es una operación binaria en $\Z$ pero no lo es en $\N$. 

\begin{definition}
	\index{Grupo}
	Un \textbf{grupo} es un conjunto no vacío $G$ junto con una operación binaria 
	en $G$ que satisface las siguientes propiedades:
	\begin{enumerate}
		\item Asociatividad: $x(yz)=(xy)z$ para todo $x,y,z\in G$.
		\item Existencia de elemento neutro: existe un elemento $e\in G$ tal que $ex=xe=x$ para todo $x\in G$.
		\item Existencia del inverso: para cada $x\in G$ existe $y\in G$ tal que $xy=yx=e$. 
	\end{enumerate}
\end{definition}

El axioma sobre asociatividad que aparece en nuestra definición de grupo es
suficiente para demostrar que todos los productos ordenados que podamos formar
con los elementos $x_1,x_2,\dots,x_n$ son iguales. Por ejemplo
\[
	(x_1x_2)((x_3x_4)x_5)=x_1(x_2(x_3(x_4x_5)))
\]
y podemos escribir sin ambigüedad $x_1x_2\cdots
x_5$, sin preocuparnos por poner paréntesis. Esta observación suele demostrarse por inducción, así se hace por ejemplo en el libro de Lang. Daremos una demostración mucho más sencilla en el capítulo~\ref{cocientes}, como aplicación del teorema de Cayley. 

\begin{proposition}
	En un grupo $G$, cada $x\in G$ admite un único inverso $x^{-1}\in G$. 
\end{proposition}

\begin{proof}
	Si $y,z\in G$ son ambos inversos del elemento $x\in G$, entonces, gracias a
	los axiomas que definen un grupo, tenemos que $z=z(xy)=(zx)y=1y=y$. 
\end{proof}

\begin{exercise}
	Demuestre que el elemento neutro de un grupo es único. 
\end{exercise}

El elemento neutro de un grupo $G$ será denotado por $1_G$ o simplemente como
$1$ cuando no haya peligro de confusión. El inverso de un elemento $x\in G$ será denotado por $x^{-1}$. 

% Si $e_1$ y $e_2$ son neutros para el grupo $G$, entonces
% $e_1=e_1e_2=e_2$. 
De la definición podemos obtener fácilmente otras propiedades de los inversos de elementos de un grupo:
\begin{enumerate}
	\item $(x^{-1})^{-1}=x$ para todo $x\in G$.
	\item $(xy)^{-1}=y^{-1}x^{-1}$ para todo $x,y\in G$. 
\end{enumerate}
	
\begin{exercise}
	Demuestre que en un grupo $G$ la ecuación $ax=b$ tiene a $x=a^{-1}b$ como
	única solución. Similarmente, $x=ba^{-1}$ es la única solución de la
	ecuación $xa=b$. 
\end{exercise}

\begin{definition}
	\index{Grupo!abeliano}
	Un grupo $G$ se dirá \textbf{abeliano} si $xy=yx$ para todo $x,y\in G$.
\end{definition}

A veces, cuando tratemos con grupos abelianos, utilizaremos la notación
aditiva. Eso significa que la operación binaria será $(x,y)\mapsto x+y$, el neutro será denotado por $0$ 
y el inverso de un cierto elemento $x$ será $-x$. 

\begin{definition}
\index{Grupo!orden de un}
\index{Grupo!finito}
\index{Grupo!infinito}
El \textbf{orden} $|G|$ de un grupo $G$ es el cardinal de $G$. Un grupo $G$ se dirá finito
si $|G|$ es finito e infinito en caso contrario. 
\end{definition}

\begin{notation}
	Sea $G$ un grupo y sea $g\in G$. Si $k\in\Z\setminus\{0\}$, escribimos 
	\begin{align*}
		& g^k=g\cdots g\quad (k-\text{veces}) && \text{si $k>0$},\\
		& g^k=g^{-1}\cdots g^{-1}\quad (|k|-\text{veces}) && \text{si $k<0$}.
	\end{align*}
	Por convención, además, $g^0=1$. 
\end{notation}

\begin{exercise}
	Si $G$ es un grupo, entonces
	\begin{enumerate}
		\item $(g^k)^l=g^{kl}$ para todo $g\in G$ y todo $k,l\in\Z$. 
		\item Si $G$ es abeliano, entonces $(gh)^k=g^kh^k$ para todo $g,h\in G$
			y todo $k\in\Z$. 
	\end{enumerate}
\end{exercise}

\begin{exercise}
	Sean $G$ un grupo y $g\in G$. Demuestre que las funciones $L_g\colon G\to
	G$, $x\mapsto gx$, y $R_g\colon G\to G$, $x\mapsto xg$, son biyectivas.
\end{exercise}

\begin{examples}
	Ejemplos de grupos abelianos:
	\begin{enumerate}
		\item $\Z$, $\Q$, $\R$ y $\C$ con la suma usual.
		\item Los enteros $\Z/n$ módulo $n$ con la suma usual.
		\item $\Q\setminus\{0\}$, $\R\setminus\{0\}$ y $\C\setminus\{0\}$ con
			la multiplicación usual. 
		\item El conjunto $\left(\Z/p\right)^\times=\Z/p\setminus\{0\}$ de enteros módulo $p$ inversibles con la multiplicación usual, donde
			$p$ es un número primo. 
	\end{enumerate}
\end{examples}

\index{Grupo!tabla (de multiplicación)}
Si $G=\{g_1,g_2,\dots,g_n\}$ es un grupo finito, la \textbf{tabla}  
del grupo $G$ es la matriz de $n\times n$ que en el lugar 
$i,j$ tiene al elemento $g_ig_j$. Esta tabla se conoce en la literatura como la \emph{tabla de multiplicación} del grupo. 
Como esta terminología puede resultar confusa en caso de trabajar con grupos aditivos, preferimos hablar simplemente
de tablas de un grupo y no hacer referencia al tipo de operación involucrada. Como ejemplo, 
vemos que la tabla del grupo
$\Z/4$ es
\begin{center}
  \begin{tabular}{l|cccc}
     &0&1&2&3 \\
    \hline
    0 & 0 & 1 & 2 & 3\\
    1 & 1 & 2 & 3 & 0\\
    2 & 2 & 3 & 0 & 1\\
    3 & 3 & 0 & 1 & 2
  \end{tabular}
\end{center}

\begin{example}
Sea $H=\{1,-1,i,-i,j,-j,k,-k\}$ con la multiplicación dada por las siguientes reglas:
\[
i^2=j^2=k^2=ijk=-1.
\]
Dejamos como ejercicio calcular la tabla de $H$ y verificar que $H$ es un grupo. 
\end{example}

\begin{example}
	Sea $n\in\N$. El conjunto $G_n=\{z\in\C:z^n=1\}$ es un grupo abeliano con
	el producto usual de números complejos.	También 
	$\cup_{n\geq1}G_n$ es un grupo abeliano.
\end{example}

\begin{example}
	Sea $n\geq2$.  El conjunto $\GL_n(\R)$ de matrices inversibles de $n\times
	n$ con la multiplicación usual de matrices es un grupo no abeliano. 
\end{example}

\begin{example}
	Sea $X$ un conjunto. El conjunto $\Sym_X$ de funciones $X\to X$ biyectivas
	con la composición de funciones es un grupo.  Si $|X|\geq3$, el grupo
	$\mathbb{S}_{X}$ no es abeliano: sean tres elementos distintos $a,b,c\in
	X$ y sean $f\colon X\to X$ biyectiva tal que $f(a)=b$, $f(b)=c$ y $f(c)=a$ y $g\colon X\to
	X$ biyectiva tal que $g(a)=b$, $g(b)=a$ y $g(x)=x$ para todo $x\in
	X\setminus\{a,b\}$.  Entonces $fg\ne gf$. 
\end{example}

\index{Permutación}
\index{Grupo!simétrico}
Si $X=\{1,2,\dots,n\}$, $\Sym_X$ será denotado por $\Sym_n$ y se denominará el
\textbf{grupo simétrico} de grado $n$. Los elementos de $\Sym_n$ serán 
denominados \textbf{permutaciones} de grado $n$. 
Notemos que $|\Sym_n|=n!$ y que $\Sym_n$
es abeliano si y sólo si $n\in\{1,2\}$. Cada elemento de $\Sym_n$ es una función
$f\colon\{1,\dots,n\}\to \{1,\dots,n\}$ y por lo tanto puede escribirse como nos resulte conveniente. 
Una notación bastante utilizada es la siguiente: escribiremos
\[
\binom{12345}{32145}
\]
para denotar a la
función $f\colon\{1,2,3,4,5\}\to\{1,2,3,4,5\}$ tal que
$f(1)=3$, $f(2)=2$, $f(3)=1$, $f(4)=4$ y $f(5)=5$. 

\begin{example}[el grupo de Klein]
\index{Grupo!de Klein}
El grupo 
\[
K=\left\{ \mathrm{id},\binom{1234}{2143},\binom{1234}{3412},\binom{1234}{4321}\right\} 
\]
es un grupo abeliano. Observar que $K$ está contenido en $\mathbb{S}_{4}$. 
Dejamos como ejercicio calcular la tabla del grupo de Klein. 
\end{example}

\begin{example}
\index{Grupo!simétrico $\Sym_3$}
	Sabemos que el conjunto $\Sym_3$ de funciones $\{1,2,3\}\to\{1,2,3\}$
	biyectivas es un grupo con la composición. El grupo $\Sym_3$ tiene orden
	seis y sus elementos son las permutaciones
	\[
	\id,\binom{123}{213},\binom{123}{321},\binom{123}{132},\binom{123}{231},\binom{123}{312}.
	\]
	Otra notación muy utilizada involucra la \emph{descomposición de una permutación en ciclos disjuntos}. 
	En este caso, los elementos de $\Sym_3$ serán escritos como 
	\[
		\id,(12),(13),(23),(123),(132),
	\]
	donde, por ejemplo, el símbolo $(12)$ representa la
	función $\{1,2,3\}\to\{1,2,3\}$ tal que
	$1\mapsto 2$, $2\mapsto 1$ y $3\mapsto 3$. Queda como ejercicio calcular la tabla del grupo $\Sym_3$. 
\end{example}

Más adelante veremos que la notación de una permutación como producto de ciclos disjuntos es de gran utilidad. 

\begin{example}
Sea $n\in\N$. Las unidades de $\Z/n$ forman un grupo con la multiplicación usual módulo $n$. La notación que utilizaremos es
\[	
\mathcal{U}(\Z/n)=\{x\in\Z/n:\gcd(x,n)=1\}.
\]
En general, el orden de $\mathcal{U}(\Z/n)$ es $\varphi(n)$, donde $\varphi$ denota a la función de Euler, es decir
\[
\varphi(n)=|\{x\in\Z:1\leq x\leq n,\,\gcd(x,n)=1\}|.
\]

Veamos un ejemplo concreto: la tabla del grupo 
$\mathcal{U}(\Z/8)=\{1,3,5,7\}$
es
\begin{center}
  \begin{tabular}{l|cccc}
     &1&3&5&7 \\
    \hline
    1 & 1 & 3 & 5 & 7\\
    3 & 3 & 1 & 7 & 5\\
    5 & 5 & 7 & 1 & 3\\
    7 & 7 & 5 & 3 & 1
  \end{tabular}
\end{center}
\end{example}

\begin{exercise}
	\index{Producto!directo de grupos}
	Sean $G$ y $H$ grupos.  
	El conjunto $G\times H$ 
	es un grupo con la operación
	\[
		(g_1,g_2)(h_1,h_2)=(g_1h_1,g_2h_2).
	\]
	Esta estructura de grupo sobre el producto cartesiano $G\times H$ se conoce como
	el \textbf{producto directo} de $G$ y $H$. 
\end{exercise}

Si se utiliza la inducción, el ejemplo anterior puede generalizarse productos finitos de tres o más grupos. 

\begin{definition}
	Un subconjunto $S$ de $G$ es un \textbf{subgrupo} de $G$ si se satisfacen
	las siguientes propiedades:
	\begin{enumerate}
		\item $1\in S$, 
		\item $x\in S\implies x^{-1}\in S$, y además 
		\item $x,y\in S\implies xy\in S$.
	\end{enumerate}
	Notación: $S$ es un subgrupo de $G$ si y sólo si $S\leq G$. 
\end{definition}

Podríamos reemplazar la primera condición de la definición de subgrupo y pedir
simplemente que el conjunto sea no vacío. 

\begin{example}
	Si $G$ es un grupo, entonces $\{1\}$ y $G$ son subgrupos de $G$. 
\end{example}

\begin{example}
$2\Z\leq\Z\leq\Q\leq\R\leq\C$. 	
\end{example}

\begin{example}
$S^1=\{z\in\C:|z|=1\}\leq\C^\times=\C\setminus\{0\}$.
\end{example}

\begin{example}
Para cada $n\in\N$, definimos el grupo de raíces $n$-ésimas de la unidad como $G_n=\{z\in\C:z^n=1\}$, es decir 
\[
G_n=\{1,\exp(2\pi i/n),\exp(4i\pi/n),\dots,\exp(2(n-1)i\pi/n)\}.
\]
Entonces
\[
G_n\leq\bigcup_{n\in\N}G_n\leq S^1\leq\C^\times.
\]
\end{example}

\begin{exercise}
	\index{Centro!de un grupo}
	Si $G$ un grupo, el \textbf{centro} 
	\[
		Z(G)=\{g\in G:gh=hg\text{ para todo	$h\in G$}\}
	\]
	de $G$ es un subgrupo de $G$.
\end{exercise}

\begin{exercise}
	\index{Centralizador!de un elemento}
	Si $G$ es un grupo y $g\in G$, entonces el \textbf{centralizador} 
	\[
		C_G(g)=\{h\in G:gh=hg\} 
	\]
	de $g$ en $G$ es un subgrupo de $G$.
\end{exercise}

\begin{exercise}
\index{Centro!de $\Sym_3$}
Demuestre que $Z(\Sym_3)=\{\id\}$ y calcule $C_{\Sym_3}((12))$. 
\end{exercise}

Una forma útil de chequear que un cierto subconjunto de un grupo 
es un subgrupo es la siguiente:

\begin{exercise}
	Sea $G$ un grupo y sea $S$ un subconjunto de $G$. Demuestre que $S$ es un subgrupo
	de $G$ si y sólo si $S$ es no vacío y para todo $x,y\in S$ vale que $xy^{-1}\in S$. 
\end{exercise}

\begin{example}
$\SL_n(\R)=\{a\in\GL_n(\R):\det(a)=1\}\leq\GL_n(\R)$. En efecto, la matriz identidad pertenece a $\SL_2(\R)$ y luego $\SL_2(\R)$ es no vacío. Además si $a,b\in\SL_n(\R)$, 
entonces $ab^{-1}\in\SL_2(\R)$ pues $\det(ab^{-1})=\det(a)\det(b)^{-1}=1	$. 
\end{example}

\begin{exercise}
	La intersección de subgrupos es también un subgrupo.
\end{exercise}

La unión de subgrupos no es, en general, un subgrupo. Para convencerse, basta por ejemplo ver qué pasa 
en el subgrupo de Klein. 

%El resultado que sigue es de fundamental importancia:

\begin{theorem}
\label{thm:Z}
	Si $S$ es un subgrupo de $\Z$, entonces 
	$S=m\Z=\{mx:x\in \Z\}$
	para algún $m\in\N_0$.  
\end{theorem}
	
\begin{proof}
	Si $S=\{0\}$, no hay nada para demostrar pues podemos
	tomar $m=0$. Supongamos entonces que $S\ne\{0\}$ y sea $m=\min\{s\in S:s>0\}$. Este mínimo existe porque, como $S$ es no nulo, $S$ contiene un elemento $n\in S\setminus\{0\}$. Existen entonces dos situaciones posibles: $n>0$ o bien $-n>0$. Y como $S$ es un subgrupo de $\Z$, $-n\in S$.
	 
	Vamos a demostrar ahora que $S=n\Z$. 
	Si $x\in S$, entonces $x=my+r$ para $y,r\in\Z$ con $r$ tal que
	$0\leq r<m$. Supongamos que $r\ne 0$. Como $x,m\in S$, entonces $r\in S$,
	una contradicción a la minimalidad de $S$.  Luego $r=0$ y entonces $x=my\in
	m\Z$. Recíprocamente, como $n\in S$, entonces $nk\in S$ para todo $k\in\Z$. En efecto, si $k=0$, $nk=0\in S$. Si $k>0$, entonces 
	\[
	\underbrace{n+\cdots+n}_{k-\text{veces}}\in S.
	\]
	Por último, si $k<0$, entonces 
	\[
	nk=\underbrace{-n+(-n)+\cdots+(-n)}_{|k|-\text{veces}}\in S.\qedhere
	\]		
\end{proof}

Como la intersección de subgrupos es un subgrupo, 
el resultado anterior tiene además aplicaciones muy interesantes. Recordemos que si $a,b\in\Z$ 
se dice que $a$ divide a $b$ (o que $b$ es divisible por $a$) 
si $b=ac$ para algún $c\in\Z$. La notación: 
\[
a\mid b\Longleftrightarrow b=ac\text{ para algún $c\in\Z$.}
\]
Si $a,b\in\Z$ son tales que $ab\ne0$, entonces 
\[
S=a\Z+b\Z=\{m\in\Z:m=ar+bs\text{ para $r,s\in\Z$}\}
\]
es un subgrupo de $\Z$ (ejercicio). El teorema anterior nos permite escribir a $S$ como $S=d\Z$ para algún entero positivo $d$. Este entero $d$ 
es el \textbf{máximo común divisor} de $a$ y $b$, es decir $d=\gcd(a,b)$. La terminología queda justificada por la siguiente proposición:

\begin{proposition}
Sean $a,b\in\Z$ tales que $ab\ne0$ y sea $d=\gcd(a,b)$. Valen entonces las siguientes afirmaciones:
\begin{enumerate}
\item $d$ divide simultáneamente a los enteros $a$ y $b$. 
\item Si $e\in\Z$ divide a los enteros $a$ y $b$, entonces $e$ también divide a $d$.
\item Existen $r,s\in\Z$ tales que $d=ar+bs$. 
\end{enumerate}
\end{proposition}

\begin{proof}
    Como $d\in S$, existen $r,s\in\Z$ tales que $d=ar+bs$, esto demuestra la tercera afirmación. Si $e\in\Z$ es tal que $e\mid a$ y $e\mid b$, 
    entonces $e\mid ar+bs=d$, lo que demuestra la segunda afirmación. Finalmente, la primera afirmación queda demostrada al observar que
    $a,b\in S$. 
\end{proof}

Dos enteros $a$ y $b$ se dirán \textbf{coprimos} si y sólo si 
el único entero positivo que divide simultáneamente a ambos es 1, es decir 
\begin{align*}
a\text{ y }b\text{ son coprimos}&\Longleftrightarrow \gcd(a,b)=1\Longleftrightarrow \Z=a\Z+b\Z\\
&\Longleftrightarrow \text{existen $r,s\in\Z$ tales que $ar+bs=1$.}
\end{align*}

% \begin{corollary}
% Dos enteros $a$ y $b$ son coprimos si y sólo si .
% \end{corollary}

% \begin{proof}
% Es consecuencia inmediata de la proposición anterior. 
% \end{proof}

\begin{proposition}
Sea $p$ un primo y sean $a,b\in\Z$. Si $p\mid ab$, entonces $p\mid a$ o bien $p\mid b$.
\end{proposition}

\begin{proof}
Si $p\nmid a$, entonces $\gcd(a,p)=1$, lo que implica que $1=ra+sp$ para ciertos $r,s\in\Z$. Al multiplicar por $b$ en ambos miembros, vemos que
$b=r(ab)+spb$ es divisible por $p$, pues $p\mid ab$ por hipótesis.
\end{proof}

Si $S$ y $T$ son subgrupos de $\Z$, entonces $S\cap T$ es también un subgrupo de $\Z$. 
Sean $a,b\in\Z$ tales que $ab\ne 0$. Como $a\Z\cap b\Z$ es un subgrupo 
no nulo de $\Z$ (pues contiene al entero $ab\ne 0$), podemos escribir $a\Z\cap b\Z=m\Z$ 
para algún $m\in\N$. Ese entero positivo $m$ 
es el \textbf{mínimo común múltiplo} de $a$ y $b$ y se denota por $m=\lcm(a,b)$. 
La terminología queda justificada por la siguinte proposición.

\begin{proposition}
Sean $a,b\in\Z\setminus\{0\}$ y sea $m=\lcm(a,b)$. Valen entonces las siguientes propiedades:
\begin{enumerate}
	\item $m$ es simultáneamente divisible por $a$ y $b$.
	\item Si $n$ es simultáneamente divisible por $a$ y $b$, entonces $n$ es divisible por $m$. 
\end{enumerate}	
\end{proposition}

\begin{proof}
    Como $m\in a\Z\cap b\Z$, entonces $a\mid m$ y además $b\mid m$. Si $a\mid n$ y además $b\mid n$, digamos $n=ax=by$ para ciertos $x,y\in\Z$, 
    entonces $n\in a\Z\cap b\Z=m\Z$, lo que implica que $m\mid n$. 
\end{proof}

\begin{proposition}
Sean $a,b\in\N$. Si $d=\gcd(a,b)$ es el máximo común divisor de $a$ y $b$ y $m=\lcm(a,b)$
es el mínimo común múltiplo de $a$ y $b$, entonces $ab=dm$.  	
\end{proposition}

\begin{proof}
Como $b/d\in\Z$, entonces $a\mid a(b/d)$. Similarmente, $b\mid a(b/d)$. Como entonces $m\mid a(b/d)$, se concluye que $dm\mid ab$. 
Sean $r,s\in\Z$ tales que $d=ra+sb$. Al multiplicar por $m$ en ambos miembros, vemos que $dm=ram+sbm$ es divisible por $ab$. 
\end{proof}

\begin{exercise}
	\index{Subgrupo!conjugado}
	Sea $S$ un subgrupo de $G$ y sea $g\in G$. Demuestre que el \textbf{conjugado} $gSg^{-1}$ 
	de $S$ por $g$ es
	también un subgrupo de $G$. Notación: $\prescript{g}{}S=gSg^{-1}$. 
\end{exercise}

\begin{definition}
	\index{Subgrupo!generado por un conjunto}
	Sean $G$ un grupo y $X$ un subconjunto de $G$. El \textbf{subgrupo
	generado} por $X$ se define como la intersección de todos los subgrupos de $G$ que contienen a $X$, es decir  
	\[
		\langle X\rangle=\bigcap\{S:S\leq G,X\subseteq S\}.
	\]
\end{definition}

Cuando el conjunto de generadores sea finito, se utilizará la siguiente notación. Si $X=\{g_1,\dots,g_k\}$, entonces $\langle
X\rangle=\langle\{g_1,\dots,g_k\}\rangle=\langle g_1,\dots,g_k\rangle$. 	

\begin{exercise}
Demuestre que $\langle
X\rangle$ es el menor subgrupo de $G$ que contiene a $X$, es decir que si $H$ es un subgrupo de $G$ tal que $X\subseteq H$, entonces $\langle X\rangle\subseteq H$.  
\end{exercise}

\begin{exercise}
Demuestre que 
\[
	\langle X\rangle=\{x_1^{n_1}\cdots x_k^{n_k}:k\in\N,\,x_1,\dots,x_k\in X,\,-1\leq n_1,\dots,n_k\leq 1\}.
\]
\end{exercise}

%Un ejemplo importante de un grupo generado por dos elementos es el grupo diedral. 

\begin{example}
El conjunto
\[
D_4=\{\id,(1234),(1432),(13)(24),(14)(23),(12)(34),(24),(13)\}
\]
es un subgrupo no abeliano de $\Sym_4$.	Observar que $D_4$ está generado por las permutaciones 
$(12)(34)$ y $(1234)$. 	
\end{example}

\begin{example}
	\index{Grupo!diedral}
	Para $n\geq2$ y $\theta=2\pi/n$ sean  
	\[
		r=\left(\begin{array}{cc}
			\cos\theta & -\sin\theta\\
			\sin\theta & \cos\theta
		\end{array}\right),\quad s=\left(\begin{array}{cc}
			1 & 0\\
			0 & -1
		\end{array}\right).
	\]
	Se define entonces al \textbf{grupo diedral} $\mathbb{D}_{n}$ como el subgrupo
	de $\mathbf{GL}_2(\mathbb{R})$ generado por $r$ y $s$, es decir
	$\mathbb{D}_{n}=\langle r,s\rangle$. Observar que 
	\[
	s^2=r^n=\begin{pmatrix}
	1 & 0\\
	0 & 1	
	\end{pmatrix},\quad
	srs=r^{-1}.
	\]
	Además $|\D_n|=2n$. 
%	\[
%		r^{i}=\left(\begin{array}{cc}
%			\cos i\theta & \sin i\theta\\
%			-\sin i\theta & \cos i\theta
%		\end{array}\right),\quad s^{2}=\left(\begin{array}{cc}
%			1 & 0\\
%			0 & 1
%		\end{array}\right).
%	\]
%	Además 
%	\[
%		sr^{i}=r^{-i}s=\left(\begin{array}{cc}
%			-\sin i\theta & \cos i\theta\\
%			\cos i\theta & \sin i\theta
%		\end{array}\right).
%	\]
%	Además, los elementos $r^{j}s$ tienen orden dos y los elementos $r^{j}$ tienen
%	orden $n/\gcd(n,j)$. Al usar que $\det(r^j)=1$ y $\det(r^js)=-1$ para todo $j$, se demuestra entonces que $\D_n$ tiene $2n$ elementos. 
\end{example}

Es conveniente mencionar que la notación que suele usarse para el grupo diedral no es estándar. Para nosotros $\D_n$ será el grupo diedral de orden $2n$. 

\begin{definition}
\index{Subgrupo!conmutador}
\index{Subgrupo!derivado}
El \textbf{conmutador}
$[G,G]$ de $G$ es el subgrupo generado por los conmutadores, es
decir 
\[
[G,G]=\langle[x,y]\mid x,y\in G\rangle,
\]	
donde $[x,y]=xyx^{-1}y^{-1}$ es el conmutador de $x$ e $y$. 
\end{definition}

El subgrupo generado por los conmutadores de un grupo $G$ a veces se conoce como el \textbf{subgrupo derivado} de $G$. 
%Más adelante se justificará esta terminología.  

\begin{example}
	$[\Z,\Z]=\{0\}$ pues $\Z$ es un grupo abeliano. Obviamente, en este ejemplo
	utilizamos la notación aditiva. 
\end{example}

\begin{exercise}
	Demuestre que $[\Sym_3,\Sym_3]=\{\id,(123),(132)\}$. 
\end{exercise}


Es natural preguntarse por qué el conmutador se define como el subgrupo
generado por los conmutadores y no directamente como el subconjunto formado por los conmutadores. En realidad, esto se hace porque no es cierto que el subconjunto formado por los conmutadores sea un subgrupo, aunque no es muy fácil conseguir ejemplos. Con ayuda de algún software de matemática que permita trabajar con grupos, se pueden verificar
los ejemplos que mencionamos a continuación. Tomamos el siguiente ejemplo del libro de Carmichael~\cite{MR0075938}.
	
\begin{example}
	Sea $G$ el subgrupo de $\Sym_{16}$ generado por 
	las permutaciones
	\begin{align*}
&a = (13)(24),&&
b = (57)(68),\\
&c = (9\,11)(10\,12),&&
d = (13\,15)(14\,16),\\
&e = (13)(57)(9\,11),&&
f = (12)(34)(13\,15),\\
&g = (56)(78)(13\,14)(15\,16),&&
h = (9\,10)(11\,12).
\end{align*}	
	Puede demostrarse que $[G,G]$ tiene orden 16 y que el conjunto de conmutadores tiene tamaño 15. 
%	El código \textsf{GAP} que
%	demuestra esta afirmación es el siguiente:
%\begin{lstlisting}
%gap> a := (1,3)(2,4);;
%gap> b := (5,7)(6,8);;
%gap> c := (9,11)(10,12);;
%gap> d := (13,15)(14,16);;
%gap> e := (1,3)(5,7)(9,11);;
%gap> f := (1,2)(3,4)(13,15);;
%gap> g := (5,6)(7,8)(13,14)(15,16);;
%gap> h := (9,10)(11,12);;
%gap> G := Group([a,b,c,d,e,f,g,h]);;
%gap> D := DerivedSubgroup(G);;
%gap> Size(D);
%16
%\end{lstlisting}
%Para ver que el conjunto de conmutadores tiene 15 elementos y que 
%$cd\in[G,G]$ y no es un conmutador: 
%\begin{lstlisting}
%gap> Size(Set(List(Cartesian(G,G), Comm)));
%15
%gap> c*d in Difference(D,Set(List(Cartesian(G,G),Comm)));
%true
%\end{lstlisting}
\end{example}

Mencionamos otro ejemplo, encontrado por Guralnick~\cite{MR673806} antes de que el uso de computadoras en teoría de grupos fuera masivo.

\begin{example}
El grupo 
\[
G=\langle (135)(246)(7\,11\,9)(8\,12\,10),(394\,10)(58)(67)(11\,12)\rangle.
\]
tiene orden 96 y su subgrupo de conmutadores de $G$ 
no es igual al conjunto de conmutadores. Puede demostrarse además que es el menor grupo finito con esta propiedad. 
\end{example}



\chapter{Grupos cíclicos}
\label{orden}

\begin{definition}
	\index{Grupo!cíclico}
	Un grupo $G$ se dice \textbf{cíclico} si $G=\langle g\rangle$ para algún
	$g\in G$.
\end{definition}

Un grupo cíclico $G$ generado por el elemento $g$ estará compuesto entonces por las potencias
de $g$, es decir $G=\langle g\rangle=\{g^k:k\in\Z\}$. Todo grupo cíclico es
entonces en particular un grupo abeliano.

\begin{examples}\
\begin{enumerate}
	\item $\Z=\langle 1\rangle=\langle -1\rangle$. 
	\item $\Z/n=\langle 1\rangle$.
	\item $G_n=\langle \exp(2i\pi/n)\rangle$.  
\end{enumerate}	
\end{examples}

\begin{example}
	$\mathcal{U}(\Z/8)\ne\langle 3\rangle$. De hecho, $\langle 3\rangle=\{1,3\}\subsetneq\{1,3,5,7\}=\mathcal{U}(\Z/8)$. 	
\end{example}

Antes de resolver el siguiente ejercicio, es conveniente recordar cómo son los subgrupos de $\Z$. 

\begin{exercise}
	Todo subgrupo de un grupo cíclico es también un grupo cíclico. 
\end{exercise}

%\begin{proof}
%	Sea $G=\langle g\rangle$ un grupo cíclico y sea $H$ un subgrupo de $G$. Sin perder generalidad podemos suponer que $H$ es no trivial, es decir $H\ne\{1\}$. Sea 
%	\[
%	n=\min\{k\in\N:g^k\in H\}.
%	\]
%	
%\end{proof}


%\begin{proof}
%	Demostremos la primera afirmación. Supongamos que $G=\langle g\rangle$. Si $k\in\Z\setminus\{-1,1\}$ es tal que $G=\langle g^k\rangle$, entonces, en particular, $g^{km}=g$ para algún $m\in\Z$. Luego $g^{km-1}=1$, con $km-1\ne0$. Esto implica que $g$ tiene orden finito, una contradicción.  
%	 
%	Demostremos ahora la segunda afirmación. Supongamos primero que $G=\langle g^k\rangle$ y que $d=\gcd(k,n)>1$. Entonces 
%	$m=n/d<n$ y luego $g^{mk}=(g^n)^{k/d}=1$.
%	
%	Recíprocamente, si $k$ y $n$ son coprimos, existen
%	$r,s\in\Z$ tales que $rk+sn=1$. Como entonces
%	\[
%		g=g^1=g^{rk+sn}=(g^k)^r(g^{n})^s=(g^k)^r,
%	\]
%	se concluye que $G=\langle g\rangle=\langle g^k\rangle$. 
%\end{proof}

\begin{definition}
	\index{Orden!de un elemento de un grupo}
	Sean $G$ un grupo y $g\in G$. El \textbf{orden} de $g$ 
	se define como el orden del subgrupo generado por $g$. Notación: 
	$|g|=|\langle g\rangle|$.
\end{definition}

\begin{theorem}
	Sean $G$ un grupo, $g\in G$ y $n\in\N$. Las siguientes afirmaciones son
	equivalentes:
	\begin{enumerate}
		\item $|g|=n$.
		\item $n=\min\{k\in\N:g^k=1\}$.
		\item Para todo $k\in\Z$, $g^k=1\Longleftrightarrow n\mid k$.
		\item $\langle g\rangle=\{1,g,\dots,g^{n-1}\}$ y los
			$1,g,\dots,g^{n-1}$ son todos distintos.
	\end{enumerate}
\end{theorem}

\begin{proof}
	Veamos que $(1)\implies(2)$. 
	Si $g=1$ entonces $n=1$. Supongamos entonces que $g\ne1$. Como $\langle g\rangle=\{g^k:k\in\Z\}$, 
	sabemos que existen enteros positivos $i>j$ tales que $g^i=g^j$, es decir $g^{i-j}=1$. En particular,
	el conjunto $\{k\in\N:g^k=1\}$ es no vacío y posee entonces elemento mínimo, digamos
	\[
	d=\min\{k\in\N:g^k=1\}.
	\] 
	Tenemos entonces que $\langle g\rangle\subseteq\{1,g,\dots,g^{d-1}\}\subseteq\langle g\rangle$. En efecto, si $g^k\in\langle g\rangle$, entonces $k=dq+r$ para $q,r\in\Z$ con $0\leq r<d$. Como $g^d=1$,  
	\[
	g^k=g^{dq+r}=(g^d)^qg^r=g^r\in\{1=g^0,g,g^2,\dots,g^{d-1}\}
	\]
	Por otro lado, es trivial observar que $\{1,g,\dots,g^{d-1}\}\subseteq \langle g\rangle$ y que 
	$\{1,g,\dots,g^{d-1}\}$ tiene $d$ elementos. 
	%Este conjunto tiene además $d$ elementos (pues si $g^i=g^j$ para $i,j\in\{0,1,\dots,d-1\}$ con $i>j$, entonces $g^{j-i}=1$ y luego 
	%$i=j$ por la minimalidad de $d$). En consecuencia, $n=d$. 
%	Tomemos el mínimo
%	$k>1$ tal que $1,g,g^{2},\dots,g^{k-1}$ son elementos distintos, y sea
%	$j\in\{0,\dots,k-1\}$ tal que $g^{k}=g^{j}$. Afirmamos que $g^{k}=1$. Si
%	$g^{k}=g^{j}$ para algún $j\geq1$, entonces $g^{k-j}=1$ con $k-j\leq
%	k-1<k$, una contradicción. Afirmamos ahora que $\langle
%	g\rangle=\{1,g,g^{2},\dots,g^{k-1}\}$.  La inclusión $\supseteq$ trivial.
%	Para probar la otra inclusión, sea $g^{l}\in\langle g\rangle$.  Escribimos
%	$l=kq+r$ con $0\leq r<k$, y entonces $g^{l}=g^{kq+r}=g^{r}$.

	Ahora demostremos que $(2)\implies(3)$. Supongamos que $g^k=1$. Si
	escribimos $k=nt+r$ con $0\leq r<n$, entonces $g^k=g^{nt+r}=g^r$. La
	minimalidad de $n$ implica entonces que $r=0$ y luego $n$ divide a $k$.
	Recíprocamente, si $k=nt$ para algún $t\in\Z$, entonces $g^k=(g^n)^t=1$.

	Demostremos que $(3)\implies(4)$. Es trivial que
	$\{1,g,\dots,g^{n-1}\}\subseteq\langle g\rangle$. Para demostrar la otra
	inclusión, escribimos $k=nt+r$ con $0\leq r\leq n-1$. Entonces
	\[
		g^k=g^{nt+r}=(g^n)^tg^r=g^r
	\]
	pues por hipótesis $g^n=1$. Para ver que los 
	$1,g,\dots,g^{n-1}$ son todos distintos, basta observar que si $g^k=g^l$ con $0\leq
	k<l\leq n-1$, entonces, como $g^{l-k}=1$ y además $0<l-k\leq n-1$, se concluye $n\leq l-k$ ya que
	por hipótesis $n$ divide a $l-k$, una contradicción. 
	 
	La implicación $(4)\implies(1)$ es trivial. 
\end{proof}

Veamos una aplicación de la proposición anterior:

\begin{corollary}
Si $G$ es un grupo y $g\in G$ tiene orden $n$, entonces 
\[
|g^m|=\frac{n}{\gcd(n,m)}.
\]		
\end{corollary}

\begin{proof}
Sea $k$ tal que $(g^m)^k=1=g^{mk}$. Esto es equivalente a decir que $n$ divide a $km$, pues $g$ tiene orden $n$. A su vez esto 
es equivalente a pedir que    
$n/d$ divida a $mk/d$, donde $d=\gcd(n,m)$. En consecuencia, como los enteros $n/d$ y $m/d$ son coprimos, $(g^m)^k=1$ es equivalente a pedir que $n/d$ divida a $k$, que implica que $g^m$ tiene orden $n/d$.  
\end{proof}

\begin{exercise}
Sea $G$ un grupo y sea $g\in G$. Demuestre que las siguientes afirmaciones son equivalentes:
\begin{enumerate}
\item $g$ tiene orden infinito.
\item El conjunto $\{k\in\N:g^k=1\}$ es vacío. 
\item Si $g^k=1$, entonces $k=0$.
\item Si $k\ne l$, entonces $g^k\ne g^l$.  	
\end{enumerate}
\end{exercise}

\begin{exercise}
Sea $G$ un grupo y sea $g\in G\setminus\{1\}$. Demuestre las siguientes afirmaciones: 
\begin{enumerate}
\item $|g|=2$ si y sólo si $g=g^{-1}$.
\item $|g|=|g^{-1}|$. 
\item Si $|g|=nm$', entonces $|g^m|=n$.  
\end{enumerate}	
\end{exercise}

\begin{exercise}
\index{Torsión!de un grupo abeliano}
	Sea $G$ un grupo abeliano. Demuestre que $T(G)=\{g\in G:|g|<\infty\}$ es un subgrupo de $G$. Calcule $T(\C^\times)$. 
\end{exercise}

\begin{exercise}
	Sea $G=\langle g\rangle$ un grupo cíclico. 
	\begin{enumerate}
		\item Si $G$ es infinito, los únicos generadores de $G$ son $g$ y $g^{-1}$.
		\item Si $G$ es finito de orden $n$, $G=\langle g^k\rangle$ si y sólo
			si $k$ es coprimo con $n$.
	\end{enumerate}
\end{exercise}

El siguiente ejercicio es un caso particular del teorema de Cauchy, que veremos más adelante. 

\begin{exercise}
\label{xca:orden2}
Demuestre que todo grupo de orden par contiene un elemento de orden dos. 	
\end{exercise}

Mostremos ahora algunos órdenes de elementos concretos: 

\begin{example}
En $\Sym_3$ tenemos los siguiente:
\[
|\id|=1,\quad
|(12)|=|(13)|=|(23)|=2,\quad
|(123)|=|(132)|=3.
\]	
\end{example}

\begin{example}
En $\Z$ todo elemento no nulo tiene orden infinito.	
\end{example}

\begin{example}
En $\Z\times\Z/6$ hay elementos de orden finito y elementos de orden infinito. Por ejemplo, $(1,0)$ tiene orden infinito y 
$(0,1)$ tiene orden seis. 
\end{example}

\begin{exercise}
Calcule los órdenes de los elementos de $\Z/6$.	
\end{exercise}

\begin{example}
La matriz $\begin{pmatrix}1&1\\0&1\end{pmatrix}\in\GL_2(\R)$ tiene orden infinito. 
\end{example}

\begin{example}
El grupo $G_\infty=\bigcup_{n\geq1}G_n$ es abeliano e infinito. Todo elemento de $G_\infty$ tiene orden finito. 
\end{example}

\begin{exercise}
Pruebe que la matrix $a=\begin{pmatrix}1&-1\\1&0\end{pmatrix}$ tiene orden cuatro, que la matrix $b=\begin{pmatrix}0&1\\-1&-1\end{pmatrix}$ tiene orden tres
y calcule el orden de $ab$.%=\begin{pmatrix}1&1\\0&1\end{pmatrix}$ tiene orden infinito. 
\end{exercise}

\begin{exercise}
Calcule el orden de la matriz $\begin{pmatrix}1&1\\-1&0\end{pmatrix}\in\GL_2(\R)$. 	
\end{exercise}

\begin{exercise}
Demuestre que en $\D_n$ se tiene $|r^js|=2$ y $|r^j|=n/\gcd(n,j)$. Demuestre además que $\D_n$ tiene orden $2n$. 	
\end{exercise}

\begin{exercise}
		Demuestre que un grupo con finitos subgrupos es finito. 	
\end{exercise}

\chapter{El teorema de Lagrange}

Sean $G$ un grupo y $H$ un subgrupo de $G$. Diremos que dos elementos $x,y\in
G$ son equivalentes a izquierda módulo $H$ si $x^{-1}y\in H$.  
Usaremos la siguiente notación:
\[
x\equiv y\bmod
H\Longleftrightarrow x^{-1}y\in H.
\]  

\begin{exercise}
	Demuestre que hemos definido una relación de equivalencia. 
	Esto significa que se tienen las siguientes propiedades:
	\begin{enumerate}
	\item $x\equiv x\bmod H$ para todo $x$.
	\item Si $x\equiv y\bmod H$, entonces $y\equiv x\bmod H$.
	\item Si $x\equiv y\bmod H$ y además $y\equiv z\bmod H$, entonces $x\equiv z\bmod H$.  	
	\end{enumerate}
\end{exercise}

Las clases de equivalencia de esta relación módulo $H$ 
son los conjuntos de la forma $xH=\{xh:h\in H\}$
pues la clase de un cierto elemento $x\in G$ es el conjunto
\[
	\{y\in G:x\equiv y\bmod H\}=\{y\in G:x^{-1}y\in H\}=\{y\in G:y\in xH\}=xH.
\]
El conjunto $xH$ se 
llama \textbf{coclase a izquierda} de $H$ en $G$. 

\begin{proposition}
Sean $G$ un grupo y $H$ un subgrupo de $G$.  
\begin{enumerate}
\item Si $xH\cap yH\ne\emptyset$, entonces $xH=yH$. 	
\item El grupo $G$ puede descomponerse como unión disjunta
de distintas coclases a izquierda de $H$. 
\end{enumerate}
\end{proposition}

\begin{proof}
Demostremos la primera afirmación. 
Si $g\in xH\cap yH$, escribimos
$g=xh$ para algún $h\in H$ y entonces
\[
gH=(xh)H=x(hH)=xH.
\]
Similarmente, $gH=yH$. En consecuencia, $xH=yH$.

La segunda afirmación se obtiene inmediatamente de la primera.  
\end{proof}

Podríamos haber definido coclases a derecha mediante la relación $x\equiv
y\bmod H$ si y sólo si $xy^{-1}\in H$. En este caso, las clases de equivalencia
serían los conjuntos $Hx$ con $x\in X$. $Hx$ se llama \textbf{coclase a derecha}
de $H$ en $G$. 

\begin{proposition}
	Si $H$ es un subgrupo de $G$, entonces $|Hx|=|H|=|xH|$ para todo $x\in G$. 
\end{proposition}

\begin{proof}
	Sea $x\in G$. La función $H\to Hx$, $h\mapsto hx$, es una biyección con
	inversa $hx\mapsto h$. Análogamente se demuestra que la función $H\to xH$,
	$h\mapsto xh$, es una biyección. 
\end{proof}

La función
\[
	\{\text{coclases a derecha de $H$ en $G$}\}\to\{\text{coclases a izquierda de $H$ en $G$}\}
\]
dada por $Hx\mapsto x^{-1}H$ es una biyección pues 
\[
	Hx=Hy
	\Longleftrightarrow xy^{-1}\in H
	\Longleftrightarrow (x^{-1})^{-1}y^{-1}\in H
	\Longleftrightarrow x^{-1}H=y^{-1}H.
\]
En particular, la cantidad de coclases a derecha de $H$ en $G$ coincide con la
cantidad de coclases a izquierda de $H$ en $G$.

\begin{example}
Si $G=\Z$ y $S=n\Z$, entonces
\[
a+S=\{a+nq:q\in\Z\}=\{k\in\Z:k\equiv a\bmod n\}.
\]	
\end{example}

\begin{example}
Los subgrupos de $\Sym_3$ son $\{\id\}$, $\Sym_3$, los subgrupos $\langle(12)\rangle$, $\langle(13)\rangle$ y $\langle(23)\rangle$ de orden dos 
y el subgrupo $\langle(123)\rangle=\{\id,(123),(132)\}$ de orden tres.  	Si $H=\langle(12)\rangle=\{\id,(12)\}$, entonces
\begin{align*}
&H=(12)H=\{\id,(12)\},\\
&(123)H=(13)H=\{(13),(123)\},\\
&(132)H=(23)H=\{(23),(132)\}.
\end{align*}
Observemos que en este caso se tiene la descomposición
\[
\Sym_3=H\cup (123)H\cup (132)H\quad\text{(unión disjunta)}.
\]
\end{example}

\begin{example}
Sea $G=\R^2$ con la suma usual y sea $v\in\R^2$. La recta $L=\{\lambda v:\lambda\in\R\}$ es un subgrupo de $G$ y 
para cada $p\in R^2$, la coclase $p+L$ es la recta paralela a $L$ que pasa por el punto $p$.  	
\end{example}


\begin{definition}
	Si $H$ es un subgrupo de $G$, se define el \textbf{índice} de $H$ en $G$
	como la cantidad $(G:H)$ de coclases a izquierda (o a derecha) de $H$ en $G$. 
\end{definition}

Tener una relación de equivalencia módulo $H$ nos permite escribir a $G$ como
unión disjunta de coclases a izquierda (o a derecha) de $H$ en $G$. Además dos
coclases cualesquiera son iguales o disjuntas. 

\begin{theorem}[Lagrange]
\index{Teorema!de Lagrange}
	Si $G$ es un grupo finito y $H$ es un subgrupo de $G$, entonces
	$|G|=|H|(G:H)$. En particular, $|H|$ divide a $|G|$. 
\end{theorem}

\begin{proof}
	Tenemos una relación de equivalencia módulo $H$ que nos permite descomponer
	en $G$ en clases de equivalencia, digamos
	\[
	G=\bigcup_{i=1}^n x_iH\quad\text{(unión disjunta)}
	\]
	para ciertos $x_1,\dots,x_n\in G$, donde $n=(G:H)$. Como cada una de esas clases tiene exáctamente
	$|H|$ elementos,  
	\[
		|G|=\sum_{i=1}^n|x_iH|=\sum_{i=1}^n|H|=|H|(G:H).\qedhere
	\]
\end{proof}

Veamos algunos corolarios. 

\begin{corollary}
	Si $G$ es un grupo finito y $g\in G$, entonces $g^{|G|}=1$. 	
\end{corollary}

\begin{proof}
	Por definición $|g|=|\langle g\rangle|$. El teorema de Lagrange aplicado al
	subgrupo $H=\langle g\rangle$ nos dice que 
	\[
		g^{|G|}=g^{|H|(G:H)}=(g^{|H|})^{(G:H)}=1.\qedhere
	\]
\end{proof}

\begin{corollary}
	Si $G$ es un grupo de orden primo, entonces $G$ es cíclico.
\end{corollary}

\begin{proof}
	Sea $g\in G\setminus\{1\}$ y sea $H=\langle g\rangle$. Por el teorema de
	Lagrange, $|H|$ divide a $|G|$ y luego $|H|=|G|$ pues $|G|$ es un número
	primo. En consecuencia, $G=H=\langle g\rangle$. 
\end{proof}

\begin{corollary}
\label{cor:ordenes_coprimos}
	Si $G$ es un grupo abeliano y $g,h\in G$ son elementos de órdenes finitos y coprimos, entonces
	$|gh|=|g||h|$.
\end{corollary}

\begin{proof}
Sean $n=|g|$, $m=|h|$ y $l=|gh|$. Como $G$ es abeliano, 
\[
(gh)^{nm}=(g^n)^m(h^m)^n=1
\]
y luego $l$ divide a $nm$. Por otro lado, como $(gh)^l=1$, 
$g^l=h^{-l}\in \langle g\rangle\cap\langle h\rangle=\{1\}$ (pues como $|\langle g\rangle|=n$ y $|\langle h\rangle|=m$ son coprimos, 
entonces $nm$ divide a $l$ gracias al teorema de Lagrange). 
\end{proof}

%\begin{example}
%Gracias al teorema de Lagrange podemos demostrar fácilmente que $(n+m)!$ divide a $n!m!$, 
%basta con observar que $\Sym_n\times\Sym_m\leq\Sym_{n+m}$
%\end{example}

El pequeño teorema de Fermat es un caso particular del teorema de Lagrange.

\begin{exercise}[pequeño teorema de Fermat]
	\index{Teorema!de Fermat}
	Sea $p$ un número primo. Demuestre que 
	$a^{p-1}\equiv1\bmod p$  
	para todo $a\in\{1,2,\dots,p-1\}$. 
\end{exercise}

El siguiente corolario utiliza la función $\varphi$ de Euler. Recordemos que
$\varphi(n)$ es la cantidad de enteros positivos $k\in\{1,\dots,n\}$ 
coprimos con $n$. El grupo de
unidades de $\Z/n$ tiene $\varphi(n)$ elementos (pues $x\in\Z/n$ es inversible
si y sólo si $x$ es coprimo con $n$). 

\begin{exercise}[teorema de Euler]
	\index{Teorema!de Euler}
	Sean $a$ y $n$ enteros coprimos. Demuestre que 
	$a^{\varphi(n)}\equiv1\bmod n$.
\end{exercise}

No vale la recíproca del teorema de Lagrange.

\begin{example}
Consideremos el grupo alternado 
\begin{multline*}
\Alt_4=\{\id,(234),(243),(12)(34),(123),(124),\\(132),(134),(13)(24),(142),(143),(14)(23)\}\leq\Sym_4.	
\end{multline*}
Vamos a demostrar que $\Alt_4$ no tiene subgrupos de orden seis. Si $H\leq\Alt_4$ es tal que 
$|H|=6$, entonces, como $(\Alt_4:H)=2$, para todo $x\not\in H$ podríamos descomponer a $\Alt_4$ como $\Alt_4=H\cup xH$ (unión disjunta). 

Afirmamos que
para todo $g\in\Alt_4$ vale que $g^2\in H$ (pues si $g\not\in H$, entonces, como $g^2\in\Alt_4=H\cup gH$, se concluye que $g^2\in H$). En particular, como
$(ijk)=(ikj)^2$, 
todos los elementos de orden tres de $\Alt_4$ están en el subgrupo $H$, una contradicción pues hay ocho elementos de orden tres.   
\end{example}

Todos deberíamos tener un grupo favorito. El mío es $\SL_2(3)$,
el grupo formado por las matrices de $2\times2$ con coeficientes en $\Z/3$ 
con determinante uno.

\begin{exercise}
Demuestre que 
\[
\SL_2(3)=\left\{\begin{pmatrix}a&b\\c&d\end{pmatrix}:ad-bc=1,\,a,b,c,d\in\Z/3\right\}
\]
es un grupo de orden 24 que no posee subgrupos de orden 12.	
\end{exercise}


%\begin{proof}
%	Si $a$ y $n$ son coprimos, entonces $a$ es una unidad de $\Z/n$. Como el
%	grupo de unidades de $\Z/n$ tiene orden $\varphi(n)$, entonces
%	$a^{\varphi(n)}\equiv1\bmod n$. 
%\end{proof}
%


%
%\section{Notas}
%
%Un teorema de Hall de 1935: Si $G$ es un grupo finito y $S$ es un subgrupo de $G$ de índice $n$, entonces
%existen $t_1,\dots,t_n\in G$ tales que los $t_1S,\dots,t_nG$ forman un conjunto de representantes de las las coclases a izquierda y los 
%$St_1,\dots,St_n$ forman un conjunto de representantes de coclases a derecha. 



\chapter{El grupo simétrico}

\index{Ciclo}
Sea $\sigma\in\Sym_n$. Diremos que $\sigma$ es un $r$-ciclo si existen $a_1,\dots,a_r\in\{1,\dots,n\}$ tales que 
$\sigma(j)=j$ para todo $j\not\in\{a_1,\dots,a_r\}$ y 
\[
\sigma(a_i)=\begin{cases}
a_{i+1} & \text{si $i<r$},\\
a_1 & \text{si $i=r$}.	
\end{cases}
\]

\begin{examples}
Por ejemplo, $(12)$, $(13)$ y $(23)$ son 2-ciclos de $\Sym_3$. Los 2-ciclos se denominan \textbf{trasposiciones}. 
Las permutaciones $(123)$ y $(132)$ son 3-ciclos de $\Sym_3$.
\end{examples}

\index{Permutaciones!disjuntas}
Dos permutaciones $\sigma,\tau\in\Sym_n$ se dicen \textbf{disjuntas} si para todo $j\in\{1,\dots,n\}$ 
se tiene que $\sigma(j)=j$ o bien $\tau(j)=j$. 

\begin{examples}
Las permutaciones $(134)$ y $(25)$ son disjuntas. En cambio, las permutaciones $(134)$ y $(24)$ no lo son. 	
\end{examples}

Si $\sigma\in\Sym_n$ y $j$ es tal que $\sigma(j)=j$, entonces $j$ es un punto fijo de $\sigma$. En cambio, los $j$ tales que
$\sigma(j)\ne j$ son los puntos movidos por $\sigma$. 

\begin{remark}
Las permutaciones disjuntas conmutan.
\end{remark}

\begin{remark}
Cada permutación puede escribirse como producto de trasposiciones. Para demostrar esta afirmación procederemos de la siguiente forma. Supongamos que las personas invitadas a un concierto se sientan en la primera fila, pero sin respetar el orden que figura en la lista de invitados. ¿Qué podemos hacer para ordenar a esas personas? Primero identificamos a la persona que debería sentarse en el primer lugar y le pedimos que intercambie asientos con la persona sentada en esa primera butaca. Luego identificamos a la persona que debería sentarse en el segundo lugar y le pedimos que intercambie asientos con la persona que ocupe la segunda butaca. Hacemos lo mismo con el tercer lugar, con el cuarto... y una vez terminado el proceso, gracias a haber utilizado finitas trasposiciones, habremos conseguido acomodar correctamente a cada una de las personas invitadas al concierto.  
\end{remark}

A continuación demostraremos que toda permutación puede escribirse como producto de ciclos disjuntos, algo que usamos en el primer capítulo en el caso particular del grupo $\Sym_3$. Necesitamos el siguiente lema:

\begin{lemma}
	Sea $\sigma=\alpha\beta\in\Sym_n$ con $\alpha$ y $\beta$ permutaciones disjuntas. Si $\alpha(i)\ne i$, entonces $\sigma^k(i)=\alpha^k(i)$ para todo $k\geq0$.
\end{lemma}

\begin{proof}
	Sin perder generalidad podemos suponer que $k>0$. En ese caso, $\sigma^k(i)=(\alpha\beta)^k(i)=\alpha^k(\beta^k(i))=\alpha^k(i)$. 
\end{proof}

Ahora sí estamos en condiciones de demostrar el teorema: 

\begin{theorem}
Toda $\sigma\in\Sym_n\setminus\{\id\}$ puede escribirse como producto de ciclos disjuntos de longitud $\geq2$. Además esta descomposición es única salvo el orden de los factores involucrados.   	
\end{theorem}

\begin{proof}
	Procederemos por inducción en el número $k$ de elementos del conjunto $\{1,\dots,n\}$ movidos por $\sigma$. Si $k=2$ el resultado es trivial. Supongamos
	entonces que el resultado es cierto para todas las permutaciones que mueven $<k$ puntos. Sea $i_1\in\{1,\dots,n\}$ tal que $\sigma(i_1)\ne i_1$. Vamos a considerar el ciclo que contiene al elemento $i_1$. 
	Sea entonces
	$i_2=\sigma(i_1)$, $i_3=\sigma(i_2)$... Sabemos que existe $r\in\N$ tal que $\sigma(i_r)=i_1$ (pues, de lo contrario, si $\sigma(i_r)=i_j$ para algún 
	$j\in\{2,\dots,n\}$, entonces $\sigma(i_{j-1})=i_j=\sigma(i_r)$, una contradicción a la biyectividad de $\sigma$). Sea $\sigma_1=(i_1\cdots i_r)$. La hipótesis
	inductiva nos dice que, como $\sigma_1^{-1}\sigma$ mueve $<k$ puntos (pues los $i_j$ son puntos fijos de $\sigma_1^{-1}\sigma$), podemos escribir $\sigma_1^{-1}\sigma=\sigma_2\cdots\sigma_s$, donde
	$\sigma_2,\dots,\sigma_s$ son ciclos disjuntos. Esto implica que $\sigma=\sigma_1\sigma_2\cdots\sigma_s$, tal como queríamos. 
	
	Demostremos ahora la unicidad. Supongamos que $\sigma=\sigma_1\cdots\sigma_s=\tau_1\cdots\tau_t$, con $s>0$. Sea $i_1\in\{1,\dots,n\}$ tal que
	$\sigma(i_1)\ne i_1$. El lema implica que $\sigma^k(i_1)=\sigma_1^k(i_1)$ para todo $k\geq0$. Existe entonces $j\in\{1,\dots,t\}$ tal que 
	$\tau_j(i_1)\ne i_1$. Como los $t_k$ conmutan, sin perder generalidad podemos suponer que $j=1$. Luego $\sigma^k(i_1)=\tau_1^k(i_1)$ para todo $k\geq0$.  Esto implica que
	$\sigma_1=\tau_1$ pues $\sigma_1$ y $\tau_1$ son ciclos y entonces $\sigma_2\cdots\sigma_s=\tau_2\cdots\tau_t$. Al repetir el argumento, vemos que $s=t$ y luego $\sigma_j=\tau_j$ para todo $j$.   
\end{proof}

\begin{corollary}\
	\begin{enumerate}
		\item $\Sym_n=\langle (ij):i<j\rangle$. 
		\item $\Sym_n=\langle (12),(13),\dots,(1n)\rangle$.
		\item $\Sym_n=\langle (12),(23),\dots,(n-1\,n)\rangle$.
		\item $\Sym_n=\langle (12),(12\cdots n)\rangle$.
	\end{enumerate}
\end{corollary}

\begin{proof}
	Ya demostramos que toda permutación puede escribirse como producto de trasposiciones. Otra demostración puede obtenerse al usar el teorema anterior ya que 
	\[
	(a_1\cdots a_r)=(a_1a_r)(a_1a_{r-1})\cdots(a_1a_2).
	\]
	En efecto, si escribimos a $\sigma\in\Sym_n$ como producto de ciclos disjuntos y usamos la fórmula anterior, tenemos que $\Sym_n\subseteq\langle (ij):i<j\rangle$. La otra inclusión es trivial. ` 
	
	Para demostrar la segunda afirmación hay que usar la primera afirmación y las fórmulas
	\[
	(1i)(1j)(1i)=(ij)
	\] 
	válidas siempre que $i\ne j$. 
	
	Para la tercera afirmación escribimos a $\sigma$ como producto de trasposiciones y luego observamos que 
	\[
	(13)=(12)(23)(12),\quad
	(1\,k+1)=(k\,k+1)(1k)(k\,k+1)
	\]
	para todo $k\geq3$. 
	
	Por último, la cuarta afirmación se obtiene al utilizar la tercera propiedad junto con la fórmula
	\[
	(12\cdots n)^{k-1}(12)(12\cdots n)^{1-k}=(k\,k+1),
	\]
	válida para todo $k\geq1$. 
\end{proof}

Cada permutación tiene asociada una matriz de permutación. Por ejemplo, para $\sigma=\id\in\Sym_3$ se tiene a $P_\sigma$ como la matriz identidad de $3\times 3$. Para la permutación $\sigma=(123)$ se tiene 
\[
P_\sigma=\begin{pmatrix}0&0&1\\1&0&0\\0&1&0\end{pmatrix}.
\]
Si $e_1,e_2,e_3$ es la base canónica de $\R^{3\times1}$, entonces $P_{\sigma}(e_1)=e_2$, $P_{\sigma}(e_2)=e_1$ y $P_{\sigma}(e_3)=e_1$. En general, la matriz de permutación $P_\sigma$ correspondiente a $\sigma\in\Sym_n$, permuta los elementos de la base canónica de $\R^{n\times1}$ tal como $\sigma$ permuta los elementos del conjunto $\{1,2,\dots,n\}$.  

%Veamos cómo actúa esta matriz en la base canónica $e_1,e_2,e_3$ de $\R^{3\times1}$. Por ejemplo
%\[
%P_{(123)}e_j=\begin{cases}
%j+1 & \text{si $j\in\{1,2\}$},\\
%1 & \text{si $j=3$}.
%\end{cases}
%\]

En general, si $\sigma\in\Sym_n$, entonces
\[
P_\sigma=\sum_{i=1}^n E_{\sigma(i),i},
\]
donde $E_{i,j}$ es la matriz con un uno en la posición $(i,j)$ e igual a cero en todas las otras entradas. Recordemos que valen las siguientes fórmulas
\begin{equation}
\label{eq:E}	
E_{i,j}E_{k,l}=\begin{cases}
E_{i,l} & \text{si $j=k$},\\
0 & \text{si $j\ne k$}.
\end{cases}
\end{equation}

Es claro que toda matriz de permutación tendrá un único uno en cada fila y cada columna y que el resto de las entradas serán todas iguales a cero. Luego
el determinante de una matriz de permutación será $\pm1$. 

\begin{proposition}
Si $\sigma,\tau\in\Sym_n$, entonces $P_{\sigma\tau}=P_\sigma P_\tau$. 
\end{proposition}

\begin{proof}
Es un cálculo directa que utiliza la fórmula~\eqref{eq:E}. 
Tenemos
\begin{align*}
P_\sigma P_\tau &=\left(\sum_{i=1}^n E_{\sigma(i),i}\right)\left(\sum_{j=1}^nE_{\tau{(j)},j}\right)\\
&=\sum_{i=1}^n\sum_{j=1}^n E_{\sigma(i),i}E_{\tau(j),j}
=\sum_{j=1}^n E_{\sigma(\tau(j)),j}=P_{\sigma\tau},
\end{align*}
ya que la suma doble será nula a menos que $i=\tau(j)$.  
\end{proof}
  
\begin{definition}
\index{Signo!de una permutación}
\index{Permutación!par}
\index{Permutación!impar}
	El \textbf{signo} de una permutación $\sigma\in\Sym_n$ se define como el determinante de la matriz $P_\sigma$, es decir $\sgn(\sigma)=\det P_\sigma$. 
	Una permutación $\sigma$ se dirá \textbf{par} si $\sgn(\sigma)=1$ e \textbf{impar} si $\sgn(\sigma)=-1$. 
\end{definition}

\begin{examples}
La identidad es una permutación par y todo 3-ciclo es también una permutación par. Cualquier trasposición es una permutación impar.   
\end{examples}


Toda permutación puede escribirse como producto de trasposiciones, aunque no de forma única. Sin embargo, puede demostrarse el siguiente resultado. Si $\sigma$ se escribe como producto de traposiciones $\sigma=\sigma_1\cdots\sigma_s$, entonces
\[
\sgn(\sigma)=(-1)^s.
\] 
En particular, $\sigma$ es una permutación par si y sólo si $s$ es par. 

\begin{proposition}
Si $\sigma,\tau\in\Sym_n$, entonces $\sgn(\sigma\tau)=(\sgn\sigma)(\sgn\tau)$. 	
\end{proposition}

\begin{proof}
	Es fácil pues 
	\[
	\sgn(\sigma\tau)=\det(P_\sigma P_\tau)=(\det P_\sigma)(\det P_\tau)=\sgn(\sigma)\sgn(\tau).\qedhere
	\]
\end{proof}

\begin{example}
\index{Centro!de $\Sym_n$}
Vamos a demostrar que si $n\geq3$ entonces $Z(\Sym_n)=\{\id\}$.
Supongamos que $Z(\Sym_n)\ne\{\id\}$ y 
sea $\sigma\in Z(\Sym_n)$ tal que $\sigma(i)=j$ para $i\ne j$. Como $n\geq3$, existe $k\in\{1,\dots,n\}\setminus\{i,j\}$ y entonces
$\tau=(jk)\in\Sym_n$. Como $\sigma$ es central, 
\[
j=\sigma(i)=\tau\sigma\tau^{-1}(i)=\tau(\sigma(i))=\tau(j)=k,
\]
una contradicción.   	
\end{example}

\index{Grupo!alternado}
El \textbf{grupo alternado} 
\[
\Alt_n=\{\sigma\in\Sym_n:\sgn(\sigma)=1\}
\]
es el subgrupo de $\Sym_n$ formado por las permutaciones de signo positivo. 

\begin{proposition}
\index{Orden!del grupo alternado}
$|\Alt_n|=n!/2$. 
\end{proposition}

\begin{proof}
Sea $\sigma=(12)\not\in\Alt_n$. Vamos a demostrar que $\Sym_n=\Alt_n\cup\Alt_n\sigma$ (unión disjunta), donde $\Alt_n\sigma=\{\tau\sigma:\tau\in\Alt_n\}$. En efecto, si $\tau\in\Sym_n$ es tal que $\tau\not\in\Alt_n$, entonces $\sgn(\tau\sigma)=(\sgn\tau)(\sgn\sigma)=1$ y luego
$\tau\sigma\in\Alt_n$. En conclusión, probamos que $\tau\in\Alt_n\sigma$. Como $|\Alt_n\sigma|=|\Alt_n|$ (por ejemplo, pues la función $\Alt_n\to\Alt_n\sigma$, $x\mapsto x\sigma$, es biyectiva), se obtiene $n!=|\Sym_n|=2|\Alt_n|$. 
\end{proof}

\begin{example}
Es fácil verificar que 	$\Alt_3=\{\id,(123),(132)\}$ y que  
\begin{multline*}
\Alt_4=\{\id,(234),(243),(12)(34),(123),(124),\\(132),(134),(13)(24),(142),(143),(14)(23)\}\end{multline*}
\end{example}

El grupo $\Alt_3$ es abeliano.   
Si $n\geq4$, el grupo $\Alt_n$ es no abeliano ya que, por ejemplo, las permutaciones $(123)$ y $(124)$ no conmutan.  

%La proposición que veremos a continuación es muy útil.

\begin{proposition}
\index{Grupo!alternado}
\label{pro:A_n3ciclos}
$\Alt_n=\langle\{\text{3-ciclos}\}\rangle$. 
\end{proposition}

\begin{proof}
Todo 3-ciclo es una permutación par pues $(ijk)=(ik)(ij)$. Demostremos entonces la otra inclusión. Sea $\sigma\in\Alt_n$. 
Escribimos $\sigma=\sigma_1\cdots\sigma_s$ para algún entero $s$ par y $\sigma_1,\dots,\sigma_s$ trasposiciones. Para completar la demostración de la proposición
basta utilizar las fórmulas 
\[
(kl)(ij)=(kl)(ki)(ki)(ij)=(kil)(ijk),\quad
(ik)(ij)=(ijk).\qedhere
\]
 \end{proof}

Veamos algunas aplicaciones sencillas:

\begin{example}
\index{Conmutador!de $\Alt_4$}
Veamos que si $n\geq5$ entonces $[\Alt_n,\Alt_n]=\Alt_n$. Vamos a demostrar la inclusión no trivial y para eso basta con observar que $\Alt_n$ está generado por 3-ciclos y que, como $n\geq5$, cada 3-ciclo puede escribirse como producto de conmutadores. En efecto, 
\[
(abc)=[(acd),(ade)][(ade),(abd)],
\] 	
donde $\#\{a,b,c,d,e\}=5$. 
\end{example}

\begin{example}
\index{Conmutador!de $\Sym_n$}
Si $n\geq3$ entonces $[\Sym_n,\Sym_n]=\Alt_n$. Primero veamos que $[\Sym_n,\Sym_n]\subseteq\Alt_n$. Si $\sigma\in[\Sym_n,\Sym_n]$, 
digamos $\sigma=[\sigma_1,\tau_1][\sigma_2,\tau_2]\cdots[\sigma_k,\tau_k]$, entonces
\[
\sgn(\sigma)=\sgn([\sigma_1,\tau_1])\cdots\sgn([\sigma_k,\tau_k])=1.
\]
Recíprocamente, si $\sigma\in\Alt_n$, la proposición anterior nos dice que podemos escribir a $\sigma$ como producto de 3-ciclos. De aquí el resultado se obtiene inmediatamente 
pues cada 3-ciclo es un conmutador, tal como vemos en la siguiente fórmula   	
\[
(abc)=(ab)(ac)(ab)(ac)=[(ab),(ac)]\in[\Sym_n,\Sym_n].\qedhere
\]
\end{example}

\chapter{Cocientes}
\label{cocientes}

Si $G$ es un grupo y $N$ es un subgrupo de $G$, nos interesa saber cuándo 
el conjunto $G/N$ de coclases es un grupo con 
la operación $G/N\times G/N\to G/N$,  
$(xN,yN)\mapsto xyN$, es decir, cuándo esta operación está 
bien definida. ¿Qué significa eso? Queremos que esa operación sea una función. 
Para eso, se necesita que si $xN=x_1N$ y además $yN=y_1N$, entonces
$xyN=x_1y_1N$. Veamos cómo puede interpretarse esa condición. Si $x^{-1}x_1\in N$ y 
$y^{-1}y_1\in N$, entonces $x_1=xn$ y además $y_1=ym$ para ciertos $m,n\in N$. Entonces
\[
(xy)^{-1}(x_1y_1)=y^{-1}x^{-1}x_1y_1=y^{-1}nym\in N
\]
si y sólo si $y^{-1}ny\in N$. 

\begin{example}
Si $G=\Sym_3$ y $H=\langle (12)\rangle$, entonces $(xN,yN)\mapsto xyN$ no es una función. Para verlo, primero recordemos que 
$G/H=\{H,(123)H,(132)H\}$, donde 
$H=(12)H$, $(123)H=(13)H$ y $(132)H=(23)H$. Tenemos
\[
(132)N=(13)(23)N=(13)N(23)N=(123)N(132)N=N,
\]
una contradiccion. 
\end{example}

\begin{definition}
	\index{Subgrupo!normal}
	Sea $G$ un grupo. 
	Un subgrupo $N$ de $G$ se dice \textbf{normal} si $gNg^{-1}\subseteq N$ para todo
	$g\in G$. Notación: si $N$ es normal en $G$, entonces $N\unlhd G$.
\end{definition}

Si $G$ es un grupo abeliano, todo subgrupo de $G$ es normal en $G$. 

\begin{proposition}
\label{pro:normalidad}
Sea $N$ un subgrupo de $G$. Las siguientes afirmaciones son equivalentes:
\begin{enumerate}
	\item $gNg^{-1}\subseteq N$ para todo $g\in G$.
	\item $gNg^{-1}=N$ para todo $g\in G$.
	\item $gN=Ng$ para todo $g\in G$.
\end{enumerate}	
\end{proposition}

\begin{proof}
Demostremos que $(1)\implies (2)$, que es la única implicación no trivial. Si $n\in N$ y $g\in G$, entonces
$n=g(g^{-1}ng)g^{-1}\in gNg^{-1}$. 	
\end{proof}

\begin{proposition}
	Sea $N$ un subgrupo de $G$. Las siguientes propiedades son equivalentes:
	\begin{enumerate}
		\item $N$ es normal en $G$.
		\item $(gN)(hN)=(gh)N$ para todo $g,h\in G$.
	\end{enumerate}
\end{proposition}

\begin{proof}
	Vamos a demostrar que $(1)\implies(2)$. Sea $g\in G$. Como $gNg^{-1}=N$,
	entonces $(gN)(hN)=g(Nh)N=g(hN)N=(gh)N$. Veamos ahora que $(2)\implies(1)$. Si $g\in G$, entonces
	$gNg^{-1}\subseteq (gN)(g^{-1}N)=(gg^{-1})N=N$. 
\end{proof}

\begin{examples}
Si $G$ es un grupo, entonces 
$\{1\}$ y $G$ son subgrupos normales de $G$.
\end{examples}

\begin{example}
\index{Centro!de un grupo}
Si $G$ es un grupo, $Z(G)$ es un subgrupo normal de $G$. Más aún, si $N\leq Z(G)$, entonces $N\unlhd G$. 	
\end{example}

\begin{example}
Si $G$ es un grupo, entonces $[G,G]$ es un subgrupo normal de $G$ pues si $x\in [G,G]$ y $g\in G$, entonces 
$gxg^{-1}=(gxg^{-1}x^{-1})x=[g,x]x\in [G,G]$. Alternativamente, 
\[
g\left(\prod_{i=1}^k[x_i,y_i]\right)g^{-1}=\prod_{i=1}^k [gx_ig^{-1},gy_ig^{-1}]
\]
para todo $g,x_1,\dots,x_k,y_1,\dots,y_k\in G$. 	
\end{example}

\begin{example}
Para todo $n\in\N$, $\Alt_n$ es un subgrupo normal de $\Sym_n$. 
De hecho, si $\sigma\in\Alt_n$ y $\tau\in\Sym_n$, entonces $\tau\sigma\tau^{-1}\in\Alt_n$ pues  
\[
\sgn(\tau\sigma\tau^{-1})=\sgn(\sigma)=1. 
\]
\end{example}

\begin{example}
Si $N$ es un subgrupo de $G$ tal que $(G:N)=2$, entonces $N$ es normal en $G$. Queremos demostrar que $gN=Ng$ para todo $g\in G$. Sea $g\in G$. Si $g\in N$, entonces $gN=Ng$. Si $g\not\in N$, entonces
$gN\ne N$. Como $(G:N)=2$, podemos escribir a $G$ como $G=N\cup gN$ (unión disjunta). En consecuencia, $gN=G\setminus N$. Similarmente se demuestra que
$Ng=G\setminus N$ y luego $gN=Ng$. 	
\end{example}

\begin{example}
El ejemplo anterior nos permite demostrar que $\langle (123)\rangle\unlhd\Sym_3$. Por otro lado, $\langle (12)\rangle$ no es normal en $\Sym_3$ pues
por ejemplo $(13)(12)(13)=(23)\not\in\langle(12)\rangle$. 
\end{example}

\begin{example}
$\SL_n(\R)$ es normal en $\GL_n(\R)$ pues si $g\in\GL_n(\R)$ y $x\in\SL_n(\R)$, entonces $\det(gxg^{-1})=(\det g)(\det x)(\det g)^{-1}=1$. 
\end{example}

\begin{example}
\index{Grupo!de Klein}
El grupo de Klein $K=\{\id,(12)(34),(13)(24),(14)(23)\}$ es normal en $\Sym_4$. 
\end{example}

%\begin{example}
%\end{example}

\index{Producto!semidirecto}
En el ejercicio siguiente nos encontramos con un caso particular del producto semidirecto de dos grupos, una construcción general que resulta de mucha utilidad. 

\begin{exercise}
\index{Producto!semidirecto}
Sea $G=\Z/p\times(\Z/p)^\times$ el grupo dado por la operación 
\[
(x,y)(u,v)=(x+yu,yv).
\]	
Demuestre que $\{(x,1):x\in\Z/p\}$ es normal en $G$ y que $\{(0,y):y\in(\Z/p)^\times\}$ no es normal en $G$. 
\end{exercise}

El siguiente ejercicio es útil:

\begin{exercise}
\index{Normalizador!de un subgrupo}
Si $S$ es un subgrupo de $G$, se define el \textbf{normalizador} de $S$ en $G$ al subgrupo
\[
N_G(S)=\{g\in G:gSg^{-1}=S\}.
\]	
Demuestre que valen las siguientes afirmaciones:
\begin{enumerate}
\item $S\unlhd N_G(S)$.
\item Si $S\leq T\leq G$ y $S\unlhd T$, entonces $T\leq N_G(S)$.
\end{enumerate}
\end{exercise}

El ejercicio anterior nos dice que el normalizador de un subgrupo $S$ en $G$ es el mayor subgrupo de $G$ que contiene a $S$ como subgrupo normal. 
 
Veamos algunos ejemplos de subgrupos normales un poco más difíciles. Primero calcularemos los subgrupos normales de $\Alt_4$. 

\begin{example}
\index{Subgrupos!normales de $\Alt_4$}
Vamos a demostrar que 
$\{\id\}$, $K=\{\id,(12)(34),(13)(24),(14)(23)\}$ y $\Alt_4$ son los únicos subgrupos normales de $\Alt_4$. 
 	
Como $\Alt_4=\{\text{3-ciclos}\}\cup K$, $K$ es el único subgrupo de $\Alt_4$ con cuatro elementos, y esto implica que $K$ es normal en $\Alt_4$ (pues cada conjugado $gKg^{-1}$ también será un subgrupo de $\Alt_4$ de cuatro elementos). Sea $N\ne\{\id\}$ un subgrupo normal de $\Alt_4$. Si $N$ contiene un 3-ciclo, digamos
$(abc)\in N$. entonces
\[
(acd)=(bcd)(abc)(bcd)^{-1}\in N
\]
y luego $N=\Alt_4$ (pues todos los 3-ciclos están en $N$). Supongamos entonces que $N$ no contiene 3-ciclos. Entonces algún elemento no trivial de $K$ 
pertenece a $N$, digamos $(ab)(cd)\in N$. En consecuencia, 
\[
(ac)(bd)=(bcd)(ab)(cd)(bcd)^{-1}\in N,\quad
(ad)(bc)=(ab)(cd)(ac)(bd)\in N
\]
y luego $N=K$. 
\end{example}
 
Es importante remarcar que la normalidad no es transitiva.

\begin{exercise}
\index{Grupo!diedral}
Sea $G=\D_4$ el grupo diedral de tamaño ocho y sean $N=\langle s,r^2\rangle$ y $H=\langle s\rangle$. Demuestre que 
$H$ es normal en $N$, $N$ es normal en $G$ pero $H$ no es normal en $G$.    
\end{exercise} 
 
Vamos a calcular ahora los subgrupos normales de $\Sym_4$. 
  
\begin{example}
\index{Subgrupos!normales de $\Sym_4$}
Vamos a demostrar que $\{\id\}$, $K$, $\Alt_4$ y $\Sym_4$ son los únicos subgrupos normales de $\Sym_4$.

Sea $N$ un subgrupo normal de $\Sym_4$. Si $N\subseteq\Alt_4$, entonces $N$ es normal en $\Alt_4$ y luego, por lo visto en el ejemplo anterior, $N=\{\id\}$, 
$N=K$ o bien $N=\Alt_4$. Supongamos entonces que $N\not\subseteq\Alt_4$, es decir $N$ contiene una permutación impar. Si $\sigma\in\Sym_4$ es una permutación impar, entonces $\sigma$ es una trasposición o $\sigma$ es un 4-ciclo. 

Si $N$ contiene una trasposición, entonces todas las trasposiciones
también pertenecen a $N$ pues
\[
\tau(ij)\tau^{-1}=(\tau(i)\,\tau(j))
\]
para todo $\tau\in\Sym_4$. En este caso, $N=\Sym_4$ pues $\Sym_4$ está generado por trasposiciones.   

Si $N$ contiene un 4-ciclo, todos los 4-ciclos también están en $N$ pues
\[
\tau(ijkl)\tau^{-1}=(\tau(i)\,\tau(j)\,\tau(k)\,\tau(l))
\]
para todo $\tau\in\Sym_4$ y además $K\subseteq N$ pues
\[
(ac)(bd)=(abcd)^2.
\]
Esto nos dice que $|N|\geq10$. Como además $K\subseteq N$, se tiene que $|N\cap\Alt_4|\geq 5$. Por otro lado, $N\cap\Alt_4$ es un subgrupo normal de $\Alt_4$. 
Por lo visto en el ejemplo anterior, $N\cap\Alt_4=\Alt_4\subseteq N$. En conclusión, $N=\Sym_4$.   
\end{example}

\begin{theorem}
\label{Grupo!cociente}
Si $N$ es un subgrupo normal de $G$, entonces $G/N$ es un grupo con la operación $(xN)(yN)=(xy)N$.  
\end{theorem}

\begin{proof}
Sabemos que la normalidad de $N$ en $G$ garantiza la buena definición de la operación. Calculos rutinarios, que dejamos como ejercicio, demuestran que 
esta operación transforma al conjunto $G/N$ en un grupo. 
\end{proof}

% todo: Mover cocientes de S4 al capítulo de isomorfismos
No estamos en condiciones de poder entender qué tipo de grupo obtenemos
como grupo cociente, ya que para eso es necesario poder entender qué significa 
que dos grupos sean ``iguales'' aunque parezcan distintos. 
 
%Veamos cómo son los posibles cocientes de $\Sym_4$. 
% todo: Listar las coclases de S4 por K, i.e. los elementos del grupo cociente#
\begin{example}
\index{Cocientes!de $\Sym_4$}
	Sabemos que $\{\id\}$, $K$, $\Alt_4$ y $\Sym_4$ son los únicos subgrupos normales de $\Sym_4$. Trivialmente 
	obtenemos que
	\[
	\Sym_4/\{\id\}\simeq\Sym_4,\quad
	\Sym_4/\Alt_4\simeq\Z/2,\quad
	\Sym_4/\Sym_4\simeq\{\id\}.
	\]
	Veamos qué podemos decir del cociente $Q=\Sym_4/K$. Sabemos que $Q$ tiene orden seis y que $Q$ es no abeliano pues
	\[
	(12)K(13)K=(12)(13)K=(132)K\ne (123)K=(13)(12)K=(13)K(12)K.
	\]
	Vimos que existe un único grupo no abeliano de orden seis. Luego $Q\simeq\Sym_3$. 
\end{example}

%Para terminar el capítulo mencionamos dos ejercicios de mucha utilidad. 

\begin{proposition}
Si $H$ es un subgrupo normal de $G$, entonces $G/H$ es abeliano si y sólo si $[G,G]\subseteq H$. 
\end{proposition}

\begin{proof}
Sean $x,y\in G$. Entonces 
\begin{align*}
    (xH)(yH)=(yH)(xH) \Longleftrightarrow (xy)H=(yx)H \Longleftrightarrow x^{-1}y^{-1}xy\in H.
\end{align*}
Luego $G/H$ es conmutativo si y sólo si $[x,y]=xyx^{-1}y^{-1}\in H$ para todo $x,y\in G$. 
\end{proof}

Veamos una pequeña aplicación:

\begin{example}
\index{Conmutador!de $\Alt_4$}
$[\Alt_4,\Alt_4]=K=\{\id,(12)(34),(13)(24),(14)(23)\}$. 
Sabemos que $K$ es normal en $\Alt_4$. Como $\Alt_4/K$ tiene tres elementos, es abeliano. El ejercicio anterior, entonces, 
nos dice que $[\Alt_4,\Alt_4]\subseteq K$. Por otro lado, como 
\[
(ab)(cd)=[(abc),(cda)],
\]   	
se concluye que $K\subseteq[\Alt_4,\Alt_4]$. 
\end{example}

Otra propiedad importante:

\begin{proposition}
Si $G/Z(G)$ es cíclico, entonces $G$ es abeliano.
\end{proposition}

\begin{proof}
Supongamos que $G/Z(G)=\langle gZ(G)\rangle$. Sean $x,y\in G$. Escribamos $xZ(G)=g^kZ(G)$ y también $yZ(G)=g^lZ(G)$, es decir
$x=g^kz_1$, $y=g^lz_2$ para ciertos $k,l\in\Z$ y $z_1,z_2\in Z(G)$. Luego $xy=yx$. 
\end{proof}

\begin{theorem}
Sea $p$ un número primo y sea $H$ un subgrupo de $G$. Si $(G:H)=p$, las siguientes afirmaciones son equivalentes:
\begin{enumerate}
\item $H$ es normal en $G$.
\item Si $g\in G\setminus H$, entonces $g^p\in H$.
\item Si $g\in G\setminus H$, entonces $g^n\in H$ para algún $n\in\N$ sin divisores primos $<p$.
\item Si $g\in G\setminus H$, entonces $g^k\not\in H$ para todo $k\in\{2,\dots,p-1\}$. 
\end{enumerate}
\end{theorem}

\begin{proof}
	La implicación $(1)\implies(2)$ es consecuencia inmediata del teorema de Lagrange, pues $|G/H|=p$. 
	
	La implicación $(2)\implies(3)$ es trivial pues $p$ es un número primo. 
	
	Demostremos que $(3)\implies(4)$. Si $g^k\in H$ para algún $k\in\{2,\dots,p-1\}$, como $\gcd(k,n)=1$, existen $r,s\in\Z$ tales que
	$rk+sn=1$. Luego
	\[
	g=g^1=g^{rk+sn}=(g^k)^r(g^n)^s\in H,
	\]
	una contradicción. 
	
	Para finalizar demostremos que $(4)\implies(1)$. Sea $x\in G\setminus H$ y sea $h\in H$. Queremos demostrar que entonces $xhx^{-1}\in H$. Si $y=xhx^{-1}\not\in H$, entonces
	$y^k\not\in H$ para todo $k\in\{2,\dots,p-1\}$. Esto implica que las coclases 
	\[
	H,yH,y^2H,\dots,y^{p-1}H
	\]
	son todas distintas (pues si $y^iH=y^jH$ para $i,j$ tales que $i<j$, entonces $y^{j-i}\in H$ con $j-i\leq p-2$). Como $y=xhx^{-1}$, 
	entonces
	\[
	(yx)H=(xh)H=xH=y^iH
	\]
	para algún $i\in\{0,1,\dots,p-1\}$. Si $i=0$, entonces $yx=xh\in H$ y luego $x\in H$, una contradicción. Luego $(yx)H=y^iH$ para algún $i\in\{1,\dots,p-1\}$ y entonces
	\[
	y^iH=xH=y^{i-1}H
	\]
	para algún $i\in\{0,\dots,p-2\}$, una contradicción.  
\end{proof}

Veamos algunas consecuencias. La primera se hará en el caso en que el grupo sea finito. 

\begin{corollary}
\label{cor:p_menor}
	Sea $p$ el menor número primo que divide al orden de un grupo finito  
	$G$ y sea $H$ es un subgrupo de $G$ índice $p$. Entonces $H$ es normal en $G$. 
\end{corollary}

\begin{proof}
	Si $g\in G\setminus H$, entonces $g^n=1\in H$, donde $n=|G|$. Como $p$ es primo, $n$ no tiene divisores primos $<p$. El teorema anterior impica entonces que $H$ es normal en $G$. 
\end{proof}

En el teorema no pedimos que $G$ sea un grupo finito. Podemos entonces obtener el siguiente resultado.

\begin{corollary}
Sea $p$ un número primo y sea $G$ un grupo tal que todo elemento tiene orden una potencia de $p$. Si $H$ es un subgrupo de $G$ de índice $p$, entonces $H$ es normal en $G$.  
\end{corollary}

\begin{proof}
Sea $g\in G\setminus H$ y sea $n=|g|$. Como todo elemento de $G$ tiene orden una potencia de $p$, 
$n$ es en particular una potencia de $p$ y, en consecuencia, $n$ no posee divisores primos $<p$. Como además $g^n=1\in H$, el teorema anterior implica que $H$ es normal en $G$. 
en particular $g^n\in H$.  	
\end{proof}

Terminamos el capítulo con una definición importante. 

\begin{definition}
\index{Grupo!simple}
Diremos que un grupo $G$ es \textbf{simple} si $G\ne\{1\}$ y sus únicos subgrupos normales
son $G$ y $\{1\}$.
\end{definition}

Por ahora, nos quedaremos conformes al observar que si $p$ es un número primo, entonces $\Z/p$ es un grupo simple. Veremos otros ejemplos más adelante.  



\chapter{Subgrupos permutables}

\index{Producto!de subgrupos}
Si $H$ y $K$ son subgrupos de un grupo $G$, definimos
\[
	HK=\{hk:h\in H,\,k\in K\}.
\]
Observemos que 
\[
H\cup K\subseteq HK\subseteq\langle H\cup K\rangle.
\]
Nos interesa saber cuándo $XY$ es un subgrupo de $G$. 
Observemos que $HK\leq G$ si y sólo si $\langle H\cup K\rangle=HK$. 

\begin{proposition}
	Sean $H$ y $K$ subgrupos de un grupo $G$. Entonces $HK$ es un subgrupo de
	$G$ si y sólo si $HK=KH$.
\end{proposition}

\begin{proof}
	Supongamos que $HK=KH$. Como $1\in H\cap K$, el conjunto $HK$ es no vacío.
	Si $h\in H$ y $k\in K$, entonces $(hk)^{-1}=k^{-1}h^{-1}\in KH=HK$. Además
	$(HK)(HK)=H(KH)K=H(HK)K=(HH)(KK)=HK$ y luego $HK$ es cerrado para la
	multiplicación. 

	Supongamos ahora que $HK$ es un subgrupo de $G$. Como $H\subseteq HK$,
	$K\subseteq HK$ y además $HK$ es cerrado para la multiplicación,
	$KH\subseteq (HK)(HK)\subseteq HK$. Recíprocamente, sea $g\in HK$. 
	Como $g^{-1}\in HK$, existen $h\in H$ y $k\in K$ tales que $g^{-1}=hk$.
	Luego $HK\subseteq KH$ pues 
	$g=k^{-1}h^{-1}\in KH$.
\end{proof}

\begin{proposition}
Sean $H$ y $K$ subgrupos de $G$. Si $H$ es normal en $G$, entonces $HK$ es un subgrupo de $G$.
\end{proposition}

\begin{proof}
Nos alcanza con demostrar que $HK=KH$. Veamos primero que $HK\subseteq KH$. Si $x=hk\in HK$, entonces $x=k(k^{-1}hk)\in KH$ pues $k^{-1}hk\in H$. Para demostrar
la otra inclusión, sea $y=kh\in KH$. Entonces $y=(khk^{-1})k\in HK$ pues $khk^{-1}\in H$. \end{proof}


\begin{example}
Sea $G=\Sym_4$. Los subgrupos $H=\langle (12)\rangle$ y $K=\langle (34)\rangle$ cumplen que $HK=KH=\{\id,(12),(34),(12)(34)\}$ es un subgrupo de $\Sym_4$. Es interesante observar que aquí ni $H$ ni $K$ son normales en $G$.  
\end{example}

\begin{exercise}
Demuestre que si $H$ y $K$ son subgrupos normales de $G$, entonces $HK$ es también normal en $G$.
\end{exercise}

\begin{exercise}
Sean $G$ un grupo y $S$ un subgrupo de $G$. 
Si $'T\leq N_G(S)$, entonces $TS$ es un grupo y además $S\leq TS$. 
\end{exercise}

\index{Subgrupos!permutables}
Dos subgrupos $H$ y $K$ de un grupo $G$ se dirán \textbf{permutables} si $HK=KH$. 
El siguiente resultado será de mucha utilidad más adelante.

\begin{theorem}
\label{thm:|HK|}
	Sean $H$ y $K$ subgrupos finitos de un grupo $G$. Entonces
	\[
		|HK|=\frac{|H||K|}{|H\cap K|}.
	\]
\end{theorem}
	
\begin{proof}
Sea $L=H\cap K$. 
Descomponemos al grupo $H$ como unión disjunta de coclases de $L$, digamos 
$H=\cup_{i=1}^k x_iL$, donde $k=(H:L)$. Observemos que $LK=K$, pues $L\subseteq K$ y además $K\subseteq 1K\subseteq LK$. 
Entonces
\[
HK=\bigcup_{i=1}^k x_iLK=\bigcup_{i=1}^k x_iK,
\]
%pues $x_iLK=x_iK$ para todo $i\in\{1,\dots,k\}$, ya que como $L\subseteq K$ y $1\in L$. 
En particular, como la unión es disjunta, 
%Veamos que esta unión es disjunta. Si $x_iLK\cap x_jLK\ne\emptyset$ para algún $i\ne j$, sea $y\in x_iLK\cap x_jLK$. Como $L\subseteq K$, 
%entonces $y\in x_iLK\cap x_jLK\subseteq x_iK\cap x_jK$ y luego $x_iK=x_jK$, una contradicción. Como la unión~\ref{eq:HK} es disjunta y 
%$|x_iLK|=|LK|$ para todo ,
\[
|HK|=\sum_{i=1}^k |x_iK|=k|K|=\frac{|H||K|}{|H\cap K|}.\qedhere
\] 
\end{proof}

%\begin{proof}
%	Sea $Q=H\cap K$ y sea 
%	\[
%		\theta\colon H\times K\to HK,\quad
%		\theta(h,k)=hk,
%	\]
%	La función $\theta$ es claramente sobreyectiva. 
%	
%	Vamos a demostrar que si $x\in HK$, entonces $|\theta^{-1}(x)|=|H\cap K|$.  Si $x\in HK$, entonces
%	$x=hk$ para algún $h\in H$ y $k\in K$. Alcanza con ver que 
%	\[
%	\theta^{-1}(x)=\{(h\gamma,\gamma^{-1} k):\gamma\in H\cap K\}.
%	\]
%    Veamos la inclusión no trivial. Si $(h_1,k_1)\in\theta^{-1}(x)$, entonces
%    \[
%    \theta(h_1,k_1)=h_1k_1=x=hk.
%    \] 
%    En consecuencia, $\gamma=h^{-1}h_1=kk_1^{-1}\in H\cap K$. Luego 
%    $(h_1,k_1)=(h\gamma,\gamma^{-1}k)$ para algún $\gamma\in H\cap K$. Como la otra inclusión es trivial, el teorema queda demostrado
%    al observar que
%    \[
%    |HK|=\frac{|H\times K|}{|H\cap K|}=\frac{|H||K|}{|H\cap K|}.\qedhere
%    \] 
%\end{proof}

Es importante remarcar que en el teorema anterior no es necesario pedir que $HK$ sea un subgrupo de $G$. 
Como una primera aplicación, daremos otra demostración del resultado que vimos en el corolario~\ref{cor:p_menor} en la página~\pageref{cor:p_menor}.

\begin{quote}
	Sea $p$ el menor número primo que divide al orden de un grupo finito  
	$G$ y sea $H$ un subgrupo de $G$ índice $p$. Entonces $H$ es normal en $G$. 
\end{quote}

Si $\{gHg^{-1}:g\in G\}=\{H\}$, entonces $H$ es normal en $G$. Supongamos que existe $g\in G$ tal que
$H\ne g^{-1}Hg=K$. Como $(H:H\cap K)$ divide al orden de $H$ y todos los divisores primos de $|G|$ son $\geq p$, sabemos que $(H:H\cap K)\geq p$. Luego
\[
|HK|=\frac{|H||K|}{|H\cap K|}\geq p|K|=|G|
\]
pues $(G:H)=p$ y $|K|=|H|$. En particular, $HK=G$. Como $K=g^{-1}Hg$, se tiene que 
$g=h(g^{-1}h_1g)$ para ciertos $h,h_1\in H$. Luego 
\[
1=hg^{-1}h_1\implies h_1h=g\in H\implies H=K,
\]
una contradicción.

% todo: agregar un ejemplo de HK tal que no sea subgrupo

\begin{example}
Sean $G=\Sym_3$, $H=\langle (12)\rangle$ y $K=\langle (23)\rangle$. En este caso, 
\[
HK=\{\id,(12),(23),(123)\}
\]
no es un subgrupo de $G$ ya que el teorema de Lagrange implica que $G$ no 
tiene subgrupos de orden cuatro. Otra forma de ver que $HK$ no es un subgrupo 
es observar que $KH=\{\id,(12),(23),(132)\}\ne HK$. 
\end{example}

\begin{example}
Sean $G=\Sym_3$, $H=\langle (12)\rangle$ y $K=\langle (123)\rangle$. 
Como $K$ es normal en $G$, entonces $HK$ es un subgrupo de $G$. El teorema
de Lagrange nos dice que $HK$ tiene orden seis y luego $G=HK$. 
Todo elemento $g\in G$ puede escribirse unívocamente como $g=hk$ para $h\in H$ y $k\in K$ (esto puede
demostrarse considerando todos los posibles casos u 
observando que $H\cap K=\{\id\}$). Esto implica que la función
\[
H\times K\to G,\quad 
(h,k)\mapsto hk,
\]
es una biyección. Es importante remarcar que esta biyección no se lleva bien 
con la multiplicación de $G$ (más adelante haremos más precisa esta observación y 
simplemente diremos que la función no es un morfismo de grupos), 
ya que en general $(h_1k_1)(h_2k_2)\ne (h_1h_2)(k_1k_2)$. 
\end{example}

\chapter{Morfismos}

\begin{definition}
	\index{Morfismo!de grupos}
	Sean $G$ y $H$ dos grupos. 
	Una función $f\colon G\to H$ es un \textbf{morfismo de grupos} si 
	$f(xy)=f(x)f(y)$ para todo $x,y\in G$. 
\end{definition}

\index{Morfismo!de grupos inyectivo}
\index{Morfismo!de grupos sobreyectivo}
\index{Morfismo!de grupos biyectivo}
\index{Monomorfismo!de grupos}
\index{Epimorfismo!de grupos}
\index{Isomorfismo!de grupos}
Si un morfismo de grupos es una función inyectiva, se denominará
\textbf{monomorfismo}. Si es una función sobreyectiva, se denominará
\textbf{epimorfismo}.  Si fuera una función biyectiva, \textbf{isomorfismo}.
Dos grupos $G$ y $H$ se dirán \textbf{isomorfos} (la notación será $G\simeq H$)
cuando exista un isomorfismo $G\to H$. 

\begin{examples}\
\begin{enumerate}
\item Si $G$ es un grupo, la función $\id\colon G\to G$ es un morfismo de grupos.
\item Si $G$ y $H$ son grupos, la función $e\colon G\to H$, $e(g)=1_H$, es un morfismo de grupos. 
\item Para cada $n\in\Z$, la función $\Z\to\Z$, $x\mapsto nx$, es un morfismo de grupos.
\item Si $G$ es un grupo abeliano y $n\in\Z$, la función $G\to G$, $g\mapsto g^n$, es un morfismo de grupos.  	
\end{enumerate}
\end{examples}

El siguiente ejemplo es particularmente importante. 

\begin{example}
\index{Morfismo!de conjugación}
\index{Conjugación}
Sea $G$ un grupo y sea $g\in G$. La función $\gamma_g\colon G\to G$, $\gamma_g(x)=gxg^{-1}$, se denomina \textbf{conjugación} por el elemento $g$ y 
es un morfismo de grupos. 	
\end{example}

\begin{example}
La función $\exp\colon\R\to\R^\times$, $\exp(x)=e^x$, es un morfismo de grupos. 
\end{example}

\begin{example}
\index{Inclusión}
La inclusión $\Z\hookrightarrow\Q$ es un morfismo inyectivo de grupos. 
\end{example}

En general, si $S$ es un subgrupo de un grupo $G$, entonces
la \textbf{inclusión} $S\hookrightarrow G$ es un morfismo de grupos. 

\begin{example}
$\det\colon\GL_2(\R)\to\R^\times$ es un morfismo de grupos.
\end{example}

\begin{example}
\index{Restricción de un morfismo}
	Sea $f\colon G\to H$ un morfismo de grupos y sea $S$ un subgrupo de $G$. 
	La \textbf{restricción} $f|_S\colon S\to H$ es también un morfismo de grupos.
\end{example}

\begin{example}
La función $f\colon\R\to\C^\times$, $f(x)=\cos x+i\sin x$, es un morfismo de grupos pues
$f(x+y)=f(x)f(y)$ para todo $x,y\in\R$. 
\end{example}

\begin{exercise}
Sea $f\colon G\to H$ un morfismo de grupos. Demuestre que $f(1)=1$, que $f(g^{-1})=f(g)^{-1}$ y que  $f(g^n)=f(g)^n$ para todo $g\in G$ y $n\in\N$.  
\end{exercise}

\begin{example}
Sea $f\colon\R_{>0}\to\R$, $f(x)=\log(x)$. La fórmula 
\[\log(xy)=\log(x)+
\log(y)
\]
nos dice que $f$ es un morfismo de grupos. Los resultados del ejercicio anterior
se traducen en las siguientes propiedades de la función logaritmo: 
\[
\log(1)=0,
\quad
\log\left(\frac{1}{x}\right)=-\log(x),
\quad
\log(x^n)=n\log(x).
\] 
\end{example}

\begin{definition}
	\index{Núcleo!de un morfismo de grupos}
	Sea $f\colon G\to H$ un morfismo de grupos. El \textbf{núcleo} de $f$ 
	es el conjunto
	$\ker f=\{x\in G:f(x)=1\}$. 
\end{definition}

La propiedad fundamental que tiene el núcleo de un morfismo $f$ es la siguiente: $f(x)=f(y)$ si y sólo si $x=yk$ para algún $k\in\ker f$. 

\begin{example}
Sea $f\colon\mathcal{U}(\Z/21)\to\mathcal{U}(\Z/21)$ el morfismo de grupos definido por $f(x)=x^3$. Entonces 
$\ker f=\{1,4,16\}$ y $f(\mathcal{U}(\Z/21))=\{1,8,13,20\}$. 
\end{example}

%\begin{example}
%Sea $f\colon\Aff(\R)\to\R$ el morfismo $\begin{pmatrix}a&b\\0&1\end{pmatrix}\mapsto a$ que
%vimos en el ejemplo~\ref{exa:afin} de la página~\pageref{exa:afin}. Entonces
%\[
%\ker f=\left\{\begin{pmatrix}
%1&b\\
%0&1
%\end{pmatrix}
%\right\}
%\]
%\end{example}


\begin{example}
\label{exa:afin}
Sea 
\[
\Aff(\R)=\left\{\begin{pmatrix}a&b\\0&1\end{pmatrix}:a\in\R^\times,\,b\in\R\right\}\leq\GL_2(\R).
\]
La función 
\[
f\colon \Aff(\R)\to\R^\times,\quad
\begin{pmatrix}
a&b\\
0&1
\end{pmatrix}
\mapsto a
\]
es un morfismo de grupos (de hecho, $f(x)=\det(x)$ para todo $x\in\Aff(\R)$) tal que
\[
\ker f=\left\{\begin{pmatrix}
1&b\\
0&1
\end{pmatrix}
\right\}.
\]
Dejamos como ejercicio verificar que la función $g\colon\Aff(\R)\to\R$, $\begin{pmatrix}a&b\\0&1\end{pmatrix}\mapsto b$, 
no es un morfismo de grupos. 
\end{example}

\begin{example}
Sea $f\colon\R\to\C^\times$, $f(x)=\cos x+i\sin x$. Entonces
\[
\ker f=\{2\pi k:k\in\Z\}=2\pi\Z.
\] 
\end{example}

\begin{definition}
\index{Imagen!de un morfismo de grupos}
La \textbf{imagen} de $f$ es 
el conjunto $f(G)=\{f(x):x\in G\}$. 
\end{definition}

\begin{proposition}
	Si  $f\colon G\to H$ un morfismo de grupos. Valen las siguientes propiedades:
	\begin{enumerate}
		\item $\ker f$ es un subgrupo normal de $G$.
		\item $f(G)$ es un subgrupo de $H$.
	\end{enumerate}
\end{proposition}

\begin{proof}
	Demostraremos solamente la primera afirmación, la segunda quedará como ejercicio. Primero debemos 
	demostrar que $\ker f$ es un subgrupo de $G$. Para eso, observamos que $1\in \ker f$ y además que si $x,y\in\ker f$ entonces $xy^{-1}\in\ker f$ (pues como $f$ es morfismo de grupos se tiene 
	que $f(xy^{-1})=f(x)f(y)^{-1}=1$). Para verificar que $\ker f$ es normal en $G$, sean $x\in\ker f$ y $g\in G$. Entonces $gxg^{-1}\in\ker f$ pues
	$f(gxg^{-1})=f(g)f(x)f(g)^{-1}=f(g)f(g)^{-1}=1$.   
\end{proof}

La imagen en general no es un subgrupo normal. 

\begin{example}
La inclusión $\langle (12)\rangle\hookrightarrow\Sym_3$ es un morfismo de grupos cuya imagen  
no es un subgrupo normal de $\Sym_3$.
\end{example}

\begin{example}
Sabemos que $\mathcal{U}(\Z/21)=\{1,2,4,5,8,10,11,13,16,17,19,20\}$ es un grupo abeliano. La función $f\colon\mathcal{U}(\Z/21)\to\mathcal{U}(\Z/21)$, $f(x)=x^3$, es un morfismo de grupos. La imagen de $f$ es igual a $\{1,8,13,20\}$, que es un subgrupo de $\mathcal{U}(\Z/21)$. 
\end{example}

\begin{example}
La función $\sgn\colon\Sym_n\to\{-1,1\}$ es un morfismo sobreyectivo de grupos tal que 
$\ker(\sgn)=\Alt_n$. En particular, $\Alt_n$ es un subgrupo normal de $\Sym_n$.   
\end{example}

\begin{example}
\index{Morfismo!canónico}
Si $N$ es un subgrupo normal de $G$, la función $\pi\colon G\to G/N$, $x\mapsto xN$, es un morfismo sobreyectivo tal que $\ker\pi=N$. La función $\pi$ se conoce como 
el \textbf{morfismo canónico} $G\to G/N$.
\end{example}

El ejemplo anterior nos dice, en particular, que cada subgrupo normal de un grupo $G$ es el núcleo de un morfismo con dominio en $G$. 

\begin{exercise}
Sea $f\colon G\to H$ un morfismo de grupos. Demuestre las siguientes afirmaciones:
\begin{enumerate}
\item Si $S\leq G$, entonces $f(S)\leq H$ y además $f^{-1}(f(S))=S\ker f$. 
\item Si $T\leq H$, entonces $\ker f\leq f^{-1}(T)\leq G$ y además $f(f^{-1}(T))=T\cap f(G)$. 
\item $f$ es inyectiva si y sólo si $\ker f=\{1\}$.
\item Si $g\in G$ tiene orden finito, entonces $|f(g)|$ divide a $|g|$. 
\end{enumerate}	
\end{exercise}

%\begin{exercise}
%	Sea $f\colon G\to H$ un morfismo de grupos y sea $Y$ un subconjunto de $H$.
%	Demuestre que $f^{-1}(Y)=\{x\in G:f(x)\in Y\}$ es un subgrupo de $G$. 
%\end{exercise}
%
%Sean $G$ y $H$ grupos. 
%Si existe un isomorfismo de grupos $G\to H$, diremos que $G$ y $H$ son grupos \textbf{isomorfos}. En ese caso, escribiremos
%$G\simeq H$. 
Si $f\colon G\to H$ es un isomorfismo de grupos, entonces $f^{-1}\colon H\to G$ es también un isomorfismo. Observemos además 
que un morfismo de grupos $f\colon G\to H$ será un isomorfismo si y sólo si existe
un morfismo de grupos $g\colon H\to G$ tal que $g\circ f=\id_G$ y $f\circ g=\id_H$. 

\begin{example}
$\Sym_2\simeq\Z/2\simeq G_2$. 	
\end{example}

\begin{example}
$\D_3\simeq\Sym_3$ y el isomorfismo está dado por la función $\D_3\to\Sym_3$, 
\[
1\mapsto \id,\quad 
r\mapsto (123),\quad r^2\mapsto(132),\quad s\mapsto(12),\quad rs\mapsto(13),\quad r^2s\mapsto(23).
\]	
\end{example}

\begin{example}
$\Z/2\times\Z/3\simeq\Z/6$ y el isomorfismo está dado por 
\[
(0,0)\mapsto 0,\quad (1,0)\mapsto 3,\quad
(0,1)\mapsto 4,\quad (1,1)\mapsto 1,\quad (0,2)\mapsto 2,\quad (1,2)\mapsto 5.
\]	
\end{example}

\begin{example}
La función $\log\colon\R_{>0}\to\R$ es un morfismo de grupos. Como $\log$ es biyectiva, 
$\R_{>0}\simeq\R$.
\end{example}

Es fácil demostrar que si $f\colon G\to H$ es un isomorfismo entonces $|g|=|f(g)|$ para todo $g\in G$. 

\begin{example}
$\Z/2\times\Z/2\not\simeq\Z/4$ pues en $\Z/2\times\Z/2$ no hay elementos de orden cuatro. 
\end{example}

\begin{example}
$\Q/\Z\not\simeq\Q$. Ambos son grupos abelianos, pero no son isomorfos. 
Para verlo, primero observamos que en $\Q$ todo elemento no trivial tiene orden 
infinito (pues si $kx=0$ con $k\in\Z$ y $x\in\Q\setminus\{0\}$ entonces $k=0$). 
En cambio, en $\Q/\Z$ todo elemento tiene orden finito. En efecto, si $x=r/s\in\Q$, entonces, como
\[
s(x+\Z)=sx+\Z=r+\Z=\Z
\]
se concluye que $|x+\Z|\leq s$. 
\end{example}

\begin{example}
Veamos que $\mathcal{U}(\Z/5)\simeq\mathcal{U}(\Z/10)$. 
En efecto, ambos grupos son cíclicos de orden cuatro pues 
$\mathcal{U}(\Z/5)=\langle 2\rangle$ y $\mathcal{U}(\Z/10)=\langle 3\rangle$. 
En cambio, $\mathcal{U}(\Z/10)\not\simeq\mathcal{U}(\Z/12)$ pues
en $\mathcal{U}(\Z/12)$ no hay elementos de orden cuatro.  
\end{example}

\begin{exercise}	
Demuestre que $F=\{\sigma\in\Sym_n:\sigma(n)=n\}\leq\Sym_n$ y que $F\simeq\Sym_{n-1}$. 
\end{exercise}

Si $G$ y $H$ son grupos, utilizaremos la siguiente notación:
\[
\Hom(G,H)=\{f\colon G\to H:f\text{ es morfismo}\}.
\]

Veamos algunos ejemplos.

\begin{example}
Veamos que 
$\Hom(\Q,\Z)=\{0\}$. Sea $f\in\Hom(\Q,\Z)$ y sea $p$ un número primo. Si fijamos $x\in\Q$ tenemos entonces, como 
\[
f(x)=f\left(p(x/p)\right)=pf(x/p),
\]	
$p$ divide a $f(x)$, de donde se concluye que $f(x)=0$ para todo $x\in\Q$ pues el primo $p$ es arbitrario.  
\end{example}

\begin{example}
Si $G$ es un grupo, entonces $\Hom(\Z,G)=\{k\mapsto g^k:g\in G\}$. 
Primero observemos que para cada $g\in G$ la función $\Z\to G$, $k\mapsto g^k$, es un morfismo de grupos, pues $k+l\mapsto g^{k+l}=g^kg^l$. Sea $f\in\Hom(\Z,G)$ y sea $g=f(1)$. Si $k>0$, 
\[
f(k)=f(\underbrace{1+\cdots+1}_{k-\text{veces}})=f(1)^k=g^k.
\]
Si, en cambio $k<0$, entonces 
\[
f(k)=f(\underbrace{(-1)+\cdots+(-1)}_{|k|-\text{veces}})=f(-1)^{-k}=(g^{-1})^{-k}=g^k.
\]	
\end{example}

\begin{example}
Vamos a demostrar que $\Hom(\Z/8,\Z/10)$ tiene dos elementos. 
Sea $f\colon\Z/8\to\Z/10$ un morfismo no nulo. Si $n=|f(1)|$, entonces
$n$ divide a $8$, es decir $n\in\{1,2,4,8\}$. Como además $f(1)\in\Z/10$ y $f$ es no nulo, 
$n=2$. Luego $f(1)=5$ y eso define únivocamente al morfismo $f$. En nuestro caso, vemos que  
$f(k)=5k$ para $k\in\{0,1,\dots,7\}$. 	
\end{example}

\begin{exercise}
Calcule $\Hom(\Z/n,G)$ para cualquier grupo $G$. 
%Demuestre que $\Hom(\Z/n,G)=\{k\mapsto g^k:g\in G,\,|g|\text{ divide a }n\}$. 
\end{exercise}

\begin{exercise}
Sean $A$, $B$ y $C$ grupos. Si $f\in\Hom(A,B)$ y $g\in\Hom(B,C)$, 
entonces $g\circ f\in\Hom(A,C)$. 	
\end{exercise}

\begin{exercise}
Demuestre que $\Z/2\times\Z/2$ y $\Z/4$ son los únicos subgrupos de orden cuatro (salvo isomorfismo).
\end{exercise}

Veamos un ejemplo de isomorfismo un poco más difícil que los anteriores.

\begin{example}
Si $G$ es un grupo de orden seis, entonces $G\simeq\Sym_3$ o bien $G$ es cíclico de orden seis. 
Para demostrar nuestra afirmación primero observamos que, como $|G|$ es par, existe en $G$ 
un elemento de orden dos, esto lo vimos en el ejercicio~\ref{xca:orden2}. Si todo elemento 
de $G\setminus\{1\}$ tuviera orden dos, entonces $xy=yx$ para todo $x,y\in G$ y luego
\[
\langle x,y\rangle=\{1,x,y,xy\}\leq G,
\]
una contradicción al teorema de Lagrange. Existe entonces $x\in G$ tal que $x$ tiene orden dos y existe $y\in G\setminus\{1\}$ tal que $y$ no tiene orden dos. Nuevamente el teorema
de Lagrange nos dice que $|y|\in\{3,6\}$ (pues el orden de $y$ es un divisor del orden del grupo $G$). Si $|y|=6$, entonces 
$G\simeq\Z/6$. En cualquier caso, existe $z\in G$ tal que 
$|z|=3$. Tenemos 
\[
\langle x,z\rangle=\{1,x,z,z^2,xz,xz^2\}=G.
\]
Para saber qué grupo es $\langle x,z\rangle$ necesitamos entender el producto $zx$. Sabemos que $zx\in\{xz,xz^2\}$. Si $xz=zx$, entonces $|xz|=6$ (pues $(xz)^k\ne1$ para todo $k\in\{1,\dots,5\}$ y 
además $(xz)^6=1$) y luego 
$G=\langle xz\rangle\simeq\Z/6$. Si, en cambio, estamos en el caso $zx=xz^2$, entonces
$G=\langle x,z:x^2=z^3=1,\,xzx^{-1}=z^2\rangle\simeq\D_3$.      
\end{example}

% \begin{example}
% Si $G$ es un grupo de orden ocho, entonces tenemos cinco posibilidades: $G\simeq(Z/2)^3$, $G\simeq\Z/4\times\Z/2$, $G\simeq \Z/8$, $G\simeq\D_4$ o bien $G\simeq Q_8$. 
% \end{example}

Lo que hicimos hasta ahora nos permite clasificar las clases de isomorfismo de grupos de orden $\leq8$.

\begin{table}[ht]
    \centering
    \begin{tabular}{|c|c|c|}
    \hline
    Orden & Cantidad & Grupos\\
    \hline
        1 & 1 & $\{1\}$ \\
        2 & 1 & $\Z/2$ \\
        3 & 1 & $\Z/3$ \\
        4 & 2 & $\Z/4$, $\Z/2\times\Z/2$ \\
        5 & 1 & $\Z/5$ \\
        6 & 2 & $\Z/6$, $\Sym_3$ \\
        7 & 1 & $\Z/7$ \\
%        8 & 5 & $(\Z/2)^3$, $\Z/4\times\Z/2$, $\Z/8$, $\D_4$, $Q_8$ \\
    \hline
    \end{tabular}
    \caption{Grupos de orden $\leq7$.}
    \label{tab:grupos<8}
\end{table}

\begin{exercise}
\label{xca:size9}
Demuestre que salvo isomorphismo los únicos grupos de orden nueve son $\Z/9$ y $\Z/3\times\Z/3$. 
\end{exercise}

Estamos en condiciones de enunciar y demostrar los teoremas de isomorfismos. 
% Primero comenzaremos con un teorema técnico pero fundamental. 

% \begin{theorem}
% \label{thm:cocientes}
% Sea $f\colon G\to H$ un morfismo de grupos y $K$ un subgrupo normal de $G$ tal que $K\subseteq\ker f$. Existe entonces
% un único morfismo $\varphi\colon G/K\to H$ tal que el diagrama
% \[
%         \xymatrix{
%         G
%         \ar[d]_\pi
%         \ar[r]^f
%         & H
%         \\
%         G/K\ar@{-->}[ur]_{\varphi}
%         }
% \]
% es conmutativo, lo que significa que $\varphi\circ\pi=f$, donde $\pi\colon G\to G/K$ es el morfismo canónico. 
% Más aún, $\ker\varphi=\pi(\ker f)$ 
% y $\varphi(G/K)=f(G)$. En particular, $\varphi$ es inyectiva si y sólo si $\ker f=K$ y $\varphi$ es sobreyectiva si y sólo si $f$ es sobreyectiva. 
% \end{theorem}

% \begin{proof}
% 	Sea $\varphi\colon G/K\to H$, $xK\mapsto f(x)$. Primero debemos demostrar que $\varphi$ está bien definida, lo que significa 
% 	demostrar que si $xK=yK$ entonces $f(x)=f(y)$. En efecto, si $xK=yK$, entonces, como $y^{-1}x\in K$, se tiene que
% 	\[
% 	f(y)^{-1}f(x)=f(y^{-1}x)\subseteq f(K)=\{1\}.
% 	\]
% 	Luego $f(x)=f(y)$. 
		
% 	Veamos que $\varphi$ es morfismo de grupos:
% 	\[
% 	\varphi(xKyK)=\varphi(xyK)=f(xy)=f(x)f(y)=\varphi(xK)\varphi(yK).
% 	\]
% 	Para calcular $\ker\varphi$ procedemos así: 
% 	\[
% 	\pi(x)=xK\in\ker\varphi\Longleftrightarrow \varphi(xK)=1
% 	\Longleftrightarrow f(x)=1
% 	\Longleftrightarrow x\in\ker f.
% 	\]
% 	En consecuencia,  $\ker\varphi=\pi(\ker f)$. 
% 	La igualdad $\varphi(G/K)=f(G)$ es trivial.
	
% 	De la definición de $\varphi$ se obtiene inmediatamente que $\varphi\circ \pi=f$. 
% 	Esta igualdad además garantiza la unicidad del morfismo $\varphi$ pues si $\psi$ 
% 	es tal que $\psi\circ\pi=f$, 
% 	\[
% 	\varphi(xK)=\varphi(\pi(x))=(\varphi\circ\pi)(x)=f(x)
% 	=(\psi\circ\pi)(x)=\psi(\pi(x))=\psi(xK).\qedhere
% 	\]
% \end{proof}

%Como corolario obtenemos:

\begin{theorem}[primer teorema de isomorfismos]
\index{Teorema!de isomorfismos I}
\index{Primer teorema de isomorfismos}	
Si $f\colon G\to H$ es un morfismo de grupos, entonces $G/\ker f\simeq f(G)$. 
\end{theorem}

\begin{proof}	
	Sean $K=\ker f$ y $\varphi\colon G/K\to H$ la función dada por $xK\mapsto f(x)$. Primero debemos demostrar que $\varphi$ está bien definida, lo que significa 
	demostrar que si $xK=yK$ entonces $f(x)=f(y)$. En efecto, si $xK=yK$, entonces, como $y^{-1}x\in K$, se tiene que
	\[
	f(y)^{-1}f(x)=f(y^{-1}x)\in f(K)=\{1\}.
	\]
	Luego $f(x)=f(y)$. 
		
	Veamos que $\varphi$ es morfismo de grupos:
	\[
	\varphi(xKyK)=\varphi(xyK)=f(xy)=f(x)f(y)=\varphi(xK)\varphi(yK).
	\]
	Para calcular $\ker\varphi$ procedemos así: 
	\[
	\pi(x)=xK\in\ker\varphi\Longleftrightarrow \varphi(xK)=1
	\Longleftrightarrow f(x)=1
	\Longleftrightarrow x\in K.
	\]
	En consecuencia,  $\ker\varphi$ es trivial y entonces $\varphi$ es inyectiva. Como es trivial verificar que $\varphi$ es sobreyectica, 
	se concluye que $G/K\simeq f(G)$. 
% 	De la definición de $\varphi$ se obtiene inmediatamente que $\varphi\circ \pi=f$. 
% 	Esta igualdad además garantiza la unicidad del morfismo $\varphi$ pues si $\psi$ 
% 	es tal que $\psi\circ\pi=f$, 
% 	\[
% 	\varphi(xK)=\varphi(\pi(x))=(\varphi\circ\pi)(x)=f(x)
% 	=(\psi\circ\pi)(x)=\psi(\pi(x))=\psi(xK).\qedhere
% 	\]
\end{proof}

\begin{examples}
Si $G$ es un grupo, entonces $G/\{1\}\simeq G$ y $G/G\simeq\{1\}$. 
\end{examples}

\begin{example}
Como $f\colon\Z\to\Z/n$, $x\mapsto x\bmod n$, es un morfismo sobreyectivo con $\ker f=n\Z$, del primer teorema de isomorfismos se concluye que 
$\Z/n\Z\simeq\Z/n$. 	
\end{example}

\begin{example}
Sea $G$ un grupo cíclico infinito, digamos $G=\langle g\rangle$. Es fácil verificar que la función $f\colon\Z\to G$, $k\mapsto g^k$, 
es un isomorfismo de grupos, es decir $G\simeq\Z$. En particular, $G=\langle g^k\rangle$ si y sólo si $k\in\{-1,1\}$. 
\end{example}

\begin{example}
Vamos a demostrar que 
$\Z/n\Z\simeq G_n$. Sea 
\[
f\colon\Z\to G_n,\quad
f(k)=\exp(2i\pi k/n).
\]
Es claro que $f$ es sobreyectiva y que $\ker f=n\Z$. El resultado que queremos demostrar se obtiene entonces inmediatemente del primer teorema de isomorfismos.	
\end{example}

\begin{example}
Observemos con $2\Z\simeq 3\Z$ (observar que ambos son cíclicos de orden infinito o considerar la función $2k\mapsto 3k$) y que 
\[
\Z/2\simeq\Z/2\Z\not\simeq\Z/3\Z\simeq\Z/3.
\]
\end{example}

\begin{example}
Como 
\[
f\colon\C^\times\to\C^\times, 
\quad
f(z)=\frac{z}{|z|},
\]
es un morfismo tal que $\ker f=\R_{>0}$ y $f(\C^\times)=S^1$, se concluye del primer teorema
de isomorfismos que $\C^\times/\R_{>0}\simeq S^1$.  
\end{example}

\begin{example}
El primer teorema de isomorfismos aplicado a $f\colon S^1\to S^1$, $f(z)=z^2$, permite demostrar que $S^1/\{\pm1\}\simeq S^1$ pues
$\ker f=\{-1,1\}$ y $f(S^1)=S^1$. 	
\end{example}

\begin{example}
Sea $f\colon\C^\times\to\C^\times$, $f(z)=|z|$. Como $\ker f=S^1$ y $f(\C^\times)=\R_{>0}$, se conluye del primer teorema
de isomorfismos que $\C^\times/S^1\simeq\R_{>0}$. 
\end{example}

\begin{example}
Veamos que $(\Z\times\Z)/\langle (1,3)\rangle\simeq\Z$. Para eso, consideramos el morfismo sobreyectivo
$f\colon\Z\times\Z\to\Z$, $f(x,y)=3x-y$. Como 
\[
\ker f=\{(x,3x):x\in\Z\}=\langle (1,3)\rangle,
\]
el primer
teorema de isomorfismos implica que $(\Z\times\Z)/\langle (1,3)\rangle\simeq\Z$. 
\end{example}

\begin{exercise}
Demuestre que $\R/\Z\simeq S^1$.	
\end{exercise}

% $t\mapsto \exp(2\pi it)$
% $z\mapsto z^2$


\begin{exercise}
Demuestre que $\Q/\Z\simeq\cup_{n\geq1}G_n$.
\end{exercise}
% x\mapsto cos (2\pi x)+i\sin (2\pi x)

\begin{exercise}
Demuestre que $(\Z\times\Z)/\langle (6,3)\rangle\simeq\Z\times(\Z/3)$.
\end{exercise}
% (x,y)\mapsto (2y-x,y\mod 3)

\begin{example}
\index{Teorema!fundamental del álgebra lineal}
Si $V$ es un espacio vectorial y $W$ es un subespacio de $V$, entonces, en particular, 
$V$ es un grupo abeliano y $W$ es un subgrupo normal de $V$. El grupo abeliano 
$V/W$ es entonces un espacio vectorial con 
\[
\lambda(v+W)=(\lambda v)+W,\quad \lambda\in\R,\,v\in V,
\]
y el morfismo canónico $\pi\colon V\to V/W$ resulta ser una transformación lineal. 
Dejamos como ejercicio 
demostrar que 
$\dim (V/W)=\dim V-\dim W$
si $\dim V<\infty$. 

Si $f\colon V\to U$ es una transformación lineal, entonces, por el primer teorema de isomorfismos, en particular, 
$V/\ker f\simeq f(V)$ como grupos abelianos. Como el morfismo del primer teorema de isomorfismos es además una transformación lineal, 
se concluye que $V/\ker f\simeq f(V)$ como espacios vectoriales. En particular, si $\dim V<\infty$, entonces 
\[
\dim V-\dim\ker f=\dim f(V).
\]
\end{example}

\begin{exercise}
\label{xca:cocientes}
Sea $f\colon G\to H$ un morfismo de grupos y $K$ un subgrupo normal de $G$ tal que $K\subseteq\ker f$. Demuestre que existe 
un único morfismo $\varphi\colon G/K\to H$ tal que el diagrama
\[\begin{tikzcd}
	G & H \\
	{G/K}
	\arrow["f", from=1-1, to=1-2]
	\arrow["\pi"', from=1-1, to=2-1]
	\arrow["\varphi"', dashed, from=2-1, to=1-2]
\end{tikzcd}
\]
es conmutativo, lo que significa que $\varphi\circ\pi=f$, donde $\pi\colon G\to G/K$ es el morfismo canónico. 
Más aún, $\ker\varphi=\pi(\ker f)$ y $\varphi(G/K)=f(G)$. 
En particular, $\varphi$ es inyectiva si y sólo si $\ker f=K$ y $\varphi$ es sobreyectiva si y sólo si $f$ es sobreyectiva. 
\end{exercise}

El segundo teorema de isomorfismos resultará de gran utilidad al estudiar series de composición y resolubilidad. 
El diagrama que vemos a continuación nos ayudará a recordar cómo funciona el segundo teorema de isomorfismos:
\[\begin{tikzcd}
	& NT \\
	N && T \\
	& {N\cap T}
	\arrow[no head, from=1-2, to=2-3]
	\arrow[no head, from=1-2, to=2-1]
	\arrow[no head, from=2-1, to=3-2]
	\arrow[no head, from=2-3, to=3-2]
\end{tikzcd}\]

\begin{theorem}[segundo teorema de isomorfismos]
\index{Teorema!de isomorfismos II}
\index{Segundo teorema de isomorfismos}	
Si $N$ es un subgrupo normal de $G$ y $T$ es un subgrupo de $G$, entonces $N\cap T$ es normal en $T$ y además 
\[
T/N\cap T\simeq NT/N.
\]	
\end{theorem}

\begin{proof}
Sea $\pi\colon G\to G/N$ el morfismo canónico. Ya vimos que la restricción $\pi|_T\colon T\to G/N$ es un morfismo de grupos con núcleo
$\ker(\pi|_T)=T\cap N$. En particular, $T\cap N$ es normal en $T$. Al aplicar el primer
teorema de isomorfismos, $T/(T\cap N)\simeq \pi(T)$. Como $N$ es normal en $G$, 
$NT$ es un subgrupo de $G$ que contiene a $N$. 
La restricción $\pi|_{NT}$ es entonces un morfismo de grupos con núcleo $NT\cap N=N$.  
Al aplicar el primer teorema de isomorfismos a $\pi|_{NT}$ obtenemos
$NT/N\simeq \pi(NT)=\pi(T)$.  
%%\[
%%T/T\cap N\simeq\pi(T).
%%\]
%%Observemos que $\pi(T)=NT/N$ pues es el subgrupo de $G/N$ formado por las coclases de $N$ en $G$ con representantes en $T$.  
\end{proof}

\begin{exercise}
Sea $N$ normal en $G$ y sea $\pi\colon G\to G/N$ el morfismo canónico. Demuestre que
si $L$ es un subgrupo de $G$, entonces $\pi^{-1}(\pi(L))=NL$. 
%%% 
%%%Por otro lado, si $L$ es un subgrupo de $G$, entonces $\pi^{-1}(\pi(L))=NL$. En efecto,
%%%si $x\in \pi^{-1}(\pi(L))$, entonces $\pi(x)\in \pi(L)$ y luego 
%%%$\pi(x)=\pi(l)$ para algún $l\in L$. Como entonces $xl^{-1}\in \ker\pi=N$, 
%%%se tiene que $x=(xl^{-1})l\in KL$. Recíprocamente, si $x=kl$ con $k\in K$ y $l\in L$, entonces
%%%$\pi(x)=\pi(kl)=\pi(l)\in \pi(L)$ y luego $x\in\pi^{-1}(\pi(L))$. Observemos que
%%%si $L$ es un subgrupo de $G$, entonces $NL$ es un subgrupo de $G$ que contiene a $N$. 
\end{exercise}

En el siguiente ejemplo utilizaremos la notación aditiva. 

\begin{example}
Sea $G=\Z/24$ y sean $H=\langle 4\rangle$ y $N=\langle 6\rangle$. Como $G$ es abeliano, $H$ y $K$ son ambos normales en $G$. Un cálculo directo nos muestra que 
$H+N=\langle 2\rangle$ y que $H\cap N=\{0,12\}$. Notemos que este ejemplo está completamente hecho en la notación aditiva. Calculemos las coclases de $N$ en $H+N$:
\[
0+N=\{0,6,12,18\},
\quad
2+N=\{2,8,14,20\},
\quad
4+N=\{4,10,16,22\}.
\]
Las coclases de $H\cap N$ en $H$ son:
\[
0+(H\cap N)=\{0,12\},
\quad
4+(H\cap N)=\{4,16\},
\quad
8+(H\cap N)=\{8,20\}.
\]
El segundo teorema de isomorfismos nos dice que $(H+N)/N\simeq H/H\cap N$. El isomorfismo 
está dado por $f\colon H/(H\cap N)\to (H+N)/N$, $h+(H\cap N)\mapsto h+N$. 
En nuestro caso, 
\begin{align*}
&f(0+(H\cap N))=0+N,\\
&f(4+(H\cap N))=4+N,\\
&f(8+(H\cap N))=8+N=2+N.
\end{align*}
\end{example}

En los ejemplos que siguen veremos que el segundo teorema de isomorfismos no es algo raro sino que nos permite obtener fórmulas ya conocidas.

\begin{example}
\index{Máximo común divisor}
\index{Mínimo común múltiplo}
Sean $a,b\in\Z$ no nulos. Sabemos que $a\Z+b\Z=\gcd(a,b)\Z$ y que $a\Z\cap b\Z=\lcm(a,b)\Z$. Al aplicar el segundo teorema de isomorfismos, 
\[
\frac{\gcd(a,b)\Z}{b\Z}=\frac{a\Z+b\Z}{b\Z}\simeq
\frac{a\Z}{a\Z\cap b\Z}=\frac{a\Z}{\lcm(a,b)\Z}.
\]
Al aplicar orden, obtenemos la fórmula 
\[
ab=\gcd(a,b)\lcm(a,b).
\] 
\end{example}

\index{Grupo!meta-abeliano}
Veamos otra aplicación. Un grupo $G$ que contiene un subgrupo normal abeliano $N$ y es tal que $G/N$ es abeliano se conoce como grupo \textbf{meta-abeliano}. Claramente, los grupos meta-abelianos no son necesariamente abelianos (el grupo simétrico $\Sym_3$ es meta-abeliano y no abeliano). El segundo teorema de isomorfismos nos permite demostrar que subgrupos de meta-abelianos son meta-abelianos. 

\begin{proposition}
Si $G$ es un grupo meta-abeliano y $H$ es un subgrupo de $G$, entonces $H$ es también meta-abeliano. 
\end{proposition}

\begin{proof}
Como $G$ es meta-abeliano, existe 
un subgrupo normal $N$ de $G$ tal que $N$ y $G/N$ son ambos abelianos. 
El subgrupo abeliano $H\cap N$ es normal en $H$. Gracias al segundo teorema de isomorfismos,
\[
H/(H\cap N)\simeq HN/N
\]
es un grupo abeliano pues $HN/N$ es un subgrupo del grupo abeliano $G/N$.   
\end{proof}

Dejamos la demostración del tercer teorema de isomorfismos como ejercicio. Primero, 
un resultado auxiliar que facilitará los cálculos. 
%Antes de demostrar otro de los teoremas de isomorfismos, necesitamos el siguiente lema técnico. 

\begin{exercise}
	\label{xca:para_3er}
	Sea $f\colon G\to H$ un morfismo de grupos y sean $U\unlhd G$ y $V\unlhd H$. Demuestre que existe 
	un morfismo de grupos $g\colon G/U\to H/V$ tal que el diagrama
\[
\begin{tikzcd}
	G & H \\
	{G/K} & {H/V}
	\arrow["f", from=1-1, to=1-2]
	\arrow["{\pi_U}"', from=1-1, to=2-1]
	\arrow["g"', dashed, from=2-1, to=2-2]
	\arrow["{\pi_V}", from=1-2, to=2-2]
\end{tikzcd}
\]
	es conmutativo si y sólo si $f(U)\subseteq V$, donde $\pi_U\colon G\to G/U$ y $\pi_V\colon H\to H/V$ son los morfismos canónicos. Además, en este caso, 
	\begin{enumerate}
	\item Si $f$ es sobreyectiva, entonces $g$ es sobreyectiva.
	\item Si $U=f^{-1}(V)$, entonces $g$ es inyectiva. 	
	\end{enumerate}
\end{exercise}

% todo: escribir bien la demo del tercero, ese lema es horrible
%Un caso particular del lema nos permite demostrar elegantemente el tercer teorema de isomorfismos.

\begin{exercise}[tercer teorema de isomorfismos]
\label{xca:3er}
\index{Teorema!de isomorfismos III}
\index{Segundo teorema de isomorfismos}	
Sean $S$ y $T$ subgrupos normales de un grupo $G$ tales que $S\subseteq T$. Demuestre que 
entonces $S$ es normal en $T$ 
y $T/S$ es normal en $G/S$. Además
\[
\frac{G/S}{T/S}\simeq G/T,
\]
donde $T/S=\{tS:t\in T\}$.
\end{exercise}

\begin{example}
Si $m$ divide a $n$, entonces $n\Z\leq m\Z\leq\Z$. Luego
\[
\frac{\Z/n\Z}{m\Z/n\Z}\simeq\Z/m\Z.
\]	
\end{example}

El teorema que sigue es también muy importante. Para recordar cómo funciona, podemos hacer uso del siguiente diagrama:
\[
\begin{tikzcd}
	&& G \\
	& L && {f(G)} \\
	N && Y \\
	& {\{1\}}
	\arrow[no head, from=1-3, to=2-4]
	\arrow[no head, from=1-3, to=2-2]
	\arrow[no head, from=2-2, to=3-1]
	\arrow[no head, from=3-1, to=4-2]
	\arrow[no head, from=2-2, to=3-3]
	\arrow[no head, from=3-3, to=4-2]
	\arrow[no head, from=2-4, to=3-3]
\end{tikzcd}
\]

\begin{theorem}[de la correspondencia]
\index{Teorema!de la correspondencia}
Sea $f\colon G\to H$ un morfismo de grupos y sea $K=\ker f$. Existe una correspondencia biyectiva entre
\[\begin{tikzcd}
	{\mathcal{A}=\{L:K\leq L\leq G\}} & {\{Y:Y\leq f(G)\}=\mathcal{B}} 
	\arrow["\sigma", shift left=1, from=1-1, to=1-2]
	\arrow["\tau", shift left=1, from=1-2, to=1-1]
\end{tikzcd}
\]
La correspondencia está dada por $\sigma(L)=f(L)$ y $\tau(Y)=f^{-1}(Y)$. Valen además las siguientes afirmaciones:
\begin{enumerate}
\item $L_1\leq L_2$ si y sólo si $\sigma(L_1)\leq \sigma(L_2)$. 
\item $L\unlhd G$ si y sólo si $\sigma(L)\unlhd f(G)$.
\end{enumerate} 
\end{theorem}

\begin{proof}
	Primero observamos que $\sigma$ y $\tau$ están ambas bien definidas pues vimos en un ejercicio
	que $f(L)\leq f(G)$ y $K\leq f^{-1}(Y)\leq G$. 
	
	Veamos que $\tau\circ\sigma=\id_\mathcal{A}$. Queremos ver que $\tau(\sigma(L))=L$ para todo $L\in\mathcal{A}$. Si $x\in f^{-1}(f(L))$ entonces
	$f(x)\in f(L)$ y luego $f(x)=f(l)$ para algún $l\in L$. Esto implica que $xl^{-1}\in K$ y entonces $x\in Kl\subseteq L$ pues $K\subseteq L$. 
	Recíprocamente, si $l\in L$ entonces $f(l)\in f(L)$ y luego $l\in f^{-1}(f(L))$. 
	
	Veamos que $\sigma\circ\tau=\id_\mathcal{B}$. Si $Y\in\mathcal{B}$, entonces $\sigma(\tau(Y))=Y$. Si $y\in Y\subseteq f(G)$, entonces
	$y=f(x)$ para algún $x\in G$, es decir $x\in f^{-1}(y)$, lo que trivialmente implica que $y=f(x)\in f(f^{-1}(Y))$. Recíprocamente, si $y\in f(f^{-1}(Y))$, entonces
	$y=f(x)$ para $x\in f^{-1}(Y)$. Pero esto significa que $y=f(x)\in Y$.   
	
	Dejamos como ejercicio demostrar que $X\leq Y$ si y sólo si $f(X)\leq f(Y)$. 
	
	Vamos a demostrar que $L\unlhd G$ si y sólo si $f(L)\unlhd f(G)$. Si $L\unlhd G$ y $x\in G$, entonces
	$xLx^{-1}=L$. Esto implica que $f(L)=f(xLx^{-1})=f(x)f(L)f(x)^{-1}$, es decir que $f(L)$ es normal en $f(G)$. Recíprocamente, si 
	$f(L)\unlhd f(G)$ y $x\in G$, entoces 
	\[
	f(xLx^{-1})=f(x)f(L)f(x)^{-1}=f(L).
	\]
	Esto implca que $xLx^{-1}\subseteq LK\subseteq L$ y luego
	$xLx^{-1}\subseteq L$, que implica la normalidad de $L$ en $G$ gracias a la proposición~\ref{pro:normalidad}.
\end{proof}

Veamos una aplicación del teorema anterior. 

\begin{proposition}
	Si $f\colon G\to f(G)$ es un morfismo sobreyectivo de grupos y $H\leq G$ es tal que $K=\ker f\subseteq H$, entonces
	$(G:H)=(f(G):f(H))$. 
\end{proposition}

\begin{proof}
Por el teorema anterior 
sabemos que existe una correspondencia biyectiva
\[
\begin{tikzcd}
	{\{L:K\leq L\leq G\}} & {\{Y:Y\leq f(G)\}}
	\arrow[shift left=1, from=1-1, to=1-2]
	\arrow[shift left=1, from=1-2, to=1-1]
\end{tikzcd}
\]
dada por $H\mapsto f(H)$ e 
inversa dada por $f^{-1}(T)\mapsfrom T$. Sea $H\leq G$ tal que $\ker f\subseteq H$ y sea 
$\alpha\colon G/H\to f(G)/f(H)$ la función dada por $\alpha(gH)=f(g)f(H)$. 
Dejamos como ejercicio verificar que $\alpha$ está bien definida. 
Veamos que $\alpha$ es una función biyectiva pues, en ese caso, 
\[
(G:H)=|G/H|=|f(G)/f(H)|=(f(G):f(H)).
\]

Veamos que $\alpha$ es sobreyectiva: si $yf(H)\in f(G)/f(H)$ entonces
$y=f(g)$ para algún $g\in G$ (pues $f$ es sobreyectiva). Luego 
\[
yf(H)=f(g)f(H)=f(gH)=\alpha(gH).
\]

Veamos ahora que $\alpha$ es inyectiva: si $\alpha(gH)=\alpha(g_1H)$, entonces, por la definición de la función $\alpha$, 
\[
f(g)^{-1}f(g_1)=f(h)\in f(H)
\]
para algún $h\in H$, es decir 
$f(g_1)=f(g)f(h)=f(gh)$ para algún $h\in H$. Esto implica que $g_1=ghk$ para algún $k\in\ker f\subseteq H$ y luego
$g_1=gh_1$ para algún $h_1\in H$, es decir $g_1H=gH$.  
\end{proof}

Es conviente enfatizar qué forma toma el teorema anterior en el caso del morfismo canónico $\pi\colon G\to G/N$.
Si $N$ es un subgrupo normal de $G$, entonces la función $K\mapsto K/N$ es una biyección entre el conjunto de subgrupos (normales) 
de $G$ que contienen a $N$ y el conjunto de subgrupos (normales) de $G/N$. 
Observemos que si $H$ es un subgrupo de $G$, entonces
\[
\pi(H)=HN/N.
\]

\begin{example}
\index{Grupo!de cuaterniones de Hamilton}
Como aplicación del teorema de la correspondencia, vamos a demostrar que todo subgrupo del grupo no abeliano 
\[
Q_8=\{1,-1,i,-i,j,-j,k,-k\}
\]
es normal en $Q_8$. Sea $N=\{-1,1\}$. Entonces $N$ es normal en $Q_8$ (pues $N\subseteq Z(Q_8)$) y además, como $Q_8/N$ tiene cuatro elementos, $Q_8/N$ es un grupo abeliano.

Afirmamos que $N$ está contenido en cualquier subgrupo no trivial de $Q_8$. En efecto, si 
$K$ es un subgrupo no trivial de $Q_8$, entonces $-1\in K$ (pues, por ejemplo, si $-i\in K$, entonces $-1=(-i)^2\in K$). 
Esto implica que cualquier subgrupo de $Q_8$ se corresponde con un subgrupo de $Q_8/N$ y allí todo subgrupo es normal pues $Q_8/N$ es abeliano. Más precisamente, si 
$S\leq Q_8$, entonces $\pi(S)\leq Q_8/N$. Como $Q_8/N$ es abeliano, $\pi(S)$ es normal en $Q_8/N$. 
Como $N\subseteq S$, se tiene que $S=\pi^{-1}(\pi(S))$. Luego $S$ es normal en $Q_8$.
\end{example}

En el ejemplo anterior, podríamos haber demostrado que 
$G/N\simeq\Z/2\times\Z/2$, ya que como sabemos que $|G/N|=4$, hubiera alcanzado con calcular el orden de cada uno de los elementos de $G/N$. 

\begin{example}
Sea $f\colon\Z/12\to\Z/6$ el morfismo dado por $1\mapsto 1$. Un cálculo sencillo nos muestra que $K=\ker f=\{0,6\}$. 
Los subgrupos de $\Z/12$ que contienen a $K$ son 
\[
\langle 1\rangle=\{0,1,\dots,11\},
\quad
\langle 2\rangle=\{0,2,4,6,8,10\},
\quad
\langle 3\rangle=\{0,3,6,9\},
\quad
\langle 6\rangle=\{0,6\},
\] 
que vía $f$ se corresponden con los subgrupos 
\[
\langle 1\rangle=\{0,1,\dots,5\},
\quad
\langle 2\rangle=\{0,2,4\},
\quad
\langle 3\rangle=\{0,3\},
\quad
\{0\}
\]
de $\Z/6$, respectivamente. Por ejemplo, 
\[
\begin{tikzcd}
	&& \Z/12 \\
	& \langle 2\rangle && {\Z/6} \\
	\{0,6\} && \langle 2\rangle \\
	& {\{0\}}
	\arrow[no head, from=1-3, to=2-4]
	\arrow[no head, from=1-3, to=2-2]
	\arrow[no head, from=2-2, to=3-1]
	\arrow[no head, from=3-1, to=4-2]
	\arrow[no head, from=2-2, to=3-3]
	\arrow[no head, from=3-3, to=4-2]
	\arrow[no head, from=2-4, to=3-3]
\end{tikzcd}
\]
\end{example}

Si se tiene un morfismo entre dos grupos,  
en cierto sentido, el teorema de la correspondencia nos permite trasladar propiedades de la
imagen del morfismo al dominio. Veamos una aplicación concreta.

\begin{example}
Sea $G$ un grupo finito que contiene un subgrupo normal $N$ tal que $N\simeq\Z/5$ y $G/N\simeq\Sym_4$. Vamos a demostrar las siguientes afirmaciones sobre $G$.
\begin{enumerate}
\item $|G|=120$
\item $G$ contiene un subgrupo normal de tamaño 20.
\item $G$ contiene tres subgrupos de orden 15, ninguno de ellos normal en $G$.
\end{enumerate}

Para demostrar la primera afirmación usamos el teorema de Lagrange pues
\[
24=|G/N|=\frac{|G|}{|N|}=|G|/5.
\]

Para la segunda afirmación, sea $K$ el subgrupo de $G/N$ isomorfo al grupo de Klein. Entonces
$K$ es normal en $G/N$ y $|K|=4$. Como
$(G/N:K)=6$,  
el subgrupo $K$ de $G/N$ se corresponde con un subgrupo normal $H$ de $G$ de índice 6. El teorema de Lagrange y el teorema de la correspondencia implican entonces que $|H|=20$ pues
\[
6=(G/N:K)=(G:H)=\frac{|G|}{|H|}.
\] 

Para demostrar la tercera afirmación observamos que $G/N\simeq\Sym_4$ tiene cuatro subgrupos de orden 3 (son los subgrupos generados por un 3-ciclo), 
ninguno de ellos normal en $G/N$. Nuevamente, el teorema de la correspondencia, nos dice que estos grupos se corresponderán con 4 subgrupos de $G$, todos de orden 15 y ninguno de ellos normal en $G$.
\end{example}

Recordemos que si $G$ es un grupo, $\Sym_G=\{f\colon G\to G:f\text{ es biyectiva}\}$. 
Terminaremos el capítulo con el siguiente teorema.

\begin{theorem}[Cayley]
Todo grupo $G$ es isomorfo a un subgrupo de $\Sym_G$. 
\end{theorem}

\begin{proof}
Sea $f\colon G\to\Sym_G$, $g\mapsto L_g$, donde $L_g\colon G\to G$, $L_g(x)=gx$. La función $f$ es un morfismo de grupos pues
\[
L_{gh}(x)=(gh)x=g(hx)=L_g(hx)=L_gL_h(x)
\]
para todo $g,h,x\in G$. Además es fácil verificar que $f$ es inyectivo (si $f(g)=f(h)$ entonces $L_g=L_h$, es decir que 
$gx=L_g(x)=L_h(x)=hx$ para todo $x\in G$, que implica que $g=h$). 
\end{proof}

Como aplicación, observamos que todo grupo finito es isomorfo a un subgrupo $\Sym_n$ para algún $n\in\N$. En particular, las matrices de permutación nos permiten observar que
todo grupo finito es un \textbf{grupo lineal}, es decir, isomorfo a un subgrupo de $\GL_n(\Z)$ para algún $n\in\N$. Veamos una aplicación un poquito más sofisticada.
  
\begin{proposition}
Todo grupo simple finito $G$ está contenido en algún $\Alt_n$.
\end{proposition}

\begin{proof}
Si $|G|=2$, el resultado es trivial pues $G\simeq\Alt_2$. Supongamos entonces que $|G|>2$.  
Sea $f\colon G\to\Sym_n$ el morfismo inyectivo obtenido del teorema de Cayley. Si $H=f(G)$, entonces $G\simeq H$ por el primer teorema de isomorfismos. Afirmamos que $H\subseteq\Alt_n$. Si   
$H$ no es un subgrupo de $\Alt_n$, existe $h\in H$ tal que $h\not\in\Alt_n$. Escribimos $h=f(g)$ para algún $g\in G$. Como $h\not\in\Alt_n$, 
\[
\sgn(f(g))=\sgn(h)=-1,
\]
entonces $g\not\in\ker(\sgn\circ f)$. 
Sea $K=\ker(\sgn\circ f)$. Entonces $K=\{1\}$ pues $G$ es simple. Además, $\sgn\circ f$ es una función biyectiva pues $\sgn(f(1))=1$ y $\sgn(f(g))=-1$. En consecuencia,
$G\simeq G/K\simeq\Z/2$, por el primer teorema de isomorfismos.  En particular, $|G|=2$, una contradicción. Luego $H\subseteq\Alt_n$.        
\end{proof}	

Como aplicación simpática del teorema de Cayley puede obtenerse que el axioma de asociatividad en un grupo permite demostrar
que ningún producto necesita llevar paréntesis.  En efecto, el teorema de Cayley afirma que $G$ es un subgrupo de $\Sym_G$. 
La composición de funciones es asociativa y es trivial observar que ninguna composición arbitraria y finita de funciones necesita llevar paréntesis, por eso escribimos 
\[
(f_1\circ\cdots\circ f_n)(g)=f_1(f_2(\cdots f_n(g))\cdots).
\]


\chapter{Grupos simples}

\index{Grupo!simple}
Recordemos que un grupo $G$ es \textbf{simple} si $G\ne\{1\}$ y sus únicos subgrupos normales
son $G$ y $\{1\}$.

\begin{example}
Si $p$ es un número primo, $\Z/p$ es simple.
\end{example}

\index{Clase de conjugación}
No es difícil demostrar que $\Alt_5$ es simple, pero necesitamos repasar algunos conceptos. Si $G$ es un grupo y $g\in G$, la clase de conjugación de $g$ en $G$ es el conjunto
$\{xgx^{-1}:x\in G\}$. Una observación sencilla pero importante: Si el subgrupo $N$ de $G$ es normal en $G$, entonces $N$ es uníón de clases de conjugación de $G$, y una de esas clases es $\{1\}$, pues 
\[
N=\bigcup_{n\in N}\{xnx^{-1}:x\in G\}.
\]  

%\begin{proposition}
%El grupo $\SL_3(2)$ es simple. 
%\end{proposition}

%\begin{proof}
%El grupo $\SL_3(2)=\{a\in\GL_3(2):\det(a)=1\}$ tiene orden... 
%\end{proof}

%Veamos otro ejemplo:
Veamos una aplicación sencilla de la afirmación anterior. 

\begin{proposition}
El grupo alternado $\Alt_5$ es simple.
\end{proposition}

\begin{proof}
Para demostrar el teorema vamos a contar los tamaños de las clases de conjugación de $\Alt_5$. Las clases de conjugación de $\Alt_5$ y sus tamaños son:
\begin{align*}
\id && 1\\
(123) && 20\\
(12)(34) && 15\\
(12345) && 12\\
(21345) && 12
\end{align*}
Si $N$ es un subgrupo normal de $\Alt_5$, entonces $N$ es unión de clases de conjugación de $\Alt_5$ y una de esas clases es $\{id\}$. Sin embargo, ninguna unión de clases de conjugación
de $\Alt_5$ que incluya $\{id\}$ tendrá tamaño un divisor de 60, a menos que $N=\{\id\}$ o bien que  $N=\Alt_5$.  
\end{proof} 

Nuestro objetivo es demostrar que los grupos alternados $\Alt_n$ son simples siempre que $n\geq5$. Para eso, necesitamos demostrar
varios lemas auxiliares. 

\index{Estructura cíclica}
Recordemos toda permutación $\rho\in\Sym_n$ puede descomponerse como producto de ciclos disjuntos, digamos
\[
\rho=(a_1\cdots a_r)(b_1\cdots b_s)\cdots (c_1\cdots c_t)
\]
donde por convención omitiremos aquellos ciclos de longitud uno. 
La estrucutra cíclica de $\rho$ será entonces la sucesión ordenada de los números $r,s,\dots t$, 
donde convenientemente omitiremos los puntos fijos. Por ejemplo, la estructura cíclica de la trasposición $(ab)$ es 2, 
del 3-ciclo $(abc)(d)$ es 3 y de la permutación $(123)(45)(789a)(bcd)(d)$ es 2,3,3,4. 
  
El primer lema que demostraremos afirma que dos permutaciones tienen la misma estructura cíclica si y sólo si son conjugadas. 

\begin{lemma}
Si $\rho_1$ y $\rho_2$ son permutaciones de $\Sym_n$ con la misma estructura cíclica, entonces
$\rho_2=\sigma\rho_1\sigma^{-1}$ para alguna permutación $\sigma\in\Sym_n$. 
\end{lemma}

\begin{proof}
Supongamos que
\begin{align*}
\rho_1=(a_1\cdots a_r)(b_1\cdots b_s)\cdots (c_1\cdots c_t), && 
\rho_2=(x_1\cdots x_r)(y_1\cdots y_s)\cdots (z_1\cdots z_t).
\end{align*}
Sean 
\begin{align*}
\Fix(\rho_1) &= \{x\in\{1,\dots,n\}:\rho_1(x)=x\}=\{k_1,\dots,k_m\},&&
\Fix(\rho_2)=\{l_1,\dots,l_m\}	
\end{align*}
los puntos fijos de las permutaciones $\rho_1$ y $\rho_2$, respectivamente. Entonces
\[
\sigma(x)=\begin{cases}
x_j & \text{si $x=a_j$ para algún $j$},\\
y_j & \text{si $x=b_j$ para algún $j$},\\
\vdots\\
z_j & \text{si $x=c_j$ para algún $j$},\\
l_j & \text{si $x=k_j$ para algún $j$},	
\end{cases}
\]
cumple que $\sigma\rho_1\sigma^{-1}=\rho_2$. 
\end{proof}

El siguiente lema es la variante del anterior correspondiente al grupo alternado. 
Nos interesa saber cuándo dos permutaciones conjugadas en $\Sym_n$ son también conjugadas en $\Alt_n$.

\begin{lemma}
Si $\rho_1,\rho_2\in\Sym_n$ son conjugados en $\Sym_n$ y además $|\Fix(\rho_1)|\geq2$, entonces
$\mu\rho_1\mu^{-1}=\rho_2$ para algún $\mu\in\Alt_n$.  
\end{lemma}

\begin{proof}
Supongamos que $\rho_2=\sigma\rho_1\sigma^{-1}$ para algún $\sigma\in\Sym_n$. 
Por hipótesis, sabemos que existen $a,b\in\{1,\dots,n\}$ tales que $\rho_1(a)=a$, $\rho_1(b)=b$ y $a\ne b$. Sea
\[
\mu=\begin{cases}
\sigma & \text{si $\sigma\in\Alt_n$,}\\
\sigma(ab) & \text{en caso contrario.}
\end{cases}
\]
Entonces $\mu\in\Alt_n$ y además $\mu\rho_1\mu^{-1}=\rho_2$ pues $(ab)$ conmuta on $\rho_1$. 
\end{proof}

Veamos algunos ejemplos.

\begin{example}
Si $\rho_1=(23)(156)$ y $\rho_2=(45)(123)$, entonces 
el lema anterior nos dice que $\rho_2=\sigma\rho_1\sigma^{-1}$ si 
\[
\sigma=\binom{123456}{145623}.
\]
\end{example}

\begin{example}
Los 3-ciclos $\rho_1=(123)$ y $\rho_2=(132)$ son conjugados en $\Sym_3$ pues 
$(123)=\sigma(132)\sigma^{-1}$ si $\sigma=(23)$. Sin embargo, $\rho_1$ y $\rho_2$ 
no son conjugados en $\Alt_3$. 
\end{example}

Estamos en condiciones de probar el teorema del capítulo.

\begin{theorem}[Jordan]
\index{Teorema!de Jordan}
\index{Simplicidad!de $\Alt_n$}
Si $n\geq5$, $\Alt_n$ es simple. 
\end{theorem}

\begin{proof}
Sea $N\ne\{\id\}$ un subgrupo normal de $\Alt_n$. Si $(abc)\in N$, entonces cualquier 3-ciclo también está en $N$ (pues todos los 3-ciclos son conjugados en $\Sym_n$ y por el lema anterior
sabemos que $(ijk)=\mu(abc)\mu^{-1}\in N$ para algún $\mu\in\Alt_n$. Luego $N=\Alt_n$. 

Vamos a demostrar ahora que nuestro $N$ siempre contiene un 3-ciclo. Como $N$ es no trivial, existe $\sigma\in N\setminus\{\id\}$. Sean $m=|\sigma|$ y $p$ un primo tal que divide a $m$. 
Entonces $\tau=\sigma^{m/p}$ tiene orden $p$ y luego $\tau=\rho_1\cdots\rho_s$, donde los $\rho_j$ son $p$-ciclos disjuntos. 

Si $p=2$, entonces $1=\sgn(\tau)=(-1)^s$ y luego $s$ es par. Escribimos 
\[
\tau=(ab)(cd)\rho_3\cdots\rho_s
\]
y entonces, como $\rho_3\cdots\rho_s$ conmuta con $(abc)$ y $(acb)$, 
\[
(ac)(bd)\tau=(abc)\tau(abc)^{-1}\in N
\]
y luego $(ab)(cd)\in N$. Sea $e\in\{1,\dots,n\}\setminus\{a,b,c,d\}$. Entonces
\[
(ae)(bd)=(aec)\underbrace{(ac)(bd)}_{\in N}(aec)^{-1}\in N
\]
y luego 
\[
(aec)=(ac)(ae)=(ac)(bd)(ae)(bd)\in N.
\]

Si $p=3$, podemos suponer sin perder generalidad que $s\geq2$ (pues de lo contrario $\tau$ sería un 3-ciclo). Entonces
$\tau=(abc)(def)\rho_3\cdots\rho_s$. Como $(bcd)$ conmuta con $\rho_3\cdots\rho_s$ y $N$ es normal en $\Alt_n$, entonces
\begin{align*}
&(adbce)=(bcd)\tau(bcd)^{-1}\tau^{-1}\in N
\shortintertext{y luego}
&(adc)=(adb)(adbce)(adb)^{-1}(adbce)^{-1}\in N.
\end{align*}

Si $p>3$, entonces $\tau=(abcd\cdots z)\rho_2\cdots\rho_s$. En particular, $(abc)$ conmuta con $\rho_2\cdots\rho_s$ y entonces
\[
(abd)=(abc)\tau(abc)^{-1}\tau^{-1}\in N.\qedhere
\]
\end{proof}

Como aplicación, vamos a calcular los subgrupos normales de $\Sym_n$ para $n\geq5$. 

\begin{proposition}
\index{Subgrupos normales!de $\Sym_n$}
Sea $n\geq5$ y sea $N$ un subgrupo normal de $\Sym_n$. Entonces $N=\{\id\}$, $N=\Alt_n$ o bien $N=\Sym_n$. 
\end{proposition}

\begin{proof}
Sea $N$ un subgrupo normal de $\Sym_n$. Como la restricción $\sgn|_N$ de la función signo a $N$ es un morfismo de grupos, 
\[
(N:\ker(\sgn|_N))=|\sgn(N)|\in\{1,2\}.
\] 
Si $(N:\ker(\sgn|_N)=1$, entonces $N\subseteq\Alt_n$ y entonces $N$ es normal en $\Alt_n$. 
Luego $N=\{\id\}$ o bien $N=\Alt_n$ pues $\Alt_n$ es un grupo simple si $n\geq5$. 

Si $(N:\ker(\sgn|_N)=2$, entonces $N\cap\Alt_n=\ker(\sgn|_N)$ es normal en $\Alt_n$ y luego, por la simplicidad de $\Alt_n$ para $n\geq5$, $N\cap\Alt_n=\{\id\}$ o bien $N\cap\Alt_n=\Alt_n$. 

En el primer caso, $|N|=2$ y entonces $N$ contiene una permutación impar que además tiene orden dos, digamos
$\tau=(ij)\tau_2\cdots\tau_s$, escrita como producto de trasposiciones disjuntas. Entonces
$\pi=(ik)\tau(ik)\in N$ si $k\not\in\{i,j\}$ y $\pi\ne \tau$ pues $\tau(j)=i$ y $\pi(j)=k$. Luego $|N|\geq3$, una contradicción. 

En el segundo caso, si $N\cap\Alt_n=\Alt_n$, entonces $N=\Alt_n$.
\end{proof}

\chapter{Grupos de automorfismos}

Si $G$ es un grupo y 
$f\colon G\to G$ es un isomorfismo, diremos que $f$ es un automorfismo de $G$. 
La composición de automorfismos de un grupo $G$ es también un automorfismo de $G$. 
Se define entonces el \textbf{grupo de automorfismos} de $G$ como
\[
\Aut(G)=\{f\colon G\to G:f\text{ es un automorfismo de $G$}\}.
\] 
Obviamente $\Aut(G)$ es un grupo con la composición. 

\begin{example}
$\Aut(\Z)\simeq\Z/2$ pues $\Aut(\Z)=\{\id,-\id\}$. 
\end{example}

\begin{example}
Sea $G$ un grupo y sea $g\in G$. La conjugación 
$\gamma_g\colon G\to G$, $x\mapsto gxg^{-1}$, 
por $g$ es un automorfismo de $G$ pues
\[
\gamma_g(xy)=g(xy)g^{-1}=(gxg^{-1})(gyg^{-1})=\gamma_g(x)\gamma_g(y).
\]
Además $\gamma\colon G\to\Aut(G)$, $g\mapsto\gamma_g$, es un morfismo de grupos pues
\[
\gamma_{gh}(x)=(gh)x(gh)^{-1}=g(\gamma_h(x))g^{-1}=\gamma_g(\gamma_h(x))=(\gamma_g\circ\gamma_h)(x).
\]

\index{Grupo!de automorfismos interiores}
El grupo de \textbf{automorfismos interiores} de $G$ se define como 
$\Inn(G)=\gamma(G)$. Observemos que $\ker\gamma=Z(G)$ pues
si $g\in G$ es tal que $\gamma_g=\id$, entonces 
\[
\gamma_g(x)=gxg^{-1}=x
\]
para todo $x\in G$. El primer teorema de isomorfismos implica entonces que
\[
G/Z(G)\simeq \gamma(G)=\Inn(G).
\]
\end{example}

Puede demostrarse que $\Inn(G)$ es un subgrupo normal de $\Aut(G)$. El cociente $\Aut(G)/\Inn(G)$ se conoce como el grupo de \textbf{automorfismos exteriores} de $G$. Observemos que
\[
\Inn(G)\text{ es cíclico}\Longleftrightarrow
|\Inn(G)|=1\Longleftrightarrow
G\text{ es abeliano.}
\]

\begin{example}
	Si $G$ es no abeliano, entonces $\Aut(G)$ no es cíclico.	 En efecto, si $\Aut(G)$ es cíclico,
	entonces $G/Z(G)\simeq\Inn(G)$ es cíclico (por ser subgrupo de un cíclico) y luego
	$G$ es abeliano. 
\end{example}

\begin{exercise}
	Si $G$ es finito, entonces $\Aut(G)$ es finito.	
\end{exercise}

\begin{example}
Veamos que $\Aut(\Sym_3)\simeq\Sym_3$. Sabemos que $Z(\Sym_3)=\{\id\}$. El ejemplo anterior nos permite entonces demostrar que
$\Inn(\Sym_3)\simeq\Sym_3/Z(\Sym_3)\simeq\Sym_3$. Observemos entonces que 
\[
\Inn(\Sym_3)=\{\gamma_g|g\in\Sym_3\}.
\]

Como $\Inn(\Sym_3)\subseteq\Aut(\Sym_3)$, sabemos que $\Aut(\Sym_3)$ tiene al menos seis elementos.
Por otro lado, como $\Sym_3=\langle (12),(13),(23)\rangle$, cada $f\in\Aut(\Sym_3)$ induce una permutación del conjunto $\{(12),(13),(23)\}$ y entonces $|\Aut(\Sym_3)|\leq6$. En conclusión,
\[
\Aut(\Sym_3)=\Inn(\Sym_3)\simeq\Sym_3.
\]   
\end{example}

\begin{example}
Si $p$ es un número primo, entonces
\[
\Aut(\Z/p\times\Z/p)\simeq\GL_2(p).
\] 
En efecto, $\Z/p\times\Z/p$ es un espacio vectorial bidimensional sobre el cuerpo $\Z/p$ y
todo automorfismo del grupo es también una transformación lineal inversible.    
\end{example}

El ejemplo anterior puede generalizarse. 

\begin{exercise}
Si $p$ es un primo, $C_p$ es el grupo cíclico de orden $p$ y $G=C_p\times\cdots\times C_p$ ($n$-veces), demuestre que
$\Aut(G)\simeq\GL_n(p)$.
%\underbrace{\Z/p\times\cdots\cdots\Z/p}_{\text{$n$-veces}})\simeq\GL_n(p).
%Si $G=\Z/p\times\cdots\times\Z/p=(\Z/p)^n$ y 
%$f\in\Aut(G)$, escribimos 
%\[
%f(e_j)=\sum_{i=1}^n a_{ij}e_i
%\]	
%para $a_{ij}\in\Z/p$ y $j\in\{1,\dots,n\}$. Definimos $\alpha\colon\Aut(G)\to\GL_n(p)$, $f\mapsto (a_{ij})$, 
%y vemos que $\alpha$ es un isomorfismo.
\end{exercise}

\begin{example}
Vamos a demostrar que 
\[
\Aut(\Z/n)\simeq\mathcal{U}(\Z/n)=\{m+n\Z:\gcd(n,m)=1\}.
\]
Sea $G=\langle g\rangle\simeq\Z/n$. Si $\alpha\in\Aut(G)$, entonces $\alpha(g)$ es algún generador del grupo $G$, es decir $|\alpha(g)|=n$. En particular, $\alpha(g)=g^m$ para algún $m$. Vimos en el capítulo~\ref{orden} que 
\[
|g^m|=\frac{n}{\gcd(n,m)}.
\]
Como consecuencia, los generadores de $G$ serán los elementos de la forma $g^m$ con $m$ tal que $\gcd(n,m)=1$. La función
\[
f\colon \Aut(G)\to\mathcal{U}(\Z/n),\quad
\alpha\mapsto m,
\]
donde $m$ es tal que $\alpha(g)=g^m$, es un morfismo de grupos: si $\alpha,\beta\in\Aut(G)$, digamos $\alpha(g)=g^m$ y $\beta(g)=g^t$, entonces 
\[
\alpha(\beta(g))=\alpha(g^t)=(g^t)^m=g^{tm},
\]
es decir $f(\alpha\circ\beta)=f(\alpha)f(\beta)$. Además puede demostrarse que $f$ no depende del generador $g$ pues si $G=\langle g_1\rangle$, entonces $g_1=g^i$ para algún $i$ y luego 
\[
\alpha(g_1)=\alpha(g^i)=\alpha(g)^i=(g^m)^i=g^{mi}=(g^i)^m=g_1^m.
\] 
Dejamos como ejercicio verificar que $f$ es biyectiva. 
\end{example}

Veamos algunos ejemplos concretos del resultado anterior. 

\begin{example}
$\Aut(\Z/8)\simeq\mathcal{U}(\Z/8)=\{1,3,5,7\}=\langle 3,5\rangle\simeq\Z/2\times\Z/2$.
\end{example}

El ejemplo siguiente es bastante más difícil. Vamos a demostrar que si $p$ es un número primo, entonces
$\Aut(\Z/p)$ es cíclico y tiene orden $p-1$. Vamos a necesitar el siguiente resultado auxiliar, que resulta ser de interés incluso en otros contextos.  

\begin{lemma}
Sean $G$ un grupo finito y abeliano y $n=\max\{|g|:g\in G\}$. Si $x\in G$, entonces $|x|$ divide a $n$. 
\end{lemma}

\begin{proof}
Sean $g\in G$ tal que $n=|g|$, $x\in G$ y $m=|x|$. Queremos ver que $m$ divide a $n$. Supongamos que $m$ no divide a $n$. Existe entonces algún número primo $p$ 
tal que $n=p^\alpha n_1$ y $m=p^\beta m_1$, donde $\gcd(p,n_1)=\gcd(p,m_1)=1$ y $\beta>\alpha$. Sabemos que
\[
|g^	{p^\alpha}|=\frac{n}{p^\alpha}
\]
no es divisible por $p$ y además
\[
|x^{\frac{m}{p^\beta}}|=p^\beta.
\]
Como $n/p^\alpha$ y $p^\beta$ son coprimos y $G$ es abeliano, 
\[
|g^{p^\alpha}x^{\frac{m}{p^\beta}}|=np^{\beta-\alpha}>n,
\]
una contradicción a la maximalidad de $n$. 
\end{proof}

Necesitamos otro resultado auxiliar, nuevamente de gran interés no solamente en este contexto. 

\begin{lemma}
\label{lem:X^n-1}
Sea $K$ un cuerpo. Si $f\in K[X]$ es un polinomio de grado $n$, entonces $f$ tiene a lo sumo $n$ raíces distintas.
\end{lemma}

\begin{proof}
Procederemos por inducción en $n$. Si $n=1$, el resultado es trivial. Supongamos entonces que el lema es válido para polinomios de grado $n-1$ y sea $f\in K[X]$. Si $f$ no tiene raíces en $K$, no hay nada para demostrar. Si, en cambio, $\alpha$ es una raíz de $f$, entonces
\[
f=(X-\alpha)q
\]
para un cierto $q\in K[X]$ de grado $n-1$. Si $\beta\ne\alpha$ es otra raíz de $f$, entonces $0=f(\beta)=(\beta-\alpha)q(\beta)$ y luego $q(\beta)=0$, es decir $\beta$ es raíz de $q$. Por hipótesis inductiva, el polinomio $q$ tiene a lo sumo $n-1$ raíces distintas. En consecuencia, $f$ tiene a lo sumo $n$ raíces distintas.   
\end{proof}

Ahora sí estamos en condiciones de demostrar el siguiente resultado. 

\begin{theorem}
\index{Unidades!de $\Z/p$}
Si $p$ es un número primo, entonces $\mathcal{U}(\Z/p)$ es cíclico de orden $p-1$. 
\end{theorem}

\begin{proof}
Sabemos que $\Aut(\Z/p)$ es un grupo abeliano. Sea 
\[
n=\max\{|g|:g\in\mathcal{U}(\Z/p)\}.
\] 

Veamos que $n=p-1$. Como $|\mathcal{U}(\Z/p)|=\varphi(p)=p-1$, tenemos $n\leq p-1$. Por otro lado, como gracias al lema anterior sabemos que el polinomio $X^n-1$ tiene a lo sumo $n$ soluciones, obtenemos $p-1\leq n$. Luego $n=p-1$. En particular, esto demuestra que $\mathcal{U}(\Z/p)$ es cíclico ya que contiene al menos un elemento de orden $p-1$.  
\end{proof}

Veamos otra aplicación importante de los resultados auxiliares que utilizamos para demostrar el teorema anterior. Primero, un lema, que bien podría quedar como ejercicio.

\begin{lemma}
Sea $G$ un grupo abeliano. Si $G$ tiene elementos de órdenes $k$ y $l$, entonces $G$ tiene un elemento de orden $\lcm(k,l)$. 
\end{lemma}

\begin{proof}
Sean $g,h\in G$ tales que $|g|=k$ y $|h|=l$. Sea 
$m=|gh|$. Si $k$ y $l$ son coprimos, 
el resultado fue demostrado en el corolario~\ref{cor:ordenes_coprimos} en la página~\pageref{cor:ordenes_coprimos} como aplicación del teorema de Lagrange. 
%entonces $m=kl$. En efecto, como $g$ y $h$ conmutan, 
%\[
%(gh)^{kl}=(g^{k})^l(h^l)^k=1,
%\]
%y entonces $m$ divide a $kl$. Por otro lado, $1=(gh)^m=g^mh^m$ y luego  
%\[
%g^m=h^{-m}\in\langle g\rangle\cap\langle h\rangle.
%\]
%Por el teorema de Lagrange $|\langle g\rangle\cap\langle h\rangle|=1$ pues el orden del subgrupo $%\langle g\rangle\cap\langle h\rangle$ divide simultáneamente a $k$ y a $l$, que son coprimos.  
Supongamos entonces que $d=\gcd(k,l)>1$. Escribimos
\begin{align*}
k&=p_1^{\alpha_1}\cdots p_r^{\alpha_r} p_{r+1}^{\alpha_{r+1}}\cdots p_s^{\alpha_s},\\
l&=p_1^{\beta_1}\cdots p_r^{\beta_r} p_{r+1}^{\beta_{r+1}}\cdots p_s^{\beta_s},
\end{align*}
donde los primos $p_1,\dots,p_s$ son todos distintos, $0\leq\alpha_j<\beta_j$ para todo $j\in\{1,\dots,r\}$ y $\alpha_j\geq\beta_j\geq0$ para todo $j\in\{r+1,\dots,s\}$. Sean
\[
x=g^{p_1^{\alpha_1}\cdots p_s^{\alpha_s}},
\quad
y=h^{p_{r+1}^{\beta_{r+1}}\cdots p_s^{\beta_s}}.
\]
Como $|x|$ y $|y|$ son coprimos, se concluye que
$|xy|=|x||y|=m$.
\end{proof}

Antes de demostrar el teorema, veamos un ejemplo que ilustra qué pasa en el lema anterior.

\begin{example}
Vamos a calcular el orden de $(8,8)\in(\Z/10)\times(\Z/80)$. Primero observamos que 
$8\in\Z/10$ tiene orden $10/\gcd(8,10)=10/2=5$ y que $8\in\Z/80$ tiene orden
$80/\gcd(8,80)=80/8=10$. Tenemos entonces que $g=(8,0)$ tiene orden 5 y $h=(0,8)$ tiene orden 10. La prueba del lema anterior nos dice que
el elemento $gh^2$ tendrá orden $\lcm(5,10)=10$.   
\end{example}

Ahora sí, el teorema. 

\begin{theorem}
\index{Subgrupos!finitos de $K^\times$} 
Sea $K$ un cuerpo. Si $G$ es un subgrupo finito de $K^\times=K\setminus\{0\}$, entonces $G$ es cíclico. En particular, si $K$ es un cuerpo finito, entonces $K^\times$ es cíclico.  
\end{theorem}

\begin{proof}
Sea $g\in G$ de orden maximal, digamos $n=|g|$. Vamos a demostrar que $G=\langle g\rangle$. Si eso no fuera cierto, 
sea $h\in G\setminus\langle g\rangle$. Sabemos que $k=|h|\leq n$. Si $k=n$, entonces los $n+1$ elementos
\[
1,g,g^2,\dots,g^{n-1},h
\]
son raíces distintas del polinomio $X^n-1$, una contradicción al lema~\ref{lem:X^n-1}.  
Luego $k<n$. Observemos ahora que $k$ divide a $n$ pues, de lo contrario, 
como $G$ es abeliano, tendríamos en $G$ un elemento de orden 
$\lcm(k,n)>n$, una contradicción a la maximalidad de $n$. Como $k$ divide $n$, tenemos 
también los $n+1$ elementos 
\[
1,g^{n/k},g^{2n/k},\dots,g^{(k-1)n/k}
\]
son raíces distintas de $X^n-1$, una contradicción al lema~\ref{lem:X^n-1}. 
\end{proof}

Terminamos el capítulo con algunos ejercicios sobre grupos de automorfismos.

\begin{exercise}
\label{xca:autgeq2}
	Si $G$ es un grupo tal que $|G|\geq3$, entonces $|\Aut(G)|\geq2$. 	
\end{exercise}


\begin{exercise}
\label{xca:aut_impar}
	No existe un grupo finito cuyo grupo de automorfismos sea no trivial cíclico y de orden impar.	
\end{exercise}

\begin{exercise}
\label{xca:p2dividesAut}
	Sea $p$ un número primo.
	Si $G$ es un $p$-grupo no abeliano, entonces $p^2$ divide a $|\Aut(G)|$. 	
\end{exercise}

\begin{exercise}
	\label{xca:Aut(H)}
	Si $G=H\times K$, entonces $\Aut(H)$ es isomorfo a un subgrupo de $\Aut(G)$. 	
\end{exercise}

\begin{exercise}
\label{xca:aut_trivial_center}
	Si $G$ tiene centro trivial, entonces $\Aut(G)$ también. 	
\end{exercise}






\chapter{Producto semidirecto}

Primero comenzaremos con una descripción alternativa del producto directo de dos grupos que vimos en el capítulo~\ref{grupos}. 

\begin{theorem}
\index{Producto!directo de grupos}
Sea $G$ un grupo y sean $H$ y $K$ subgrupos normales de $G$. Si $G=HK$ y $H\cap K=\{1\}$, entonces $G\simeq H\times K$.
\end{theorem}

\begin{proof}
Sea $f\colon G\to H\times K$, $f(g)=(h,k)$, donde $h\in H$ y $k\in K$ son únicos tales que $g=hk$. Esto tiene sentido pues si $g\in G$ entonces $g=hk$ para algún $h\in H$ y $k\in K$; si además $g=h_1h_1$ para $h_1\in H$ y $k_1\in K$, entonces, como $hk=h_1k_1$, se tiene que $h_1^{-1}h=k_1k^{-1}\in H\cap K=\{1\}$ y luego $h=h_1$ y $k=k_1$.

Veamos que si $g=hk$ y $g_1=h_1k_1$ para $h,h_1\in H$ y $k,k_1\in K$, entonces $kh_1=h_1k$. En efecto, $[k,h_1]=kh_1k^{-1}h_1^{-1}\in H\cap K=\{1\}$ pues la normalidad de $H$ y $K$ implican que $ kh_1k^{-1}\in H$ y $h_1k^{-1}h_1^{-1}\in K$.

La observación anterior nos permite demostrar que $f$ es un morfismo de grupos. Si $g=hk$ y $g_1=h_1k_1$ con $h,h_1\in H$ y $k,k_1\in K$, entonces, como $f(g)=(h,k)$ y $f(g_1)=(h_1,k_1)$, tenemos que
\[
f(gg_1)=f((hk)(h_1k_1))=f(h(kh_1)k_1)=f((hh_1)(kk_1))=(hh_1,kk_1).
\]
Queda como ejercicio demostrar que $f$ es biyectiva. 
\end{proof}

\index{Factorización exacta!de grupos}
Un grupo $G$ se dice que admite una factorización exacta mediante los subgrupos $H$ y $K$ si $G=HK$ y adenás $H\cap K=\{1\}$. 
El teorema anterior puede entonces enunciarse con la siguiente terminología: Si un grupo admite una factorización exacta mediante dos subgrupos normales, entonces es isomorfo al producto directo de esos subgrupos. 

\begin{example}
Sea $G=\Sym_3$ y sean $H=\langle (123)\rangle\unlhd G$ y $K=\langle (12)\rangle$. Observemos que $K$ no es normal en $G$, no podemos utilizar el teorema anterior. Tenemos $G=HK$ y $H\cap K=\{\id\}$, pero $H\times K\simeq\Z/3\times\Z/2\not\simeq\Sym_3$ pues $\Z/3\times\Z/2$ es un grupo abeliano y $\Sym_3$ no lo es.
\end{example}

Mencionamos a continuación un corolario sencillo. La demostración quedará como ejercicio. 

\begin{corollary}
Sean $A$ un subgrupo normal de $H$ y $B$ un subgrupo normal de $K$. Entonces $A\times B$ es un subgrupo normal de $H\times K$ y vale además que
\[
\frac{H\times K}{A\times B}\simeq(H/A)\times(K/B).
\]	
\end{corollary}

\begin{proof}[Bosquejo de la demostración]
Sea $\varphi\colon H\times K\to(H/A)\times(K/B)$, $\varphi(h,k)=(hA,kB)$. Dejamos como ejercicio verficar que $\varphi$ es un isomorfismo de grupos tal que $\ker\varphi=A\times B$. Al aplicar el primer teorema de isomorfismos tendremos entonces el resultado deseado. 
\end{proof}

Veremos a continuación qué pasa cuando solamente uno de los factores es normal. Nos encontraremos con un grupo que admite una factorización exacta donde uno de los subgrupos es normal.  

\begin{definition}
\index{Complemento}
Sea $G$ un grupo y sean $K$ un subgrupo normal de $G$ y $Q$ un subgrupo de $G$. Diremos que $Q$ es un \textbf{complemento} de $K$ en $G$ si $K\cap Q=\{1\}$ y $G=KQ$.
\end{definition} 

\begin{example}
Sea $G=\Sym_3$ y sea $K=\langle (123)\rangle\unlhd G$. Los subgrupos $\langle (12)\rangle$, $\langle (13)\rangle$ y $\langle (23)\rangle$ son complementos de $K$ en $G$. 
\end{example}

El ejemplo anterior nos muestra que los complementos no son únicos. Sin embargo, sí son únicos salvo isomorfismos pues cualquier complemento será isomorfo a $G/K$. En efecto, gracias a los teoremas de isomorfismo, 
\[
G/K\simeq KQ/K\simeq Q/K\cap Q=Q/\{1\}\simeq Q.
\] 

\begin{definition}
\index{Producto!semidirecto de grupos}
Diremos que un grupo $G$ es un \textbf{producto semidirecto} de $Q$ en $K$ si $K$ es normal en $G$ y además $K$ admite un complemento en $G$ isomorfo a $Q$. La notación que utilizaremos será $G=K\rtimes Q$
\end{definition}

Veamos algunas caracterizaciones del producto semidirecto. 

\begin{proposition}
Sea $K$ un subgrupo normal de $G$. Las siguientes afirmaciones son equivalentes:
\begin{enumerate}
\item $K$ admite un complemento en $G$.
\item Existe 	un subgrupo $Q$ de $G$ tal que cada $g\in G$ se escribe unívocamente como $g=xy$ con $x\in K$ e $y\in Q$.
\item Existe un morfismo $s\colon G/K\to G$ tal que $\pi\circ s=\id_{G/K}$, donde $\pi\colon G\to G/K$, $g\mapsto Kg$,' es el morfismo canónico.
\item Existe un morfismo $\rho\colon G\to G$ tal que $\ker\rho=K$ y la restricción $\rho|_{\rho(G)}$ es igual a la identidad.  
\end{enumerate}
\end{proposition}

\begin{proof}
Veamos que $(1)\implies(2)$. Si $Q$ es un complemento de $K$, entonces $G=KQ$ y $K\cap Q=\{1\}$. En particular, si $g\in G$, entonces $g=xy$ para $x\in K$ e $y\in Q$. Y la escritura es única pues si además $g=x_1y_1$ con $x_1\in K$ e $y_1\in Q$, entonces $x_1^{-1}x=yy_1^{-1}\in K\cap Q=\{1\}$ y luego $x=x_1$ y también $y=y_1$.  

Veamos que $(2)\implies(3)$. Sea $s\colon G/K\to G$, $s(Kg)=y$ si $g=xy$ con $x\in K$ e $y\in Q$. (Es importante observar que 
acá, para definir $s$, nos es conveniente utilizar coclases a derecha.) Veamos que $s$ está bien definida. Para eso, tenemos que ver que si $Kg=Kg_1$, entonces $s(Kg)=s(Kg_1)$. Si escribimos $g=xy$ y $g_1=x_1y_1$ con $x,x_1\in K$ e $y,y_1\in Q$, entonces
Como $Kg=Kg_1$, sabemos que $xyy_1^{-1}x_1^{-1}=gg_1^{-1}\in K$, es decir $yy_1^{-1}\in x^{-1}Kx_1=K$ pues $x,x_1\in K$. Luego $yy_1^{-1}\in K\cap Q=\{1\}$ y entonces $y=y_1$. Veamos ahora que $\pi\circ s=\id_{G/K}$. Si $g=xy$ con $x\in K$ e $y\in Q$, entonces  
$(\pi\circ s)(Kg)=\pi(y)=Ky=Kxy=Kg$.

Veamos ahora que $(3)\implies(4)$. Sea $\rho=s\circ\pi$. Es claro que $\rho$ es un morfismo, pues es composición de morfismos. Calculamos:
\[
\rho(\rho(g))=\rho( (s\circ\pi)(g))=\rho(s(Kg))=((s\circ\pi)\circ s)(Kg)=s(Kg)=\rho(g).
\]
Por último, calculamos $\ker\rho$. Si $g\in\ker\rho$, entonces $s(\pi(g))=\rho(g)=1$. Luego
\[
\pi(g)=\pi(s(\pi(g)))=\pi(1)=1_{G/K},
\]  
es decir $g\in\ker\pi=K$. 

Por último, demostremos que $(4)\implies(1)$. Afirmamos que $Q=\rho(G)$ es un complemento para $K$ en $G$. Veamos primero que $K\cap Q=\{1\}$: si $x\in K\cap Q$, entonces $x=\rho(g)$ para algún $g\in G$ y además 
\[
1=\rho(x)=\rho(\rho(g))=\rho(g).
\]
Luego $g\in\ker\rho=K$ y entonces $x=1$. Veamos ahora que $G=KQ$. Para demostrar que $G\subseteq KQ$ observamos que
\[
g=(g\rho(g^{-1}))\rho(g)
\]
y que $g\rho(g^{-1})\in K=\ker\rho$ pues $\rho(g\rho(g^{-1}))=	\rho(g)\rho(g^{-1})=1$.  
\end{proof}

\begin{example}
$\Sym_n=\Alt_n\rtimes\Z/2$ pues $Q=\langle (12)\rangle\simeq\Z/2$ es un complemento para el subgrupo normal $\Alt_n$ de $\Sym_n$. 
\end{example}

La siguiente proposición permite construir productos semidirectos. La demostración quedará como ejercicio. 

\begin{proposition}
Sean $K$ y $Q$ grupos y sea $\theta\colon Q\to\Aut(K)$, $x\mapsto\theta_x$, un morfismo de grupos. 
El conjunto $K\times Q$ con la operación
\[
(a,x)(b,y)=(a\theta_x(b),xy)
\]
es un grupo. Este grupo será denotado por $K\rtimes_\theta Q$. 
\end{proposition}

\begin{proof}[Bosquejo de la demostración]
Dejamos como ejercicio verificar que la operación es asociativa. Hay que verificar además que el elemento neutro de $K\rtimes_\theta Q$ será $(1,1)$ y
que el inverso de $(a,x)\in K\rtimes_\theta Q$ será $(\theta_{x^-1}(a^{-1}),x^{-1})$. 
\end{proof}

El grupo que construimos en la proposición anterior es, de hecho, un producto semidirecto. En efecto, es un producto semidirecto de los subgrupos  
\begin{align*}
K\times\{1\}=\{(a,1):a\in K\}\simeq K,&&
\{1\}\times Q=\{(1,x):x\in Q\}\simeq Q
\end{align*}
de $K\rtimes_\theta Q$. Observar que $K\times\{1\}$ es normal en $K\rtimes_\theta Q$. Es importante remarcar que si identificamos al subgrupo normal $K\rtimes\{1\}$ con $K$ y al subgrupo  
$\{1\}\rtimes Q$ con $Q$, podemos escribir 
\[
\theta_x(a)=xax^{-1}
\]
para todo $x\in Q$ y $a\in K$.

%\begin{proposition}
%Sean $K$ y $Q$ dos grupos y sea $\theta\colon Q\to\Aut(K)$ un morfismo de grupos. Entonces $K\rtimes_\theta Q$ es un 
%producto semidirecto tal que
%\[
%(\theta_x(a),1)=(1,x)(a,1)(1,x)^{-1}.
%\]
%\end{proposition}
%
%\begin{proof}[Bosquejo de la demostración]
%Sea $\pi\colon K\rtimes_\theta Q\to Q$, $\pi(a,x)=x$. Entonces $\pi$ es un morfismo sobreyectivo. Como
%\begin{align*} 
%\ker\pi&=\{(a,1):a\in K\}\simeq K,\\
%\{1\}\times Q&=\{(1,x):x\in Q\}\simeq Q,
%\end{align*}
%podemos identificar estos grupos con $K$ y $Q$, respectivamente. Luego, gracias a esta identificación,  
%$G=(\ker\pi)\rtimes (\{1\}\times Q)=K\rtimes Q$.  %y además $\theta_x(a)=xax^{-1}$.   
%\end{proof}
%
% explicar

\begin{proposition}
Si $G$ es un producto semidirecto del subgrupo normal $K$ con el subgrupo $Q$, existe un morfismo de grupos $\theta\colon Q\to\Aut(K)$
tal que $G\simeq K\rtimes_\theta Q$.  
\end{proposition}

\begin{proof}[Bosquejo de la demostración]
Para $x\in Q$ sea $\theta_x\colon K\to K$, $\theta_x(a)=xax^{-1}$. Ya vimos que $\theta_x\in\Aut(K)$ y que $Q\to\Aut(K)$, $x\mapsto\theta_x$ es un morfismo de grupos. Queda verificar que 
la función $K\rtimes_\theta Q\to G$, $(a,x)\mapsto ax$, es un morfismo biyectivo de grupos. 
\end{proof}

Veamos algunos ejemplos.

\begin{example}
Sean $N\simeq \Z/n$ y $H\simeq\Z/2=\{0,1\}$. La función $\theta\colon H\to\Aut(N)$, $1\mapsto (x\mapsto x^{-1})$, es un morfismo de grupos. Sea $G=N\rtimes_\theta H$. 
Entonces $G\simeq\D_n$, el grupo diedral de orden $2n$. 

Recordemos que 
\[
\D_n=\langle r,s:r^n=s^2=1, srs^{-1}=r^{-1}\rangle.
\]
Supongamos que $N=\langle x\rangle$ y que $H=\langle y\rangle$. Entonces $|(x,1)|=n$ y $|(1,y)|=2$. Además
\begin{align*}
(1,y)(x,1)(1,y)^{-1} &= (\varphi_y(x),y)(1,y)=(\varphi_y(x),y^2)\\
&=(\varphi_y(x),1)=(x^{-1},1)=(x,1)^{-1}.
\end{align*}
Si $u=(x,1)$ y $v=(1,y)$, entonces $u^n=v^2=(1,1)$ y además $vuv^{-1}=u^{-1}$. Esto significa que existe un morfismo de grupos
$\D_n\to G$ que además es sobreyectivo (pues $G$ está generado por $u$ y $v$). Además $|G|=|N||H|=2n$, luego $G$ también tiene orden $2n$ y en consecuencia $G\simeq\D_n$. 
\end{example}

\begin{example}
Sea $K=\{\id,(12)(34),(13)(24),(14)(23)\}$, que sabemos es normal en $\Alt_4$ y sea $H=\langle (123)\rangle\simeq\Z/3$. Como $K\cap H$ es un subgrupo de $H$ y $K$ y además
los órdenes de $K$ y $H$ son coprimos, $H\cap K=\{\id\}$. Luego $\Alt_4=K\rtimes H$. 	
\end{example}

\begin{example}
Tal como en el ejercicio anterior, sea $K$ el subgrupo de Klein de $\Sym_4$. 
Sea $H=\{\sigma\in\Sym_4:\sigma(4)=4\}$, que es un subgrupo de $\Sym_4$ isomorfo a $\Sym_3$. Simplemente al observar los elementos vemos que $H\cap K=\{\id\}$ y luego
$\Sym_4=K\rtimes H$.   	
\end{example}

\begin{example}
Si $n\geq5$, $\Alt_n$ no es un producto semidirecto de subgrupos propios (pues $\Alt_n$ es simple para todo $n\geq5$). 	
\end{example}

\begin{example}
Sea $K\simeq\Z/3$ y sea $Q=\Z/4$. Como $\Hom(Q,\Aut(K))=\{1,\tau\}$, donde 
\[
\tau\colon\Z/4\to\Aut(\Z/3)=\{\id,\rho\}\simeq\Z/2,\quad 1\mapsto\rho,
\]
el producto semidirecto $T=K\rtimes_t Q$ es un grupo no abeliano de orden 12. Además $T\not\simeq\Alt_4$ pues, por ejemplo, $|(2,2)|=6$ y sabemos que en $\Alt_4$ no existen elementos de orden seis. 
\end{example}



% todo: El grupo Aff(R) es un producto semidirecto
% todo: Necesitamos más ejemplos de producto semidirecto

\chapter{Acciones}

\begin{definition}
\index{Acción!de un grupo en un conjunto}
	Sean $G$ un grupo y $X$ un conjunto. Una acción (a izquierda) de $G$ en $X$ es una función
	$G\times X\to X$, $(g,x)\mapsto g\cdot x$, tal que
	\begin{enumerate}
		\item $1\cdot x=x$ para todo $x\in X$, y 
		\item $g\cdot (h\cdot x)=(gh)\cdot x$ para todo $g,h\ n G$ y $x\in X$.
	\end{enumerate}
\end{definition}

Cuando un grupo $G$ actúa en un conjunto $X$, se dice también que $X$ es un $G$-conjunto. 

\begin{example}
\index{Acción!trivial}
Todo grupo $G$ actúa en $G$ trivialmente: $g\cdot h=h$ para todo $g,h\in G$.	
\end{example}

\begin{example}
\index{Acción!por multiplicación a izquierda}
Todo grupo $G$ actúa en $G$ por multplicación a izquierda, es decir $g\cdot h=gh$ para todo $g,h\in G$.	
\end{example}

\begin{example}
Todo grupo $G$ actúa en $G$ por $g\cdot h=hg^{-1}$ para todo $g,h\in G$.	
\end{example}

\begin{example}
\index{Acción!por conjugación}
Si $N$ es un subgrupo normal de $G$, entonces $G$ actúa en $N$ por conjugación, es decir $g\cdot x=gxg^{-1}$ para todo $g\in G$ y $x\in N$. En particular, 
todo grupo $G$ actúa en $G$ por conjugación.		
\end{example}

\begin{example}
\index{Acción!en el conjunto de coclases}
Sea $G$ un grupo y sea $H$ un subgrupo de $G$. Entonces $G$ actúa en el 
conjunto de coclases $G/H$ por multiplicación a izquierda, es decir
$g\cdot (xH)=(gx)H$ para todo $g,x\in G$. 
\end{example}

Toda acción a izquierda de $G$ en $X$ se corresponde biyectivamente con un morfismo 
$\rho\colon G\to\Sym_X$. La correspondencia está dada por la fórmula
\[
\rho(g)(x)=g\cdot x,\quad g\in G,x\in X.
\]
Para simplificar, utilizaremos la notación $\rho_g=\rho(g)$. 

Como ejemplo veamos que si $G\times X\to X$, $(g,x)\mapsto x$, es una acción de $G$ en $X$, entonces
cada $\rho_g\colon X\to X$ es una función biyectiva con inversa $(\rho_g)^{-1}=\rho_{g^{-1}}$. Además $\rho$ es un morfismo de grupos pues
\[
\rho(gh)(x)=(gh)\cdot x=g\cdot (h\cdot x)=\rho_g(h\cdot x)=\rho_g(\rho_h(x))
\]
para todo $g,h\in G$ y todo $x\in X$. 

\begin{example}
Sea $G=\Sym_3$ y sea $H=\langle (123)\rangle=\{\id,(123),(132)\}$. Si hacemos actuar al grupo $G$ en el conjunto $X=G/H=\{H,(12)H\}$ 
por multiplicación a izquierda, y escribimos  
$x_1=H$ y $x_2=(12)H$, tenemos entonces
\begin{align*}
&(12)\cdot x_1=x_2,
&&(12)\cdot x_2=x_1,
&&(123)\cdot x_1=x_1,
&&(123)\cdot x_2=x_2.
\end{align*}
Como $G=\langle (12),(123)\rangle$, queda definido el morfismo 
$\rho\colon G\to\Sym_{X}\simeq\Sym_2$ de la siguiente forma: $(12)\mapsto (12)$, $(123)\mapsto\id$. 
\end{example}

\begin{example}
Sea $G=\Sym_3$ y sea $H=\langle (12)\rangle=\{\id,(12)\}$. Si hacemos actuar a $G$ en $X=G/H=\{H,(123)H,(132)H\}$ por multiplicación a izquierda y 
escribimos $x_1=H$, $x_2=(123)H$ y $x_3=(132)H$, entonces
\begin{align*}
(12)\cdot x_1=x_1,&& (12)\cdot x_2=x_3, && (12)\cdot x_3=x_2,\\
(123)\cdot x_1=x_2, && (123)\cdot x_2=x_3, &&(123)\cdot x_3=x_1.
\end{align*}
Como $G=\langle (12),(123)\rangle$, queda definido el morfismo 
$\rho\colon G\to\Sym_{X}\simeq\Sym_3$ de la siguiente forma: $(12)\mapsto (23)$, $(123)\mapsto (123)$. 
\end{example}

\begin{example}
Sea $G=Q_8=\{1,-1,i,-i,j,-j,k,-k\}$ y sea $N=\{1,-1,i,-i\}$. Como $N$ es normal en $G$, el grupo $G$ actúa por conjugación en $X=N$. 
Si $x_1=1$, $x_2=-1$, $x_3=i$ y $x_4=-i$, entonces $i\cdot x=x$ para todo $x\in N$ y además 
\begin{align*}
j\cdot x_1=x_1, && j\cdot x_2=x_2, && j\cdot x_3=x_4, && j\cdot x_4=x_3. 
\end{align*}
Como $G=\langle i,j\rangle$, el morfismo $\rho\colon G\to\Sym_X\simeq\Sym_4$ queda determinado por
$\rho_i=\id$ y $\rho_j=(34)$. 
\end{example}

El ejemplo siguiente es particularmente importante, ya que suele generar confusión.

\begin{example}
    El grupo $\Sym_n$ actúa en $\R^n$ por 
    \[
    \sigma\cdot (x_1,\dots,x_n)=(x_{\sigma^{-1}(1)},\dots,x_{\sigma^{-1}(n)}).
    \]
    Es muy importante remarcar que debe usarse $\sigma^{-1}$ en la definición y no $\sigma$, ya que lo que queremos
    es permutar los elementos de la base canónica de $\R^3$. 
    
    Como ejemplo, veamos que la operación 
    $\sigma\cdot (x_1,x_2,x_3)=(x_{\sigma(1)},x_{\sigma(2)},x_{\sigma(3)})$ 
    no define una acción de $\Sym_3$ en $\R^3$. 
    Si $\sigma=(12)$ y $\tau=(23)$, entonces $\sigma\tau=(123)$. Como 
    \begin{align*}
    &(123)\cdot (5,6,7)=(6,7,5),\\
    &(12)\cdot ((23)\cdot (5,6,7))=(1,2)\cdot (5,7,6)=(7,5,6),
    \end{align*}
    no tenemos una acción. En general, no se tiene una acción porque 
    si calculamos
    \begin{align*}
        \sigma\cdot (\tau\cdot (x_1,\dots,x_n))
        =\sigma\cdot (x_{\tau(1)},\dots,x_{\tau(n)})
    \end{align*}
    y para cada $i\in\{1,\dots,n\}$ hacemos $y_i=x_{\tau(i)}$, entonces
    \[
    \sigma\cdot (\tau\cdot (x_1,\dots,x_n))=\sigma\cdot (y_1,\dots,y_n)=(y_{\sigma(1)},\dots,y_{\sigma(n)})
    =(x_{\tau\sigma(1)},\dots,x_{\tau\sigma(n)}),
    \]
    aunque $\sigma$ y $\tau$ no conmuten. 
    %Tenemos que hacer el cambio de variables porque los elementos vienen ordenados.
    
    Veamos que se tiene una acción si usamos el inverso. 
    Para cada $j\in\{1,\dots,n\}$ sea $y_j=x_{\tau(j)}$, 
    es decir 
    \[
    (y_1,y_2,\dots,y_n)=\tau\cdot (x_1,x_2,\cdots,x_n)=(x_{\tau^{-1}(1)},x_{\tau^{-1}(2)},\dots,x_{\tau^{-1}(n)}).
    \]
    Calculamos entonces 
    \begin{align*}
        \sigma\cdot (\tau\cdot (x_1,x_2,\dots,x_n))&=\sigma\cdot (y_1,y_2,\dots,y_n)\\
        &=\left(y_{\sigma^{-1}(1)},y_{\sigma^{-1}(2)},\dots,y_{\sigma^{-1}(n)}\right)\\
        &=\left(x_{\tau^{-1}(\sigma^{-1}(1))},x_{\tau^{-1}(\sigma^{-1}(2))},\dots,x_{\tau^{-1}(\sigma^{-1}(n))}\right)\\
        &=\left(x_{(\sigma\tau)^{-1}(1))},x_{(\sigma\tau)^{-1}(2))},\dots,x_{(\sigma\tau)^{-1}(n))}\right).
    \end{align*}
\end{example}

El ejemplo anterior y el siguiente están relacionados.

\begin{example}
    El grupo simétrico $\Sym_n$ actúa en el conjunto de polinomios de $n$ variables $X_1,\dots,X_n$
    permutando las variables. Por ejemplo, en el caso de tres variables, si 
    $\sigma=(123)$ y $f=X_2X_3-X_1+5X_2X_3^2X_1$, entonces
    $\sigma\cdot f=X_2^2X_3-X_1+5X_2X_3^2X_1$. 
    
    Al restringir la acción, vemos que 
    $\Sym_n$ actúa en el conjunto 
    \[
    \{\lambda_1X_1+\cdots\lambda_nX_n:\lambda_1,\dots,\lambda_n\in\R\}.
    \]
    Podemos escribir entonces
    \begin{align*}
    \sigma \cdot (\lambda_1X_1+\cdots+\lambda_nX_n) &= (\lambda_1X_{\sigma(1)}+\cdots+\lambda_nX_{\sigma(n)})
    =(\lambda_{\sigma(1)}X_1+\cdots+\lambda_{\sigma(n)}X_n)
    \end{align*}
    y vemos cuál es la relación que tiene esta acción con la que vimos en el ejemplo anterior.
\end{example}

Es importante poder calcular el núcleo de la acción. 

\begin{example}
Sea $G$ un grupo y sea $H$ un subgrupo de $G$. Entonces $G$ actúa en el conjunto de coclases $G/H$ por multiplicación a izquierda, es decir
$g\cdot (xH)=(gx)H$ para todo $g,x\in G$. Sea $\rho\colon G\to\Sym_{G/H}$ el morfismo inducido por la acción.

 Veamos que $\ker\rho=\cap_{x\in G}xHx^{-1}$. Demostremos primero $\supseteq$. Si $g\in xHx^{-1}$ para todo $x\in G$, entonces
 fijado $x\in G$, 
 \[
 \rho(g)(xH)=g\cdot (xH)=(gx)H=(xhx^{-1})xH=(xh)H=xH
 \]
 pues $g=xhx^{-1}$ para algún $h\in H$. Luego $\rho(g)=\id$ y entonces $g\in\ker\rho$. Veamos ahora que vale $\subseteq$. Si $g\in\ker\rho$, entonces
 $\rho(g)=\id$, es decir que, para todo $x\in G$,  
 \begin{align*}
\rho(g)(xH)=xH
\Longleftrightarrow (gx)H=xH
\Longleftrightarrow x^{-1}gx\in H
\Longleftrightarrow g\in xHx^{-1}.
 \end{align*}
Dejamos como 
ejercicio demostrar que $\ker\rho$ es el mayor subgrupo normal de $G$ contenido en $H$. 
\end{example}

Podemos utilizar el ejemplo anterior para dar una aplicación. Daremos una tercera demostración 
del corolario~\ref{cor:p_menor} en la página~\pageref{cor:p_menor}.

\begin{quote}
	Sea $p$ el menor número primo que divide al orden de un grupo finito  
	$G$ y sea $H$ un subgrupo de $G$ índice $p$. Entonces $H$ es normal en $G$. 
\end{quote}

Hacemos actuar a $G$ en $G/H$ por multiplicación a 
izquierda y tenemos un morfismo $\rho\colon G\to\Sym_p$ que tiene núcleo
\[
K=\ker\rho=\bigcap_{x\in G}xHx^{-1}\subseteq H.
\]
Por el primer teorema de isomorfismos, $G/K\simeq\rho(G)\lesssim\Sym_p$ (aquí la notación nos dice que $\rho(G)$ es isomorfo a un subgrupo de $\Sym_p$). Luego $|G/K|$ divide a $p!$. 
Sea $m=(H:K)$. Por el teorema de Lagrange, 
\[
(G:K)=(G:H)(H:K)=pm
\]
y luego $pm$ divide a $p!$, lo que implica que $m$ divide a $(p-1)!$. Si $q$ es un primo que divide a $m$, entnoces $q\geq p$, por la minimalidad de $p$. Además los factores primos de $(p-1)!$ son todos $<p$. En consecuencia, $m=1$ y luego $H=K$.
 

Una acción de un grupo en un conjunto permite definir una relación de equivalencia. Si $G$ actúa en $X$, sobre el conjunto $X$ 
definimos $x\sim y$ si y sólo si existe $g\in G$ tal que $g\cdot x=y$. 

\begin{definition}
\index{Órbita}
Sea $G$ un grupo que actúa en un conjunto $X$. Si $x\in X$, la órbita
de $x$ es el conjunto
\[
G\cdot x=\{g\cdot x:g\in G\}.
\]	
\end{definition}

Las órbitas de la acción de $G$ en $X$ son entonces las clases de equivalencia de la relación inducida por la acción. En particular, dos órbitas cualesquiera serán disjuntas o iguales. Además $X$ podrá descomponerse como unión disjunta de órbitas.  

\begin{definition}
	\index{Estabilizador}
	Sea $G$ un grupo que actúa en un conjunto $X$. Si $x\in X$, el estabilizador de $x$ en $G$ 
	es el subgrupo 
	\[
	G_x=\{g\in G:g\cdot x=x\}.
	\]	
\end{definition}

Queda como ejercicio demostrar que el estabilizador es un subgrupo. 

\index{Acción!transitiva}
El ejemplo anterior es un ejemplo típico de \textbf{acción transitiva}, 
esto significa que dados
$xH,yH\in G/H$, existe $g\in G$ tal que $(gx)H=yH$ (basta tomar $g=yx^{-1}$). Veamos la definición 
general. 

\begin{definition}
\index{Acción!transitiva}
Diremos que una acción de un grupo $G$ en un conjunto $X$ 
es \textbf{transitiva} si dados $x,y\in X$ existe $g\in G$ tal que $g\cdot x=y$. 
\end{definition}

\begin{example}
Por evaluación, 
el grupo simétrico $\Sym_n$ actúa transitivamente en el conjunto $\{1,\dots,n\}$.  	
\end{example}

En la definición de acción transitiva, no hay condiciones sobre la cantidad de $g$ tales que $g\cdot x=y$. 

\begin{definition}
\index{Acción!fiel}
Diremos que una acción de un grupo $G$ en un conjunto $X$ es \textbf{fiel} si 
$\{g\in G:g\cdot x=x\text{ para todo $x\in X$}\}=\{1\}$. 
\end{definition}

La definición anterior equivale a pedir que el morfismo inducido por la acción sea inyectivo. 

% \begin{definition}
% 	Si $G$ es un grupo y $X$ e $Y$ son $G$-conjuntos, diremos que una función $\varphi\colon X\to Y$ es un morfismo de $G$-conjuntos
% 	si $\varphi(g\cdot x)=g\cdot \varphi(x)$ para todo $g\in G$ y $x\in X$. 
% \end{definition}

\begin{theorem}[principio fundamental del conteo]
\index{Principio fundamental del conteo}
Sea $G$ un grupo finito que actúa en un conjunto finito $X$. Si $x\in X$, entonces $|G\cdot x|=(G:G_x)$. 
\end{theorem}

\begin{proof}
	Sea $\varphi\colon G/G_x\to G\cdot x$, $gG_x\mapsto g\cdot x$. La función $\varphi$ está bien definida pues
	\[
	gG_x=hG_x\implies h^{-1}g\in G_x
	\implies h^{-1}g\cdot x=x\implies g\cdot x=h\cdot x.
	\]
	La función $\varphi$ es inyectiva pues 
	\[
	\varphi(gG_x)=\varphi(hG_x)\implies
	g\cdot x=h\cdot x\implies
	h^{-1}g\in G_x\implies gG_x=hG_x.
	\]
	La función $\varphi$ es trivialmente sobreyectiva. En consecuencia, $|G/G_x|=|G\cdot x|$. 
\end{proof}

Si $G$ es un grupo y $X$ e $Y$ son $G$-conjuntos, diremos que una función $\varphi\colon X\to Y$ es 
un \textbf{morfismo} de $G$-conjuntos
si $\varphi(g\cdot x)=g\cdot \varphi(x)$ para todo $g\in G$ y $x\in X$. La 
biyección $\varphi$ que construimos en la demostración del teorema anterior 
es en realidad un morfismo de $G$-conjuntos, 
donde la acción de $G$ en $G/G_x$ es por multiplicación a izquierda, 
pues 
\[
\varphi(g\cdot hG_x)=\varphi((gh)G_x)=(gh)\cdot x=g\cdot (h\cdot x)=g\cdot\varphi(hG_x).
\]
Luego $G\cdot x\simeq G/G_x$ como $G$-conjuntos.
% El corolario que daremos a continuación bien puede denominarse el principio fundamental del conteo. 

% \begin{corollary}[principio fundamental del conteo]
% \index{Principio fundamental del conteo}
% 	Si $G$ es un grupo finito que actúa en un conjunto $X$, entonces
% 	$|G\cdot x|=(G:G_x)$ para todo $x\in X$. 
% \end{corollary}

% \begin{proof}
% 	El resultado es consecuencia inmediata del teorema anterior ya que $G$ actúa transitivamente en cada órbita. 
% \end{proof}
%Vamos a dar dos ejemplos que utilizaremos frecuentemente. 

\begin{example}
	Si $G$ actúa en $G$ por conjugación, es decir $g\cdot x=gxg^{-1}$, las órbitas de esta acción son las \textbf{clases de conjugación} del grupo, es decir los conjuntos de la forma
	\[
	G\cdot x=\{gxg^{-1}:g\in G\}.
	\]
	Los estabilizadores son los centralizadores pues
	\[
	G_x=\{g\in G:g\cdot x=x\}=\{g\in G:gxg^{-1}=x\}=C_G(x).
	\]
	Luego $|G\cdot x|=(G:C_G(x))$. 
\end{example}

\begin{example}
	Sea $H$ un subgrupo de $G$ y sea $X$ el conjunto de subconjuntos de $G$. Hacemos actuar a $G$ en $X$ por conjugación, es decir si $S\in X$, entonces 
	$g\cdot S=gSg^{-1}$. La órbita de $H$ es entonces
	\[
	G\cdot H=\{g\cdot H:g\in G\}=\{gHg^{-1}:g\in G\},
	\]
	el conjunto de conjugados de $H$. El estabilizador de $H$ en $G$ es 
	\[
	G_H=\{g\in G:g\cdot H=H\}=\{g\in G:gHg^{-1}=H\}=N_G(H),
	\]
	el normalizador de $H$ en $G$. Luego $H$ tiene exactamente $(G:N_G(H))$ conjugados en $G$. Observemos que, en particular, la cantidad de conjugados de $H$ en un grupo finito $G$ es un divisor del orden de $G$. 
\end{example}

Como aplicación daremos una demostración de la fórmula que vimos en el teorema~\ref{thm:|HK|}, 
que permite calcular el tamaño del conjunto $HK$ si $H$ y $K$ 
son subgrupos de un grupo $G$. 

\begin{example}
Si $G$ es un grupo y $H$ y $K$ son subgrupos de $G$, entonces 
el grupo $L=H\times K$ actúa en $X=HK$ por
\[
(h,k)\cdot x=hxk^{-1},\quad x\in X,\,h\in H,\,k\in K.
\] 
Observemos que $1\in HK$ y que la acción de $L$ en $X$ es transitiva, 
es decir que tiene una única órbita, pues
$(h,k^{-1})\cdot 1 = hk$. Como además    
\[
L_1=\{(h,k)\in H\times K: (h,k)\cdot 1=1\}=\{(h,k)\in H\times K:h=k\},
\]
entonces $|L_1|=|H\cap K|$, pues $L_1$ y $H\cap K$ están en biyección. Luego 
el principio teorema fundamental del conteo 
nos dice que  
\[
|HK|=(L:L_1)=\frac{|H\times K|}{|H\cap K|}=\frac{|H||K|}{|H\cap K|}.
\]
\end{example}

El ejemplo anterior puede generalizarse, lo que nos da una descomposición 
de un grupo como unión disjunta de \textbf{coclases dobles}. Veremos más adelante
demostraciones alternativas de los teoremas de Sylow basadas en 
coclases dobles.

\begin{example}
\index{Coclase!doble}
Sea $G$ un grupo y sean $H$ y $K$ subgrupos de $G$. Hacemos
que el grupo $L=H\times K$ actúe en $G$ por
\[
(h,k)\cdot g=hgk^{-1}.
\]
Las órbitas son los conjuntos de la forma
\[
HgK=\{hgk:h\in H,\,k\in K\},
\]
estos conjuntos se llaman $(H,K)$-coclases dobles. 
En particular, dos $(H,K)$-coclases dobles son disjuntas o iguales. Más aún, 
$G$ se descompone como unión disjunta 
\[
G=\bigcup_{i\in I}Hg_iK,
\]
para algún conjunto $I$, es decir $G$ 
es unión disjunta de $(H,K)$-coclases dobles. 
Calculamos ahora 
\[
L_g=\{(h,k)\in H\times K:hgk^{-1}=g\}=\{(h,g^{-1}hg)\in H\times K\}
\]
y vemos que $|L_g|=|H\cap gKg^{-1}|$, pues los conjuntos $L_g$ y $H\cap gKg^{-1}$ están en biyección. 
Luego, 
gracias al principio fundamental del conteo, 
\[
|HgK|=(L:L_g)=\frac{|H\times K|}{|H\cap gKg^{-1}|}=\frac{|H||K|}{|H\cap gKg^{-1}|}.
\]
\end{example}

Veamos otra aplicación. 
Calculemos ahora el orden del grupo $\GL_n(p)$ para $n\in\N$ y $p$ un número primo. El mismo argumento
nos permite calcular $\GL_n(q)$ para $q$ una potencia del primo $p$. 

\begin{example}
Sea $K=\Z/p$. 
Vamos a demostrar que 
\[
|\GL_n(p)|=(p^n-1)p^{n-1}|\GL_{n-1}(p)|,
\]
lo que implica que
\[
|\GL_n(p)|=(p^n-1)(p^n-p)\cdots (p^n-p^{n-1}).
\]
La fórmula es cierta en el caso $n=1$ y por lo tanto también cuando $n=2$. Supongamos
entonces que vale para $n-1\geq1$. 
El grupo $G=\GL_{n}(p)$ actúa en $K^{n}$ por multiplicación a izquierda y hay dos órbitas, es decir
\[
X=\{0\}\cup (K^{n}\setminus\{0\}),
\]
pues si $v,w\in K^{n}\setminus\{0\}$, entonces existe $g\in G$ tal que $gv=w$. 
El principio 
fundamental del conteo nos dice que
\[
p^{n+1}-1=|K^{n+1}\setminus\{0\}|=(G:G_{e_1}),
\]
donde $e_1=(1,0,\dots,0)^T$. Si $g=(g_{ij})\in G$ es tal que $ge_1=e_1$, entonces 
\[
g=
\begin{pmatrix}
1 & g_{12} & \cdots & g_{1n}\\
0 & g_{22} & \cdots & g_{2n}\\
\vdots & \vdots & \ddots &\vdots\\
0 & g_{n1} & \cdots & g_{nn}	
\end{pmatrix}.
\]
Luego $|G_{e_1}|=p^{n-1}|\GL_{n-1}(p)|$, ya que la submatriz
$(g_{ij})_{2\leq i,j\leq n}$ es inversible y los 
$g_{1j}$ pueden elegirse arbitrariamente para todo $j\in\{2,\dots,n\}$. 
Luego
\[
p^{n}-1=\frac{|G|}{|G_{e_1}|}=\frac{|\GL_n(p)|}{p^{n-1}|\GL_{n-1}(p)|},
\]
que es esencialmente la fórmula que queríamos demostrar.
\end{example}


%Como aplicación de las acciones de grupos, 
%daremos una tercera demostración del corolario~\ref{cor:p_menor} que vimos en la página~\pageref{cor:p_menor}.
%
%\begin{quote}
%	Sea $p$ el menor número primo que divide al orden de un grupo finito  
%	$G$ y sea $H$ es un subgrupo de $G$ índice $p$. Entonces $H$ es normal en $G$. 
%\end{quote}
%
%Hacemos actuar a $G$ en $G/H$ por multiplicación a izquierda y tenemos entonces un morfismo de grupos $\rho\colon G\to\Sym_p$ con núcleo
%\[
%K=\cap_{x\in G}xHx^{-1}\subseteq H,
%\]
%pues $|G/H|=p$. 
%
%Observemos que $|G/K|$ divide a $p!$ pues $G/K\simeq\rho(G)\leq\Sym_p$ por el primer teorema de isomorfismos. 
%Si $m=(H:K)$, entonces $(G:K)=(G:H)(H:K)=pm$. Luego $pm$ divide a $p!$ y entonces $m$ divide a $(p-1)!$. Si $q$ es un primo que divide a $m$, entonces $q\geq p$ (por la minimalidad de $p$ que tenemos en el enunciado). Como los factores primos de $(p-1)!$ son todos $<p$, $m=1$ y entonces $H=K$.  

\chapter{El teorema de Cauchy}

\index{Puntos fijos!de una acción}
\index{Ecuación de clases}
Si $X$ es un $G$-conjunto finito, sabemos que $X$ puede descomponerse como unión disjunta de órbitas. Sea 
\[
\Fix(X)=\{x\in X:g\cdot x=x\text{ para todo $g\in G$}\}
\]
el \textbf{conjunto de puntos fijos} de $X$. Al tomar cardinal en la descomopsición que tenemos del conjunto $X$, 
agrupar las órbitas que tienen únicamente un elemento y 
utilizar el principio fundamental del conteo para las órbitas con $\geq2$ elementos, obtenemos
\begin{equation}
\label{eq:clases}	
|X|=|\Fix(X)|+\sum_{i=1}^k|G\cdot x_i|
=|\Fix(X)|+\sum_{i=1}^k(G:G_{x_i}),
\end{equation}
donde los $x_j$ son los representantes de las órbitas que tienen $\geq2$ elementos. La fórmula~\eqref{eq:clases} es muy útil y se conoce como
\textbf{ecuación de clases}. 

\begin{example}
Si un grupo finito $G$ actúa en $G$ por conjugación, un cálculo directo nos muestra que $\Fix(G)=Z(G)$ y luego la ecuación de clases queda
\[
|G|=|Z(G)|+\sum_{i=1}^k(G:C_G(x_i)),	
\]
para ciertos $x_1,\dots,x_k\in G$ tales que 
$(G:C_G(x_i))\geq2$ para todo $i\in\{1,\dots,k\}$. 
\end{example}


\begin{definition}
Sea $p$ un número primo. Diremos que $G$ es un $p$-grupo si $|G|=p^m$ para algún $m\in\N_0$.  	
\end{definition}

\begin{theorem}
Sea $p$ un número primo y 
sea $G$ un $p$-grupo. Si $\{1\}\ne N\unlhd G$, entonces $N\cap Z(G)\ne\{1\}$. 	
\end{theorem}

\begin{proof}
Como $N$ es normal en $G$, $G$ actúa en $N$ por conjugación. El teorema fundamental del conteo nos dice que cada órbita de la acción es una potencia del primo $p$. Escribamos
\[
N=\underbrace{\mathcal{O}_1\cup\cdots\cup \mathcal{O}_k}_{\text{órbitas de un elemento}}\cup\underbrace{\mathcal{O}_{k+1}\cup\cdots\cup\mathcal{O}_m}_{\text{órbitas de tamaño $>1$}},
\]	
Como $N\cap Z(G)=\mathcal{O}_1\cup\cdots\cup\mathcal{O}_k$, los números $k=|N\cap Z(G)|$ y $|N\setminus(N\cap Z(G))|$ son divisibles por el primo $p$. Luego
$|N|\equiv|N\cap Z(G)|\bmod p$. Como $1\in N\cap Z(G)$, entonces $|N\cap Z(G)|>1$. En particular, $N\cap Z(G)\ne\{1\}$. 
\end{proof}

\begin{corollary}
Sea $p$ un número primo. Si  
$G$ es un $p$-grupo, entonces $Z(G)\ne\{1\}$.  
\end{corollary}

\begin{proof}
Tomar $N=G$ en el teorema anterior.
\end{proof} 

\begin{corollary}
	Sea $p$ un número primo. Si $|G|=p^2$, entonces $G$ es abeliano.  
\end{corollary}

% todo: página?

\begin{proof}
Por el teorema de lagrange, $|Z(G)|\in\{1,p,p^2\}$. Además, como $G$ es un $p$-grupo, $Z(G)\ne\{1\}$. Si $|Z(G)|=p$, entonces $Z(G)$ es cíclico y luego $G$ es abeliano (es un ejercicio que hicimos en la página...), una contradicción. Luego $|Z(G)|=p^2$ y en consecuencia $G=Z(G)$. 	
\end{proof}

\begin{theorem}[Cauchy]
\index{Teorema!de Cauchy}
Si $G$ es finito y $p$ es un primo que divide al orden de $G$, entonces existe $g\in G$ de orden $p$. 	
\end{theorem}

\begin{proof}
Sea $C=\Z/p$ y sea
\[
X=\{(x_1,\dots,x_p)\in G\times\cdots\times G:x_1\cdots x_p=1\}.
\]
Entonces $C$ actúa en $X$ por $k\cdot (x_1,\dots,x_p)=(x_{k+1},\dots,x_{k+p})$, donde los índices se toman módulo $p$. Para ver que esto es realmente una acción alcanza con observar
que 
\[
x_{i_1}\cdots x_{i_p}=1
\implies (x_{i_1}^{-1}x_{i_1})x_{i_2}\cdots x_{i_p}=x_{i_1}^{-1}
\implies x_{i_2}\cdots x_{i_p}x_{i_1}=1.
\]	
Una vez que $x_1,\dots,x_{p-1}$ están fijos, $x_p=x_{p-1}^{-1}\cdots x_{1}^{-1}$ es la única posibilidad para $x_p$. Luego $|X|=|G|^{p-1}$. Cada $C$-órbita tiene $1$ o $p$ elementos pues $|C|=p$. Escribamos
\[
X=\underbrace{\mathcal{O}_1\cup\cdots\cup \mathcal{O}_k}_{\text{órbitas de un elemento}}\cup\underbrace{\mathcal{O}_{k+1}\cup\cdots\cup\mathcal{O}_m}_{\text{órbitas de tamaño $p$}}.
\] 
Entonces $0\equiv |G|^{p-1}=|X|\equiv k\bmod p$, es decir $p$ divide a $k$. Como además $(1,1,\dots,1)\in X$, $k\geq 1$. Luego $p\leq k$. En particular, 
existe $x\in G\setminus\{1\}$ tal que $(x,x,\dots,x)\in X$. Luego $|x|=p$. 
\end{proof}

\begin{corollary}
	Sea $p$ un primo y $G$ un grupo finito. Entonces $G$ es un $p$-grupo si y sólo si todo elemento de $G$ tiene orden una potencia de $p$. 
\end{corollary}

\begin{proof}
Si $G$ es un $p$-grupo, entonces todo elemento tiene orden una potencia de $p$ por el teorema de Lagrange. Recíprocamente, si $q$ es un primo que divide al orden de $G$, el teorema
de Cauchy nos dice que existe $g\in G$ de orden $q$. Luego $q=p$. 	
\end{proof}

\begin{corollary}
Si $p$ es un primo impar y $G$ es un grupo de orden $2p$, entonces $G\simeq\Z/2p$ o bien $G\simeq\D_p$. 	
\end{corollary}

\begin{proof}
Por el teorema de Cauchy sabemos que existen $r,s\in G$ tales que $|r|=p$ y $|s|=2$. Sea $H=\langle r\rangle$. Entonces $(G:H)=2$ y luego $H\unlhd G$. Escribimos $G=H\cup Hs$ (unión disjunta) pues $s\not\in H$.
En particular,  
\[
G=\{1,r,\dots,r^{p-1},s,rs,\dots,r^{p-1}s\}.
\]
Como $srs^{-1}\in H$, entonces $srs^{-1}=r^k$ para algún $k\in\{0,1,\dots,p-1\}$. Como $s^2=1$,
\[
r=s^2rs^{-2}=s(srs^{-1})s^{-1}=sr^ks^{-1}=r^{k^2}.
\]	
Luego $k^2\equiv 1\bmod p$ y entonces $k\equiv 1\bmod p$ o bien $k\equiv -1\bmod p$. Si $k\equiv -1\bmod p$, entonces $srs^{-1}=r^{-1}$ y luego $G\simeq\D_p$. 
Si $k\equiv 1\bmod p$, entonces $rs=sr$ y luego, como $G$ es abeliano, $G\simeq\Z/{2p}$.
\end{proof}

\begin{theorem}
	Un grupo de orden $p^m$ tiene un subgrupo normal de orden $p^n$ para todo $n\leq m$. 
\end{theorem}

\begin{proof}
Procederemos por inducción en $m$. El caso $m=1$ es trivial. Supongamos entonces que el resultado vale para los grupos de orden $p^m$ y sea $G$ un grupo de orden $p^{m+1}$. 
Queremos ver que si $n\leq m$, entonces $G$ contiene un subgrupo normal de orden $p^n$. Como $Z(G)\ne\{1\}$, existe $g\in Z(G)\setminus\{1\}$ de orden $p$. Sea 
$N=\langle g\rangle\unlhd G$. El grupo $G/N$ tiene orden $p^m$ y entonces, por la hipótesis inductiva, existe un subgrupo normal $Y$ de $G/N$ de orden $p^n$. El teorema de la correspondencia
nos permite afirmar entonces que $G$ contiene un subgrupo normal $K$ de $G$ que contiene a $N$, es decir $N\leq K\leq G$. En efecto, $Y=\pi(K)$ y además 
$(G:K)=(\pi(G):\pi(K))=p^{m-n}$. En consecuencia, $|K|=p^n$.   
\end{proof}



\chapter{Los teoremas de Sylow}

\begin{definition}
\index{Subgrupo!de Sylow}
Sea $G$ un grupo de orden $p^\alpha m$, donde $p$ es un primo coprimo con $m$. Un subgrupo $S$ de $G$ es 
un $p$-subgrupo	de Sylow de $G$ si $|S|=p^\alpha$. 
\end{definition}

Observemos que un subgrupo $S$ de $G$ será entonces un $p$-subgrupo de Sylow de $G$ si y sólo si $S$ es un $p$-grupo y además $p$ no divide a $(G:S)$. 

\begin{examples}\
\begin{enumerate}
\item Si $p$ no divide a $|G|$, entonces $\{1\}$ es un $p$-subgrupo de Sylow de $G$. 
\item Si $G$ es un $p$-grupo, entonces $G$ es un $p$-subgrupo de Sylow de $G$.
\end{enumerate}	
\end{examples}

\begin{example}
Sea $G=\Sym_3$. Entonces $\langle (12)\rangle$, $\langle (13)\rangle$ y $\langle (23)\rangle$ son los $2$-subgrupos de Sylow de $G$. Además
$\langle (123)\rangle$ es el único $3$-subgrupo de Sylow de $G$.	
\end{example}

\begin{example}
Si $G=\Sym_4$, el subgrupo $\langle (1234),(13)\rangle$ es un $2$-subgrupo de Sylow de $G$ y
el subgrupo $\langle (123)\rangle$ es un $3$-subgrupo de Sylow de $G$. 	
\end{example}

\begin{example}
Si $G=\Z/18$, el subgrupo $\langle 2\rangle =\{0,2,4,6,8,10,12,14,16\}$ es el único $3$-subgrupo
de Sylow de $G$ y el subgrupo $\langle 9\rangle=\{0,9\}$ es el único $2$-subgrupo de Sylow de $G$. 	
\end{example}

\begin{example}
Sea $p$ un número primo y sea 
$G=\GL_n(p)$. Como 
\begin{align*}
|\GL_n(p)|&=(p^n-1)(p^n-p)\cdots (p^n-p^{n-1})\\
&=p^{1+2+\cdots+n}(p^n-1)(p^{n-1}-1)\cdots (p-1),
\end{align*}
podemos escribir $|\GL_n(p)|=p^\alpha m$, donde $\alpha=1+2+\cdots+n$ y $m$ es un entero no divisible por $p$. El subgrupo
de matrices de la forma 
\[
\begin{pmatrix}
1 & * & \cdots & *\\
0 & 1 & \cdots & *\\
\vdots & \vdots & \ddots & \vdots\\
0 & 0 & \cdots & 1 	
\end{pmatrix},
\]
es decir el conjunto de matrices $(g_{ij})$ con 
\[
g_{ij}=\begin{cases}
1 & \text{si $i=j$},\\
0 & \text{si $i>j$},
\end{cases}
\]
tiene orden $p^\alpha$ y luego es un $p$-subgrupo de Sylow de $\GL_n(p)$. 
\end{example}

El primer objetivo del capítulo es demostrar el primer teorema de Sylow, que garantiza la existencia de $p$-subgrupos de Sylow para todo primo $p$. 
Antes de ir al teorema, necesitamos un resultado auxiliar.

\begin{lemma}
	Si $p$ es un primo, $\alpha\geq0$ y $m\geq 1$, entonces 
	\[
	\binom{p^\alpha m}{p^\alpha}\equiv m\bmod p.
	\]
\end{lemma}

\begin{proof}
	Por el teorema del binomio, 
	\[
	(1+X)^p=\sum_{j=0}^p\binom{p}{j}X^{p-j}\equiv 1+X^p\bmod p,
	\]
	ya que $\binom{p}{j}$ es divisible por $p$ para todo $j\in\{1,\dots,p-1\}$. 
	Por inducción, demostramos ahora que 
	\begin{gather*}
	(1+X)^{p^j}\equiv 1+X^{p^j}\bmod p	\\
	\shortintertext{vale para todo $j$. Luego}
	(1+X)^{p^\alpha m}\equiv (1+X^{p^\alpha})^m\bmod p.
	\end{gather*}
Al comparar el coeficiente de $X^{p^\alpha}$ en ambos miembros de la fórmula anterior, obtenemos el resultado que queríamos demostrar.
\end{proof}

\begin{theorem}[primer teorema de Sylow]
\index{Teorema!de Sylow I}	
Si $G$ es un grupo finito y $p$ es un número primo, existe un $p$-subgrupo de Sylow de $G$. 
\end{theorem}

\begin{proof}
Escribimos $|G|=p^\alpha m$, con $\gcd(p,m)=1$ y $\alpha\geq1$. Sea 
\[
X=\{S:S\subseteq G\text{ subconjunto de tamaño $p^\alpha$}\}.
\]
Hacemos actuar a $G$ en $X$ por multiplicación a izquierda, pues $|g\cdot S|=|gS|=|S|$ para todo $g\in G$ y todo $S\in X$. Descomponemos a $X$
en $G$-órbitas y observamos que, gracias al lema anterior,  
\[
|X|= \binom{p^\alpha m}{p^\alpha}\equiv m\not\equiv 0\bmod p,
\] 	
lo que implica que existe una órbita $\mathcal{O}$ de tamaño no divisible por $p$. Si $S\in\mathcal{O}$, sea $G_S$ el estabilizador de $S$ en $G$. Como
$|\mathcal{O}|=(G:G_S)$ y $|\mathcal{O}|$ no es divisible por $p$, tenemos que $p^\alpha$ divide a $|G_S|$. En particular, $p^\alpha\leq |G_S|$. Si $g\in G_S$, entonces
$gS=S$. Si $x\in S$, entonces $G_Sx\subseteq S$. Luego
\[
|G_S|=|G_Sx|\leq |S|=p^\alpha
\]
pues $S\in X$. En conclusión $G_S$ es un $p$-subgrupo de Sylow de $G$. 
\end{proof}

Nos resultará conveniente introducir la siguiente notación. 
Si $G$ es un grupo finito y $p$ es un número primo que divide al orden de $G$, escribiremos
\[
\Syl_p(G)=\{\text{$p$-subgrupos de Sylow de $G$}\}.
\]


\begin{quote}
Veamos una demostración alternativa del primer teorema de Sylow 
que utiliza coclases dobles. 
Primero demostraremos un resultado auxiliar. Si $P\in\Syl_p(G)$ y $H\leq G$, entonces  
existe un $g\in G$ tal que $H\cap gPg^{-1}\in\Syl_p(H)$. En efecto, supongamos que
$|H|=p^\beta t$ con $p$ coprimo con $t$. 
Si descomponemos a $G$ en $(H,P)$-coclases dobles, 
\[
|G|=\sum_{i=1}^k \frac{|H||P|}{|H\cap x_iPx_i^{-1}|}.
\]
Al simplificar $|P|=p^\alpha$, tenemos que $m=\sum_{i=1}^k(H:H\cap x_iPx_i^{-1})$, lo que nos dice 
que existe $i\in\{1,\dots,k\}$ tal que 
$(H:H\cap x_iPx_i^{-1})$ no es divisible por $p$. Esto significa que que 
$p^\beta$ divide a $|H\cap x_iPx_i^{-1}|$ y en consecuencia $H\cap x_iPx_i^{-1}\in\Syl_p(H)$. 
Por el teorema de Cayley podemos suponer que nuestro subgrupo $G$ es un subgrupo
de $\GL_n(p)$ para algún $n\in\N$ y algún primo $p$. Sea $P$ un subgrupo de Sylow
del grupo $\GL_n(P)$. La observación que demostramos
aplicada al grupo $G$ nos dice que existe $g\in \GL_n(p)$ tal que $G\cap gPg^{-1}$ es un subgrupo de Sylow de $G$.  
\end{quote}

Antes de demostrar el segundo teorema de Sylow, vamos a demostrar un resultado similar, aunque levemente más técnico.

\begin{theorem}
Si $P$ es un $p$-subgrupo de $G$ y $S\in\Syl_p(G)$, entonces $P\subseteq gSg^{-1}$ para algún $g\in G$.	
\end{theorem}

\begin{proof}
	Sea $X=\{xS:x\in G\}$ el conjunto de coclases de $S$ en $G$. Entonces $|X|=(G:S)$ no es divisible por el primo $p$. Si hacemos actuar a $G$ en $X$ por multiplicación a izquierda, en particular, el subgrupo $P$ también actuará en $X$ por multiplicación a izquierda. Si descomponemos entonces a $X$ en $P$-órbitas, existirá una $P$-órbita $\mathcal{O}$ de tamaño 
	no divisible por $p$, pues $|X|$ no es divisible por $p$. Como $|\mathcal{O}|$ divide al orden de $P$ y $p$ no divide al tamaño de $\mathcal{O}$, necesariamente se tiene $|\mathcal{O}|=1$, 
	es decir $\mathcal{O}=\{gS\}$ para algún $g\in G$. Como entonces $P(gS)=gS$, en particular, $xg\in gS$ para todo $x\in P$, es decir: si $x\in P$, entonces $x\in gSg^{-1}$. Luego $P\subseteq gSg^{-1}$. 
\end{proof}

\begin{corollary}
	Sea $p$ un número primo. 
	Si $G$ es un grupo finito y $P$ es un $p$-subgrupo de $G$, entonces $P$ está contenido en algún $p$-subgrupo de Sylow de $G$.
\end{corollary}

\begin{proof}
Si $S\in\Syl_p(G)$, entonces $gSg^{-1}\in\Syl_p(G)$ pues $|gSg^{-1}|=|S|$. El teorema anterior nos da el corolario pues $P\subseteq gSg^{-1}$ para algún $g\in G$  	
\end{proof}

Ahora sí, el segundo teorema de Sylow, que afirma que dos $p$-subgrupos de Sylow siempre serán conjugados.

\begin{theorem}[segundo teorema de Sylow]
\index{Teorema!de Sylow II}
Sea $G$ un grupo finito y $p$ un número primo. 
Si $S,T\in\Syl_p(G)$, entonces existe $g\in G$ tal que $gSg^{-1}=T$. 
\end{theorem}

\begin{proof}
Utilizamos el teorema anterior con $P=T$ y entonces $T\subseteq gSg^{-1}$ para algún $g\in G$. Como $|S|=|T|$ y además  
$|T|\leq |gSg^{-1}|=|S|$, se concluye que $T=gSg^{-1}$.  	
\end{proof}

\begin{corollary}
Sea $G$ un grupo finito, $p$ un número primo y $S\in\Syl_p(G)$. Si $S$ es normal en $G$, entonces $\Syl_p(G)=\{S\}$. 
\end{corollary}

\begin{proof}
Si $T\in\Syl_p(G)$, entonces $T=gSg^{-1}=S$ para algún $g\in G$. 	
\end{proof}

\begin{quote}
Veamos una 
demostración alternativa del segundo teorema de Sylow que usa coclases dobles. 
Si $P,Q\in\Syl_p(G)$ y 
descomponemos a $G$ en $(P,Q)$-coclases dobles, tenemos
\[
p^\alpha m=\sum_{i=1}^k\frac{|P||Q|}{|P\cap x_iQx_i^{-1}|}
\implies
m=\sum_{i=1}^k\frac{|P|}{|P\cap x_iQx_i^{-1}|}
\]
para ciertos $x_1,\dots,x_k\in G$. 
Como $m$ no es divisible por $p$, existe algún $i\in\{1,\dots,k\}$ tal que $|P|=|P\cap x_iQx_i^{-1}|$, lo que
implica que $P=x_iQx_i^{-1}$ para algún $i\in\{1,\dots,k\}$. 
\end{quote}

Antes de enunciar y demostrar el tercer teorema de Sylow introduciremos la siguiente notación. Si $p$ es un número primo y $G$ es un grupo finito de orden $p^\alpha m$ con $\gcd(p,m)=1$, entonces
$n_p(G)=|\Syl_p(G)|$. Observar que entonces 
\[
n_p(G)=(G:N_G(P))
\]
para cualquier $P\in\Syl_p(G)$. Veremos que además $n_p(G)$ divide a $m$. 

\begin{theorem}[tercer teorem de Sylow]
Sea $G$ un grupo finito y $p$ un número primo. Entonces $n_p(G)\equiv 1\bmod p$. 	
\end{theorem}

\begin{proof}
	Sea $P\in\Syl_p(G)$, que sabemos que existe por el primer teorema de Sylow, y sea $n=n_p(G)$, Consideramos
	el conjunto
	\[
	X=\{gPg^{-1}:g\in G\}=\{P=P_1,P_2,\dots,P_n\}.
	\] 
	Si hacemos actuar a $G$ en $X$ por conjugación, $P$ también actúa en $X$ por conjugación. 
	Entonces todas las $P$-órbitas tiene tamaño una potencia del primo $p$. 
	
	Afirmamos que $\{P\}$ es la única $P$-órbita de tamaño 1. En efecto, como $xPx^{-1}=P$ si $x\in P$, tenemos que $\{P\}$ es una $P$-órbita. Sea $P_i$ una $P$-órbita de tamaño 1. Entonces
	$xP_ix^{-1}=P_i$ para todo $x\in P$ y luego $P\subseteq N_G(P_i)$. El grupo
	$N_G(P_i)/P_i$ tiene orden no divisible por $p$, pues $P_i\in\Syl_p(G)$. Si $xP_i\in N_G(P_i)/P_i$ con $x\in P$, entonces $xP_i=P_i$, es decir $x\in P_i$. Esto implica
	que $P\subseteq P_i$ y luego $P=P_i$. Ahora tenemos 
	\[
	X=\{P\}\cup \underbrace{\mathcal{O}_1\cup\mathcal{O}_2\cup\cdots\cup\mathcal{O}_k}_{\text{de tamaño $>1$ divisible por $p$}},
	\]
	de donde obtenemos $n_p(G)=|X|\equiv 1\bmod p$. 
\end{proof}

\begin{quote}
Una demostración alternativa del tercer teorema de Sylow basada en coclases dobles. 
Sean $P\in\Syl_p(G)$ y $N=N_G(P)$. Recordemos
que $n_p(G)=(G:N)$. Si descomponemos
a $G$ en $(P,N)$-coclases dobles, 
\[
|G|=\sum_{i=1}^k\frac{|P||N|}{|N\cap x_iPx_i^{-1}|}
\] 
para ciertos $x_1,\dots,x_k\in G$. Sin perder generalidad podemos suponer que $x_1=1$, entonces la fórmula anterior queda
\[
n_p(G)=1+\sum_{i=2}^k\frac{|P|}{|N\cap x_iPx_i^{-1}|},
\] 
pues $(G:N)=n_p(G)$. 
El teorema quedará demostrado si vemos que la suma del miembro de la derecha es divisible por $p$. Si esto no pasa, 
es decir si existe $i\in\{2,\dots,k\}$ tal que $|N\cap x_iPx_i^{-1}|=|P|$, entonces
$x_iPx_i^{-1}=N\cap x_iPx_i^{-1}\subseteq N$. Como entonces $P$ y también $x_iPx_i^{-1}$ son ambos $p$-subgrupos de Sylow de $N$,
el segundo teorema de Sylow afirma que estos subgrupos tienen que ser conjugados en $N$. Por definición del normalizador, $P$ es normal en $N$. 
En consecuencia, $x_iPx_i=P$, es decir $x_i\in N$, una contradicción pues como $i>1$ se tiene que 
$Px_iN$ y $Px_1N=PN$ son coclases dobles disjuntas.  
\end{quote}

Veamos algunas aplicaciones sencillas de los teoremas de Sylow. 

\begin{example}
Si $G$ es un grupo de orden 15, entonces $G$ es cíclico. 

Sean $n_3=n_3(G)$ y $n_5=n_5(G)$. Sabemos que $n_3\equiv1\bmod 3$ y que además $n_3$ divide a 5, luego $n_3=1$. Esto nos dice
que existe un único $H\in\Syl_3(G)$, que resulta ser normal en $G$ e isomorfo a $\Z/3$. Similarmente, $n_5=1$ y existe 
entonces un único $K\in\Syl_5(G)$ tal que $K\unlhd G$ y $K\simeq\Z/5$. Como $H\cap K=\{1\}$ por el teorema de Lagrange, tenemos
\[
|HK|=\frac{|H||K|}{|H\cap K|}=|H||K|=15=|G|.
\] 	
Luego $G=HK\simeq H\times K\simeq \Z/3\times\Z/5\simeq\Z/15$. 
\end{example}

El siguiente ejemplo, es bastante más difícil que el anterior.

\begin{example}
Si $G$ es un grupo de orden 455, entonces $G$ es cíclico. 

Para cada primo $p$ que divide al orden de $G$, sea $n_p=n_p(G)$. Como $n_5$ divide a $7\times 13$ y $n_5\equiv 1\bmod 5$, entonces $n_5\in\{1.91\}$. En cambio,
un cálculo sencillo nos da $n_7=n_{13}=1$. Sea $P\in\Syl_7(G)$ y sea $Q\in\Syl_{13}(G)$, ambos son subgrupos normales en $G$. Como $P$ y $Q$ tienen órdenes coprimos, el teorema de Lagrange implica que $P\cap Q=\{1\}$. 

Estudiaremos ahora los subgrupos de Sylow
de los cocientes $G/P$ y $G/Q$. 
Sea $m_5=n_5(G/P)$ y $m_{13}=n_{13}(G/P)$. Como $m_5$ divide a 13 y además $m_5\equiv1\bmod 5$, entonces $m_5=1$. Similarmente, $m_{13}=1$ y entonces $G/P\simeq\Z/5\times\Z/13$. De la misma forma vemos que $G/Q\simeq\Z/5\times\Z/7$ y entonces $G/P$ y $G/Q$ son ambos abelianos. Esto significa que 
$[G,G]\subseteq P\cap Q=\{1\}$ y luego $G$ también es un grupo abeliano. En particular, $n_5=1$ y luego
\[
G\simeq\Z/5\times\Z/7\times\Z/13\simeq\Z/455.
\]  	
\end{example}



\begin{example}
Si $G$ es un grupo de orden 	21, entonces 
\[
G\simeq\Z/21\text{ o bien }G\simeq\langle x,y:x^7=y^3=1,\,yx=x^2y\rangle.
\] 

Sean $n_3=n_3(G)$ y $n_7=n_7(G)$. Como $n_7\equiv1\bmod 7$ y $n_3$ divide a $3$, entonces $n_7=1$. Existe entonces
un único $H\in\Syl_7(G)$. Ese subgrupo $H$ es tal que $H\unlhd G$ y $H\simeq\Z/7$. Entonces $H=\langle x\rangle$ donde $x^7=1$. 
Sea $K\in\Syl_3(G)$. Como $n_3$ divide a 7 y $n_3\equiv1\bmod 3$, entonces $n_3\in\{1,7\}$. En cualquier caso, $K\simeq\Z/3$ y 
entonces $K=\langle y\rangle$ donde $y^3=1$. Por el teorema de Lagrange, $H\cap K=\{1\}$ y luego $G=HK$. En particular,
\[
G=\{x^iy^j:0\leq i\leq 6,\,0\leq j\leq 2\}.
\]
Como $H$ es normal en $G$, $yxy^{-1}\in H$, es decir $yxy^{-1}=x^i$ para algún $i\in\{1,\dots,6\}$. Tenemos entonces
que $x^7=y^3=1$ y además $yx=x^iy$ para un cierto $i\in\{1,\dots,6\}$. Para ver qué podemos decir de ese $i$ observamos que
\[
x=y^3xy^{-3}=y^2(yxy^{-1})y^{-2}=y^2x^iy^{-2}=y(x^i)^2y^{-1}=(x^i)^3
\]
y luego $i^3\equiv 1\bmod 7$, es decir $i\in\{1,2,4\}$.  Tenemos entonces tres casos para analizar.
\begin{enumerate}
	\item[(a)] Si $yxy^{-1}=x$, entonces $xy=yx$ y luego $K\unlhd G$. Esto implica que $G\simeq H\times K\simeq\Z/21$. 
	\item[(b)] Si $yxy^{-1}=x^2$, entonces tenemos todo lo que necesitamos para conocer $G$. De hecho, no solamente tenemos la descripción prometida sino que 
	podemos escribir la tabla de multiplicación e intentar reconocer este grupo. Para poder tener una idea más concreta del grupo $G$ que encontramos en este caso, mencionamos que puede presentarse como un cierto subgrupo de $\GL_2(\Z/7)$. En efecto,  
	\[
	x=\begin{pmatrix}
	1&1\\
	0&1\end{pmatrix},
	\quad	
	y=\begin{pmatrix}
	2&0\\
	0&1\end{pmatrix},
	\quad
	G\simeq\langle x,y\rangle\leq \GL_2(\Z/7).
\]
\item[(c)] Si $yxy^{-1}=x^4$, entonces $y^2xy^{-2}=x^2$. Como $|y^2|=|y|=3$, si $z=y^2$, entonces $H=\langle y\rangle=\langle z\rangle$, lo que nos dice
que, en realidad, estamos en el caso anterior. 
\end{enumerate}
\end{example}

\begin{example}
Si $G$ es un grupo de orden $5\cdot 7\cdot 17$, entonces $G$ es cíclico.

Si $p\in\{5,7,17\}$, sea $n_p=n_p(G)$. Como $n_5\equiv 1\bmod 5$ y $n_5$ divide a $7\cdot 17$, entonces $n_5=1$. Sea $H\in\Syl_5(G)$. Al ser el único $5$-subgrupo de Sylow de $G$, $H$ es normal en $G$. Sean además $K\in\Syl_7(G)$ y $L\in\Syl_{17}(G)$. Como $H$ es normal en $G$, $HK$ es un subgrupo de $G$. Por el teorema de Lagrange, $H\cap K=\{1\}$ pues $H$ y $K$ tienen órdenes coprimos. Entonces $|HK|=5\cdot 7$.   

Usaremos ahora la teoría de Sylow pero en el grupo $HK$. Si $m_7=n_7(HK)$, entonces $m_7=1$. En particular, $K\in\Syl_7(HK)$ y además $K$ es normal en $HK$, es decir
$HK\subseteq N_G(K)$, lo que implica que $|HK|\leq |N_G(K)|$. Como
\[
n_7=(G:N_G(K))=\frac{|G|}{|N_G(K)|}\leq \frac{|G|}{|HK|}=\frac{5\cdot 7\cdot 17}{5\cdot 7}=17
\]
y además $n_7\in\{1,5\cdot 17\}$, se concluye que $n_7=1$. Dejamos como ejercicio utilizar la misma técnica para demostrar que $n_{17}=1$. En conclusión, $K$ y $L$ son ambos normales en $G$. El teorema de Lagrange implica que $L\cap H=H\cap K=L\cap K=\{1\}$ y entonces 
\[
L\cap (HK)=H\cap (LK)=K\cap (LH)=\{1\}.
\]
Luego $G=HKL\simeq\Z/5\times\Z/7\times\Z/17\simeq\Z/5\cdot 7\cdot 17$. 
\end{example}


\begin{example}
Si $G$ es un grupo de orden 12 tal que $n_3(G)\ne1$, entonces $G\simeq\Alt_4$.

Sea $P\in\Syl_3(G)$ y sea $n_3=n_3(G)=4$. Claramente $P$ no es normal en $G$. Hacemos actuar a $G$ en el conjunto de coclases $G/P$ por multiplicación a izquierda y 
obtenemos un morfismo 
\[
\rho\colon G\to\Sym_{G/P}\simeq\Sym_4.
\]
Afirmamos que $\rho$ es inyectivo. Primero vemos que $\ker\rho\subseteq P$ pues 
\[
x\in\ker\rho\implies
\rho_x=\id\implies
xP\subseteq P\implies
x\in P.
\]
Como $P$ no es normal en $G$, $P\ne \ker\rho$. Luego $\ker\rho$ es un subgrupo propio de $P$. En consecuencia, $\ker\rho=\{1\}$ pues $|P|=3$.   

Sean $S,T\in\Syl_3(G)$. Como $S\simeq T\simeq\Z/3$, el teorema de Lagrange implica que $S\cap T=\{1\}$. Esto implica que $G$ contiene exactamente ocho elementos de orden tres. Como los elementos de orden tres de $\Sym_4$ están todos en $\Alt_4$, el subgrupo $\rho(G)\cap\Alt_4$ de $\Sym_4$ contiene al menos ocho elementos. Luego $G\simeq\rho(G)\simeq\Alt_4$.  
\end{example}

Otra aplicación bastante común de los teoremas de Sylow es a la (no) simplicidad de grupos. 

\begin{example}
Si $G$ es un grupo de orden 36, entonces $G$ no es simple.

Si $G$ fuera simple, entonces $n_3=n_3(G)=4$. Sea $P\in\Syl_3(G)$. Si hacemos actuar a $G$ en $X=\{gPg^{-1}:g\in G\}$ por conjugación, 
tenemos un morfismo de grupos 
\[
\rho\colon G\to\Sym_X\simeq\Sym_4.
\]
Como $G$ es simple, $\ker\rho=\{1\}$ o bien $\ker\rho=G$. Si $\ker\rho=G$, $P$ es normal en $G$, una contradicción. Luego $\ker\rho=\{1\}$ y entonces $\rho$ es inyectivo. En particular, gracias al primer teorema de isomorfismos, 
\[
G\simeq G/\ker\rho\simeq\rho(G)\lesssim\Sym_4,
\]
que implica que 36 divide a 24, una contradicción.   	
\end{example}

\begin{example}
Si $G$ es un grupo de orden 30, entonces $G$ no es simple.

Para cada primo $p$ que divide a 30, sea $n_p=n_p(G)$. Supongamos que $n_2>1$, $n_3>1$ y que $n_5>1$. Entonces $n_3=10$. Tenemos así
diez $3$-subgrupos de Sylow, todos ellos con intersección trivial. En efecto, para ver que la intersección de dos $3$-subgrupos de Sylow es trivial procedemos de la siguiente forma: Si $P,Q\in\Syl_3(G)$ son tales que $P\ne Q$, entonces $P\cap Q\leq P$ y luego
$|P\cap Q|\in\{1,3\}$. Si $|P\cap Q|=3$, entonces $P\cap Q=P$ y luego $P=Q$, un contradicción. De la misma forma, tenemos seis $5$-subgrupos de Sylow de $G$, todos con interección trivial. En conclusión, 
\[
|G|\geq 1+10\times 2+6\times 4>30,
\]
una contradicción. 
\end{example}

Al terminar la demostración del primer teorema de Sylow, usamos coclases dobles para demostrar que  
si $H$ es un subgrupo de un grupo finito $G$ y 
$P\in\Syl_p(G)$, entonces $g\in G$ tal que $gPg^{-1}\cap H\in\Syl_p(H)$. Otra demostración
puede obtenerse al considerar la acción de $H$ en $G/P$ por multiplicación a izquierda.

%\begin{theorem}
%	Sea $H$ un subgrupo de un grupo finito $G$ y 
%	sea $P\in\Syl_p(G)$. Existe entonces $g\in G$ tal que $gPg^{-1}\cap H\in\Syl_p(H)$.
%\end{theorem}
%
%\begin{proof}
%	El grupo $H$ actúa en $G/P$ por multiplicación a izquierda. Como $p$ no
%	divide al índice $(G:H)$ de $H$ en $G$, alguna órbita, digamos $H\cdot
%	(gP)$, tiene tamaño no divisible por $p$, es decir: existe $g\in G$ tal que
%	$|H\cdot (gP)|$ no es divisible por $p$. El tamaño de esta órbita es
%	\[
%		|H\cdot (gP)|=(H:H_{gP}),
%	\]
%	donde $H_{gP}=\{h\in H:h\cdot (gP)=gP\}$ es el estabilizador en $H$ de $gP$. Si $|H|=p^{\alpha}m_1$, entonces
%	$H_{gP}|=p^{\alpha}m_2$. Además
%	\[
%		H_{gP}=\{h\in H:(hg)P=gP\}=\{h\in H:g^{-1}hg\in P\}=gPg^{-1}\cap H,
%	\]
%	que es un $p$-grupo por ser un subgrupo del $p$-grupo $gPg^{-1}$. 
%\end{proof}
%
%
%\begin{example}
%	Si $G=\GL_2(p)$ y $H=\begin{pmatrix} * & 0\\ * & *\end{pmatrix}$, $P=\begin{pmatrix} 1 & *\\ 0 & 1\end{pmatrix}\in\Syl_p(G)$.
%	Si $g=\begin{pmatrix}0 & 1\\-1&0\end{pmatrix}$, entonces
%	\[
%		g\begin{pmatrix}
%			1 & x\\
%			0 & 1\end{pmatrix}g^{-1}=\begin{pmatrix}
%			1 & 0\\
%			-x & 1
%		\end{pmatrix}
%	\]
%	y luego $gPg^{-1}\in\Syl_p(H)$. 
%\end{example}

\begin{theorem}
	Sea $N$ un subgrupo normal de un grupo finito $G$ y sea $P\in\Syl_p(N)$. Entonces 
	$P\cap N\in\Syl_p(N)$ y todo los $p$-subgrupos de Sylow de $N$ se obtienen
	de esa forma. 
\end{theorem}

\begin{proof}
	Como $N$ es normal, 
	sabemos por el teorema anterior que existe un $g\in G$ tal que
	\[
		g(P\cap N)g^{-1}=gPg^{-1}\cap gNg^{-1}=gPg^{-1}\cap N\in\Syl_p(N).
	\]
	Luego $P\cap N$ es un $p$-subgrupo de Sylow de $g^{-1}Ng=N$. 
	
	Sea $Q\in\Syl_pN$ y sea $P\in\Syl_p(G)$ tal que $Q\subseteq P$. Entonces
	$Q$ está contenido en el $p$-subgrupo de Sylow $P\cap N$ de $N$. Luego
	$Q=P\cap N$.
\end{proof}

Como corolario obtenemos que si $N$ es un subgrupo normal de un grupo finito $G$, entonces 
$n_p(N)\leq n_p(G)$. 

\begin{theorem}
	Sean $N$ un subgrupo normal de un grupo finito $G$, $\pi\colon G\to
	G/N$ el morfismo canónico y $P\in\Syl_p(G)$. Entonces
	$\pi(G)\in\Syl_p(G/N)$ y todos los $p$-subgrupos de Sylow de $G/N$ se
	obtienen de esa forma. 
\end{theorem}

\begin{proof}
	Como $\pi(P)=\pi|_{P}(P)\simeq P/N\cap P$ por el segundo teorema de
	isomorfismo, $\pi(P)$ es un $p$-grupo.  Como además $|PN|=|P||N|/|P\cap
	N|$, 
	\[
		(G/N:\pi(P))=(G:PN)
	\]
	no es divisible por $p$. Luego $\pi(P)\in\Syl_p(G/N)$. 
	
	Si $Q\in\Syl_p(G/N)$, entonces $Q=\pi(H)$ para algún subgrupo $H$ de $G$
	tal que $N\subseteq H$. En particular, 
	\[
		|Q|=|\pi(H)|=\frac{|H|}{|H\cap N|}=\frac{|H|}{|N|}
	\]
	y luego
	\[
		(G:H)=\frac{|G|/|N|}{|H|/|N|}=(G/N:Q)
	\]
	no es divisible por $p$. Esto nos dice que si $X\in\Syl_p(H)$, entonces
	$X\in\Syl_p(G)$. Luego $\pi(X)\subseteq\pi(H)=Q$ y entonces $\pi(X)=Q$ ya
	que $\pi(X)\in\Syl_p(G/N)$.
\end{proof}

Como corolario, si $N$ es un subgrupo normal de un grupo finito $G$, entonces 
$n_p(G/N)\leq n_p(G)$. 

\begin{corollary}
Si un grupo finito $G$ tiene un único $p$-subgrupo de Sylow para algún
primo $p$, entonces todo subgrupo y todo cociente de $G$ también tienen un
único $p$-subgrupo de Sylow.
\end{corollary}

\begin{proof}
	Si $H$ es un subgrupo de $G$, entonces $n_p(H)\leq n_p(G)=1$. Si $N$ es un subgrupo
	normal de $G$, entonces $n_p(G/N)\leq n_p(G)=1$. 	
\end{proof}

%Veamos una aplicación. 
%
%\begin{theorem}[Wilson]
%	Sea $n\in\N$. Entonces $n$ es primo si y sólo si
%	$(n-1)!\equiv -1\bmod n$.
%\end{theorem}
%
%\begin{proof}
%	Sea $p$ un número primo. 
%	El grupo $\Sym_p$ tiene $(p-1)!$ elementos de orden $p$. Cada 
%	$p$-subgroupo de Sylow de $\Sym_p$ está generado por un $p$-ciclo, y luego
%	$n_p\equiv (p-2)!$. Por el tercer teorema de Sylow,
%	$(p-2)!=n_p\equiv 1\bmod p$. Al multiplicar por $p-1$, tenemos
%	$(p-1)!\equiv -1\bmod p$. 
%\end{proof}
%
%

%\chapter{El teorema de Burnside}

\begin{theorem}[Burnside]
\index{Teorema!de Burnside}
Sea $G$ un grupo finito que actúa en un conjunto $X$. Si $m$ es la cantidad de órbitas, entonces
\[
m=\frac{1}{|G|}\sum_{g\in G}|Fix(g)|,
\]
donde $Fig(g)=\{x\in X:g\cdot x=x\}$. 	
\end{theorem}

Veamos algunas aplicaciones.

\begin{theorem}[Jordan]
	
\index{Teorema!de Jordan}
\end{theorem}

\begin{corollary}
	Si $G$ es un grupo finito y $H$ es un subgrupo propio de $G$, entonces $G\ne\cup_{g\in G}gHg^{-1}$.
\end{corollary}


\chapter{El teorema de Jordan--Hölder}

\begin{definition}
\index{Serie de composición}	
\index{Factores!de una serie de composición}
\index{Longitud!de una serie de composición}
Una sucesión de subgrupos $G=G_0\supseteq G_1\supseteq\cdots\supseteq G_r=\{1\}$ de un grupo $G$ es una
\textbf{serie de composición} si cada $G_{i+1}$ es normal en $G_i$ y cada cociente 
$G_i/G_{i+1}$ es simple. En ese caso, el entero $r$ es la \textbf{longitud} de la serie de composición
y los cocientes son los \textbf{factores} de la serie de composición. 
\end{definition}

\begin{example}
$\Sym_5\supseteq\Alt_5\supseteq\{1\}$ es una serie de composición de $\Sym_5$.	
\end{example}

\begin{example}
$\Z$ no admite una serie de composición pues $\Z$ no es simple y cada subgrupo $S$ de $\Z$ cumple que 
$S\simeq n\Z\simeq\Z$.  	
\end{example}

Vimos en el ejemplo anterior que no todo grupo admite una serie de composición. Sin embargo, todo grupo
finito sí lo hará. Esa será nuestra primera observación. 

% todo: ejercicio, N maximal-normal <=> G/N simple

\begin{proposition}
Si $G$ es un grupo finito, entonces $G$ admite una serie de composición.	
\end{proposition}

\begin{proof}
Procederemos por inducción en el orden de $G$. 
Si $G$ es simple, entonces $G\supseteq\{1\}$ es una serie de composición. Si $G$ no es simple, $G$ contiene un subgrupo normal propio $N\ne\{1\}$, que además puede tomarse maximal sobre los subgrupos normales de $G$. La maximal de $N$ entro los normales de $G$ implica que $G/N$ es un grupo simple. Por hipótesis inductiva, $N$ admite una serie de composición
\begin{gather*}
N=N_0\supseteq N_1\supseteq\cdots\supseteq N_r=\{1\}
\shortintertext{y entonces}
G\supseteq N=N_0\supseteq N_1\supseteq\cdots\supseteq N_r=\{1\}
\end{gather*}
es una serie de composición para $G$ pues $G/N$ es simple.
\end{proof}

\begin{definition}
\index{Equivalencia!de series de composición}
Dos series de composición
\begin{align*}
G=G_0\supseteq G_1\supseteq\cdots\supseteq G_r=\{1\}, &&
G=H_0\supseteq H_1\supseteq\cdots\supseteq H_r=\{1\}	
\end{align*}
de un grupo $G$ se dirán \textbf{equivalentes} si existe $\sigma\in\Sym_r$ tal que 
\[
G_{i-1}/G_i\simeq H_{\sigma(i)-1}/H_{\sigma(i)}
\]
para todo $i\in\{1,\dots,n\}$.  
\end{definition}

\begin{example}
Sea $G=\langle x\rangle\simeq\Z/6$. Las series de composición
\[
G\supseteq\langle x^2\rangle=G_1\supseteq\{1\},\quad
G\supseteq\langle x^3\rangle=H_1\supseteq\{1\}
\]
son equivalentes por la permutación $\sigma=(12)\in\Sym_2$. Observar que $G/G_1\simeq\Z/3$ y $G/H_1\simeq\Z/2$. 	
\end{example}

Nuestro objetivo es demostrar el teorema de Jordan--Hölder, que afirma que todo grupo que admita una serie de composición, tendrá esencialmente 
una única serie de composición módulo equivalencia. Antes de ir directamente al teorema, necesitamos un resultado previo. 

\begin{lemma}
Sea $G=G_0\supseteq G_1\supseteq \cdots\supseteq G_r=\{1\}$ una serie de composición para $G$ y sea $N$ un subgrupo normal de $G$. Entonces $N$ 
también admite una serie de composición.	
\end{lemma}

\begin{proof}
Para cada $i$, sea $N_i=G_i\cap N$. Como ejercicio, se demuestra que  
\[
N=N_0\supseteq N_1\supseteq\cdots\supseteq N_r=\{1\}
\]
y que además $N_{i+1}$ es normal en $N_i$ para todo $i$. Como
\[
N\cap G_{i+1}=N\cap (G_i\cap G_{i+1})=(N\cap G_i)\cap G_{i+1}
\]
para todo $i$, entonces
\[
\frac{N_i}{N_{i+1}}=\frac{N\cap G_i}{N\cap G_{i+1}}=\frac{N\cap G_i}{(N\cap G_i)\cap G_{i+1}}
\simeq \pi(N\cap G_i)%\unlhd\pi(G_i)=\frac{G_i}{G_{i+1}}.
\]
donde $\pi\colon G_i\to G_i/G_{i+1}$ es el morfismo canónico. En efecto, la restricción $\pi|_{N_i}$ tiene núcleo
$(N_i\cap G_i)\cap G_{i+1}=(N\cap G_{i+1})$ y entonces, por el primer teorema de isomorfismos, $\pi(N_i)=\pi(N\cap G_i)\simeq (N\cap G_i)/(N\cap G_{i+1})$.   
Como $\pi(N\cap G_i)$ es un subgrupo normal del grupo simple $\pi(G_i)=G_i/G_{i+1}$, se sigue que $N_i=N_{i+1}$ o bien $N_i/N_{i+1}=G_i/G_{i+1}$. Luego de remover
las posibles repeticiones obtememos entonces una serie de composición para $N$.   	
\end{proof}

Ahora sí, el teorema.

\begin{theorem}[Jordan--Hölder]
\index{Teorema!de Jordan--Hölder}
Si $G$ es un grupo que admite una serie de composición, entonces todas las series de composición de $G$ tienen la misma longitud y son además equivalentes. 	
\end{theorem}

% Falta la referencia al ejercicio

\begin{proof}
Sean 
\begin{align*}
G=G_0\supseteq G_1\supseteq\cdots\subseteq G_r=\{1\}, &&
G=H_0\supseteq H_1\supseteq\cdots\supseteq H_s=\{1\}	
\end{align*}
dos series de composición de $G$. Procederemos por inducción en $r$. Si $r=1$, entonces $G$ es simple y el teorema queda demostrado trivialmente. Si $r>1$, 
supongamos que el resultado vale para todos los grupos que admiten una series de composición de longitud $<r$. 

Si $G_1=H_1$, entonces $G_1$ admite dos series
de longitudes $r-1$ y $s-1$, respectivamente, y entonces, por hipótesis inductiva, $r=s$ y además las series de composición son equivalentes.  

Si $G_1\ne H_1$, como $G_1$ y $H_1$ son ambos normales en $G$, entonces $G_1H_1$ es también normal en $G$ (esto es un ejercicio que dejamos en la página ??). Como
$G/G_1$ es simple y $G_1\unlhd G_1H_1\unlhd G$, entonces $G_1=G_1H_1$ o bien $G_1H_1=G$, pues $G_1$ es maximal entre todos los subgrupos normales de $G$. Como $G/H_1$ es simple, $H_1$ es maximal entre todos los subgrupos normales de $G$, y entonces $H_1=G_1H_1$ o bien $G_1H_1=G$. En conclusión, $G_1H_1=G$. Sea $K=G_1\cap H_1$.
Entonces $K$ es normal en $G$ y además
\[
G/G_1=\frac{G_1H_1}{G_1}\simeq H_1/K,
\quad
G/H_1=\frac{G_1H_1}{H_1}\simeq G_1/K.
\]
Luego $H_1/K$ y $G_1/K$ son ambos grupos simples. Gracias al lema anterior sabemos que $K$ también admite una serie de composición, digamos
\[
K=K_0\supseteq K_1\supseteq\cdots\supseteq K_t=\{1\}.
\]
Luego 
\[
G_1\supseteq G_2\supseteq\cdots\supseteq G_r=\{1\},
\quad
G_1\supseteq K=K_0\supseteq K_1\supseteq\cdots\supseteq K_t=\{1\},
\]
son ambas series de composición para $G_1$. La hipótesis inductiva implica entonces que $r-1=t+1$, es decir $t=r-2$, y además que estás series de composición son equivalentes. Similarmente, 
\[
H_1\supseteq H_2\supseteq\cdots\supseteq H_s=\{1\},\quad
H_1\supseteq K=K_0\supseteq K_1\supseteq\cdots\supseteq K_t=\{1\}
\]
son series de composición para $H_1$ y además $s-1=t+1=r-1$, lo que implica que $r=s$. Como las series de composición 
\[
G=G_0\supseteq G_1\supseteq K_0\supseteq\cdots\supseteq K_t=\{1\},\quad
G=H_0\supseteq H_1\supseteq K_0\supseteq\cdots\supseteq K_t=\{1\},
\]
son equivalentes, las series
\[
G=G_0\supseteq G_1\supseteq\cdots\supseteq G_r=\{1\},
\quad
G=H_0\supseteq H_1\supseteq\cdots\supseteq H_r=\{1\},
\]
también son equivalentes. 
\end{proof}

Veamos un corolario muy simpático.

\begin{corollary}[teorema fundamental de la aritmética]
	Los primos y sus multiplicidades que aparecen en la factorzación de un entero $n\geq2$ están unívocamente determinados por $n$.
\end{corollary}

\begin{proof}
Escribamos $n=p_1\cdots p_k$, donde los primos $p_1,\dots,p_k$ no son necesariamente distintos. Si $G=\langle x\rangle\simeq\Z/n$ es cíclico de orden $n$, entonces
\[
G=\langle x\rangle\supseteq \langle x^{p_1}\rangle\supseteq \langle x^{p_1p_2}\rangle\supseteq\cdots\supseteq \langle x^{p_1\cdots p_{k-1}}\rangle\supseteq\{1\}
\]
es una serie de composición con factores de orden $p_1p_2,\dots,p_k$, respectivamente. Si $n=q_1\dots q_l$ es otra factorización de $n$ como producto de primos, entonces
\[
G=\langle x\rangle\supseteq \langle x^{q_1}\rangle\supseteq \langle x^{q_1q_2}\rangle\supseteq\cdots\supseteq \langle x^{q_1\cdots q_{l-1}}\rangle\supseteq\{1\}
\]
es también una serie de composición. Por el teorema de Jordan--Hölder, $k=l$ y además las series son equivalentes, algo que se traduce en poder reordenar los primos
$q_1,\dots,q_k$ y obtener como $p_1,\dots,p_k$. 
\end{proof}


 
\chapter{Grupos resolubles}

Una forma de atacar ciertos aspectos de la estrucura de los grupos se basa en estudiar propiedades de los factores que aparecen en las series de composición. El caso que estudiaremos en este capítulo involucra factores abelianos. Nos concentraremos en el caso de los grupos finitos. 

\begin{definition}
\index{Grupo!resoluble}
Diremos que un grupo finito $G$ es \textbf{resoluble} si los factores de su serie de composición son abealianos. 	
\end{definition}

La definición anterior trivialmente implica que todo grupo abeliano finito será resoluble. En cambio, grupos simples finitos no abelianos no serán resolubles. 

\begin{example}
Si $p$ es primo, $\D_p$ es resoluble pues $\D_p\supseteq\langle r\rangle\supseteq\{1\}$ es una serie de composición. 	
\end{example}

\begin{example}
El grupo simétrico $\Sym_3$ es resoluble pues $\Sym_3\supseteq\Alt_3\supseteq\{\id\}$ es una serie de composición.	
\end{example}

\begin{example}
	Si $n\geq5$ el grupo simétrico $\Sym_n$ no es resoluble. Una serie de composición para $\Sym_n$ es $\Sym_n\supseteq\Alt_n\supseteq\{\id\}$. 
\end{example}

\begin{example}
Todo grupo $G$ de orden 20 es resoluble. Por el primer teorema de Sylow sabemos que existe $P\in\Syl_2(G)$, es decir
$|P|=4$. Por el teorema de Cauchy sabemos que existe $x\in P$ tal que $x$ tiene orden 2. Entonces $G$ es resoluble pues la serie de composición  
$G\supseteq P\supseteq\langle x\rangle\supseteq \{1\}$ tiene factores abelianos. 
\end{example}

Vamos a dar una caracterización de la resolubilidad de un grupo $G$. Para eso definimos la siguiente sucesión de conmutadores:
\[
G^{(0)}=G,\quad
G^{(k+1)}=[G^{(k)},G^{(k)}]
\]
para $k\geq0$. La sucesión $G=G^{(0)}\supseteq G^{(1)}\supseteq\cdots$ se conoce como la \textbf{serie derivada} de $G$ o sucesión de conmutadores de $G$. 

\begin{theorem}
Sea $G$ un grupo finito. Las siguientes afirmaciones son equivalentes:
\begin{enumerate}
	\item $G$ es resoluble.
	\item $G$ admite una sucesión $G=G_0\supseteq G_1\supseteq\cdots\supseteq G_n=\{1\}$ de subgrupos tal que $G_i\unlhd G_{i-1}$ para todo $i$ y además $G_{i-1}/G_i$ es abeliano para todo $i$. 
	\item $G^{(n)}=\{1\}$ para algún $n\in\N$. 
\end{enumerate}	
\end{theorem}

\begin{proof}
	La implicación $(1)\implies(2)$ es trivial, solamente basta con utilizar una serie de composición para el grupo.
	
	Demostremos ahora que $(2)\implies(3)$. Veamos por inducción que $G^{(i)}\subseteq G_i$ para todo $i\geq0$. El caso $i=0$ es trivial. Si el resultado
	vale para algún $i\geq0$, entonces, como $G_i/G_{i+1}$ es abeliano, $[G_i,G_i]\subseteq G_{i+1}$ y luego
	\[
	G^{(i+1)}=[G^{(i)},G^{(i)}]\subseteq [G_i,G_i]\subseteq G_{i+1}.
	\]
	En particular, $G^{(n)}\subseteq G_n=\{1\}$.   
	
	La implicación $(3)\implies(2)$ es trivial.
	
	Por último, demostraremos que $(2)\implies(1)$. Sea 
	\begin{equation}
	\label{eq:sucesion}
		G=H_0\supseteq H_1\supseteq\cdots\supseteq H_n=\{1\}		
	\end{equation}
	una sucesión de subgrups de $G$ tal que
	$H_i\unlhd H_{i-1}$ para todo $i$ y además $H_{i-1}/H_i$ es abeliano para todo $i$, donde $n$ se tomará lo mayor posible. Entonces cada
	cociente $H_{i-1}/H_i$ es simple, pues de lo contrario existirá $N\unlhd H_{i-1}$ tal que $H_i\subsetneq N\subsetneq H_{i-1}$ 
	y la sucesión
	\[
	G=H_0\supseteq H_1\supseteq\cdots\supseteq H_{i-1}\supseteq N\supseteq H_i\supseteq\cdots\supseteq H_n=\{1\}
	\]
	tendrá longitud $>n$, ya que $N/H_i$ es abeliano por ser un subgrupo de $H_{i-1}/H_i$ y 
	\[
	H_{i-1}/N\simeq \frac{H_{i-1}/H_i}{N/H_i}
	\]
	es también abeliano,
	una contradicción. En conclusión, la sucesión~\eqref{eq:sucesion} es una serie de composición con factores abelianos. 
\end{proof}

Un grupo infinito se dirá resoluble si se satisface el segundo o tercer ítem del teorema anterior. El teorema siguiente ya no requiere la finitud del grupo $G$.

\begin{theorem}
Sea $G$ un grupo y $H$ un subgrupo de $G$.
\begin{enumerate}
	\item Si $G$ es resoluble, $H$ es resoluble.
	\item Si $K\unlhd G$. Entonces $G$ es resoluble si y sólo si $K$ y $G/K$ son resolubles. 
\end{enumerate}	
\end{theorem}

\begin{proof}
La primera afirmación se obtiene de la inclusión $H^{(i)}\subseteq G^{(i)}$ para todo $i\geq0$. Para demostrar la segunda, sea $Q=G/K$ y sea
$\pi\colon G\to Q$ el morfismo canónico. 

Afirmamos que $\pi(G^{(i)})=Q^{(i)}$ para todo $i\geq 0$. El caso $i=0$ es fácil pues $\pi$ es sobreyectivo. Si el resultado vale para un cierto $i\geq0$, entonces
\[
\pi(G^{(i+1)})=\pi([G^{(i)},G^{(i)}])=[\pi(G^{(i)}),\pi(G^{(i)})]=[Q^{(i)},Q^{(i)}]=Q^{(i+1)}
\]

Supongamos primero que $K$ y $Q$ son resolubles. Entonces que $Q^{(n)}=\{1\}$ para algún $n\in\N$. Como $\pi(G^{(n)})=Q^{(n)}=\{1\}$, entonces $G^{(n)}\subseteq K=\ker\pi$. Como $K$ es resoluble, 
existe $m\in\N$ tal que $K^{(m)}=\{1\}$. Luego
\[
G^{(n+m)}\subseteq \left(G^{(n)}\right)^{(m)}\subseteq K^{(m)}=\{1\}.
\]

Supngamos ahora que $G$ es resoluble, es decir $G^{(n)}=\{1\}$ para algún $n\in\N$. Entonces $Q^{(n)}=\pi(G^{(n)})=\pi(\{1\})=\{1\}$ y luego $Q$ es resoluble. La resolubiildad 
del subgrupo $K$ se obtiene a partir del primer ítem del teorema.
\end{proof}

La siguiente proposición nos da muchos ejemplos de grupos resolubles.

\begin{proposition}
Sea $p$ un número primo. Si $G$ es un $p$-grupo, entonces $G$ es resoluble.
\end{proposition}

\begin{proof}
Procederemos por inducción en el order de $G$. Si $|G|=1$, el resultado es trivialmente cierto. Supongamos entonces que la proposición vale para todos los $p$-grupos de tamaño $<|G|$. Si $G$ es abeliano, $G$ es resoluble. De lo contrario, $1<|Z(G)|<|G|$ pues $Z(G)\ne\{1\}$ porque $G$ es un $p$-grupo. Como $Z(G)$ es resoluble (por ser abeliano) y $G/Z(G)$ es resoluble (por hipótesis inductiva), $G$ es resoluble gracias al teorema anterior. 
\end{proof}



\part{Anillos}
\chapter{Anillos}

En esta parte trataremos sobre algunas propiedades básicas de ciertas
estructuras que se conocen como anillos conmutativos. Si bien los anillos no conmutativos tienen gran importancia dentro de la matemática, 
las aplicaciones que daremos en este curso, estarán basadas en la teoría de anillos conmutativos.

\begin{definition}
Un anillo es un conjunto $R$ con dos operaciones binarias, una suma $(x,y)\mapsto x+y$ y un producto $(x,y)\mapsto xy$ de forma tal que  
se cumplen las siguientes propiedades:
\begin{enumerate}
\item $(R,+)$ es un grupo abeliano (escrito aditivamente).
\item $(xy)z=x(yz)$ para todo $x,y,z\in R$.
\item Existe $e\in R$ tal que $xe=ex=x$ para todo $x\in R$.
\item $x(y+z)=xy+xz$ para todo $x,y,z\in R$.
\item $(x+y)z=xz+yz$ para todo $x,y,z\in R$.
\end{enumerate}
\end{definition}

\begin{definition}
Un anillo $R$ se dirá \textbf{conmutativo} si $xy=yx$ para todo $x,y\in R$.
\end{definition}

\begin{examples}
$\N$ no es un anillo. En cambio, 
$\Z$, $\Q$, $\R$ y $\C$ sí son anillos conmutativos.
\end{examples}

\begin{example}
$\Z/n$ es un anillo conmutativo.
\end{example}

\begin{example}
Si $R$ es un anillo, entonces 
\[
R[X]=\left\{\sum_{i=0}^n a_iX^i:n\in\N_0,\,a_0,\dots,a_n\in R\right\}
\]
es un anillo con las operaciones usuales.    	
\end{example}

Como dijimos al principio del capítulo, los anillos no conmutativos tienen gran importancia dentro de la matemática. Para convencernos, 
tenemos el siguiente ejemplo. Si $n\geq2$, entonces 
$M_n(\R)=R^{n\times n}$ es un anillo no conmutativo con las operaciones usuales. 

\begin{exercise}
Si $A$ es un grupo abeliano, entonces $\End(A)$ es un anillo conmutativo con 
las operaciones
\[
(f+g)(x)=f(x)+g(x),\quad
(fg)(x)=f(g(x)).
\]	
\end{exercise}

Si $R$ es un anillo, es fácil verificar las siguientes propiedades:
\begin{enumerate}
	\item $x0=0x=0$ para todo $x\in R$.
	\item $x(-y)=-xy$ para todo $x,y\in R$.
	\item Si $1=0$, entonces $|R|=1$. Este anillo se conoce como el \textbf{anillo nulo}.
\end{enumerate}

Si $R$ es un anillo y $S$ es un subconjunto de $R$, diremos que 
$S$ es cerrado por multiplicación si $S\cdot S\subseteq S$, 
es decir $st\in S$ si $s,t\in S$. 

\begin{definition}
\index{Subanillo}
Sea $R$ un anillo. 
Un \textbf{subanillo} de $R$ es un subconjunto $S$ de $R$ tal que $(S,+)$ es un subgrupo abeliano de $(R,+)$, $S$ es cerrado por multiplicación y además $1\in S$. 
\end{definition}

\begin{examples}\
\begin{enumerate}
\item $\Z\subseteq \Q\subseteq\R\subseteq\C$ son subanillos. 
\item $\Z$ es un subanillo de $\Z$.
\item $\Z[i]=\{a+bi:a,b\in\Z\}$ es un subanillo de $\C$. 
\item $\Q[\sqrt{2}]=\{a+b\sqrt{2}:a,b\in\Q\}$ es un subanillo de $\R$.
\end{enumerate}
\end{examples}

\begin{example}
\index{Centro!de un anillo}
Si $R$ es un anillo, $Z(R)=\{x\in R:xy=yx\text{ para todo $y\in R$}\}$ es un subanillo de $R$. Se denomina el \textbf{centro} de $R$.
\end{example}

\begin{exercise}
Si $S$ es un subanillo de $R$, entonces $0_S=0_R$. 
Además si $x\in S$, el inverso aditivo de $x$ en $S$ coincide con el inverso aditivo de $x$ en $R$.	
\end{exercise}

\begin{exercise}
Si $S$ y $T$ son subanillos de $R$, entonces $S\cap T$ es también un subanillo de $R$.	
\end{exercise}

El resultado del ejercicio anterior puede generalizarse a una intersección arbitraria de subanillos. 

\begin{exercise}
Si $R_1\subseteq R_2\subseteq\cdots$ es una sucesión de subanillos de un anillo $R$, entonces $\cup_{i\geq1}R_i$ es un subanillo de $R$.	
\end{exercise}

\begin{definition}
\index{Unidad!de un anillo}
Sea $R$ es un anillo. Un elemento $x\in R$ es una \textbf{unidad} si existe $y\in R$ tal que $xy=yx=1$.   	
\end{definition}

Si un elemento $x$ es una unidad, entonces el inverso $y$ tal que $xy=yx=1$ es único. Esto nos permite 
escribir $y=x^{-1}$.  
El \textbf{grupo de unidades} de $R$ se define como
\[
\mathcal{U}(R)=\{x\in R:x\text{ es una unidad}\}.
\]

En anillos no conmutativos conviene distinguir unidades, unidades a derecha y unidades a izquierda, ya que todos
estos conceptos son diferentes. Veamos un ejemplo. Sea $R=\End(V)$, donde $V$ es un espacio vectorial
con base $e_1,e_2,\dots$. Sea $f\in R$ tal que $f(e_i)=e_{i+1}$ para todo $i$ y sea $g\in R$ tal que
\[
g(e_i)=\begin{cases}
0 & \text{si $i=1$},\\
e_{i-1} & \text{si $i>1$}.	
\end{cases}
\]
Entonces $g\circ f=\id$ pero $f\circ g\ne\id$ pues $f(g(e_1))=f(0)=0$. 
Luego $f$ no es una unidad. En caso contrario, existe $h\in R$ tal que $f\circ h=h\circ f=\id$ y entonces
\[
g=g\circ\id=g\circ (f\circ h)=(g\circ f)\circ h=\id\circ h=h,
\]
una contradicción. 

\begin{definition}
\index{Anillo!de división}
Diremos que un anillo $R$ es de \textbf{división} si $\mathcal{U}(R)=R\setminus\{0\}$.  	
\end{definition}

\begin{example}
Sea 
\[
R=\R+\R i+\R j+\R k=\{a+bi+cj+dk:a,b,c,d\in\R\}
\]
Entonces $R$ es un anillo de división no conmutativo con la suma usual y 
la multiplicación inducida por 
$i^2=j^2=k^2=-1$, $ij=k$, $jk=i$, $ki=j$. 
\end{example}

\begin{definition}
\index{Cuerpo}
Un \textbf{cuerpo} es un anillo de división conmutativo. 	
\end{definition}

\begin{examples}
$\Q$, $\R$ y $\C$ son cuerpos. Si $p$ es un número primo, entonces $\Z/p$ es un cuerpo.	
\end{examples}

\begin{example}
$\Q[\sqrt{2}]=\{x+\sqrt{2}y:x,y\in\Q\}$ es un cuerpo con las operaciones usuales. 	
\end{example}







\chapter{Ideales}

\begin{definition}
\index{Ideal!a izquierda}
\index{Ideal!a derecha}
Sea $R$ un anillo. Un subconjunto $I$ de $R$ es un \textbf{ideal a izquierda} de $R$ si 
$(I,+)\leq (R,+)$ y además $RI\subseteq I$ (es decir $xu\in I$ para todo $x\in R$ y $u\in I$).  	
\end{definition}

Análogamente pueden definirse ideales a derecha, simplemente hay que reemplazar la condición $RI\subseteq I$ por $IR\subseteq I$.

\begin{example}
Si $R=M_{2}(\mathbb{R})$, entonces  
\[
I=\left(\begin{array}{cc}
\mathbb{R} & \mathbb{R}\\
0 & 0
\end{array}\right)=\left\{ \left(\begin{array}{cc}
x & y\\
0 & 0
\end{array}\right):x,y\in\mathbb{R}\right\} 
\]
es un ideal a derecha de $R$ que no es un ideal a izquierda. 
\end{example}

Dejamos como ejercicio encontrar un ideal a izquierda de $M_{2}(\mathbb{R})$ que no sea un ideal a derecha.

\begin{definition}
\index{Ideal}
\index{Ideal!bilátero}
Sea $R$ un anillo. Un \textbf{ideal} es un subconjunto $I$ de $R$ tal que $I$ es simultáneamente un ideal a derecha y a izquierda. 
\end{definition} 

Los ideales de la definición anterior también se llaman \textbf{ideales biláteros}.

\begin{example}
Si $g(X)\in\mathbb{R}[X]$, entonces el conjunto 
\[
(g(X))=\{f(X)g(X):f(X)\in\mathbb{R}[X]\}
\]
de múltiplos de $g(X)$ es un ideal de $\mathbb{R}[X]$. 
\end{example}

Dejamos como ejercicio verificar los siguientes ejemplos. 
 
\begin{examples}\
\begin{enumerate}
\item $\{0\}$ y $R$ son siempre ideales de $R$.
\item $n\Z$ es un ideal de $\Z$.
\item Si $\{I_\lambda:\lambda\}$ es una colección de ideales de $R$, entonces $\cap_{\lambda}I_\lambda$ es un ideal de $R$.
\item Si $I_1\subseteq I_2\subseteq\cdots$ es una sucesión de ideales de $R$, entonces $\cup_{i\geq1}I_i$ es un ideal de $R$.
\end{enumerate}	
\end{examples}

\begin{exercise}
\label{xca:PcapQ}
    Sean $R$ un anillo conmutativo y $J$, $P$ y $Q$ ideales de $R$. Pruebe
    que si $J\subseteq P\cup Q$ entonces $J\subseteq P$ o bien $J\subseteq Q$.
\end{exercise}

Tal como hicimos para grupos, pueden definirse ideales generados por un subconjunto del anillo $R$. Por ejemplo, 
podemos definir el ideal a izquierda generado por el subconjunto $X$ como
\[
(X)_L = \bigcap\{I:X\subseteq I,\,I\text{ ideal a izquierda}\}.
\]
Análogamente se define el ideal a derecha $(X)_R$ generado por $X$.  
Puede demostrarse que
\[
(X)_L=\left\{\sum_{i=1}^nr_ix_i:n\in\N_0,\,r_1,\dots,r_n\in R,\,x_1,\dots,x_n\in X\right\}. 
\]
De la misma forma podemos definir el ideal $(X)$ generado por $X$. En este caso, 
\[
(X) = \left\{\sum_{i=1}^nr_ix_is_i:n\in\N_0,\,r_1,\dots,r_n,s_1,\dots,s_n\in R,\,x_1,\dots,x_n\in X\right\}
\]

\begin{example}
Veamos que los ideales de $\Z/n$ son de la forma $(d)$ con $d$ un divisor de $n$. En efecto, si 
$d$ es un divisor de $n$, entonces $(d)$ es un ideal de $\Z/n$. Demostremos entonces
que todo ideal $I$ de $\Z/n$ es de la forma $(d)$ para algún divisor $d$ de $n$. Como
$I$ es en particular un subgrupo del grupo aditivo de $\Z/n$, el teorema de Lagrange nos dice que 
$m=|I|$ es un divisor de $n$. Como además el grupo
aditivo de $\Z/n$ es cíclico, digamos $\Z/n=\langle 1\rangle$, el grupo aditivo de $I$ también será cíclico, digamos
$I=\langle n/m\rangle$. Luego $I=(n/m)$, donde $d=n/m$ es un divisor de $n$.  
\end{example}

Usaremos ideales para poder definir cocientes. 
Antes de proceder a explicar esta construcción es 
conveniente repasar nociones básicas sobre morfismos de anillos. 

\begin{definition}
\index{Morfismo!de anillos}
Si $R$ y $S$ son anillos, diremos que una función $f\colon R\to S$ es un morfismo de anillos si $f(1)=1$, 
$f(x+y)=f(x)+f(y)$ y $f(xy)=f(x)f(y)$ para todo $x,y\in R$. 	
\end{definition}

Si $f\colon R\to S$ es una función tal que $f(x+y)=f(x)+f(y)$ y $f(xy)=f(x)f(y)$ para todo $x,y\in R$, 
no es necesariament cierto que $f(1)=1$. Se pide esta condición en la definición para evitar patologías. Veremos 
otra posible explicación en el capítulo~\ref{modulos}.

\begin{example}
Sea $f\colon\Z/6\to\Z/6$, $f(x)=3x$. Un cálculo sencillo muestra que $f(x+y)=f(x)+f(y)$ y 
$f(xy)=f(x)f(y)$ para todo $x,y\in\Z/6$, pero $f(1)=3\ne 1$.	
\end{example}

\index{Núcleo!de un morfismo de anillos}
Tal como hicimos en el caso de grupos, podemos definir el núcleo de un morfismo $f\colon R\to S$ como
\[
\ker f=\{x\in R:f(x)=0\}.
\]
Queda como ejercicio demostrar que $\ker f$ es un ideal de $R$. Además $f$ es inyectivo si y sólo si $\ker f=\{0\}$.  

\begin{examples}
Las siguientes funciones son ejemplos de morfismos de anillos:
\begin{enumerate}
	\item La identidad $\id\colon R\to R$.
	\item Las inclusiones $\Q\hookrightarrow\R\hookrightarrow\C$.
	\item $\Z\to R$, $k\mapsto k1_R$.
	\item La evaluación $\operatorname{ev}_{x_0}\colon R[X]\to R$, $f\mapsto f(x_0)$, donde $x_0\in R$ es un elemento fijo. 
\end{enumerate}
\end{examples}

\begin{example}
La función $f\colon\Z[i]\to\Z/5$, $f(a+bi)=a+2b\bmod 5$, es un morfismo de anillos.	El núcleo de $f$
es el conjunto
\[
\ker f=\{a+bi:a+2b\text{ es divisible por 5}\}.
\] 
\end{example}

\begin{exercise}
\label{xca:Zsqrtd}
Sea $R$ un anillo conmutativo y sea $d\in\Z$ libre de cuadrados.  
Demuestre que $\Hom(\Z[\sqrt{d}],R)$ está en biyección con $\{r\in R:r^2=d1_R\}$. 
\end{exercise}

El ejercicio anterior nos dice por ejemplo que $\Hom(\Z[i],\Z)=\emptyset$. 

\begin{example}
La función $\C\to\R^{2\times 2}$, $a+bi\mapsto\begin{pmatrix}a&b\\-b&a\end{pmatrix}$, es un morfismo inyectivo  
de anillos. 
\end{example}

La idea
del ejemplo anterior nos permite también encontrar 
un morfismo de anillos 
$\Z[i]\to\Z^{2\times 2}$.  	

\begin{example}
La función $\Z\to 2\Z$, $x\mapsto 2x$, es un morfismo de grupos abelianos pero no es un morfismo de anillos. 	
\end{example}

\begin{example}
Sean $D$ un anillo de división y $R$ un anillo no trivial. Si $f\colon D\to R$ es un morfismo de anillos, entonces, como $f(1)=1\ne0$,
el núcleo $\ker f$ es un ideal propio de $D$. Luego $f$ resulta ser inyectivo.   	 
\end{example}

\begin{exercise}
\label{xca:sqrt2sqrt3}
Demuestre que los anillos $\Q(\sqrt{2})$ y $\Q(\sqrt{3})$ no son isomorfos.	
\end{exercise}

\begin{exercise}
\label{xca:Z6Z15}
Demuestre que no hay morfismos de anillos $\Z/6\to\Z/15$. 	
\end{exercise}

\begin{exercise}
Demuestre que si $f\colon \Q\to\Q$ es un morfismo de anillos, entonces $f=\id$.  	
\end{exercise}

\begin{exercise}
Demuestre que si $f\colon\R[X_1,\dots,X_n]\to\R$ es un morfismo de anillos tal que la restricción $f|_{\R}$ es la identidad, entonces $f=\operatorname{ev}_{p}$ para algún $p\in\R^n$. 	
\end{exercise}

Para definir cocientes de anillos utilizaremos ideales. Si $I$ es
un ideal de $R$, entonces $(I,+)$ es un subgrupo normal de $(R,+)$, pues el grupo aditivo de $R$ es abeliano. Esto implica que
$R/I$ es un grupo abeliano con la operación
\[
(x+I)+(y+I)=(x+y)+I.
\]
Hasta acá, solamente se necesita que $(I,+)$ sea un subgrupo normal de $(R,+)$. Si queremos que $R/I$ sea un anillo, necesitamos
determinar cómo tiene que ser el producto. La 
estrucuturea de anillo sobre $R/I$ tiene que ser tal que el morfismo canónico   
$\pi\colon R\to R/I$, $\pi(x)=x+I$, sea un morfismo de anillos. Esto nos dice que la multiplicación 
tiene que estar dada por
\[
(x+I)(y+I)=(xy)+I.
\]
Veamos cómo demostrar que esta operación está bien definida. 

Usaremos que $I$ es un ideal. 
Sean $x+I=x_{1}+I$ e $y+I=y_{1}+I$. Queremos demostrar que $xy-x_{1}y_{1}\in I$.
Como $x-x_{1}\in I$ y además $I$ es un ideal a derecha, 
\begin{equation}
xy-x_{1}y=(x-x_{1})y\in I.\label{eq:right_ideal}
\end{equation}
Similarmente, como $y-y_{1}\in I$ y además $I$ es un ideal a izquierda,  
\begin{equation}
x_{1}y-x_{1}y_{1}=x_{1}(y-y_{1})\in I.\label{eq:left_ideal}
\end{equation}
Entonces, al combinar las fórmula~\eqref{eq:left_ideal} con~\eqref{eq:right_ideal} y 
que $I$ es un subgrupo aditivo de $R$, obtenemos que 
la multiplicación de $R/I$ está bien definida pues  
\[
xy-x_{1}y_{1}=xy-x_{1}y+x_{1}y-x_{1}y_{1}\in I.
\]

\begin{theorem}
Sea $R$ un anillo. Si $I$ es un ideal de $R$, entonces existe una 
única estructura de anillo en $R/I$ tal que
$\pi\colon R\to R/I$ es un morfismo de anillos sobreyectivo. 	
\end{theorem}

\begin{proof}
Ya sabemos que las operaciones
\[
(x+I)+(y+I)=(x+y)+I,\qquad(x+I)(y+I)=(xy)+I
\]
están bien definidas gracias a que $I$ es un ideal de $R$. Sabemos también que 
$(R/I,+)$ es un grupo abeliano. Nos queda por demotrar entonces que $R/I$ es un anillo, algo que dejaremos como ejercicio.
\end{proof}

\begin{example}
Sea $R=(\Z/3)[X]$ y sea $I=(2X^2+X+2)$. Sabemos que todo $f\in R$ se escribe como 
\[
f=(2X^2+2X+2)q+r,
\]
donde $q,r\in R$ y $r=0$ o bien $\deg r<2$. Podemos escribir entonces $r=aX+b$ para ciertos $a,b\in\Z/3$. Esto
nos dice que entonces
\[
f+I=( (2X^2+2X+2)q+r)+I=(aX+b)+I.
\]	
Como $a,b\in\Z/3$, tenemos entonces nueve posibilidades. Luego $|R/I|=9$. 

Como ejemplo, calculemos $( (2X+1)+I)( (X+1)+I)$. En efecto, si usamos el algoritmo de división, 
\[
(2X+1)(X+1)=2X^2+3X+1=2X^2+1=(2X^2+X+2)\cdot 1+(2X+2)
\]
y luego $(2X^2+1)+I=(2X+2)+I$. 
\end{example}

Valen además los teoremas de isomorfismos. Dado que no hay mucha diferencia entre lo que se hizo en el caso de grupos y lo que debe hacerse en el caso de anillos, 
enunciaremos los teoremas más importantes y dejaremos las demostraciones como ejercicio. 

\begin{theorem}
Sea $f\colon R\to S$ un morfismo de anillos y $I$ un ideal de $R$ tal que $I\subseteq\ker f$. Existe entonces
un único morfismo $\varphi\colon R/I\to S$ tal que el diagrama
\[
        \xymatrix{
        R
        \ar[d]_\pi
        \ar[r]^f
        & S
        \\
        R/I\ar@{-->}[ur]_{\varphi}
        }
\]
es conmutativo, lo que significa que $\varphi\circ\pi=f$, donde $\pi\colon R\to R/I$ es el morfismo canónico. 
Más aún, $\ker\varphi=\ker f/I$ 
y $\varphi(R/I)=f(R)$. En particular, $\varphi$ es inyectiva si y sólo si $\ker f=I$ y $\varphi$ es sobreyectiva si y sólo si $f$ es sobreyectiva. 
\end{theorem}

\begin{proof}
Queda como ejercicio, ya que es muy similar a la demostración hecha en el caso de grupos.
\end{proof}

Como corolario obtenemos:

\begin{corollary}[primer teorema de isomorfismos]
Si $f\colon R\to S$ es un morfismo de anillos, entonces $R/\ker f\simeq f(R)$. 
\end{corollary}

% \begin{proof}
% Es consecuencia inmediata del teorema anterior.
% \end{proof}

Veamos algunas aplicaciones del primer teorema de isomorfismos. 

% todo: hay que tener las definiciones de idales generados!

\begin{example}
Con el morfismo $\R[X]\to\C$, $f\mapsto f(i)$, se demuestra que 
\[
\R[X]/(X^2+1)\simeq\C.
\]  	
\end{example}

\begin{example}
Con el morfismo $\Z[X]\to(\Z/7)[X]$, $\sum_{i=0}^na_iX^i\mapsto \sum_{i=0}^n\overline{a_i}X^i$, donde $\overline{a}$ es $a$ módulo 7, se demuestra que
\[
\Z[X]/(7)\simeq (\Z/7)[X].
\]	
\end{example}

\begin{example}
Sea $R$ el anillo de funciones continuas $[0,2]\to\R$. Veamos que el conjunto $I=\{f\in R:f(1)=0\}$ es un ideal de $R$ y calculemos
el cociente $R/I$. Para ver que $I$ es un ideal
consideramos la evaluación $\varphi\colon R\to\R$, $\varphi(f)=f(1)$. Sabemos que $\varphi$ es un morfismo de anillos y además 
podemos verificar que
\[
\ker\varphi=\{f\in R:\varphi(f)=0\}=\{f\in R:f(1)=0\}=I.
\]
Como $\varphi$ es sobreyectiva (basta tomar funciones constantes), el teorema de isomorfismos implica que $R/I\simeq\R$.  	
\end{example}

\begin{example}
Sea $R$ el anillo de matrices de la forma $\begin{pmatrix}a&b\\0&a\end{pmatrix}$, donde $a,b\in\Q$. La función 
\[
f\colon R\to\Q,
\quad
\begin{pmatrix}a&b\\0&a\end{pmatrix}\mapsto a,
\]
es un morfismo sobreyectivo de anillos tal que $I=\ker f$ es el ideal formado por las matrices de la forma
$\begin{pmatrix}0&b\\0&0\end{pmatrix}$, donde $b\in\Q$. Entonces $R/I\simeq\Q$.  
\end{example}

\begin{example}
Sea $R=\Z[\sqrt{10}]$ y sea 
\[
I=(2,\sqrt{10})=\{a+b\sqrt{10}:a\equiv 0\bmod 2\}.
\]
La función
\[
f\colon R\to\Z/2,\quad
a+b\sqrt{10}\mapsto a\bmod 2,
\]
es un morfismo sobreyectivo tal que $\ker f=I$. Luego $R/I\simeq\Z/2$. 
\end{example}

\begin{example}
Si $I$ es un ideal de $R$, entonces $M_n(I)$ es un ideal de $M_n(R)$ y además 
$M_n(R)/M_n(I)\simeq M_n(R/I)$. Un cálculo sencillo muestra que
$M_n(I)$ es un subgrupo de $M_n(R)$. 
Además si $a=(a_{ij})\in M_n(R)$ y $y\in M_n(I)$, entonces
\[
(ay)_{ij}=\sum_{k=1}^n a_{ik}y_{kj}\in I
\]
para todo $i,j\in\{1,\dots,n\}$. Similarmente vemos que $ya\in M_n(I)$. Sea $\pi\colon R\to R/I$ el morfismo
canónico y sea $\varphi\colon M_n(R)\to M_n(I)$, $(a_{ij})\mapsto (\pi(a_{ij}))$. Entonces
$\varphi$ es un morfismo sobreyectivo de anillos tal que 
\[
\ker\varphi=\{(a_{ij})\in M_n(R):a_{ij}\in I\text{ para todo $i,j\in\{1,\dots,n\}$}\}. 
\]
Por el primer teorema de isomorfismos, $M_n(R)/M_n(I)\simeq M_n(R/I)$. 
\end{example}

\begin{example}
Vamos a demostrar que $\Z[i]/(1+3i)\simeq\Z/10$. Sea $f$ la composición
\[
\Z\hookrightarrow\Z[i]\xrightarrow{\pi} \Z[i]/(1+3i),
\]
donde $\pi$ es el morfimso canónico. Claramente $f$ es morfismo de anillos por ser composición de morfismos. 

Veamos que $f$ es sobreyectiva. Para eso, alcanza con encontrar un entero $b\in\Z$ tal que $f(b)=i$, es decir: queremos $b\in\Z$ 
tal que $b-i\in (1+3i)$. Observemos que
\[
b-i\in(1+3i)\Longleftrightarrow b-i=(1+3i)(x+yi)=(x-3y)+i(3x+y)
\]
para $x,y\in\Z$. Si tomamos $x=1$, $y=-4$ y $b=13$, entonces vemos que
$f$ es sobreyectiva pues $f(a+13b)=f(a)+f(b)f(13)=a+bi$. 

Calculemos ahora el núcleo de $f$. Afirmamos que $\ker f=(10)$. Primero observamos que, como $10=(1+3i)(1-3i)$, entonces
$f(10)=\pi(1+3i)\pi(1-3i)=0$ y luego $(10)\subseteq\ker f$. Recíprocamente, si $m\in\ker f$, entonces $m\in (1+3i)$, es decir
\[
m=(1+3i)(x+iy)=(x-3y)+i(3x+y)
\]
para ciertos $x,y\in\Z$. Pero entonces $3x+y=0$, lo que implica que $y=-3x$ y que entonces $m=x-3(-x)=10x$, es decir $m\in(10)$. 
En conclusión, por el primer teorema de isomorfismos de anillos, $\Z/(1+3i)\simeq\Z/10$. 
\end{example}

\begin{exercise}
Demuestre que $\Z[i]/(2+3i)\simeq\Z/13$. 
\end{exercise}

\begin{exercise}
Demuestre que no existe un ideal $I$ de $\Z[i]$ tal que $\Z[i]/I\simeq\Z/15$.
\end{exercise}

% pues si $f$ es ese isomorfismo, $f(1)=1$, $f(-1)=-1=14$ y $f(-1)=f(i^2)=f(i)^2$, luego
% $m^2=14$ para algún $m\in\Z/15$, una contradicción.  


%\begin{example}
%Vamos a demostrar que $\Z[i]/(2+3i)\simeq\Z/13$. Sea $f$ la composición
%\[
%\Z\hookrightarrow\Z[i]\xrightarrow{\pi} \Z[i]/(2+3i),
%\]
%donde $\pi$ es el morfimso canónico. Claramente $f$ es morfismo de anillos por ser composición de morfismos. 
%Veamos que $f$ es sobreyectiva...	
%\end{example}
%

\begin{theorem}
\index{Teorema!de la correspondencia}
Si $f\colon R\to S$ es un morfismo sobreyectivo de anillos con $K=\ker f$, existe una correspondencia biyectiva entre los ideales de $R$ que contienen a $K$ y los ideales de $S$. 
La correspondencia está dada por $I\mapsto f(I)$ y $f^{-1}(J)\mapsfrom J$. Más aún, si 
$f(I)=J$, entonces $R/I\simeq S/J$.  
\end{theorem}

\begin{proof}[Bosquejo de la demostración]
Hay que demostrar las siguientes afirmaciones:
\begin{enumerate}
\item $f(I)\subseteq S$ es un ideal.
\item $f^{-1}(J)\subseteq R$ es un ideal que contiene a $K$.
\item $f(f^{-1}(J))=J$ y además $f^{-1}(f(I))=I$.
\item Si $f(I)=J$, entonces $R/I\simeq S/J$.	
\end{enumerate}
Las primeras tres afirmaciones quedarán como ejercicio. 
Demostremos (4). Primero obervamos que la tercera afirmación implica que
$f(I)=J$ si y sólo si $I=f^{-1}(J)$. Sea $\pi\colon S\to S/J$ 
el morfismo canónico y sea 
$g=\pi\circ f$. Como
\[
\ker g=\{x\in R:g(x)=0\}=\{x\in R:f(x)\in J\}=\{x\in R:x\in f^{-1}(J)=I\}=I,
\]
existe un morfismo de anilos $h\colon R/I\to S/J$ tal que $h\circ p=g$, donde $p\colon R\to R/I$ es el morfismo canónico. Dejamos como ejercicio verificar que $h$ es biyectivo.
\end{proof}


\begin{definition}
\index{Ideal!principal}
Un ideal $I$ de $R$ se dice \textbf{principal} si existe $x\in R$ tal que $I=(x)$. 
\end{definition}

Análogamente pueden definirse ideales a izquierda principales e ideales a derecha principales. 

\begin{example}
Todo ideal de $\Z$ es principal.	
\end{example}
 
\begin{exercise}
Sea $I$ un ideal (a izquierda) de $R$. Demuestre que $I=R$ si y sólo si existe $x\in I\cap\mathcal{U}(R)$.	
\end{exercise}

El ejercicio anterior nos permite demostrar que un anillo $R$ es de 
división si y sólo si $R$ admite únicamente dos ideales a izquierda.

\begin{remark}
$u\in\mathcal{U}(R)\Longleftrightarrow (u)=R$.
\end{remark}

\chapter{Polinomios}

\index{Polinomio!en una variable}
Sea $R$ un anillo conmutativo. 
Un polinomio con coeficientes en $R$ es una expresión
de la forma
\[
	f=a_nX^n+a_{n-1}X^{n-1}+\cdots+a_1X+a_0=\sum_{i=0}^n a_nX^i,
\]
donde $n\in\N_0$. El conjunto de polinomios en una variable 
con coeficientes en $R$ será denotado por $R[X]$. 

Diremos que $f$ y $g=b_nX^n+b_{n-1}X^{n-1}+\cdots+b_1X+b_0$ son iguales si y
sólo si $ a_i=b_i$ para todo $i\in\{0,1,\dots,n\}$. 

\index{Grado!de un polinomio}
\index{Polinomio!constante}
El grado de un polinomio $f=\sum_{i=0}^na_iX^i$ se define como el menor entero positivo tal que $a_n\ne0$. Un polinomio de grado cero
será denominado \textbf{polinomio constante}.

\index{Polinomio!mónico}
\index{Coeficiente principal!de un polinomio}
Un polinomio $f=\sum_{i=0}^na_iXî$ de grado $n$ se dice \textbf{mónico} si su \textbf{coeficiente principal} es igual a uno, es decir $a_n=1$.   
Si 
\[
f=\sum a_iX^i,\quad
g=\sum b_jX^j,
\]
donde omitimos los índices de la sumatoria para aliviar un poco la notación, se definen la
suma y el producto como
\[
f+g=\sum (a_k+b_k)X^k,
gh=\sum\sum a_ib_jX^{i+j}.
\]
Con estas operaciones, $R[X]$ es un anillo conmutativo. Además $R$ puede pensarse como un subanillo de $R[X]$ pues
existe un morfismo inyectivo de anillos $R\to R[X]$.  

\begin{example}
Si $f$ es un polinomio mónico y $g$ es un polinomio, existen entonces únicos $q\in R[X]$ y $r\in R[X]$ tales que
\[
g=fq+r,
\]
donde $r=0$ o bien $\deg r<\deg f$. Por ejemplo, si
\[
f=X^5+X^4-3X^3+4X^2+2X,\quad
g=X^4+3X^3-X^2-6X-2
\]
entonces $q=X-2$ y $r=4X^3+8X^2-8X-4$ pues $f=(X-2)g+(4X^3+8X^2-8X-4)$. 
\end{example}

El algoritmo de división podrá hacerse siempre que el coeficiente principal de $f$ sea una unidad del anillo $R$. En particular, siempre podremos
utilizar el algoritmo de división cuando el anillo $R$ es en realidad un cuerpo y $f\ne 0$.  

\begin{proposition}
Si $g\in R[X]$ y $\alpha\in R$, el resto de dividir al polinomio $g$ por $X-\alpha$ es $g(\alpha)$. En particular, $X-\alpha$ divide al polinomio  
$g$ en $R[X]$ si y sólo si $g(\alpha)=0$. 
\end{proposition}

\begin{proof}
Hay que evaluar en $\alpha$ la expresión $g=(X-\alpha)q+r$. 	
\end{proof}


\begin{proposition}
\label{pro:polinomios}
Sea $\varphi\colon R\to S$ un morfismo de anillos y sea $\alpha\in S$. Existe un único morfismo de anillos $\Phi\colon R[X]\to S$ 
tal que $\Phi(X)=\alpha$ y tal que $\Phi$ coincide con $\varphi$ en los polinomios constantes.  	
\end{proposition}

\begin{proof}[Bosquejo de la demostración]
Si $f=\sum a_iX^i$, definimos 
\[
\Phi(f)=\Phi(\sum a_iX^i)=\sum\Phi(a_i)\Phi(X)^i=\sum\Phi(a_i)\alpha^i,
\] 	
y así $\Phi$ quedaría unívocamente determinado. Dejamos como ejercicio demostrar que $\Phi$ es un morfismo de anillos. 
%Veamos que $\Phi$ es un morfismo de anillos:
%\begin{align*}
%\Phi&\left((\sum a_iXî)(\sum b_jX^j)\right)
%=\Phi\left(\sum\sum a_ib_jX^{i+j}\right)
%=\sum\sum\Phi(a_ib_jX^{i+j})\\
%&=\sum\sum\Phi(a_i)\Phi(b_j)\alpha^{i+j}
%=\left(\sum\Phi(a_i)\alpha^i\right)\left(\sum\Phi(b_j)\alpha^j\right)
%=\Phi(f)\Phi(g).   	
%\end{align*}
\end{proof}

\begin{example}
Sea $\varphi\colon\R[X]\to\R$, $X\mapsto 2$. Entonces $\ker\varphi$ es el ideal   
$(X-2)$ de $\R[X]$ generado por $X-2$ pues 
\[
\ker\varphi=\{g\in\R[X]:g(2)=0\}
=\{g\in\R[X]:g=(X-2)q\text{ para algún $q\in\R[X]$}\}.
\] 	
\end{example}

El resultado que sigue es análogo al teorema~\ref{thm:Z}.

\begin{theorem}
Sea $K$ un cuerpo. Todo ideal de $K[X]$ es principal. Más aún, todo ideal $I$ no nulo de $K[X]$
está generado por el único polinomio mónico de menor grado que está contenido en $I$.  	
\end{theorem}

\begin{proof}
	Sea $I$ un ideal de $K[X]$. Si $I=\{0\}$, entonces $I$ es principal. Supongamos que $I\ne\{0\}$ 
	y sea $f\in I\setminus\{0\}$ de grado mínimo. Sin perder generalidad, podemos suponer que $f$ es mónico pues
	si no lo fuera, digamos $f=a_nX^n+\cdots$ con $a_n\ne 0$, entonces solamente hay que reemplaar a $f$ 
	por el polinomio $a_n^{-1}f$. 
	
	Veamos que $I=(f)$. Vamos a demostrar la inclusión no trivial. Sea $g\in I$. Escribimos
	$g=fq+r$ para ciertos $q,r\in K[X]$, donde $r=0$ o bien $\deg r<\deg f$. Si $r\ne 0$, entonces $r=g-fq\in I$, 
	una contradicción a la minimalidad del grado de $f$. Luego $r=0$ y entonces $f$ divide a $g$, es decir
	$g\in (f)$.   
\end{proof}


\index{Monomio}
Nos interesará también estudiar anillos de polinomios en varias variables. Un \textbf{monomio} 
en las variables $X_1,\dots,X_n$ es un producto de la forma
\[
X_1^{i_1}X_2^{i_2}\cdots X_n^{i_n}
\]
donde $i_1,\dots,i_n\geq0$. A veces conviene utilizar la siguiente notación: si $i=(i_1,\dots,i_n)$, entonces 
\[
X^i=X_1^{i_1}X_2^{i_2}\cdots X_n^{i_n}.
\]  
El monomio $X^0$, donde $0=(0,0,\dots,0)$, será denotado por $1$. 

Un polinomio en las variables $X_1,\dots,X_n$ 
con coeficientes en $R$ es una combinación lineal (sobre $R$) de finitos 
monomios, es decir
\[
f=f(X_1,\dots,X_n)=\sum a_iX^i,
\]
donde la suma se recorre sobre todos los multi-índices $i=(i_1,\dots,i_n)$, los coeficientes $a_i$ son elementos del anillo $R$
y solamente finitos de esos coeficientes son distintos de cero. El conjunto de los polinomios en las variables
$X_1,\dots,X_n$ con coeficientes en $R$ será denotado por $R[X_1,\dots,X_n]$. 

\begin{exercise}
Demuestre que 
la proposición~\ref{pro:polinomios} vale también en el caso de polinomios en varias variables.
\end{exercise} 

%\index{Polinomio!homogéneo}
%Un polinomio donde todos sus monomios con coeficiente no nulo son de grado $d$ se denomina \textbf{homogéneo}

\begin{example}
Si $R$ es un anillo, $R[X]$ es un anillo. Podemos considerar entonces el anillo de polinomios $R[X,Y]=(R[X])(Y)=R[X][Y]$. 
Sabemos que si utilizamos las identificaciones pertinentes, podemos pensar que $R\subseteq R[X]\subseteq (R[X])[Y]$ son subanillos. Si 
	$\varphi\colon R\to R[X][Y]$ la inclusión, existe entonces un único morfismo de anillos $\Psi\colon R[X,Y]\to R[X][Y]$ que extiende a $\varphi$ y tal que
	$\Psi(X)=X$ y $\Phi(Y)=Y$. Puede demostrarse que $\Psi$ es biyectivo y luego, en consecuencia,
	\[
R[X,Y]\simeq R[X][Y].
\]
\end{example}

\begin{example}
Veamos que $\R[X,Y]/(X)\simeq\R[Y]$. Consideramos el morfismo sobreyectivo $\varphi\colon\R[X,Y]\to\R[Y]$, $\varphi(f(X,Y))=f(0,Y)$. Afirmamos
que $\ker\varphi=(X)$. En efecto, para probar que $\ker\varphi\subseteq (X)$ observamos que si
\[
f(X,Y)=f_0(Y)+f_1(Y)X+\cdots+f_n(Y)X^n,
\]
donde $f_i(Y)\in\R[Y]$ para todo $i\in\{0,1,\dots,n\}$, entonces 
\[
0=\varphi(f(X,Y))=f(0,Y)=f_0(Y).
\] 
Luego $f(X,Y)=f_1(Y)X+\cdots+f_n(Y)X^n=X(f_1(Y)+\cdots+f_n(Y)X^{n-1})\in (X)$. La otra inclusión es trivial. 
El primer teorema de isomorfismos, entonces, implica que $\R[X,Y]/(X)\simeq\R[Y]$.  
\end{example}

\begin{exercise}
Demuestre que $\R[X]/(X^2-1)\simeq\R\times\R$.	
\end{exercise}

\begin{exercise}
Demuestre que $\Q[X]/(X-2)\simeq\Q$. 	
\end{exercise}
	

\chapter{El teorema chino del resto}

En este capítulo veremos una generalización del conocido teorema chino del resto válida para anillos conmutativos. 
Empezaremos con algunas observaciones básicas. 

\begin{definition}
\index{Ideales!coprimos}
Sea $R$ un anillo conmutativo y sean $I$ y $J$ ideales de $R$. Diremos que $I$ y $J$ son \textbf{coprimos}
si $I+J=R$. 	
\end{definition}

La terminología está justificada por la siguiente observación.

\begin{example}
Si $R=\Z$, $I=(a)$ y $J=(b)$, entonces $I$ y $J$ serán coprimos si y sólo si existen $r,s\in\Z$ tales que $ra+sb=1$, es decir si y sólo si $\gcd(a,b)=1$. 	
\end{example}

Si $I$ y $J$ son ideales de $R$, entonces
\[
IJ=\left\{\sum_{i=1}^mu_iv_i:m\in\N_0,\,u_i\in I,\,v_i\in J\right\}
\]
es también un ideal de $R$. Vale además que $IJ\subseteq I\cap J$. La igualdad no siempre vale, tal como nos muestra el siguiente ejemplo.

\begin{example}
Si $R=\Z$ y además $I=J=(2)$, entonces $IJ=(4)\subsetneq (2)=I\cap J$. 	
\end{example}

Sin embargo, si $I$ y $J$ son ideales coprimos de un anillo conmutativo $R$, entonces $IJ=I\cap J$. En efecto, para demostrar la inclusión no trivial, 
sea $x\in I\cap J$. Entonces $1=u+v$ para ciertos $u\in I$ y $v\in J$ pues los ideales son coprimos, y luego 
\[
x=x1=x(u+v)=xu+xv\in IJ. 
\]


\begin{theorem}[teorema china del resto]
\index{Teorema!chino del resto}
Sea $R$ un anillo conmutativo y sean $I$ y $J$ ideales coprimos de $R$. Si $u,v\in R$, existe $x\in R$ tal que
$\pi_I(x)=\pi_I(u)$ y $\pi_J(x)=\pi_J(v)$, donde $\pi_I\colon R\to R/I$ y $\pi_J\colon R\to R/J$ son los morfismos canónicos. 
\end{theorem}

\begin{proof}
	Como $I$ y $J$ son coprimos, existen $a\in I$ y $b\in J$ tales que $1=a+b$. Si $x=av+bu$, entonces 
	\[
	x-u=av+(b-1)u=av-au=a(v-u)\in I.
	\]
	Similarmente, $x-v\in J$. Luego $\pi_I(x-u)=0$ y entonces $\pi_I(x)=\pi_I(u)$. Análogamente, como $\pi_J(x-v)=0$, se tiene que $\pi_J(x)=\pi_J(v)$.   
\end{proof}

Si escribimos 
$x\equiv u\bmod I\Longleftrightarrow x-u\in I$ y 
$x\equiv v\bmod J\Longleftrightarrow x-v\in J$, 
entonces el teorema chino del resto garantinza la existencia de $x\in R$ tal que 
\[
\begin{cases}
x\equiv u\bmod I,\\
x\equiv v\bmod J.
\end{cases}
\]

\begin{corollary}
Si $R$ es un anillo conmutativo y $I$ y $J$ son ideales coprimos de $R$, entonces
$R/(I\cap J)\simeq (R/I)\times(R/J)$.	
\end{corollary}

\begin{proof}
Sea $f\colon R\to (R/I)\times (R/J)$, $f(a)=(\pi_I(a),\pi_J(a))$. Claramente, $f$ es un morfismo tal que $\ker f=I\cap J$. 
Gracias al teorema anterior, $f$ es sobreyectivo. En efecto,
Si $(u+I,v+J)\in R/I\times R/J$, entonces existe $x\in R$ tal que $x-u\in I$ y $x-v\in J$, es decir
$f(x)=(u+I,v+J)$. El primer teorema de isomorfismos concluye la demostración.  	
\end{proof}

Veamos qué pasa con el resultado anterior en el caso particular $R=\Z$. 
Sean $a,b\in\Z$ tales que $\gcd(a,b)=1$. Si $u,v\in\Z$, entonces, al utilizar el corolario con $I=(a)$ y $J=(b)$, 
tenemos garantizada la existencia de $x\in\Z$ tal que
\[
\begin{cases}
x\equiv u\bmod a,\\
x\equiv v\bmod b,	
\end{cases}
\]
para todo $u,v\in\Z$. 

\begin{exercise}
Si $I_1,\dots,I_n$ son ideales de un anillo conmutativo $R$, entonces
\[
I_1\cdots I_n=\left\{\sum_{j=1}^m a_{1j}\cdots a_{nj}:m\in\N_0,\,a_{ij}\in I_i,\,1\leq j\leq m,\,1\leq i\leq n\right\}
\]
es un ideal de $R$. Puede demostrarse además que si $I_1$ es un ideal coprimo con $I_j$ para todo $j\in\{2,\dots,n\}$ entonces
$I_1$ y $I_2\cdots I_n$ son también ideales coprimos. 
\end{exercise}

El ejercicio anterior nos permite extender el teorema chino del resto a finitos ideales. Supongamos que $R$ es un anillo conmutativo y que
$I_1,\dots,I_n$ son ideales de $R$ tales que $I_i$ e $I_j$ son coprimos siempre que $i\ne j$. Si $x_1,\dots,x_n\in R$, puede demostrarse que
entonces existe $x\in R$ 
tal que $\pi_i(x_i)=\pi_i(x)$ para todo $i\in\{1,\dots,n\}$, donde $\pi_i\colon R\to R/I_i$ es el morfismo canónico. En este caso, además,
\[
R/(I_1\cap\cdots\cap I_n)\simeq (R/I_1)\times\cdots\times (R/I_n).
\]

Un hecho sorprendente. El teorema de interpolación de Lagrange es en realidad un caso particular del teorema chino del resto en el anillo de polinomios $R=\R[X]$. En efecto,
si $x_1,\dots,x_k\in\R$ son tales que $x_i\ne x_j$ y fijamos elementos $y_1,\dots,y_k\in\R$, entonces, gracias a la versión abstracta del teorema chino del resto aplicado
a los ideales coprimos $I_j=(X-x_j)$ para $j\in\{1,\dots,k\}$, se garantizará la existencia de una única solución 
módulo $(X-x_1)\cdots (X-x_k)$ del sistema
\[
\begin{cases}
f\equiv y_1\bmod (X-x_1),\\
f\equiv y_1\bmod (X-x_2),\\
\vdots\\
f\equiv y_1\bmod (X-x_k).	
\end{cases}
\]
El sistema tendrá en particular una única solución de grado $k-1$, 
que es lo que conocemos como el polinomio interpolador de Lagrange.


\chapter{Anillos noetherianos}

\begin{definition}
\index{Anillo!noetheriano}
Sea $R$ un anillo. Diremos que $R$ es \textbf{noetheriano} si toda sucesión de ideales $I_1\subseteq I_2\subseteq\cdots $ se estabiliza, es decir que existe $m\in\N$ tal que 
$I_n=I_m$ para todo $n\geq m$. 
\end{definition}

Análogamente se definen anillos noetherianos a izquierda y a derecha. 

%\begin{definition}
%Sea $R$ un anillo. Diremos que $R$ es \textbf{artiniano} si toda sucesión de ideales $I_1\supseteq I_2\supseteq\cdots $ se estabiliza, es decir que existe $m\in\N$ tal que 
%$I_n=I_m$ para todo $n\geq m$. 	
%\end{definition}

\begin{example}
$\Z$ es noetheriano.
\end{example}

%\begin{example}
%$\Z$ no es artiniano pues la sucesión $(2)\supsetneq (4)\supsetneq (8)\supsetneq\cdots$ no se estabiliza.	
%\end{example}
%
\begin{example}
Sea $R$ el anillo de funciones $[0,1]\to\R$ 
con las operaciones usuales punto a punto. Para cada $n\in\N$ sean  
\[
I_n=\{f\in R:f|_{[0,1/n]}=0\}.
%\quad
%J_n=\{f\in A:f|_{[1/n,1]}=0\}.
\]
Como $I_1\subsetneq I_2\subsetneq\cdots$, $R$ no es noetheriano. 
%Como $J_1\supsetneq J_2\supsetneq\cdots$, $A$ no es artiniano.  	
\end{example}

\begin{theorem}
	Sea $R$ un anillo. Entonces $R$ es noetheriano si y sólo si todo ideal de $R$ es finitamente generado.
\end{theorem}

\begin{proof}
	Vamos a demostrar primero que vale $\implies$. Sea $I$ un ideal de $R$ que no es finitamente generado. En particular, $I\ne\{0\}$. Existe entonces
	$x_1\in I\setminus\{0\}$. Si $I_1=(x_1)$, entonces $\{0\}\subsetneq I_1\subsetneq I$. Si los ideales 
	$I_0,I_1,\dots,I_{k-1}$ fueron construidos, sea $x_k\in I\setminus I_{k-1}$ y sea $I_k=(x_1,\dots,x_k)$. De esta forma pudimos construir una sucesión
	\[
	I_0\subsetneq I_1\subsetneq I_2\subsetneq\cdots 
	\]
	que no se estabiliza.   
	
	Demostremos ahora la recíproca. Supongamos que tenemos una sucesión
	\[
	I_0\subsetneq I_1\subsetneq I_2\cdots
	\]
	de ideales de $R$. Vimos que entonces $I=\cup_{i\geq0}I_i$ es un ideal de $R$. Como por hipótesis todo ideal es finitamente generado, podemos
	escribir $I=(x_1,\dots,x_n)$ para ciertos $x_1,\dots,x_n\in R$. Sin perder generalidad podemos suponer además que $x_j\in I_{i_j}$ para todo $j\in\{1,\dots,n\}$. Si 
	$N=\max\{i_1,\dots,i_n\}$ y $n\geq N$, entonces 
	$I_N\subseteq I\subseteq I_N\subseteq I_n$ y el resultado queda demostrado. 
\end{proof}

\begin{example}
Todo anillo de ideales principales es noetheriano. En particular, $\Z$, $\Z/n$ y $\R[X]$ son noetherianos. 	
\end{example}

\begin{exercise}
Sea $R$ un anillo noetheriano. Si $I$ es un ideal de $R$, entonces $R/I$ también es noetheriano.	
\end{exercise}

\begin{theorem}[Hilbert]
\index{Teorema!de Hilbert}
Si $R$ es noetheriano entonces $R[X]$ también lo es.
\end{theorem}

\begin{proof}
	Tenemos que demostrar que todo ideal $I$ de $R[X]$ es finitamente generado. Sin perder generalidad podemos suponer que
	$I$ es no nulo. Sea entonces $f_1\in I$ de grado mínimo. Para $i>1$, 
	sea $f_i\in I$ de menor grado tal que $f_i\not\in (f_1,\dots,f_{i-1})$. Para cada $i$, sea $a_i$ el coeficiente principal del polinomio $f_i$. 
	Si la sucesión de los $f_i$ 	nunca se estabiliza, sea $J=(a_1,a_2,\dots)$ el ideal generado por los coeficientes principales de los $f_i$.  Como $R$ es noetheriano, 
	$J=(a_1,\dots,a_m)$ para algún $m$. Luego
	\[
	a_{m+1}=\sum_{i=1}^m u_ia_i
	\]
	para ciertos $u_1,\dots,u_m\in R$. Escribimos 
	\[
	f_i=a_iX^{n_i}+\cdots\text{ (términos de grado menor)} 
	\]
	para cada $i$ y entonces
	\[
	g=\sum_{i=1}^m u_if_iX^{\deg(f_{m+1})-n_i}\in (f_1,\dots,f_m).
	\]
	El coeficiente principal de $g$ es entonces $a_{m+1}$ y además $\deg g=\deg(f_{m+1})$ pues
	\begin{align*}
	g&=\sum_{i=1}^m u_i(a_iX^{n_i}+\cdots)X^{\deg(f_{m+1})-n_i}
	=\sum_{i=1}^m u_ia_i(X^{\deg(f_{m+1})}+\cdots)
%	=X^{\deg(f_{m+1})}\sum_{i=1}^m u_ia_i+\cd
\end{align*}
	donde nuevamente los puntos suspensivos representantes un polinomio de grado menor. 
	Pero $g-f_{m+1}\not\in(f_1,\dots,f_m)$ pues por construcción $f_{m+1}\not\in (f_1,\dots,f_m)$ y además $\deg(g-f_{m+1})<\deg(f_{m+1})$, una contradicción
	a la minimalidad del grado de $f_{m+1}$. 
\end{proof}

\begin{corollary}
	Si $R$ es un anillo noetheriano, entonces $R[X_1,\dots,X_n]$ también es noetheriano.
\end{corollary}

\begin{proof}[Bosquejo de la demostración]
La demostración quedará como ejercicio, hay que utilizar inducción y que $R[X_1,\dots,X_{n-1}][X_n]\simeq R[X_1,\dots,X_n]$.  	
\end{proof}

Veamos algunas aplicaciones sencillas.

\begin{example}
$\Z[\sqrt{N}]$ es noetheriano pues $\Z[\sqrt{N}]\simeq\Z[X]/(X^2-N)$ y $\Z[X]$ es noetheriano gracias al teorema de Hilbert.	
\end{example}

\begin{example}
$\Z[X,X^{-1}]\simeq\Z[X,Y]/(XY-1)$ es noetheriano ya que $\Z[X,Y]$ es noetheriano por el teorema de Hilbert. 	
\end{example}

\begin{proposition}
Si $R$ es noetheriano y $f\colon R\to R$ es un morfismo de anillos sobreyectivo, entonces $f$ es un isomorfismo.	
\end{proposition}

\begin{proof}
	Si escribimos $f^n=f\circ\cdots\circ f$ ($n$-veces), entonces $f^n$ es sobreyectivo. Para cada $n$ sea $K_n=\ker(f^n)$. Entonces 
	\[
	K_1\subseteq K_2\subseteq K_3\subseteq\cdots
	\] 
	es una sucesión de ideales de $R$. Como $R$ es noetheriano, $K_m=K_{m+1}$ para algún $m$. Sea $y\in\ker f=K_1$. Como $f^m$ es
	sobreyectivo, $y=f^m(x)$ para algún $x\in R$. Entonces
	$0=f(y)=f(f^m(x))$ 
	y luego 
	\[
	x\in \ker(f^{m+1})=K_{m+1}=K_m=\ker f^m,
	\]
	es decir $y=0$ y luego $\ker f=\{0\}$. 
\end{proof}

Veamos ahora un ejemplo de ideal no finitamente generado.

\begin{example}
	Sea $R=\C[X_1,X_2,\dots]$ es anillo de polinomios en infinitas variables. Vamos a demostrar que el ideal 
	$I=(X_1,X_2,\dots)$ generado por esas infinitas variables no es finitamente generado. Observemos que $I$ es el conjunto de polinomios con término constante nulo.  
	
	Si $I$ fuera finitamente generado, digamos 
	$I=(f_1,\dots,f_n)$ para ciertos $f_i$, hay que observar que cada $f_i$ involucra únicamente una cantidad finita de variables, por lo que entonces
	existe $m\in\N$ tal que todas las $X_i$ aparecen en alguno de los $f_1,\dots,f_n$ con $i<m$. Sea $\varphi\colon R\to\C$ 
	dado por
	\[
	\varphi(X_i)=
	\begin{cases}
		0 & \text{si $i<m$},\\
		1 & \text{si $i\geq m$}.
	\end{cases}
	\]
	Entonces $\varphi(f_i)=0$ para todo $i\in\{1,\dots,n\}$, lo que implica que $\varphi(I)=0$. Por otro lado, $\varphi(X_m)=1$, lo que implica que $X_m\not\in I$, una contradicción. 		
\end{example}



\chapter{Factorización}

Nuestro objetivo es utilizar ciertos aspectos de la teoría de anillos conmutativos para demostrar algunos resultados de la teoría de números. La idea básica será intentar reconocer similitudes entre
ciertos anillos conmutativos y $\Z$. Comenzaremos entonces 
estudiando divisibilidad en anillos conmutativos. 

\begin{definition}
\index{Dominio!íntegro}
Un anillo conmutativo $R$ será un \textbf{dominio íntegro} si $xy=0\implies x=0$ o bien $y=0$.
\end{definition}

\begin{example}
$\Z$ es un dominio íntegro y $\Z/4$ no lo es.
\end{example}

\begin{example}
\index{Enteros de Gauss}
$\Z[i]$ es un dominio íntegro. Este anillo se conoce como el anillo de \textbf{enteros de Gauss}.
\end{example}

\index{Divisibilidad!en anillos}
\index{Divisor}
\index{Divisor!propio}
Sea $R$ un dominio íntegro.
Vamos a extender algunas nociones de la divisibilidad en $\Z$ 
en el contexto del dominio íntegro $R$. Diremos que
$x$ \textbf{divide} al elemento $y$ si y sólo si $y=xz$ para algún $z\in R$, lo que resulta ser equivalente a pedir $(y)\subseteq (x)$. 
Diremos además que $x$ es un \textbf{divisor propio} de $y$ si y sólo si $(y)\subsetneq (x)\subsetneq R$. Tal como se hace en el caso
de los enteros, a veces utilizaremos la notación $x\mid y$ para referirnos a que el elemento $y$ es divisible por $x$. Escribiremos
$x\nmid y$ cuando $y$ no es divisible por $x$.

\begin{example}
\label{exa:Z[i]div}
Sea $R=\Z[i]$ y sean $d\in\Z$ y $a+bi\in R$. Entonces $d\mid a+bi$ si y sólo si $d\mid a$ y $d\mid b$ en $\Z$. 
En efecto, si existen $e,f\in\Z$ tales que 
\[
a+bi=d(e+fi)=de+dfi,
\] 	
entonces $a=de$ y $b=df$, es decir $d\mid a$ y $d\mid b$. 
\end{example}

Las siguientes definiciones extienden propiedades que conocemos de $\Z$.

\begin{definition}
\index{Elementos!asociados}
Sea $R$ un dominio íntegro y sean $x,y\in R$. Diremos que 	
$x$ e $y$ son \textbf{asociados} si y sólo si $(x)=(y)$.
\end{definition}

Observar que $x$ e $y$ son asociados si y sólo si $x=yu$ para alguna unidad $u$. 

\begin{examples}
En $\Z$ los enteros $2$ y $-2$ son asociados. En $\R[X]$ los polinomios 
$f\ne 0$ y $\lambda f$, donde $\lambda\in\R^\times$, son asociados. 
\end{examples}

\begin{example}
En $\Z[i]$ los elementos $2+5i$ y $-5+2i=(2+5i)i$ son asociados. 
\end{example}

\begin{definition}
\index{Elemento!irreducible}
Sea $R$ un dominio íntegro y sea $x\in R\setminus\{0\}$. Diremos que  
$x$ es \textbf{irreducible} si y sólo si $(x)\ne R$ y $(x)$ es maximal en el conjunto de ideales principales de $R$, es decir que 
no existe ningún ideal principal $(y)$ tal que $(x)\subsetneq (y)\subsetneq R$.
\end{definition}

Para entender mejor la definición de elementos irreducibles observemos que en un dominio íntegro $R$, los divisores de un irreducible 
son sus asociados y las unidades de $R$. En efecto, si $z$ es irreducible y $x\mid z$, entonces $(z)\subseteq (x)$. Esto nos 
da dos posibilidades, $(z)=(x)$ o bien $(x)=R$, es decir $x$ y $z$ son asociados o bien $x\in\mathcal{U}(R)$.   

\begin{example}
Si $K$ es un cuerpo y $f\in K[X]$ de grado $n>0$.
Entonces $f$ es irreducible en $K[X]$ si y solo si los únicos divisores de $f$ son de la forma $g=\lambda$ o bien 
$g=\lambda f$ para $\lambda\in K^\times=K\setminus\{0\}$, es decir cuando los divisores de $f$ 
son unidades de $K[X]$ o los asociados a $f$. 
\end{example}

Del ejemplo anterior se desprende que un polinomio $f$ será reducible (es decir, no irreducible) si $\deg(f)>0$ y además 
$f$ tiene algún divisor $g\in K[X]$ no nulo 
tal que $0<\deg(g)<\deg(f)$.  

\begin{definition}
\index{Elemento!primo}
Sea $R$ un dominio íntegro y sea $p\in R\setminus\{0\}$. Diremos que  
$p$ es \textbf{primo} si y sólo si $(p)\ne R$ y además $xy\in (p)\implies x\in(p)$ o bien $y\in(p)$.   		
\end{definition}

En $\Z$ primos e irreducibles coinciden, algo que no pasará en otros anillos. Más adelante 
caracterizaremos todos los primos en $\Z[i]$. 

\begin{proposition}
En un dominio íntegro, todo elemento primo es irreducible.  
\end{proposition}

\begin{proof}
Como $p$ es primo, $p\ne 0$ y $p\not\in\mathcal{U}(R)$. Sea $x$ un divisor de $p$, digamos $p=xy$ para $y\in R$. Como $xy\in(p)$ y $p$ es primo,
entonces $x\in(p)$ o $y\in(p)$. Si $x\in(p)$, entonces $x=pz$ y luego 
\[
p=xy=(pz)y=p(yz)\implies p(1-yz)=0.
\]
Como $p\ne 0$ y $R$ es un dominio, $y,z\in\mathcal{U}(R)$ y luego $x=pz$ es asociado a $p$. Si $y\in(p)$, una cuenta similar a la anterior nos muestra
que $x\in\mathcal{U}(R)$.
%Sea $x\in R$ un elemento primo. Observemos que $x$ es irreducible si y sólo si $x$ no es una unidad y además no posee divisores propios. Por otro lado, $x$ es primo si y sólo si $x$ no es una unidad y $x\mid yz\implies x\mid y$ o bien $x\mid z$.  
%Si $x$ es primo y además $x=yz$, entonces $x\mid yz$. Esto implica que $x\mid y$ o bien $x\mid z$. Si $x\mid y$, entonces
%$y=xa$ para algún $a$. Luego $x=yz=(xa)z=x(az)$ y entonces $x(1-az)=0$. Como estamos en un dominio íntegro y $x\ne 0$, se concluye
%que $az=1$, es decir $z$ es una unidad.  
\end{proof}

Veremos que la afirmación recíproca no vale con total generalidad, aunque sí en el caso de dominios principales. Para poder hacer algunas
cuentas con mayor facilidad, utilizaremos la siguiente herramienta sobre el anillo $R=\Z[\sqrt{d}]$, donde $d$ es un entero 
libre de cuadrados. Utilizaremos la siguiente notación: si $n\in\N$, entonces $\sqrt{-n}=\sqrt{n}i$. 

\begin{lemma}
Sea $d\in\Z$ libre de cuadrados. Para cada $\alpha=a+b\sqrt{d}\in\Z[\sqrt{d}]$ 
sea $N(\alpha)=|a^2-db^2|$. 
\begin{enumerate}
\item $N(\alpha)=0\Longleftrightarrow \alpha=0$.
\item $N(\alpha\beta)=N(\alpha)N(\beta)$ para todo $\alpha,\beta\in\Z[\sqrt{d}]$. 
\item $\alpha\in\Z[\sqrt{d}]$ es una unidad $\Longleftrightarrow N(\alpha)=1$.
\item Si $N(\alpha)$ es primo, entonces $\alpha$ es irreducible en $\Z[\sqrt{d}]$.  	
\end{enumerate}
\end{lemma}

\begin{proof}
Dejamos las primeras dos afirmaciones como ejercicio. Demostremos (3). Si $\alpha\in\mathcal{U}(\Z[\sqrt{d}])$, 
entonces $\alpha\beta=1$ para algún $\beta\in\Z[\sqrt{d}]$. Como
\[
1=N(1)=N(\alpha\beta)=N(\alpha)N(\beta),
\]
entonces $N(\alpha)=1$. Recíprocamente, si $\alpha\ne0$, entonces $\alpha$ es una unidad con inversa
$\alpha^{-1}=\overline{\alpha}/N(\alpha)$, 
donde $\overline{a+b\sqrt{d}}=a-b\sqrt{d}$.

Demostremos ahora (4). Si $\alpha=\beta\gamma$, entonces $N(\alpha)=N(\beta\gamma)=N(\beta)N(\gamma)$ y luego
$N(\beta)=1$ o bien $N(\gamma)=1$ (pues $N(\alpha)$ es primo). Luego
$\beta\in\mathcal{U}(\Z[\sqrt{d}])$ o bien $\gamma\in\mathcal{U}(\Z[\sqrt{d}])$ por el ítem anterior.
\end{proof}

\begin{example}
Sea $R=\Z[i]$. 
\begin{enumerate}
    \item $\mathcal{U}(R)=\{-1,1,i,-i\}$ pues si $a+bi\in R$ es una unidad, entonces, gracias al lema anterior, $N(a+bi)=a^2+b^2=1$.
    \item $3$ es irreducible en $R$. Si $3=\alpha\beta$, entonces
$9=N(\alpha\beta)=N(\alpha)N(\beta)$. Luego $N(\alpha)\in\{1,3,9\}$. Supongamos que $\alpha=a+bi$. Si $N(\alpha)=3$, entonces $|a^2+b^2|=3$, una contradicción pues $a,b\in\Z$. Luego $N(\alpha)\in\{1,9\}$. Si $N(\alpha)=1$, entonces $\alpha$ es una unidad. Si no, $N(\beta)=1$ y $\beta$ es una unidad. 	
\item $2\in R$ no es irreducible. En efecto, alcanza con observar que $2=(1+i)(1-i)$, que 
$1+i$ y $1-i$ no son unidades pues $N(1+ i)=N(1-i)=2\ne 1$. 
\end{enumerate}
\end{example}

\begin{example}
Sea $R=\Z[\sqrt{-5}]$. 
\begin{enumerate}
\item $1+\sqrt{-5}$ es irreducible en $R$. Dejamos como ejercicio verificar que $1+\sqrt{-5}$ no es una unidad. 
Si $1+\sqrt{-5}=\alpha\beta$ para ciertos $\alpha,
\beta\in R$, entonces 
\[
6=N(1+\sqrt{-5})=N(\alpha)N(\beta).
\]
Esto implica que $N(\alpha)\in\{1,2,3,6\}$. Si $N(\alpha)\in\{1,6\}$, entonces
$\alpha$ es una unidad o bien $\beta$ es una unidad. Si 
$\alpha=x+y\sqrt{-5}$, entonces  
$x^2+5y^2=N(\alpha)\in\{2,3\}$, una contradicción. De la misma forma puede demostrarse que $1-\sqrt{-5}$ 
es irreducible. 
\item $2$, $3$, $1+\sqrt{-5}$ y $1-\sqrt{-5}$ no son asociados en $R$. En efecto, $2$, $3$ y $1+\sqrt{-5}$ no son asociados
pues tienen todos distinta norma. Tampoco son asociados $1+\sqrt{-5}$ y $1-\sqrt{-5}$, pues $1+\sqrt{-5}=u(1-\sqrt{-5})$ 
para $u\in\mathcal{U}(R)$
da una contradicción. %$u\in\mathcal{U}(R)=\{-1,1\}$. 
\end{enumerate}
\end{example}

\begin{example}
Sea $R=\Z[\sqrt{-3}]$ y sea $x=1+\sqrt{-3}$. Entonces $x$ es irreducible. En efecto,
si $1+\sqrt{-3}=\alpha\beta$, entonces $4=N(\alpha)N(\beta)$. 
Supongamos que $N(\alpha)=2$. Si $\alpha=a+b\sqrt{-3}$, entonces
$a^2+3b^2=2$. Obviamente, los enteros $a$ y $b$ tienen que tener la misma paridad. 
Si $a\equiv b\equiv 0\bmod 2$, digamos $a=2k$ y $b=2l$, entonces
\[
2=a^2+3b^2=(2k)^2+3(2l)^2=4k^2+12l^2
\]
es divisible por $4$, una contradicción. Si $a\equiv 1\bmod 2$ y además $b\equiv 1\bmod 2$, digamos
$a=2k+1$ y $b=2l+1$, entonces
\[
2=a^2+3b^2=(2k+1)^2+3(2l+1)^2=4k^2+4k+12l^2+12l+4
\]
es un múltilo de $4$, una contradicción. 

Sin embargo, $x$ no es primo. Como
\[
(1+\sqrt{-3})(1-\sqrt{-3})=4,
\]
$x$ divide a $4=2\times 2$, pero $1+\sqrt{-3}$ no divide a $2$ pues 
\[
(1+\sqrt{-3})(a+b\sqrt{-3})=2\implies
(a-3b)+(a+b)\sqrt{-3}=2
\]
que implica que $a-3b=2$ y además $a+b=0$. En conclusión, $a=1/2\not\in\Z$, una contradicción. 
\end{example}

\begin{exercise}
Demuestre que $1+\sqrt{5}\in\Z[\sqrt{5}]$ es un irreducible que no es primo.	
\end{exercise}

\begin{definition}
\index{Dominio!principal}
\index{Dominio!de ideales principales}
Un dominio íntegro $R$ se dirá un \textbf{dominio de ideales principales} (o simplemente dominio principal) si 
todo ideal de $R$ es principal. 	
\end{definition}

Vimos que $\Z$ es un dominio de ideales principales. Si $K$ es un cuerpo, entonces
$K[X]$ es también un dominio de ideales principales.

\begin{example}
Veamos que $\Z[X]$ no es principal. Sea $I=(X,2)$.
Primero observemos que $I\ne\Z[X]$. En efecto, si $I=\Z[X]$, entonces
\[
1=2f+Xg
\]
para ciertos $f,g\in\Z[X]$. En particular, $-1/2=f(0)\in\Z$, una contradicción. 
Si existe $h\in\Z[X]$ tal que $I=(h)$, entonces, en particular, $2=hg$ y además $X=hf$ para ciertos $f,h\in\Z[X]$. 
En particular, $\deg h=0$ y luego $h$ es la constante $h(1)$. Como $2=h(1)g(1)$, tenemos 
$h=h(1)\in\{-1,1,2,-2\}$. Como $I$ es un ideal propio,
$h=h(1)\not\in\{-1,1\}$. Luego  
$X=\pm 2f$. Si escribimos 
\[
f=a_0+a_1X+\cdots+a_nX^n,
\]
donde $a_0,a_1,\dots,a_n\in\Z$, entonces, al comparar el coeficiente de $X$ en $X=\pm 2f$ 
vemos que 
$1=\pm 2a_1$, una contradicción pues $a_1\in\Z$. 
\end{example}

\begin{example}
Veamos que $\Z[\sqrt{-5}]$ no es principal. Sea $I=(2,1+\sqrt{-5})$. Primero
observamos que $I\ne\Z[\sqrt{-5}]$. En efecto, si $I=\Z[\sqrt{-5}]$, entonces
\[
1=2(x+\sqrt{-5}y)+(1+\sqrt{-5})(u+\sqrt{-5}v)=(2x+u-5v)+\sqrt{-5}(2y+u+v)
\]
para ciertos $x,y,u,v\in\Z$. Luego
\[
1=2x+u-5v,\quad
0=2y+u+v,
\]
que implica que $1=2(x+y+u-2v)$, una contradicción pues $x+y+u-2v\in\Z$. 
Si $I=(\alpha)$, entonces 
$N(\alpha)\mid N(2)$, pues 
$\alpha\mid 2$, y además  
$N(\alpha)\mid N(1+\sqrt{-5})$, pues 
$\alpha\mid 1+\sqrt{-5}$. 
Observemos que entonces se tiene que $N(\alpha)\in\{1,2\}$, pues $N(2)=4$ y $N(1+\sqrt{-5})=6$. 
Si $\alpha=a+b\sqrt{-5}$, entonces, como $N(\alpha)=a^2+5b^2$, se concluye que $N(\alpha)=1$. Luego 
$\alpha$ es una unidad y entonces $I=\Z[\sqrt{-5}]$, una contradicción.   
\end{example}

\begin{proposition}
Sea $R$ un dominio principal y sea $x\in R$. Entonces $x$ es irreducible si y sólo si $x$ es primo.
\end{proposition}

\begin{proof}
	Vimos en la proposición anterior que todo primo es irreducible. Supongamos entonces que $x$ es irreducible y
	que $x\mid yz$. Sea $I=(x,y)$ el ideal generado por $x$ e $y$. Como $R$ es principal, existe $a\in R$ tal que
	$I=(a)$. En particular, $x=ab$ para algún $b\in R$. La irreducibilida de $x$ implica que $a\in\mathcal{U}(R)$ o bien
	$b\in\mathcal{U}(R)$. Si $a\in\mathcal{U}(R)$, entonces $I=R$ y luego $1=xr+ys$ para ciertos $r,s\in R$, lo que 
	implica que
	\[
	z=z1=z(xr+ys)=xzr+yzs
	\]
	y entoces $x\mid z$. Si $b\in\mathcal{U}(R)$, entonces $I=(x)=(a)$ y luego, como $y\in I$, existe
	$t\in R$ tal que $xt=y$, es decir $x\mid y$. 	
\end{proof}
	
\begin{example}
$\Z[\sqrt{-3}]$ 
no es principal ya que existen irreducibles que no son primos.
\end{example}

\begin{example}
Como $\Z$ es principal, en $x\in \Z$ es primo si y sólo si $x\in\Z$ es irreducible.  
\end{example}

\begin{definition}
\index{Dominio!euclidiano}
Sea $R$ un dominio íntegro. Diremos que $R$ es un \textbf{dominio euclidiano} 
si existe una función $\varphi\colon R\setminus\{0\}\to\N_0$ 
tal que para cada $x,y\in R$ con $y\ne0$ existen $q,r\in R$ tales que $x=yq+r$, donde $r=0$ o bien $\varphi(r)<\varphi(y)$.  
\end{definition}

Es importante remarcar que en la definición de dominio euclidiano no pedimos la unicidad que
tenemos en $\Z$. En muchos libros de texto, en la definición de dominio euclidiano, a la función $\varphi$ se le pide además 
que cumpla $\varphi(x)\leq\varphi(xy)$ para todo $x,y\in R\setminus\{0\}$. Esta condición no se usa para demostrar los resultados básicos, 
por eso no se incluye. Además,
si $R$ es euclidiano (con nuestra definición y nuestro $\varphi$), siempre puede reemplazarse $\varphi$ por una función $\psi(x)=\min_{y\ne0}\varphi(xy)$ y esta $\psi$ satisface 
$\psi(x)\leq\psi(xy)$ para todo $x,y\in R\setminus\{0\}$. 


\begin{examples}\
\begin{enumerate}
\item $\Z$ es un dominio euclidiano con $\varphi(x)=|x|$. Cuidado que acá tampoco tenemos unicidad. 
\item Si $K$ es un cuerpo, $K[X]$ es euclidiano con $\varphi(f)=\deg f$. Este ejemplo es el que motiva que la función $\varphi$ de la definición de dominio euclidiano 
esté definida para elementos no nulos del anillo. 
\end{enumerate}
\end{examples}

Veamos otro ejemplo de dominio euclidiano.	
	
\begin{example}
Sea $\Z[i]=\{a+bi:a,b\in\Z\}$. Dejamos como ejercicio demostrar que $\Z[i]$ es un dominio íntegro. 

Sea $N(x+iy)=x^2+y^2$. Veamos que $N$ es un función multiplicativa: Si $\alpha=x+iy$ y $\beta=u+iv$, entonces 
\begin{gather*}
\alpha\beta=(xu-yv)+i(xv+yu)
\shortintertext{y luego}
N(\alpha\beta)=(xu-yv)^2-(xv+yu)^2=(x^2+y^2)(u^2+v^2)=N(\alpha)N(\beta).
\end{gather*}
	
Vamos a demostrar ahora que $\Z[i]$ es un dominio euclidiano con $\varphi(\alpha)=N(\alpha)$. 
Sean $\alpha=a+ib$ y $\beta=c+id\ne 0$. Entonces
\[
\frac{\alpha}{\beta}=\frac{a+bi}{c+di}=r+is,
\]
donde $r=(ac+bd)/(c^2+d^2)$ y $s=(bc-ad)/(c^2+d^2)$. 
Sean $m,n\in\Z$ tales que $|r-m|\leq 1/2$ y $|s-n|\leq 1/2$. Si $\delta=m+in$ y $\gamma=\alpha-\beta\delta$, entonces
$\delta,\gamma\in\Z[i]$ y además $\alpha=\beta\delta+\gamma$. Si $\gamma\ne0$, entonces
\begin{align*}
\varphi(\gamma)&=\varphi\left(\beta\left(\frac{\alpha}{\beta}-\delta\right)\right)
=\varphi(\beta)\varphi\left(\frac{\alpha}{\beta}-\delta\right)\\
&=\varphi(\beta)\varphi( (r-m)+i(s-n))
=\varphi(\beta)\left( (r-m)^2+(s-n)^2\right)\\
&\leq\varphi(\beta)\left(\frac14+\frac14\right)=\frac12\varphi(\beta)<\varphi(\beta).
\end{align*}
\end{example}

En $\Z[i]$ hay algoritmo de división pero no tenemos unicidad. De hecho, 
por ejemplo, podemos escribir 
\[
1+8i=(2-4i)(-1+i)+(-1+2i)
=(2-4i)(-2+i)+(1-2i)
\] 
y entonces
$N(1-2i)=N(-1+2i)=5<20=N(2-4i)$.

\begin{example}
Si $\omega=\frac{-1+\sqrt{3}i}{2}$, entonces $\Z[\omega]$ es un dominio euclidiano.	Primero observamos que
\[
\Z[\omega]=\left\{\frac{a}{2}+\frac{b}{2}\sqrt{-3}:a,b\in\Z,\,a\equiv b\bmod 2\right\}
\]
pues $a+b\omega=\frac{2a+b}{2}+\frac{b}{2}\sqrt{-3}$. 

Veamos que $\Z[\omega]$ es euclidiano con la norma  
$N(a+b\sqrt{-3})=a^2+3b^2$, donde $a,b\in\frac{1}{2}\Z$. Sean $\alpha=a_1+a_2\omega$ y $\beta=b_1+b_2\omega\ne 0$, donde
$a_1,a_2,b_1,b_2\in\frac{1}{2}\Z$. Queremos ver que existen $\gamma,\delta\in\Z[\omega]$ tales que $\alpha=\beta\gamma+\delta$, donde $N(\delta)<N(\beta)$. 
En $\Q[\sqrt{-3}]$ podemos dividir, entonces 
\[
\frac{\alpha}{\beta}=c_1+c_2\sqrt{-3},
\]
para ciertos $c_1,c_2\in\Q$. Sea $q_2\in\Z$ tal que $|2c_2-q_2|\leq 1/2$ y sea $t\in\Z$ el entero más cercano al número $c_1-\frac{q_2}{2}$. 
Si $q_1=2t+q_2$, entonces $|c_1-\frac{q_1}{2}|\leq 1/2$. Si 
\[
\gamma=\frac{q_1}{2}+\frac{q_2}{2}\sqrt{-3},
\]
entonces $\gamma\in\Z[\omega]$ pues $q_1-q_2=2t$ es par. Si 
\[
\delta=\beta\left( (c_1-\frac{q_1}{2})+(c_2-\frac{q_2}{2})\sqrt{-3}\right),
\]
entonces $\alpha=\gamma\beta+\delta$. 
Como  
\begin{multline*}
N\left( (c_1-\frac{q_1}{2})+(c_2-\frac{q_2}{2})\sqrt{-3}\right)
\leq ( c_1-\frac{q_1}{2})^2+3(c_2-\frac{q_2}{2})^2\leq \frac{1}{4}+3\frac{1}{16}<1, 
\end{multline*}
se concluye que $N(\delta)<N(\beta)$ pues $N$ es multiplicativa. 
\end{example}

\begin{exercise}
Demuestre que $\Z[\sqrt{d}]$ es euclidiano si $|d|\leq 2$.
\end{exercise}

El siguiente ejemplo no es sencillo. Simplemente lo mencionamos para mayor completitud en la presentación.  Para más información ver \cite{MR967349,MR3665445,MR314831}.

\begin{example}
Si $\theta=\frac{1+\sqrt{-19}}{2}$, entonces $\Z[\theta]$ no es euclidiano.	Sin embargo, $\Z[\theta]$ es 
un dominio de ideales principales. 
\end{example}

Tal como pasa en $\Z$ y $K[X]$, el tener algoritmo de división nos permite demostrar que todo ideal es principal.

\begin{theorem}
	Si $R$ es euclidiano, entonces $R$ es principal. 
\end{theorem}

\begin{proof}
	Sea $I$ un ideal no nulo de $R$ y sea $y\in I$ no nulo donde la función $x\mapsto\varphi(x)$ alcanza su mínimo. 
	Si $z\in I$, entonces $z=yq+r$ donde $r=0$ o bien $\varphi(r)<\varphi(y)$. La minimalidad de $y$ implica que $r=0$ y 
	luego $z=yq$. Tenemos entonces $I\subseteq Ry\subseteq (y)\subseteq I$ y luego $I=(y)$.     
\end{proof}

\begin{example}
El anillo $\Z[\sqrt{-5}]$ no es euclidiano ya que no es principal. 
\end{example}

Nos interesa poder reconocer anillos de la forma $\Z[\sqrt{d}]$, con $d$ libre de cuadrados, que se parezcan al anillo $\Z$. Ya vimos
que hay muchas similitudes, pero en $\Z[\sqrt{d}]$ no siempre valdrá el teorema de la factorización única.

\begin{definition}
\index{Dominio!de factorización única}
Diremos que un dominio íntegro $R$ es un \textbf{dominio factorización única} si valen las siguientes propiedades:
\begin{enumerate}
\item Cada $x\ne0$ que no es una unidad puede escribirse 
como $x=c_1\dots c_n$ para ciertos irreducibles $c_1,\dots,c_n$.  
\item Si $x=c_1\cdots c_n=d_1\cdots d_m$ con los $c_i$ y los $d_j$ irreducibles, entonces $n=m$ y además existe 
una permutación $\sigma\in\Sym_n$ tal que $c_i$ y $d_{\sigma(i)}$ son asociados para todo $i\in\{1,\dots,n\}$. 	
\end{enumerate}
\end{definition}

El ejemplo típico de domino de factorización única es $\Z$. 

Intentaremos explicar mejor la diferencia entre tener factorización y tener factorización única. Supongamos que $R$ es un dominio íntegro noetheriano. Podemos demostrar entonces que $R$ tendrá factorización, aunque no necesariamente única. Sea $x\in R$ no nulo tal que
$x\not\in\mathcal{U}(R)$. Si $x$ es irreducible, no hay nada para demostrar. En caso contrario, podemos
escribir 
$x=x_1x_2$ con $x_1,x_2\not\in\mathcal{U}(R)$. Si $x_1$ y $x_2$ son ambos irreducibles, 
significa que $x$ pudo factorizarse en irreducibles. En caso
contrario, digamos si $x_1$ no es irreducible, entonces
podemos escribir $x_1=x_{11}x_{12}$ para $x_{11},x_{12}\not\in\mathcal{U}(R)$. Nos interesa demostrar 
que este procedimiento en algún momento tiene que terminarse en una cantidad finita de pasos y eso
pasa porque, así como la factorización $x=x_1x_2$ nos da la sucesión 
$(x)\subsetneq (x_1)\subsetneq (x_{11})$, el procedimiento general
nos da la sucesión
$(x)\subsetneq (x_1)\subsetneq (x_{11})\subseteq\cdots$, 
que tiene que estabilizarse pues $R$ es noetheriano. 

\begin{example}
Gracias al teorema de Hilbert sabemos que $\Z[i]$ y $\Z[\sqrt{-6}]$ son ambos noetherianos, lo que nos dice que 
en en esos anillos existirá factorización. Sin embargo, veremos que en $\Z[i]$ hay factorización única y que en $\Z[\sqrt{-6}]$ no. 
\end{example}

\begin{theorem}
Sea $R$ un dominio de ideales principales. 
Entonces $R$ es un dominio de factorización única.
\end{theorem}

\begin{proof}
Primero demostraremos que $R$ es noetheriano. Como todos los ideales de $R$ son principales, toda sucesión de ideales es
de la forma $(a_1)\subsetneq (a_2)\subsetneq\cdots$. Fijada esa sucesión, la unión 
$J=\cup_{i\geq 1}(a_i)$ es un ideal de $R$. Como $R$ es principal, $J=(x)$ para algún $x\in R$. 
En particular, como $x\in (a_i)$ para algún $i\geq1$, 
podemos concluir que $(a_k)\subseteq J=(x)\subseteq (a_i)$ para todo $k$. 

Como $R$ es noetheriano, $R$ admite 
factorización en irreducibles. 

Nos falta demostrar la unicidad. Sea $x\in R$ y supongamos que 
\[
x=c_1\cdots c_n=d_1\cdots d_m
\]
son factorizaciones de $x$ en irreducibles, donde $n\leq m$. 
Si $m=1$, entonces $n=1$ y luego $c_1=d_1$. Si $m>1$, como $c_1$ es primo (pues sabemos 
que en $R$ los irreducibles son primos) y además $c_1\mid d_1\cdots d_m$, 
entonces $c_1$ divide a alguno de los $d_j$, digamos $c_1\mid d_1$, sin perder generalidad. 
Como $d_1$ es irreducible y $c_1\not\in\mathcal{U}(R)$, 
$c_1$ y $d_1$ son asociados, es decir $c_1=ud_1$ para algún $u\in\mathcal{U}(R)$. Como entonces 
\[
c_1c_2\cdots c_n=(ud_1)c_2\cdots c_n=d_1d_2d_3\cdots d_m,
\]
se sigue que 
\[
d_1(c_2\cdots c_n-u^{-1}d_2\cdots d_m)=0.
\]
Como $R$ es un dominio y además $d_1\ne 0$, 
después de reemplazar, sin perder generalidad, $u^{-1}d_2$ por $d_2$, nos quedamos con $c_2\cdots c_n=d_2\cdots d_m$. Por inducción, queda entonces
demostrada la implicación que queríamos probar.
% Supongamos ahora que $R$ es un dominio de factorización única. Si $p$ es irreducible y no es primo, existen $x,y\in R$ tales que
% $p\mid xy$, $p\nmid x$ y $p\nmid y$. Como $xy=pz$ para algún $z\in R$, al factorizar $x$, $y$ y $z$ en irreducibles, obtenemos dos factorizaciones
% distintas para $xy$, una contradicción a la unicidad de la factorización. 	
\end{proof}

% \begin{corollary}
% 	Un dominio de ideales principales es un dominio de factorización única.
% \end{corollary}

% \begin{proof}
% Sea $R$ un dominio de ideales principales. Vimos que en $R$ primos e irreducibles son equivalentes. Por el teorema anterior 
% basta con demostrar que $R$ es noetheriano. Como todos los ideales de $R$ son principales, toda sucesión de ideales es
% de la forma $(a_1)\subsetneq (a_2)\subsetneq\cdots$. Fijada esa sucesión, la unión 
% $J=\cup_{i\geq 1}(a_i)$ es un ideal de $R$. Como $R$ es principal, $J=(x)$ para algún $x\in R$. En particular, como $x\in (a_i)$ para algún $i\geq1$, 
% podemos concluir que $(x)\subseteq (a_i)$. Por otro lado, $(a_i)\subseteq (a_{i+1})\subset (x)$. Luego
% $(x)=(a_i)=(a_{i+1})$ y entonces la sucesión de ideales principales se estabiliza. 
% \end{proof}

\begin{example}
$\Z[\sqrt{-6}]$ no es un dominio de factorización única. En efecto, primero observamos que 
\[
10=2\cdot 5=(2+\sqrt{-6})(2-\sqrt{-6}).
\]
Recordemos que $N(a+b\sqrt{-6})=a^2+6b^2$.	
Primero veamos que $2$ es irreducible. Si $2=\alpha\beta$ con $\alpha,\beta\not\in\mathcal{U}(R)$, entonces 
\[
4=N(2)=N(\alpha)N(\beta)
\]
y luego $N(\alpha)=N(\beta)=2$, una contradicción pues $a^2+6b^2\ne 2$ para todo $a,b\in\Z$. 
De la misma forma se demuestra que $5$ 
es también irreducible en $\Z[\sqrt{-6}]$. 

Queda como ejercicio demostrar que $2+\sqrt{-6}$ y $2-\sqrt{-6}$ son irreducibles,.
\end{example}

% \begin{exercise}
% Demuestre que $\Z[\sqrt{-5}]$ no es un dominio de factorización única.
% \end{exercise}
% 6=2\cdot 3=(1+\sqrt{-5})(1-\sqrt{-5})

Terminamos el capítulo con una aplicación a la teoría de números.

\begin{theorem}[Fermat]
\index{Teorema!de Fermat}
Para $p$ un número primo, son equivalentes:
\begin{enumerate}
	\item $p=2$ o bien $p\equiv1\bmod 4$.
	\item Existe $a\in\Z$ tal que $a^2\equiv-1\bmod p$.
	\item $p$ no es irreducible en $\Z[i]$.
	\item $p$ es suma de dos cuadrados.
\end{enumerate}	
\end{theorem}

\begin{proof}
Veamos primero que $(1)\implies (2)$. Si $p=2$, entonces $a=1$. Si $p=4k+1$, el pequeño teorema de Fermat nos dice que 
las raíces del polinomio $X^{p-1}-1$ 
con coeficientes en $\Z/p$ son $1,2,\dots,p-1$. Escribimos
\[
X^{p-1}-1=X^{4k}-1=(X^{2k}-1)(X^{2k}+1)=(X-1)(X-2)\cdots (X-(p-1))
\]
en $(\Z/p)[X]$. Como $p$ es primo, $\Z/p$ es un cuerpo y luego $(\Z/p)[X]$ es un dominio de factorización única, pues
$(\Z/p)[X]$ es un dominio de ideales principales por ser un dominio euclidiano. Existe entonces $\alpha\in\Z/p$
tal que $\alpha^{2k}+1=0$ y para terminar la demostración alcanza con tomar $a=\alpha^{2k}$. 

Demostremos ahora que $(2)\implies(3)$. Si $a^2\equiv -1\bmod p$, entonces $a^2+1=kp$ para algún $k\in\Z$. Como
$(a-i)(a+i)=a^2+1=kp$, entonces $p\mid (a-i)(a+i)$. Afirmamos que $p\nmid a+i$ en $\Z[i]$. Si $p\mid a+i$, entonces
$a+i=p(e+fi)$ para ciertos $e,f\in\Z$. Luego $1=pf$, una contradicción. De la misma forma se demuestra que $p\nmid a-i$. 
Sabemos entonces que $p\mid (a-i)(a+i)$ pero $p\nmid a-i$ y $p\nmid a+i$, es decir $p$ no es un primo de $\Z[i]$, por lo que 
tampoco será un irreducible de $\Z[i]$ (recordemos que $\Z[i]$ es un dominio de ideales principal y en dominios de ideales
principales primos e irreducibles son equivalentes).  

Veamos ahora que $(3)\implies(4)$. Como $p$ no es irreducible en $\Z[i]$, escribimos
$p=(a+bi)(c+di)$ con $a+bi\not\in\mathcal{U}(\Z[i])$ y $c+di\not\in\mathcal{U}(\Z[i])$, 
es decir $N(a+bi)\ne 1$ y $N(c+di)\ne 1$, y entonces
\[
p^2=N(p)=N(a+bi)N(c+di)=(a^2+b^2)(c^2+d^2).
\]
Como $\Z$ es un dominio de factorización única y $p$ es irreducible en $\Z$, $p=a^2+b^2$. 

Para finalizar, demostremos que $(4)\implies(1)$. Como $p$ es un número primo, entonces $p=2$, $p\equiv 1\bmod 4$ o bien $p\equiv 3\bmod 4$. Si $p\equiv 3\bmod 4$ y $p=a^2+b^2$, entonces
$a^2+b^2\equiv 3\bmod 4$, una contradicción pues los únicos casos posibles son $a^2+b^2\equiv 0\bmod 4$, $a^2+b^2\equiv 1\bmod 4$ o bien $a^2+b^2\equiv 2\bmod 4$.     
\end{proof}

%Veamos otra aplicación del anillo $\Z[i]$. Primero, un resultado auxiliar.
%
%\begin{lemma}
%	Si $a,b\in\Z\setminus\{0\}$ son coprimos y $m\in\Z$, entonces $a+bi$ divide a $m$ en $\Z[i]$ si y sólo si $N(a+bi)$ divide a $m$ en $\Z$.
%\end{lemma}
%
%\begin{proof}
%	Supongamos que $N(a+bi)\mid m$ en $\Z$, es decir $m=N(a+bi)k$ para algún $k$. Entonces $a+bi\mid m$ en $\Z[i]$, pues 
%	$m=(a^2+b^2)k=(a+bi)(a-bi)k$.
%	
%	Supongamos ahora que $a+bi$ divide a $m$, es decir 
%	\[
%	m=(a+bi)(c+di)=(ac-bd)+i(bc+ad)
%	\]
%	para ciertos $c,d\in\Z$. Como entonces $bc+ad=0$, tenemos $ad=-bc$, de donde se obtiene que $a\mid bc$ y $b\mid ad$. Como $a$ y $b$ son coprimos,
%	$a\mid c$ y $b\mid d$, es decir $c=ax$ y $d=by$ para ciertos $x,y\in\Z$. Como entonces
%	$a(by)=ad=-bc=-b(ax)$, entonces $ab(y+x)=0$ y luego $x=-y$. En conclusión, 
%	\[
%	m=(a+bi)(c+di)=a^2x-b^2y+i(abx+aby)=x(a^2+b^2)=xN(a+bi).\qedhere
%	\]
%\end{proof}
	
Veamos ahora como aplicación que $\Z[i]$ puede utilizarse para resolver ecuaciones en $\Z$. 	
	
\begin{proposition}
La ecuación $y^3-1=x^2$ tiene únicamente una solución en $\Z$.
\end{proposition}

\begin{proof} 
Si $x$ es impar, entonces $x^2\equiv 1\bmod 4$. Luego $2\mid x^2+1$, pero además $4\nmid x^2+1$. Como además $y$ es par, tenemos
que $y^3=x^2+1$ es divisible por $8$, una contradicción. Luego $x$ es par e $y$ es impar. Escribimos
\[
y^3=x^2+1=(x-i)(x+i)
\]
Observemos que $x-i$ y $x+i$ no tienen factores en común pues si $d\in\Z[i]$ es tal que
$d\mid x+i$ y $d\mid x-i$, entonces $d=1$ por lo que vimos en el ejemplo~\ref{exa:Z[i]div} de la página~\pageref{exa:Z[i]div}. La factorización
única del anillo $\Z[i]$ implica que $x+i=((a+bi)u)^3$ para ciertos $a,b\in\Z$ y $u\in\mathcal{U}(\Z[i])=\{1,-1,i,-i\}$. 
Como entonces $u^3\in\{-1,1,i,-i\}$, sin perder generalidad podemos suponer que 
\[
x+i=(a+bi)^3=(a^3-3ab^2)+i(3a^2b-b^3).
\]
En particular, $1=3a^2b-b^3=b(3a^2-b^2)$, lo que implica que $b=1$ y $a=0$, es decir $(x,y)=(0,1)$. 
\end{proof}

\chapter{El lema de Zorn}

\index{Conjunto!parcialmente ordenado}
\index{Poset}
Un conjunto no vacío $R$ se dirá \textbf{parcialmente ordenado} si posee una relación $X$ en $R$ (es decir, $X\subseteq R\times R$)
tal que 
\begin{enumerate}   	
\item $(r,r)\in X$ para todo $r\in R$,
\item $(r,s)\in X$ y $(s,t)\in X$ implican que $(r,t)\in X$, y además
\item $(r,s)\in X$ y $(s,r)\in X$ implican que $r=s$.
\end{enumerate}

La notación que utilizaremos es la siguiente: $(r,s)\in X\Longleftrightarrow r\leq s$. Un conjunto parcialmente ordenado será denotado entonces como el par $(R,\leq)$. Las 
tres condiciones anteriores pueden reescribirse así:
\begin{enumerate}
\item $r\leq r$ para todo $r\in R$.
\item $r\leq s$ y $s\leq t$ implican que $r\leq t$.
\item $r\leq s$ y $s\leq r$ implican que $r=s$.
\end{enumerate}

Otra notación que utilizaremos frecuentemente: $r<s\Longleftrightarrow r\leq s$ y además $r\ne s$.

\index{Elementos!comparables en un poset}
Sea $(R,\leq)$ un conjunto parcialmente ordenado y sean $r,s\in R$. Diremos que $r$ y $s$ son \textbf{comparables}
si $r\leq s$ o bien $s\leq r$. 	

\begin{example}
Sea $U=\{1,2,3,4,5\}$ y sea $T$ el conjunto de subconjuntos de $U$. Definimos 
la relación $C\leq D\Longleftrightarrow C\subseteq D$. Luego $T$ es un conjunto parcialmente ordenado. Los
subconjuntos $\{1,2\}$ y $\{3,4\}$ no son comparables. 
\end{example}

Si $(R,\leq)$ es un conjunto parcialmente ordenado, diremos que un elemento $r\in R$ es \textbf{maximal} en $R$ 
si para todo $t\in R$ comparable con $r$, se tiene que $t\leq r$, es decir: para todo $t\in R$ tal que $r\leq t$ se tiene $r=t$. 	

\begin{example}
$(\Z,\leq)$ no tiene elementos maximales.
\end{example}

\begin{example}
Sea $R=\{(x,y)\in\R^2:y\leq 0\}$. Definimos la relación de orden parcial 
\[
(x_1,y_1)\leq (x_2,y_2)\Longleftrightarrow x_1=x_2\text{ y además }y_1\leq y_2.
\]
Entonces $(R,\leq)$ es un conjunto parcialmente ordenado. Cada elemento de la forma $(x,0)$ 
es un elemento maximal, pues si $(x,0)\leq (x_1,y_1)$, entonces $x=x_1$ y además $y_1=0$. En conclusión, $R$ tiene
una infinidad de elementos maximales. 	
\end{example}


\index{Cota superior}
\index{Lema!de Zorn}
\index{Cadena}
Si $R$ es un conjunto parcialmente ordenado, una \textbf{cota superior} de un subconjunto no vacío $S$ de $R$ 
será un elemento $u\in R$ tal que $s\leq u$ para todo $s\in S$. 


\begin{example}
El conjunto $S=\{6\Z,12\Z,24\Z\}$ de subgrupos de $\Z$ 
es totalmente ordenado con la inclusión. El elemento $6\Z=6\Z\cup 12\Z\cup 24\Z$ es
cota superior de $S$. 	
\end{example}

Un subconjunto no vacío $S$ de $R$ 
será una \textbf{cadena} si dos elementos cualesquiera de $S$ son comparables. El \textbf{lema de Zorn} 
afirma lo siguiente:


\begin{quote}
Si $R$ es un conjunto parcialmente ordenado tal que 
toda cadena en $R$ admite una cota superior en $R$, entonces
$R$ contiene un elemento maximal.	
\end{quote}

Como se ve, el lema de Zorn nada tiene de intuitivo. Curiosamente, es lógicamente equivalente al axioma de elección y al principio de buena ordenación. 
En realidad, el lema 
de Zorn es un axioma y no un resultado que debe demostrare. 

En vez de profundizar más en los aspectos lógicos del lema de Zorn y sus equivalencias, nos contentaremos con dar una aplicación.

\begin{definition}
\index{Ideal!maximal}
	Sea $R$ un anillo. Diremos que un ideal $I\ne R$ es \textbf{maximal} si dado un ideal $J$ de $R$ tal que 
	$I\subseteq J$, entonces $I=J$ o bien $J=R$.  
\end{definition}

\begin{example}
Si $p$ es un número primo, entonces $p\Z$ es un ideal maximal de $\Z$.
\end{example}

\begin{exercise}
Sea $R$ un anillo. Entonces $R$ es un cuerpo si y sólo si $\{0\}$ es un ideal maximal. 	
\end{exercise}

\begin{exercise}
Sea $R$ un anillo. Un ideal $I$ de $R$ es maximal si y sólo si $R/I$ es un cuerpo.
\end{exercise}

Ahora sí, el teorema.

\begin{theorem}
Sea $R$ un anillo no nulo. Todo ideal $I\ne R$ está contenido en un ideal maximal. 
En particular, todo anillo no nulo tiene ideales maximales. 
\end{theorem}

\begin{proof}
Sea 
\[
X=\{J\subseteq R:J\text{ es un ideal tal que }I\subseteq J\subsetneq R\}.
\]
Como $I\in X$, entonces $X$ es no vacío. Luego 
$X$ es un conjunto parcialmente ordenado con la inclusión. Si $C$ es una cadena en $X$, entonces $\cup_{J\in C}J\in X$
es una cota superior para $C$, pues $\cup_{J\in C}J$ es un ideal de $R$ y $1\not\in \cup_{J\in C}J$, lo que
implica que $\cup_{J\in C}J\ne R$. Por el lema de Zorn, existe entonces $M\in X$ un elemento maximal. 

Afirmamos que $M$ es un ideal maximal de $R$. En efecto, si $M_1$ es un ideal propio de $R$ tal que $M\subseteq M_1$, entonces
$I\subseteq M_1$ y luego $M_1\in X$, lo que implica que $M=M_1$ pues $M$ es maximal en $X$.   
\end{proof}

En el teorema anterior, el hecho crucial es la existencia de la unidad del anillo. 
De hecho, el álgebra no conmutativa nos muestra que existen anillos (no son anillos unitarios, obviamente) 
que no poseen ideales maximales.

\begin{example}
Sea $K$ un cuerpo. 
Los ideales maximales de $K[X]$ son los ideales principales generados por los polinomios mónicos irreducibles. En efecto,
si $I$ es un ideal maximal, entonces $I=(f)$ para algún $f\in K[X]$, pues sabemos que $K[X]$ es principal. Si $\deg f=n$ y 
$a$ es el coeficiente principal de $f$, entonces $g=a^{-1}f$ es un polinomio mónico tal que
\[
(a^{-1}f)=(f).
\] 
Podemos suponer entonces, sin perder generalidad, que $f$ es mónico. Si $f$ es irreducible y $(f)\subseteq J\subseteq K[X]$, escribimos
$J=(g)$ para algún $g\in K[X]$. Entonces $f=gh$ para algún $h\in K[X]$, lo que implica que $g$ o $h$ son constantes. Si $g$ es constante, 
entonces $(g)=K[X]$. Si $h$ es constante, digamos $h=a\in K$, entonces $f=ga$, lo que implica que $g=a^{-1}f\in (f)$ y luego $I=J$.   
\end{example}

%\begin{definition}
%\end{definition}
\begin{exercise}
Sea $R$ un dominio de ideales principales. Demuestre que $p\in R$ es irreducible si y sólo si $(p)$ es un ideal maximal. 	
\end{exercise}

Un caso particular del ejercicio anterior. Si $K$ es un cuerpo, $f\in K[X]$ es irreducible si y sólo si el
ideal $(f)$ es maximal.  

\begin{example}
Sean $K$ un cuerpo y $S$ un dominio íntegro. Si $\varphi\in K[X]\to S$ es un morfismo de anillos, entonces $\ker \varphi=\{0\}$ o bien $\ker\varphi$ es maximal. En efecto,
como $K[X]$ es principal, sabemos que $\ker\varphi=(f)$ para algún $f\in K[X]$. Si $\ker\varphi=0$, no hay nada que demostrar. 
Si $f$ es irreducible, entonces $(f)$ es maximal. En caso contrario, 
digamos $f=gh$, tenemos 
\[
0=\varphi(f)=\varphi(gh)=\varphi(g)\varphi(h).
\]
Como $S$ es un dominio íntegro, $\varphi(g)=0$ o bien $\varphi(h)=0$. Sin perder generalidad, podemos suponer que $\varphi(g)=0$,
es decir $g\in\ker \varphi$. Luego $(f)\subseteq (g)=\ker\varphi$ y entonces $h$ es una unidad, una contradicción. 
\end{example}

\begin{example}
El ideal $(X^2+2X+2)$ es maximal en $\Q[X]$ pues 
\[
X^2+2X+2=(X+1)^2+1>0
\]
es irreducible en $\Q[X]$, por ser un polinomio de grado dos sin raíces racionales.  	
\end{example}

\begin{exercise}
Demuestre que $R/I$ es un cuerpo si y sólo si $I$ es un ideal maximal.	
\end{exercise}
	
\begin{example}
Sea $R=(\Z/2)[X]$. Como $X^2+X+1$ es irreducible en $R$, el ideal $I=(X^2+X+1)$ es maximal. Luego $R/I$ es un cuerpo.
\end{example}

\begin{exercise}
\label{xca:Jacobson}
Sea $R$ un anillo conmutativo y sea $J(R)$ la intersección de todos
los ideales maximales de $R$. Pruebe que $x\in J(R)$ si y sólo si
$1-xy\in\mathcal{U}(R)$ para todo $y\in R$. 
\end{exercise}

\begin{exercise}
\label{xca:maxZn}
Los ideales maximales de $\Z/n$ son de la forma $I=\Z/p$ donde $p$ es un primo que divide a $n$.
\end{exercise}

Veamos ahora una aplicación a la teoría de grupos. 

\index{Subgrupo!maximal}
Un subgrupo propio $M$ de un grupo $G$ se dice \textbf{maximal} si $M\subseteq H\subseteq G$ para algún subgrupo $H$ de $G$ implica que
$M=H$ o $H=G$, es decir que el subgrupo $M$ es maximal con respecto a la inclusión entre los subgrupos propios de $G$. Quizá sería
mejor definir estos subgrupos como maximal-propio, pero, tal como se hace en la literatura, nos quedaremos con la terminología estándar.  

\begin{exercise}
	Demuestre que $M\leq\Z$ es maximal si y sólo si $M=p\Z$ para algún primo $p$. 
\end{exercise}

\begin{exercise}
    Demuestre que $\Q$ no tiene subgrupos maximales.        
\end{exercise}

\begin{exercise}
        Demuestre que todo subgrupo propio de un grupo finito está contenido en
        algún subgrupo maximal.
\end{exercise}

El resultado del ejercicio anterior puede extenderse a grupos finitamente generados gracias al lema de Zorn. 

\begin{theorem}
Sea $G$ un grupo no trivial y finitamente generado. Todo subgrupo propio de $G$ está contenido en un subgrupo maximal.  	
\end{theorem}

\begin{proof}
	Supongamos que $G=\langle g_1,\dots,g_n\rangle$ y sea $K$ un subgrupo propio de $G$. Para cada $j\in\{0,\dots,n\}$ se define
	$G_j=\langle K,g_1,\dots,g_j\rangle$. Como
	\[
	K=G_0\subseteq G_1\subseteq G_2\subseteq\cdots\subseteq G_n=G,
	\]
	existe entonces $l=\max\{j:0\leq j\leq n-1,\,G_j\ne G\}$. Luego $G_{l+1}=\langle G_l,g_{l+1}\rangle=G$ y
	además $g_{l+1}\not\in G_l$. Sea
	\[
	S=\{H:H\leq G,\,G_l\subseteq H,\,g_{l+1}\not\in H\}
	\]
	ordenado parcialmente con la inclusión. Como $G_l\in S$, entonces $S\ne\emptyset$. Dejamos como ejercicio
	demostrar que si $\{H_i:i\in I\}\subseteq S$ es
	totalmente ordenado, entonces $H=\cup_{i\in I}H_i$ es una cota superior de $S$. Por el lema de Zorn sabemos
	entonces que $S$ tiene un elemento maximal, digamos $M$, es decir que $M$ es maximal con 
	respecto a las siguientes propiedades: $M\leq G$, $G_l\subseteq M$ y $g_{l+1}\not\in M$.   
	
	Vamos a demostrar ahora que $M$ es un subgrupo maximal de $G$ que contiene a $K$. 
	Como $K\subseteq G_l\subseteq M$, entonces $M$ contiene a $K$. Para ver que $M$ es maximal, supongamos que
	$M\leq L\leq G$. Si $g_{l+1}\not\in L$, entonces, por definición, $L	\in S$, pero esto contradice la maximalidad
	del conjunto $M$. Luego $g_{l+1}\in L$, lo que implica que $\langle M,g_{l+1}\rangle\subseteq L$. Como $G_l\subseteq M$, entonces
	\[
	G=G_{l+1}=\langle G_l,g_{l+1}\rangle\subseteq \langle M,g_{l+1}\rangle.   
	\]
	En consecuencia, $G\subseteq L$ y luego $L=G$. 
\end{proof}


\chapter{Álgebras}

En este capítulo veremos cierto tipo de anillos que además son espacios vectoriales de forma que
la acción por escalares y la estructura de anillo son compatibles. 
Este concepto es de gran importancia en álgebra. 

\begin{definition}
	\label{Álgebra}
	\label{Álgebra!asociativa}
	\label{Álgebra}
	Un espacio vectorial $A$ sobre un cuerpo $K$ es un \textbf{algebra} sobre $K$
	(o una $K$-álgebra) si posee una multiplicación asociativa $A\times A\to A$,
	$(a,b)\mapsto ab$, tal que
	$(\lambda a+\mu b)c=\lambda(ac)+\mu(bc)$ y 
	$a(\lambda b+\mu c)=\lambda(ab)+\mu(ac)$ 
	para todo $a,b,c\in A$ y $\lambda,\mu\in K$. Existe además un 
	elemento $1_A\in A$ tal que $1_Aa=a1_A=a$
	para todo $a\in A$.
\end{definition}

\index{Álgebra!conmutativa}
Un álgebra $A$ se dirá \textbf{conmutativa} si $ab=ba$ para todo $a,b\in A$. 

\index{Álgebra!dimensión}
La \textbf{dimensión} de un álgebra $A$ es la dimensión de $A$ como $K$-espacio
vectorial. Justamente esta es quizá una de las claves de la definición, 
un álgebra es en particular un espacio vectorial y cuando sea necesario podremos 
utilizar argumentos que involucren el concepto de dimensión. 

\begin{example}
	Todo cuerpo $K$ es una $K$-álgebra. 
\end{example}

\begin{example}
\index{Álgebra!de polinomios}
	Si $K$ es un cuerpo, $K[X]$ es una $K$-álgebra. 
\end{example}

Similarmente, el anillo de polinomios $K[X,Y]$ y el anillo $K[[X]]$ de series de potencias son ejemplos de álgebras sobre el cuerpo $K$. 

\begin{example}
\index{Álgebra!de matrices}
	Si $A$ es un álgebra, entonces $M_n(A)$ es un álgebra. 
\end{example}

\begin{example}
El conjunto de funciones continuas $[0,1]\to\R$ es un álgebra sobre $\R$ con las operaciones usuales, $(f+g)(x)=f(x)+g(x)$, $(fg)(x)=f(x)g(x)$. 
\end{example}


\index{Morfismo!de álgebras}
Un \textbf{morfismo de álgebras} es un morfismo de anillos $f\colon A\to B$ que es además una transformación lineal. Observemos 
que es necesario pedir que un morfismo de álgebras sea una transformación lineal, por ejemplo, la conjugación 
$\C\to \C$, $z\mapsto\overline{z}$, es un morfismo de anillos que no es un morfismo de álgebras sobre $\C$. 

\begin{definition}
	\index{Álgebra!ideal de un}
	Un \textbf{ideal} de un álgebra es un ideal del anillo que además es un
	subespacio. 
\end{definition}

Análogamente se definen ideales a izquierda y a derecha de un álgebra.

	Si $A$ es un álgebra, entonces todo ideal a izquierda 
	del anillo $A$ es un ideal a izquierda del
	álgebra $A$. Si $L$ es un ideal de $A$ y $\lambda\in K$ y $x\in L$,
	entonces
	\[
		\lambda x=\lambda (1_Ax)=(\lambda 1_A)x
	\]
	y luego, como $\lambda 1_A\in A$, se concluye que $\lambda L=(\lambda
	1_A)L\subseteq L$. Análogamente se demuestra que todo ideal a derecha del
	anillo unitario $A$ es también un ideal de $A$ como álgebra. 

\begin{exercise}
	Demuestre que si $A$ es un álgebra, entonces todo ideal a derecha
	del anillo $A$ es un ideal a derecha del álgebra $A$.
\end{exercise}

Puede demostrarse que si 
$A$ es un álgebra e $I$ es un ideal de $A$, entonces el anillo cociente $A/I$ 
tiene una única estructura de álgebra que hace que el morfismo canónico 
$A\to A/I$, $a\mapsto a+I$, sea un morfismo de álgebras. 

\begin{example}
\index{Álgebra!de polinomios truncados}
Si $n\in\N$, entonces $K[X]/(X^n)$ es un álgebra de dimensión finita, se conoce como el \textbf{álgebra de polinomios truncados}.  \end{example}

\index{Elemento!algebraico}
\index{Álgebra!algebraica}
Sea $A$ un álgebra. Un elemento $a\in A$ se dice
\textbf{algebraico} sobre $A$ si existe un polinomio no nulo $f\in K[X]$
tal que $f(a)=0$. Si todo elemento de $A$ es algebraico, $A$ se dice
\textbf{algebraica}. Por ejemplo, sabemos que en la $\Q$-álgebra $A=\R$ el elemento $\sqrt{2}$ es algebraico, pues $\sqrt{2}$ es raíz del polinomio $X^2-2\in\Q[X]$,  
y que $\pi$ no lo es. Todo elemento de $\R$ como $\R$-álgebra es algebraico.

\begin{proposition}
	\label{lem:algebraica}
	Toda álgebra de dimensión finita es algebraica. 
\end{proposition}

\begin{proof}
   Sea $A$ un álgebra de dimensión finita $n$    
	y sea $a\in A$. Como el conjunto 
	$\{1,a,a^2,\dots,a^n\}$ es linealmente dependendiente, existe un polinomio
	no nulo $f\in k[X]$ tal que $f(a)=0$.
\end{proof}

\index{Álgebra!de grupo}
Sea $K$ un cuerpo y sea $G$ un grupo finito. El \textbf{álgebra de grupo} $K[G]$ es el
$K$-espacio vectorial con base $\{g:g\in G\}$ con la estructura de álgebra dada
por el producto
\[
	\left(\sum_{g\in G}\lambda_gg\right)\left(\sum_{h\in G}\mu_hh\right)
	=\sum_{g,h\in G}\lambda_g\mu_h(gh).
\]
Observemos que el álgebra $K[G]$ es conmutativa si y sólo si $G$ es abeliano. 
Además $\dim K[G]=|G|$. 

\begin{example}
Sea $G=\{1,g,g^2\}$ el grupo cíclico de orden tres y sea $A=\C[G]$ el álgebra (compleja) del grupo $G$. Si 
$\alpha=a_11+a_2g+a_3g^2$ y $\beta=b_11+b_2g+b_3g^2\in A$, donde $a_1,a_2,a_3,b_1,b_2,b_3\in\C$, 
entonces la suma de $A$ está dada por
\begin{gather*}
\alpha+\beta=(a_1+b_1)1+(a_2+b_2)g+(a_3+b_3)g^2
\shortintertext{y el producto por}
\alpha\beta=(a_1b_1+a_2b_3+a_3b_2)1+(a_1b_2+a_2b_1+a_3b_3)g+(a_1b_3+a_2b_2+a_3b_1)g^2.
\end{gather*}
\end{example}

\index{Ideal!de aumentación}
Si $G$ es un grupo finito no trivial,  
entonces $K[G]$ posee ideales propios no triviales. 
Esto es porque el conjunto 
\[
	I(G)=\left\{\sum_{g\in G}\lambda_gg\in K[G]:\sum_{g\in G}\lambda_g=0\right\}
\]
es un ideal propio y no nulo de $K[G]$ (pues $\dim I(G)=\dim K[G]-1$). Este
conjunto se conoce como el \textbf{ideal de aumentación} de $K[G]$.

\begin{exercise}
	Sea $G=C_n$ el grupo ciclico de orden $n$ (escrito multiplicativamente).
	Demuestre que $K[G]\simeq K[X]/(X^n-1)$. 
\end{exercise}

\begin{proposition}
Si $G$ es un grupo finito no trivial, entonces $K[G]$ tiene divisores de cero.	
\end{proposition}

\begin{proof}
Sea $g\in G\setminus\{1\}$ y sea $n$ el orden de $g$. Para ver que $K[G]$ tiene divisores
de cero alcanza con observar que 
$(1-g)(1+g+\cdots+g^{n-1})=0$. 
\end{proof}

Si $A$ es un álgebra, entonces $\mathcal{U}(A)$ es el grupo de unidades del anillo $A$. 
La proposición que sigue se conoce como la propiedad universal del álgebra de grupo.

\begin{proposition}
Sean $A$ un álgebra y $G$ un grupo finito. Si $f\colon G\to\mathcal{U}(A)$ es un morfismo de grupos, entonces
existe un único morfismo $\varphi\colon K[G]\to A$ de álgebras tal que la restricción
$\varphi|_G$ de $\varphi$ al grupo $G$ es igual a $f$, es decir 	$\varphi|_G=f$. 
\end{proposition}

\begin{proof}
Como $G$ es base de $K[G]$, puede verificarse que 
el morfismo $\varphi$ de álgebras 
queda unívocametne determinado por 
\[
\varphi\left(\sum_{g\in G}\lambda_gg\right)=\sum_{g\in G}\lambda_gf(g).\qedhere
\]	
\end{proof}

La proposición anterior nos dice que si $G$ es un grupo finito y $A$ es un álgebra, 
para definir un morfismo de álgebras $K[G]\to A$ 
alcanza con tener un morfismo de grupos $G\to\mathcal{U}(A)$.  


%\chapter{El teorema de los ceros de Hilbert}

En este capítulo daremos una versión elemental de un
caso particular del teorema de los ceros de Hilbert.
Primero necesitamos un lema.

\begin{lemma}
\begin{enumerate}
    \item Si $\C$ es un subanillo de $R$, entonces $R$ es un espacio vectorial complejo.
    \item Si $M$ es un ideal maximal de $\C[X_1,\dots,X_n]$, entonces $E=\C[X_1,\dots,X_n]/M$ está generado por una cantidad numerable de elementos.
    \item Si $V$ es un espacio vectorial generado por una cantidad numerable de elementos, entonces cada subconjunto de $V$ 
    linealmente independiente es finito o infinito numerable. 
    \item Si $\C(X)=F(\C[X])$ es un espacio vectorial complejo y el conjunto $\{(X-\alpha)^{-1}:\alpha\in\C\}$ es linealmente independiente.
\end{enumerate}
\end{lemma}

\begin{proof}

\end{proof}

Necesitamos además el siguiente lema sobre el cuerpo de fracciones de un dominio. 

\begin{lemma}
Sea $R$ un dominio íntegro. 
\begin{enumerate}
    \item La función $\varphi\colon R\to F(R)$, $x\mapsto \frac{x}{1}$, es un morfismo inyectivo de anillos.
    \item Si $f\colon R\to S$ es un morfismo de anillos tal que $f(x)\in\mathcal{U}(S)$ para todo $x\in R$, entonces existe
    un único morfismo $g\colon F(R)\to S$ de anillos tal que $g\circ\varphi=f$.
    \item Si $\varphi\colon R\to F(R)$ es la inclusión y $E$ es un cuerpo y $f\colon R\to E$ es un morfismo inyectivo de anillos, entonces existe un único morfismo 
    $g\colon F(R)\to E$ de anillos tal que $g|_R=f$. 
\end{enumerate}
\end{lemma}

\begin{proof}

\end{proof}

\begin{theorem}[de los ceros de Hilbert]
\index{Teorema!de los ceros de Hilbert}
Los ideales maximales de $\C[X_1,\dots,X_n]$ están en biyección con 
los puntos de $\C^n$.
\end{theorem}

\begin{proof}
Sea $M$ un idela maximal de $\C[X_1,\dots,X_n]$. Como $E=\C[X_1,\dots,X_n]/M$ es un cuerpo, la restricción
$f_i=\pi|_{\C[X_i]}\colon\C[X_i]\to E$ es un morfismo inyectivo de anillos. Entonces $\ker f_i=\{0\}$ o bien $\ker f_i$ es un ideal maximal. Los ideales 
maximales de $\C[X_i]$ son de la forma $X_i-a_i$ para $a_i\in\C$. 

Afirmamos que $\ker f_i=\{0\}$. De lo contrario...

\end{proof}

\part{Módulos}
\chapter{Módulos}
\label{modulos}

Un módulo sobre un anillo $R$ será un grupo aditivo junto con un morfismo
de anillos $R\to\End(M)$, que será la acción de $R$ en $M$. Al traducir
qué significa tener tal morfismo de anillos, obtenemos la siguiente definición:

\begin{definition}
\index{Módulo!sobre un anillo}
Sea $R$ un anillo. Un grupo abeliano aditivo $M$ junto con una operación
$R\times M\to M$, $(x,m)\mapsto x\cdot m$, 
será un \textbf{módulo} (a izquierda) sobre $R$ (o también $R$-módulo a izquierda) si
se cumplen las siguientes propiedades:
\begin{enumerate}
\item $(r_1+r_2)\cdot m=r_1\cdot m+r_2\cdot m$ para todo $r_1,r_2\in R$ y $m\in M$.
\item $r\cdot (m_1+m_2)=r\cdot m_1+r\cdot m_2$ para todo $r\in R$ y $m_1,m_2\in M$.
\item $r_1\cdot (r_2\cdot m)=(r_1r_2)\cdot m$ para todo $r_1,r_2\in R$ y $m\in M$.
\item $1\cdot m=m$ para todo $m\in M$.	
\end{enumerate}
\end{definition}

Similarmente uno puede definir módulos a derecha. 

Tabajaremos con módulos a izquierda, por lo tanto los llamaremos simplemente módulos y no habrá peligro de confusión.  Muchas veces no haremos referencia al anillo
sobre el que se define el módulo.

\begin{example}
Si $R$ es un cuerpo, entonces un $R$-módulo 	es un espacio vectorial. 
\end{example}

\begin{example}
Todo grupo abeliano es un $\Z$-módulo.	
\end{example}

\begin{example}
\index{Representación!regular de un anillo}
Si $R$ es un anillo, entonces $R$ es un $R$-módulo con $x\cdot m=xm$. 
Este módulo es la \textbf{representación regular (a izquierda)} de $R$.  
La notación que utilizaremos para este módulo será $M=\prescript{}{R}R$. 
\end{example}

\begin{example}
Si $R$ es un anillo, $R^n=\{(x_1,\dots,x_n):x_1,\dots,x_n\in R\}$ 
es un $R$-módulo con 
$r\cdot (x_1,\dots,x_n)=(rx_1,\dots,rx_n)$. 
\end{example}

\begin{example}
Si $R$ es un anillo, $M_{m,n}(R)$ es un $R$-módulo.
\end{example}

El siguiente ejemplo explica por qué es útil 
pedir que un morfismo $f$ 
de anillos cumpla con la condición $f(1)=1$. 

\begin{example}
\label{exa:f(1)=1}
Si $f\colon R\to S$ es un morfismo de anillos y $M$ es un $S$-módulo con la acción $(s,m)\mapsto sm$, entonces
$M$ es un $R$-módulo con $r\cdot m=f(r)m$ para $r\in R$ y $m\in M$. En efecto,
\begin{align*}
&1\cdot m=f(1)m=1m=m,\\
&r_1\cdot (r_2\cdot m)=f(r_1)(r_2\cdot m)=f(r_1)(f(r_2)m)=(f(r_1)f(r_2))m=f(r_1r_2)m
\end{align*}
para todo $r_1,r_2\in R$ y $m\in M$.	  	
\end{example}

El ejemplo siguiente es particularmente importante.

\begin{example}
Sean $R=\R[X]$, $T\colon\R^n\to\R^n$ una transformación lineal y $M=\R^n$, entonces $M$ es un $R$-módulo con 
\[
\left(\sum_{i=0}^na_iX^i\right)\cdot v=\sum_{i=0}^na_iT^i(v).
\]	
\end{example}

\begin{example}
\index{Producto directo!de módulos}
Si $\{M_i|i\in I\}$ es una familia de módulos, entonces  	
\[
\prod_{i\in I}M_i=\{(m_i)_{i\in I}:m_i\in M_i\text{ para todo $i\in I$}\}
\]
es un módulo con
la operación 
$x\cdot (m_i)_{i\in I}=(x\cdot m_i)_{i\in I}$, 
donde el símbolo $(m_i)_{i\in I}$ denota a la función $I\to M_i$, $i\mapsto m_i$.
Este módulo se conoce como el \textbf{producto directo} de los $M_i$.
\end{example}

\begin{example}
\index{Suma directa!de módulos}
Si $\{M_i|i\in I\}$ es una familia de módulos, entonces  	
\[
\bigoplus_{i\in I}M_i=\{(m_i)_{i\in I}:m_i\in M_i\text{ para todo $i\in I$ y $m_i=0$ salvo finitos $i\in I$}\}
\]
es un módulo con la operación
$x\cdot (m_i)_{i\in I}=(x\cdot m_i)_{i\in I}$. 
Este módulo se conoce como la \textbf{suma directa} de los $M_i$. 
\end{example}

\begin{exercise}
Si $M$ es un módulo, entonces
\begin{enumerate}
\item $0\cdot m=0$ para todo $m\in M$,
\item $x\cdot 0=0$ para todo $x\in R$ y además
\item $-m=(-1)\cdot m$ para todo $m\in M$. 	
\end{enumerate}
\end{exercise}

\begin{example}
$M=\Z/6$ es un $\Z$-módulo tal que $3\cdot 2=0$ pero $3\ne 0$ (en $\Z$) y $2\ne 0$ (en $\Z/6$).  
\end{example}

\begin{definition}
\index{Submódulo}
	Sea $M$ un $R$-módulo. Un subconjunto $S$ de $M$ será un \textbf{submódulo} de $M$ si $(S,+)$ es un subgrupo de $(M,+)$ y además
	$x\cdot s\in S$ para todo $x\in R$ y $s\in S$. 
\end{definition}

\begin{examples}
Sea $M$ un $R$-módulo. 
\begin{enumerate}
\item $\{0\}$ y $M$ son submódulos de $M$.
\item Si $R$ es un cuerpo, $S$ es un submódulo de $M$ si y sólo si $S$ es un subespacio de $M$.
\item Si $R=\Z$, entonces $S$ es un submódulo si y sólo si $S$ es un subgrupo de $M$. 	
\end{enumerate}
\end{examples}

\begin{example}
Si $M=\prescript{}{R}R$, entonces $S\subseteq M$ es un submódulo si y sólo si $S$ es un ideal a izquierda de $R$
\end{example}

\begin{example}
Si $V$ es un espacio vectorial y $T\colon V\to V$ es una transformación lineal, entonces
$V$ es un $K[X]$-módulo con 
\[
\left(\sum_{i=0}^na_iX^i\right)\cdot v=\sum_{i=0}^na_iT^i(v).
\]
Un submódulo $W$ será entonces un subespacio $T$-invariante de $V$, es decir un subspacio vectorial de $V$ 
tal que $T(W)\subseteq W$. 
\end{example}

\begin{exercise}
Demuestre que un subconjunto $S$ de $M$ es un submódulo si y sólo si $r_1s_1+r_2s_2\in S$ para
todo $r_1,r_2\in R$ y $_1,s_2\in S$. 	
\end{exercise}

\begin{exercise}
Si $S$ y $T$ son submódulos de $M$, entonces 
\[
S+T=\{s+t:s\in S,\,t\in T\}
\]
es un submódulo de $M$.
\end{exercise}

\begin{definition}
\index{Morfismo!de módulos}
Sean $M$ y $N$ módulos sobre $R$. 
Un \textbf{morfismo} de módulos es una función $f\colon M\to N$ tal que $f(x+y)=f(x)+f(y)$ y 
$f(r\cdot x)=r\cdot f(x)$ para todo $x,y\in M$ y $r\in R$. 
\end{definition}

\begin{exercise}
Sea $f\colon M\to N$ un morfismo de módulos. 
\begin{enumerate}
\item Si $S$ es un submódulo de $M$, entonces $f(S)$ es un submódulo de $N$.
\item Si $T$ es un submódulo de $N$, entonces $f^{-1}(T)$ es un submódulo de $M$.
\end{enumerate}
\end{exercise}

\index{Núcleo!de un morfismo de módulos}
\index{Monomorfismo!de módulos}
\index{Epimorfismo!de módulos}
\index{Isomorfismo!de módulos}
Si $f\in\Hom_R(M,N)$, se define el \textbf{núcleo} de $f$ como el submódulo 
\[
\ker f=f^{-1}(\{0\})=\{m\in M:f(m)=0\}
\]
de $M$. Diremos que $f$ es un \textbf{monomorfismo} si $f$ es inyectiva, que
es un \textbf{epimorfismo} si $f$ es sobreyectiva y que es un \textbf{isomorfismo} 
si $f$ es biyectiva. 

\begin{exercise}
Sea $f\in\Hom_R(M,N)$. Son equivalentes:
\begin{enumerate}
\item $f$ es monomorfismo.
\item $\ker f=\{0\}$.
\item Para todo módulo $T$ y todo $g,h\in\Hom_R(T,M)	$, $f\circ g=f\circ h\implies g=h$.
\item Para todo módulo $T$ y todo $g\in\Hom(T,M)$, $f\circ g=0\implies g=0$.
\end{enumerate}
\end{exercise}


\begin{example}
	Sea $R=
		\begin{pmatrix}
			\R & 0\\
			0 & \R
		\end{pmatrix}$. 
	Veamos que
	$\begin{pmatrix}
			\R\\
			0
		\end{pmatrix}
		\not\simeq\begin{pmatrix}
			0\\
			\R
		\end{pmatrix}$
	como $R$-módulos, donde la estructura de módulos está dada por la multiplicación usual de matrices. 
	Sea 
	$f\colon\begin{pmatrix}
			0\\
			\R
		\end{pmatrix}
		\to\begin{pmatrix}
			\R\\
			0
		\end{pmatrix}$  
	un isomorfismo de módulos y sea 
	$x_0\in\R\setminus\{0\}$ 
	tal que $f\begin{pmatrix}0\\1\end{pmatrix}=\begin{pmatrix}x_0\\0\end{pmatrix}$. Entonces
	\[
	f\begin{pmatrix}
	0\\
	1\end{pmatrix}
	=f\left(\begin{pmatrix}
	0&0\\
	0&1\end{pmatrix}
	\cdot 
	\begin{pmatrix}
	0\\
	1
	\end{pmatrix}\right)
	=\begin{pmatrix}
	0&0\\
	0&1\end{pmatrix}\cdot f\begin{pmatrix}0\\1\end{pmatrix}
	=\begin{pmatrix}
	0&0\\
	0&1
	\end{pmatrix}
	\cdot 
	\begin{pmatrix}		
	x_0\\
	0
	\end{pmatrix}
	=\begin{pmatrix}
	0\\
	0
	\end{pmatrix},
	\]	
	una contradicción pues $f$ es inyectiva.   
\end{example}

\index{Suma directa!de módulos}
\index{Complemento!de un submódulo}
\index{Sumando directo!de un módulo}
Si $S$ y $T$ son submódulos de $M$, diremos que $M$ es \textbf{suma directa} de $S$ y $T$ si 
$M=S+T$ y además $S\cap T=\{0\}$. La notación que utilizaremos en este caso será $M=S\oplus T$. Observemos que
si $M=S\oplus T$, entonces 
todo $m\in M$ puede escribirse unívocamente como $m=s+t$ para ciertos $s\in S$ y $t\in T$. En efecto,
la descomposición existe gracias a que $M=S+T$. Si $m\in M$ se descompone como
$m=s+t=s_1+t_1$, donde $s,s_1\in S$ y $t,t_1\in T$, entonces
$-s_1+s=t_1-t\in S\cap T=\{0\}$ y luego $s=s_1$ y $t=t_1$.  Si $M=S\oplus T$, el submódulo $S$ es un \textbf{sumando directo} de $M$ y el submódulo $T$ es un \textbf{complemento} para $S$ en $M$.   	
\begin{examples}\
	\begin{enumerate}
	\item Para todo módulo $M$, los submódulos $\{0\}$ y $M$ son sumandos directos de $M$.
	\item Si $M=\R^2$ con la estructura usual de espacio vectorial, entonces todo subespacio de $M$ es un sumando directo. 	
	\end{enumerate}
\end{examples}

\begin{example}
Si $M=\Z$ como $\Z$-módulo, $m\Z$ es sumando directo de $M$ si y sólo si $m\in\{0,1\}$, pues $n\Z\cap m\Z=\{0\}$ si y sólo si $nm=0$.
\end{example}

\begin{proposition}
\index{Proyector}
Un módulo $N$ es isomorfo a un sumando directo del módulo $M$ si y sólo si 
existen morfismos $i\colon N\to M$ y $p\colon M\to N$ 
tales que $p\circ i=\id_N$. En este caso, $M=\ker p\oplus i(N)$.  
\end{proposition}

\begin{proof}
	Supongamos que $N$ es isomorfo a un sumando directo de $M$, es decir $M=S\oplus T$ y sea $s\colon N\to S$ un isomorfismo. Para
	cada $m\in M$ existen únicos $s\in S$ y $t\in T$ tales que $m=s+t$. Definimos entonces el epimorfismo 
	$q\colon M\to S$, $m\mapsto s$. Observemos que $q(m)=m$ si y sólo si $m\in S$, 
	es decir que $q$ es un \textbf{proyector} de $M$ sobre $S$ con respecto a $T$. 
	Definimos además el morfismo 
	$i\colon N\to M$, $n\mapsto s(n)$, y el morfismo $p\colon M\to N$, $m\mapsto s^{-1}(q(m))$. Como $s(n)\in S$, 
	\[
	p(i(n))=p(s(n))=s^{-1}(q(s(n)))=s^{-1}(s(n))=n
	\]
	para todo $n\in N$. 
	
	Demostremos ahora la recíproca. Afirmamos que $i$ es monomorfismo: si $i(n)=0$, entonces
	$n=p(i(n))=p(0)=0$. Luego $i\colon N\to i(N)$ es un isomorfismo. 
	Veamos ahora que $M=\ker p\oplus i(N)$. Si $m\in M$, entonces
	\[
	m=m-i(p(m))+i(p(m))\in\ker p+i(N),
	\] 
	pues
	$p(m-i(p(m)))=p(m)-p(m)=0$. Si $m\in\ker p\cap i(N)$, entonces $0=p(m)$ y además $m=i(n)$ para algún $n\in N$. Entonces
	$0=p(m)=p(i(n))=n$ y luego $m=0$.      
\end{proof}

La \textbf{suma directa} de submódulos puede extenderse a un número finito de sumandos. 
Si $S_1,\dots,S_n$ son submódulos de $M$, diremos que
$M=S_1\oplus\cdots\oplus S_n$ si todo $m\in M$ puede escribirse unívocamente
como $m=s_1+\cdots+s_n$ para ciertos $s_1\in S_1,\dots,s_n\in S_n$. 

\begin{exercise}
Demuestre que $M=S_1\oplus\cdots\oplus S_n$ si y sólo si
$M=S_1+\cdots+S_n$ y además 
\[
S_i\cap\left(\sum_{j\ne i}S_j\right)=\{0\}
\]	
para todo $i\in\{1,\dots,n\}$.
\end{exercise}

\begin{exercise}
Si $\{S_i:i\in I\}$ es una familia de submódulos de $M$, entonces $\cap_{i\in I}S_i$ es un submódulo de $M$.
\end{exercise}

\begin{example}
Sea $T\colon\R^2\to\R^2$, $T(x,y)=(0,y)$ y sea $M=\R^2$ con la estructura de $\R[X]$-módulo dada por
\[
\left(\sum_{i=0}^n a_iX^i\right)\cdot (x,y)=\sum_{i=0}^n a_iT(x,y).
\]
Vamos a demostrar que
$\{0\}$, $M$, $\R\times\{0\}$ y $\{0\}\times\R$ son los únicos submódulos de $M$. Si $N$ es un submódulo no nulo de $M$, sea
$(x_0,y_0)\in N\setminus\{(0,0)\}$. Si $(x,y)\in M$ es tal que $xy\ne 0$, entonces
\[
\left(\frac{x}{x_0}+\left(\frac{y}{y_0}-\frac{x}{x_0}\right)X\right)\cdot (x_0,y_0)=(x,y) 
\]
y luego $N=M$. Si $y_0=0$, entonces $N=\R\times\{0\}$, pues $\frac{x}{x_0}\cdot (x_0,0)=(x,0)$. Si $x_0=0$, entonces
$N=\{0\}\times\R$, pues 
$\frac{y}{y_0}\cdot (0,y_0)=(0,y)$ 
\end{example}


\begin{example}
Sea $M=\R^2$ como $\R[X]$-módulo con la acción
\[
\left(\sum_{i=0}^n a_iX^i\right)\cdot (x,y)=\sum_{i=0}^n a_iT^i(x,y),
\]
donde $T\colon M\to M$, $T(x,y)=(y,x)$. 
Vamos a calcular todos los submódulos de $M$. 
Si $N\subseteq M$ es un submódulo entonces $N$ es un espacio vectorial real. Supongamos que $N\ne\{(0,0)\}$ y que $N\ne\R^2$. Como entonces $\dim N=1$, 
sea $\{(a_0,b_0)\}$ una base de $N$. Como $N$ es un submódulo,
$(b_0,a_0)=X\cdot (a_0,b_0)\in N$. En particular, existe $\lambda\in\R$ tal que $(b_0,a_0)=\lambda (a_0,b_0)$. Como $(a_0,b_0)\ne(0,0)$, sin perder generalidad
podemos suponer que $a_0\ne 0$. Esto implica que
$\lambda^2 a_0=\lambda (\lambda a_0)=\lambda b_0=a_0$ y entonces $\lambda^2=1$. Si $\lambda=1$, entonces
$a_0=b_0$. Si $\lambda=-1$, entonces $a_0=-b_0$. En conclusión, $N$ está generado por $(1,1)$ o por $(1,-1)$.  
\end{example}

\begin{example}
Si $V$ es un espacio vectorial y $T\colon V\to V$ es una transformación lineal, entonces
$V$ es un $K[X]$-módulo con 
\[
\left(\sum_{i=0}^n a_iX^i\right)\cdot v=\sum_{i=0}^n a_iT^i(v).
\]
Si $g\colon V\to V$ es un morfismo de $K[X]$-módulos, entonces $g$ conmuta con $T$, pues
\[
(g\circ T)(v)=g(T(v))=g(X\cdot v)=X\cdot g(v)=T(g(v))=(T\circ g)(v)
\]
para todo $v\in V$.
%Sea $K$ un cuerpo y sea $M$ un $K[X]$-módulo. Entonces $M$ es un $K$-espacio vectorial con la restricción 
%de la acción de $K[X]$ en $M$. 
%
%Si $f\colon M\to M$, $f(m)=X\cdot m$, entonces $f$ es una transformación lineal. 
%
%Si $V$ es un $K$-espacio vectorial y $f\colon V\to V$ es una transformación lineal, definimos
%$p\cdot v=p(f)(v)$ para $p\in K[X]$ y $v\in V$. Entonces $M$ es un $K[X]$-módulo si y sólo si $M$ es un $K$-espacio vectorial
%y $x\cdot v=f(v)$ para alguna transformación lineal $f\colon V\to V$. 
%
%Además, si $g\colon M\to M$ es un morfismo de $K[X]$-módulos, entonces $g\colon M\to M$ es una transformación lineal que conmuta con $f$
%pues 
%\[
%g(f(v))=g(X\cdot v)=X\cdot g(v)=f(g(v)).
%\]
\end{example}

Sea $\Hom_R(M,N)$ el conjunto de morfismos de módulos $M\to N$. 

\begin{example}
\label{exa:Hom}
Veamos que $\Hom_R(M,N)$ es un $Z(R)$-módulo. Si $f\in\Hom_R(M,N)$ y $r\in R$, definimos la función 
$r\cdot f\colon M\to N$, $m\mapsto f(r\cdot m)$, que es un morfismo de grupos abelianos. 
Si $r,s\in Z(R)$, entonces $f$ es un morfismo pues
\begin{align*}
(r\cdot (s\cdot f))(m)&=
(s\cdot f)(r\cdot m)\\
&=f(s\cdot (r\cdot m))=f((sr)\cdot m)=f( (rs)\cdot m)=((rs)\cdot f)(m). 
\end{align*} 	
\end{example}

\index{Módulo!cociente}
\index{Epimorfismo!canónico de módulos}
Si $M$ es un módulo y $N$ es un submódulo, entonces $M/N$ es un grupo abeliano y el morfismo
canónico $\pi\colon M\to M/N$, $x\mapsto x+N$, es un morfismo sobreyectivo de grupos. Veamos que 
el \textbf{cociente} $M/N$ es un módulo con 
\[
r\cdot (x+N)=(r\cdot x)+N,
\]
donde $r\in R$ y $x\in M$. Para esto, tenemos que ver la buena definición de la acción en $M/N$. Si $x+N=y+N$, entonces, como
$x-y\in N$, se tiene que 
\[
r\cdot x-r\cdot y=r\cdot (x-y)\in N,
\]
es decir $r\cdot (x+N)=r\cdot (y+N)$. Dejamos como ejercicio demostrar que la función $\pi\colon M\to M/N$, $x\mapsto x+N$, es
un morfismo sobreyectivo de módulos. 

\begin{example}
Si $R=M=\Z$ y $N=2\Z$, entonces $M/N\simeq\Z/2$. 
\end{example}

\begin{example}
Sea $R$ un anillo conmutativo. Veamos que $M\simeq\Hom_R(\prescript{}{R}R,M)$. 
Como $R$ es un anillo conmutativo, $\Hom_R(\prescript{}{R}R,M)$ es un módulo, ver ejemplo~\ref{exa:Hom}.
Sea $\varphi\colon M\to\Hom_R(\prescript{}{R}R,M)$, $m\mapsto f_m$, donde $f_m\colon R\to M$, $r\mapsto r\cdot m$. Para ver que $\varphi$ está bien definida
hay que observar que $\varphi(m)\in\Hom_R(\prescript{}{R}R,M)$, es decir 
\[
f_m(r+s)=(r+s)\cdot m=r\cdot m+s\cdot m,\quad
f_m(rs)=(rs)\cdot m=r\cdot (s\cdot m)=r\cdot f_m(s).
\]

Para ver que $\varphi$ es morfismo primero vemos que $\varphi(m+n)=\varphi(m)+\varphi(n)$ para todo $m,n\in M$, pues 
\begin{align*}
\varphi(m+n)(r)&=f_{m+n}(r)=r\cdot (m+n)\\
&=r\cdot m+r\cdot n=f_m(r)+f_n(r)=\varphi(m)(r)+\varphi(n)(r).
\end{align*}
Además $\varphi(r\cdot m)=r\cdot\varphi(m)$ para 
todo $r\in R$ y $m\in M$, pues 
\begin{align*}
%\varphi(r\cdor m)(s)=f_{r\cdot m}(s)=(r\cdot \varphi(m))(s).
%
\varphi(r\cdot m)(s)&=f_{r\cdot m}(s)
%(r\cdot f_m)(s)=f_m(rs)\\
=s\cdot (r\cdot m)
=(sr)\cdot m\\
&=(rs)\cdot m=f_m(rs)=\varphi(m)(rs)=(r\cdot\varphi(m))(s).
%&=(rs)\cdot m=(sr)\cdot m=s\cdot (r\cdot m)\\
%&=f_{r\cdot m}(s)=\varphi(r\cdot m)(s)=(r\cdot \varphi)(m)(s). 
\end{align*}

Falta ver que $\varphi$ es un isomorfismo. 
Veamos primero que $\varphi$ es monomorfismo. Si $\varphi(m)=0$, entonces $r\cdot m=\varphi(m)(r)=0$ para todo $r\in R$. En particular, 
$m=1\cdot m=0$. 	Veamos ahora que $\varphi$ es epimorfismo. Si $f\in\Hom_R(\prescript{}{R}R,M)$, sea $m=f(1)$. Entonces $\varphi(m)=f$ pues
$\varphi(m)(r)=r\cdot m=r\cdot f(1)=f(r)$.
\end{example}

Tal como hicimos con grupos, puede demostrarse que si $M$ es un módulo y $N$ es un submódulo de $M$, el par
$(M/N,\pi\colon M\to M/N)$ tiene las siguientes propiedades:
\begin{enumerate}
\item $N\subseteq \ker \pi$.
\item Si $f\colon M\to T$ es un morfismo tal que $N\subseteq \ker f$, entonces existe un único morfismo $\varphi\colon M/N\to T$ tal que $\varphi\circ\pi =f$.  
\end{enumerate}

\index{Teoremas de isomorfismos!para módulos}
Recordemos que si $S$ y $T$ son submódulos de un módulo $M$, entonces
$S\cap T$ y $S+T=\{s+t:s\in S,\,t\in T\}$ son ambos submódulos de $M$. 
Se tienen entonces los teoremas de isomorfismos. 
\begin{enumerate}
	\item Si $f\in\Hom_R(M,N)$, entonces $M/\ker f\simeq f(M)$.
	\item Si $T\subseteq N\subseteq M$ son submódulos, entonces 
	\[
	\frac{M/T}{N/T}\simeq M/N
	\]
	\item Si $S$ y $T$ son submódulos de $M$, entonces $(S+T)/S\simeq T/(S\cap T)$. 
\end{enumerate}

\begin{example}
Si $R$ es un cuerpo y $V$ es un $R$-módulo, entonces $V$ es un espacio vectorial. 
Si $S$ y $T$ son subespacios de $V$, entonces son submódulos de $V$. 
El segundo teorema de isomorfismos nos dice que $(S+T)/T\simeq S/(S\cap T)$, 
un isomorfismo de espacios vectoriales. Al aplicar dimensión, 
\[
\dim(S+T)-\dim T=\dim(S)-\dim(S\cap T).
\]
\end{example}

\begin{example}
Si $S$ es un sumando directo de $M$ y $T$ es un complemento para $S$, entonces $T\simeq M/S$, pues
\[
M/S=(S\oplus T)/S\simeq T/(S\cap T)=T/\{0\}\simeq T
\]
por el segundo teorema de isomorfismos. Luego todos los complementos
de $S$ en $M$ serán isomorfos.  	
\end{example}

Puede demostrarse además el teorema de la correspondencia, que afirma que existe una correspondencia biyectiva 
entre los submódulos de $M/N$ y los submódulos de $M$ que contienen a $N$. La correspondencia está dada
por $S\mapsto \pi^{-1}(S)$ y $\pi(T)\mapsfrom T$. 

\begin{exercise}
Sea $f\in\Hom_R(M,N)$. Son equivalentes: 
\begin{enumerate}
\item $f$ es epimorfismo.
\item $N/f(M)\simeq\{0\}$. 
\item Para todo módulo $T$ y todo $g,h\in\Hom_R(N,T)$, $g\circ f=h\circ f\implies g=h$.
\item Para todo módulo $T$ y todo $g\in\Hom_R(N,T)$, $g\circ f=0\implies g=0$. 
\end{enumerate}
\end{exercise}

\begin{exercise}
\label{xca:mod_iso_max}
    Sea $R$ un anillo conmutativo y sean $M_1$ y $M_2$ ideales
    maximales de $R$. Pruebe que $R/M_1\simeq R/M_2$ como $R$-módulos 
    si y sólo si existe $r\in R\setminus M_2$ tal que $rM_1\subseteq M_2$. 
\end{exercise}



\chapter{El teorema de Maschke}

\index{Representación!de un grupo}
Si $G$ es un grupo finito, un morfismo de grupos $G\to\GL(V)$, donde $V$ es un espacio
vectorial complejo de dimensión finita, se dice una 
\textbf{representación} de $G$. Si el espacio vectorial $V$ tiene dimensión $n$, al fijar una base 
para $V$ podemos considerar $G\to\GL(V)\simeq\GL_n(\C)$.
 
\begin{example}
Como $\Sym_3=\langle (12),(123)\rangle$, la función $\rho\colon \Sym_3\to\GL_3(\C)$,
\[
(12)\mapsto\begin{pmatrix}
0 & 1 & 0\\
1 & 0 & 0\\
0 & 0 & 1
\end{pmatrix},\quad
(123)\mapsto\begin{pmatrix}
0 & 0 & 1\\
1 & 0 & 0\\
0 & 1 & 0
\end{pmatrix}
\] 
es una representación de $\Sym_3$. 
\end{example}

\begin{example}
Como el grupo de cuaterniones $Q_8=\{1,-1,i,-i,j,-j,k,-k\}$ está generado por $\{i,j\}$, 
la función $\rho\colon G\to\GL_2(\C)$, 
\[
i\mapsto\begin{pmatrix}
i & 0\\
0 & i
\end{pmatrix},
\quad
j\mapsto\begin{pmatrix}
0 & -1\\
1 & 0	
\end{pmatrix}
\]
es una representación de $Q_8$.
\end{example}

\begin{example}
Sea $G=\langle g\rangle$ cíclico de orden seis. 
La función $\rho\colon G\to\GL_2(\C)$, 
\[
g\mapsto
\begin{pmatrix}
1&1\\
-1&0
\end{pmatrix}
\] 
es una representación del grupo $G$ cíclico de orden seis. 
\end{example}

\begin{example}
Sea $G=\langle g\rangle$ cíclico de orden cuatro. 
La función $\rho\colon G\to\GL_2(\C)$, 
\[
g\mapsto
\begin{pmatrix}
0&-1\\
1&0
\end{pmatrix}
\] 
es una representación del grupo $G$ cíclico de orden cuatro. 
\end{example}

Observemos que existe una correspondencia biyectiva 
\[
\{\text{representaciones de $G$}\}\leftrightarrow\{\text{$\C[G]$-módulos de dimensión finita}\}.
\]
Si $\rho\colon G\to\GL(V)$ es una representación, entonces 
$V$ es un $\C[G]$-módulo con
\[
\left(\sum_{g\in G}\lambda_gg\right)\cdot v=\sum_{g\in G}\lambda_g\rho(g)(v).
\]
Recíprocamente, si $V$ es un $\C[G]$-módulo, entonces $\rho\colon G\to\GL(V)$, 
$\rho(g)(v)=g\cdot v$, es una representación de $G$ en $V$. Puede verificarse que estas
construcciones son una la inversa de la otra.  

\begin{definition}
\index{Módulo!simple}
\index{Módulo!irreducible}
Un módulo $M$ se dice \textbf{simple} (o irreducible) si $M\ne\{0\}$ y $M$ no tiene
submódulos propios no triviales.  
\end{definition}

\begin{example}
Si $A$ es un álgebra, vimos que todo $A$-módulo es un espacio vectorial. Los módulos
de dimensión uno serán entonces módulos simples.
\end{example}

\begin{example}
Sea $G=\langle g\rangle$ cíclico de orden tres y sea $M=\R^3$ con la estructura de $\R[G]$-módulo 
dada por $g\cdot (x,y,z)=(y,z,x)$. El conjunto
\[
N=\{(x,y,z)\in\R^3:x+y+z=0\}
\]
es un submódulo de $M$. Veamos que $N$ es simple. Si $N$ contiene un submódulo no trivial $S$, 
sea $(x_0,y_0,z_0)\in S\setminus\{(0,0,0)\}$. Como $S$ es un submódulo, 
\[
(y_0,z_0,x_0)=g\cdot (x_0,y_0,z_0)\in S.
\]
Afirmamos
que $\{(x_0,y_0,z_0),(y_0,z_0,x_0)\}$ es un conjunto linealmente independiente. 	Si existe $\lambda\in\R$ 
tal que $\lambda(x_0,y_0,z_0)=(y_0,z_0,x_0)$, entonces $x_0=\lambda^3 x_0$. Como $x_0=0$ implica que 
$y_0=z_0=0$, entonces $\lambda=1$. En particular, $x_0=y_0=z_0$, una contradicción, pues $x_0+y_0+z_0=0$. 
Luego $\dim S=2$ y entonces
$S=N$. 
\end{example}

\begin{example}
Sea $M=\R^2$ con la estructura de $\R[X]$-módulo dada por
\[
\left(\sum_{i=0}^n a_iX^i\right)\cdot m=\sum_{i=0}^n a_iT^i(m).
\]
donde $T\colon\R^2\to\R^2$, $T(x,y)=(y,-x)$. 
Veamos que $M$ es simple. Si $N$ es un submódulo no nulo, sea $(x_0,y_0)\in N\setminus\{(0,0)\}$. Si $(x,y)\in M$, veamos que 
existen
$\alpha,\beta\in\R$ tales que
\[
(\alpha+\beta X)\cdot (x_0,y_0)=(x,y).
\]
En efecto, basta tomar 
\[
\alpha=\frac{x_0x+y_0y}{x_0^2+y_0^2},\quad
\beta=\frac{y_0x-x_0y}{x_0^2+y_0^2},
\]
pues
\begin{align*}
(\alpha+\beta X)\cdot (x_0,y_0)&=\alpha(x_0,y_0)+\beta\cdot (X\cdot (x_0,y_0))\\
&=(\alpha x_0,\alpha y_0)+(\beta y_0,-\beta x_0)\\
&=(\alpha x_0+\beta y_0,\alpha y_0-\beta x_0)\\
&=(x,y).	
\end{align*}
\end{example}

\begin{definition}
\index{Módulo!semisimple}
\index{Módulo!completamente reducible}
Un módulo $M$ se dice \textbf{semisimple} (o completamente reducible) 
si es suma directa de módulos simples.
\end{definition}

Vimos en el capítulo anterior que si $M$ es un módulo, se dice que un submódulo $S$ de $M$ se complementa en $M$ si existe un submódulo $T$ de $M$ tal que $M=S\oplus T$.

\begin{lemma}
Si $p\colon M\to M$ es un morfismo tal que $p^2=p$, entonces 
\[
M=\ker p\oplus p(M).
\]
\end{lemma}

\begin{proof}
	Como $p$ es un morfismo, $\ker p$ y $p(M)$ son submódulos de $M$. 
	Para ver que $M=\ker p+p(M)$ alcanza con observar que todo $m\in M$ puede escribirse como
	$m=(m-p(m))+p(m)$ 
	y que $m-p(m)\in\ker p$ pues 
	\[
	p(m-p(m))=p(m)-p^2(m)=p(m)-p(m)=0. 
	\]
	Veamos ahora que $\ker p\cap p(M)=\{0\}$. Si $m\in\ker p\cap p(M)$, escribimos $m=p(m_1)$ para algún $m_1\in M$. Como entonces 
	$0=p(m)=p^2(m_1)=m_1$, se concluye que $m=0$.   
\end{proof}

\index{Proyección}   
Recordemos que 
una \textbf{proyección} (o proyector) de un módulo $M$ es un   
morfismo $p\colon M\to M$ tal que $p^2=p$. 
 
\begin{lemma}
Si $A$ es un álgebra y $M$ es un $A$-módulo de dimensión finita tal que
todo submódulo de $M$ se complementa, entonces $M$ es semisimple.
\end{lemma}

\begin{proof}
Procederemos por inducción en $\dim M$. Si $M=\{0\}$ el resultado es trivial. Si $M\ne\{0\}$, 
sea $S$ un submódulo no nulo de $M$ de dimensión minimal. En particular, $S$ es simple. Por hipótesis sabemos que existe un submódulo $T$ de $M$ tal que $M=S\oplus T$. Como $\dim T<\dim M$, la hipótesis inductiva implica que $T$ es suma directa de módulos simples. Luego $M$ también lo es. 
\end{proof}

El lema anterior vale también para módulos arbitrarios sobre anillos. Sin embargo, la demostración 
requiere el uso del lema de Zorn.

\begin{example}
Sea $R=M_2(\C)$ y sea $M=\prescript{}{R}R$. Los subconjuntos
\[
I=\begin{pmatrix}
\C&0\\
\C&0
\end{pmatrix},\quad
J=\begin{pmatrix}
0&\C\\
0&\C
\end{pmatrix}
\]
son submódulos de $M$ tales que $M\simeq I\oplus J$. Veamos que $M$ es semisimple, es decir que
$I$ y $J$ son simples. 

Si $S$ es un submódulo no nulo de $I$, sea 
$\begin{pmatrix}
a&0\\
c&0
\end{pmatrix}\in S$ no nulo. Supongamos que 
$a\ne 0$, el caso $c\ne 0$ es similar. Entonces
\[
\begin{pmatrix}
a^{-1} & 0\\
0 & 0\end{pmatrix}
\begin{pmatrix}
a&0\\
c&0
\end{pmatrix}
=\begin{pmatrix}
1&0\\
0&0
\end{pmatrix}\in S.
\]
Análogamente se demuestra que $\begin{pmatrix}0&0\\1&0\end{pmatrix}\in S$. Luego $S=I$ y entonces $I$ es simple.  
La misma técnica nos permite demostrar que $J$ es simple.
\end{example}

Es importante observar que si $A$ es un álgebra y $M$ es un módulo
semisimple de dimensión finita, entones $M$ es suma directa de finitos simples. 

\begin{theorem}[Maschke]
\index{Teorema!de Maschke}
Sea $G$ un grupo finito y sea $M$ un $\C[G]$-módulo de dimensión finita. Entonces
$M$ es semisimple.
\end{theorem}

\begin{proof}
Gracias al lema anterior, alcanza con demostrar que todo submódulo $S$ de $M$ se complementa. 
Como, en particular, $S$ es un subespacio de $M$, existe un subespacio $T_0$ de $M$ 
tal que $M=S\oplus T_0$ (como espacios vectoriales). Vamos a usar el espacio vectorial
$T_0$ para construir un submódulo $T$ de $M$ que complementa a $S$. Como $M=S\oplus T_0$, 
cada $m\in M$ puede escribirse unívocamente como $m=s+t_0$ para ciertos $s\in S$ y $t_0\in T$. 
Podemos definir entonces la transformación lineal 
\[
p_0\colon M\to S,\quad
p_0(m)=s,
\]
donde $m=s+t_0$ con $s\in S$ y $t_0\in T$. 
Observemos que si $s\in S$, entonces $p_0(s)=s$. En particular, $p_0^2=p_0$ pues
$p_0(m)\in S$. 

El problema 
es que $p_0$ no es, en general, un morfismo de $\C[G]$-módulos. Promediamos
sobre el grupo $G$ para conseguir un morfismo de grupos: Sea 
\[
p\colon M\to S,\quad
p(m)=\frac{1}{|G|}\sum_{g\in G}g^{-1}\cdot p_0(g\cdot m).
\]

Primero demostramos que $p$ es un morfismo de $\C[G]$-módulos. Alcanza con ver que
$p(g\cdot m)=g\cdot p(m)$ para todo $g\in G$ y $m\in M$. En efecto,
\[
p(g\cdot m)=\frac{1}{|G|}\sum_{h\in G}h^{-1}\cdot p_0(h\cdot (g\cdot m))
=\frac{1}{|G|}\sum_{h\in G}(gh^{-1})\cdot p_0(h\cdot m)=g\cdot p(m).
\]

Veamos ahora que $p(M)=S$. La inclusión $\subseteq$ es trivial, pues $S$ es un submódulo de $M$ 
y además $p_0(M)\subseteq S$. Recíprocamente, si $s\in S$, entonces $g\cdot s\in S$, pues
$S$ es un submódulo. Luego 
$s=g^{-1}\cdot (g\cdot s)=g^{-1}\cdot p_0(g\cdot s)$ y en consecuencia
\[
s=\frac{1}{|G|}\sum_{g\in G}g^{-1}\cdot (g\cdot s)=\frac{1}{|G|}\sum_{g\in G}g^{-1}\cdot (p_0(g\cdot s))=p(s).
\]
Como $p(m)\in S$ para todo $m\in M$, entonces $p^2(m)=p(m)$, es decir que $p$ es un proyector en $S$. Luego $S$ se complementa en $M$, es decir $M=S\oplus\ker(p)$.
\end{proof} 

La misma demostración del teorema de Maschke vale para el álgebra de grupo real o racional. 
La descomposición de un módulo sobre el álgebra de grupo dependerá
fuertemente del cuerpo sobre el que se trabaje. 

\begin{example}
Sea $G=\langle g\rangle$ el grupo cíclico de orden cuatro y sea $\rho_g=\begin{pmatrix}
0&-1\\
1&0\end{pmatrix}$. 
Sea $M=\C^{2\times 1}$ con la estructura de $\C[G]$-módulo dada por 
\[
g\cdot\begin{pmatrix}u\\v\end{pmatrix}
%\begin{pmatrix}0&-1\\1&0\end{pmatrix}\begin{pmatrix}u\\v\end{pmatrix}
=\begin{pmatrix}-v\\u\end{pmatrix},
\]
es decir, si $a,b,c,d\in\C$, entonces 
\[
(a1+bg+cg^2+dg^3)\cdot\begin{pmatrix}u\\v\end{pmatrix}
=\begin{pmatrix}
(a-d)u+(c-b)v\\
(1-b)u+(a-d)v
\end{pmatrix}.
\]
Sabemos por el teorema de Maschke que $M$ es semisimple. Veamos cómo descomponer el módulo $M$ como suma directa de simples. 
Como $\dim M=2$, tendremos que $M$ es suma directa de dos submódulos de dimensión uno. 
Observemos que si $S$ es un submódulo tal que $\{0\}\subsetneq S\subsetneq M$, 
entonces $\dim S=1$. Además 
\[
S=\left\{\lambda\begin{pmatrix}
u_0\\
v_0
\end{pmatrix}:\lambda\in\C\right\}
\text{ es un submódulo de $M$}
\Longleftrightarrow
\begin{pmatrix}
u_0\\
v_0
\end{pmatrix}
\text{ es autovector de $\rho_g$}.
\]
Como la matriz $\rho_g$ tiene polinomio característico $X^2+1$, se sigue 
que  
$\begin{pmatrix}
i\\
1\end{pmatrix}$ es autovector de $\rho_g$ de autovalor $-i$ y que
$\begin{pmatrix}
-i\\
1\end{pmatrix}$ es autovector de autovalor $i$. 
Luego $M$ se descompone en suma directa de simples como 
\[
M=\C\begin{pmatrix}
i\\
1\end{pmatrix}
\oplus
\C
\begin{pmatrix}
-i\\
1\end{pmatrix}
\]
\end{example}

Observar que en ejemplo anterior pudimos descomponer a la matriz $\rho_g$ 
gracias a la existencia de autovectores, algo 
que no pasaría si consideramos módulos sobre el álgebra de grupo real.   

\begin{example}
Sea $G=\langle g\rangle$ el grupo cíclico de orden cuatro y sea $\rho_g=\begin{pmatrix}
0&-1\\
1&0\end{pmatrix}$. 
Sea $M=\R^{2\times 1}$ con la estructura de $\R[G]$-módulo dada por 
\[
g\cdot\begin{pmatrix}u\\v\end{pmatrix}
=\begin{pmatrix}-v\\u\end{pmatrix}.
\]
Tal como hicimos en el ejemplo anterior, 
como $\dim M=2$, si $S$ es un submódulo de $M$ tal que $\{0\}\subsetneq S\subsetneq M$, entonces $\dim S=1$. 
Pero como $\rho_g$ no tiene autovectores reales, $M$ no tendrá submódulos de dimensión uno.  
En consecuencia, $M$ es simple como $\R[G]$-módulo. 
\end{example}
\chapter{Sucesiones exactas}

\begin{definition}
\index{Sucesión exacta}	
Sea $M_1,M_2,\dots$ una sucesión de $R$-módulos y para cada $n\in\N$ sea
$f_n\colon M_n\to M_{n-1}$ un morfismo. Diremos que la sucesión
\[
\cdots\xrightarrow{f_{n+2}}M_{n+1}\xrightarrow{f_{n+1}}M_n\xrightarrow{f_n}M_{n-1}\xrightarrow{f_{n-1}}\cdots
\]
de módulos y morfismos 
es \textbf{exacta} si $\ker f_n=f_{n+1}(M_{n+1})$ para todo $n\in\N$. 
\end{definition}

\index{Sucesión exacta corta}
En general nos encontraremos con \textbf{sucesiones exactas cortas}, es decir 
decir, sucesiones de la forma
	\begin{equation}
	\label{eq:exacta1}	
		\xymatrix{
        0\ar[r]
        & M
        \ar[r]^f
        & N
        \ar[r]^g
        & T\ar[r]
        & 0,
        }
  	\end{equation}
donde la exactitud significa que $f$ es monomorfismo, $g$ es epimorfismo y que $f(M)=\ker g$. 

\begin{exercise}
Demuestre que si la sucesión~\eqref{eq:exacta1} es exacta, entonces $M\simeq\ker f$ y $T\simeq M/\ker f$.
\end{exercise}

\begin{examples}
La sucesión 
\[
		\xymatrix{
        0\ar[r]
        & M
        \ar[r]^f
        & N
        }
\]
es exacta si y sólo si $f$ es un monomorfismo. Similarmente, la sucesión
\[
		\xymatrix{
        M
        \ar[r]^g
        & N
        \ar[r]
        & 0
        }
\]
es exacta si y sólo si $g$ es un epimorfismo.
\end{examples}

\begin{example}
La sucesión
\begin{equation}
\label{eq:split}
			\xymatrix{
        0\ar[r]
        & M
        \ar[r]^f
        & M\oplus N
        \ar[r]^g
        & N\ar[r]
        & 0,
        }
\end{equation}
donde $f(m)=(m,0)$ y $g(m,n)=n$, es exacta.
\end{example}

Nos interesa saber cuándo una sucesión exacta~\ref{eq:exacta1} es de la forma~\eqref{eq:split}. Para eso
necesitamos algunas definiciones.

\begin{definition}
\index{Sección}	
Sea $f\in\Hom_R(M,N)$. Diremos que el morfismo $f$ es una \textbf{sección} si existe $g\in\Hom_R(N,M)$ tal que $g\circ f=\id_M$.  
\end{definition}

\begin{definition}
\index{Retracción}	
Sea $f\in\Hom_R(M,N)$. Diremos que el morfismo $f$ es una \textbf{retracción} si existe $g\in\Hom_R(N,M)$ tal que $f\circ g=\id_N$. 
\end{definition}

Dejamos como ejercicio demostrar que una sección es siempre inyectiva. La afirmación recíproca no es cierta, ya que 
la inclusión $2\Z\hookrightarrow\Z$ de $Z$-módulos es un monomorfismo que no es una sección.  
Similarmente, una retracción es siempre sobreyectiva y la recíproca no es cierta 
ya que por ejemplo $\Z\to\Z/2$ es un epimorfismo de $\Z$-módulos que no es una retracción. 
 
\begin{definition}
	\index{Sucesiones exactas!euivalentes}
	La sucesión exacta 
	\[  
		\xymatrix{
        0\ar[r]
        & A
        \ar[r]
        & B
        \ar[r]
        & C\ar[r]
        & 0,
        }
      \]
      y la sucesión exacta 
      \[
        \xymatrix{
        0\ar[r]
        & A_1
        \ar[r]
        & B_1
        \ar[r]
        & C_1\ar[r]
        & 0	
        }
     \]
	se dirán \textbf{equivalentes} (o isomorfas) si 
	existen isomorfismos $\alpha$, $\beta$ y $\gamma$ tales que
	el diagrama 
	    \begin{equation}
        \xymatrix{
        0\ar[r] 
        & A
        \ar@{->}[d]^{\alpha}
        \ar[r]
        & B
        \ar[r]
        \ar[d]^\beta
        & C\ar[r]
        \ar@{->}[d]^\gamma 
        & 0
        \\
        0\ar[r] 
        & A_1
        \ar[r]
        & B_1
        \ar[r]
        & C_1\ar[r]
        & 0
        }
        \end{equation} 
     es conmutativo. 
\end{definition}

El siguiente lema es de gran utilidad, aunque bastante técnico.

\begin{lemma}[de los cinco]
\index{Lema!de los cinco}
Si el diagrama
	    \begin{equation}
        \xymatrix{
        A\ar[r]^r
        \ar@{->}[d]^{\alpha}
        & B
        \ar@{->}[d]^{\beta}
        \ar[r]^s
        & C
        \ar[r]^t
        \ar[d]^\gamma
        & D\ar[r]^u
        \ar@{->}[d]^\delta 
        & E
		\ar@{->}[d]^\epsilon 
        \\
        A_1\ar[r]^{r_1} 
        & B_1
        \ar[r]^{s_1}
        & C_1
        \ar[r]^{t_1}
        & D_1\ar[r]^{u_1}
        & E_1
        }
        \end{equation} 	
       es conmutativo y con filas exactas, valen las siguientes afirmaciones:
       \begin{enumerate}
       	\item Si $\alpha$ es epimorfismo y $\beta$ y $\delta$ son monomorfismos, entonces $\gamma$ es monomorfismo.
       	\item Si $\epsilon$ es monomorfismo y $\beta$ y $\delta$ son epimorfismos, entonces $\gamma$ es epimorfismo.
       	\item Si $\alpha$, $\beta$, $\delta$ y $\epsilon$ son isomorfismos, entonces $\gamma$ es isomorfismo.
       \end{enumerate}
\end{lemma}

\begin{proof}
	Demostremos la primera afirmación. Sea $c\in C$ tal que $\gamma(c)=0$. Queremos ver que $c=0$. Como $\gamma(c)=0$, 
	entonces $\delta(t(c))=t_1(\gamma(c))=0$. Como $\delta$ es inyectiva, $t(c)=0$, es decir $c\in \ker t=s(B)$. En particular, 
	$c=s(b)$ para algún $b\in B$. Si $b_1=\beta(b)$, entonces
	\[
	s_1(b_1)=s_1(\beta(b))=\gamma(s(b))=\gamma(c)=0
	\]
	y entonces $b_1\in\ker s_1=r_1(A_1)$. En particular, $b_1=r_1(a_1)$ para algún $a_1\in A_1$. Como $\alpha$ es epimorfismo, 
	$a_1=\alpha(a)$ para algún $a\in A$. Entonces
	\[
	\beta(b)=b_1=r_1(a_1)=r_1(\alpha(a))=\beta(r(a))
	\]
	y luego $b-r(a)\in\ker\beta=\{0\}$, es decir $b=r(a)$. En conclusión,  
	\[
	c=s(b)=s(r(a))=0.
	\] 
	
	Demostremos la segunda afirmación. Sea $c_1\in C_1$. Queremos ver que $c_1=\gamma(c)$ para algún $c\in C$. Sea
	$d_1=t_1(c_1)$. Como $\delta$ es epimorfismo, $d_1=\delta(d)$ para algún $d\in D$. Entonces
	\[
	u_1(\delta(d))=u_1(t_1(c_1))=0
	\]
	y luego $\delta(d)\in\ker u_1$. Como $0=u_1(\delta(d))=\epsilon(u(d))$ y $\epsilon$ es un monomorfismo, 
	entonces $u(d)=0$, es decir $d\in\ker u=t(C)$. En consecuencia, $d=t(c)$ para algún $c\in C$. Como
	\[
	t_1(c_1)=d_1=\delta(d)=\delta(t(c))=t_1(\gamma(c)),
	\]
	entonces $c_1-\gamma(c)\in\ker t_1=s_1(B_1)$, lo que significa que $c_1-\gamma(c)=s_1(b_1)$ para algún $b_1\in B_1$. Como
	$\beta$ es un epimorfismo, $b_1=\beta(b)$ para algún $b\in B$. Luego
	$c_1-\gamma(c)=s_1(\beta(b))=\gamma(s(b))$ y entonces
	$c_1=\gamma(c)+\gamma(s(b))=\gamma(c+s(b))$. 
\end{proof}

\begin{exercise}
	Consideremos el diagrama conmutativo
	\[
		\xymatrix{
		X
		\ar[d]^{\alpha}
		\ar[r]^-{f}
		& Y
		\ar[r]^-{g}
		\ar[d]^{\beta}
		& Z
		\ar[d]^{\gamma}
		\\
		X_1
		\ar[r]^-{f_1}
		& Y_1
		\ar[r]^-{g_1}
		& Z_1
		}
	\]
	y supongamos que tiene filas exactas. Demuestre las siguientes afirmaciones:
	\begin{enumerate}
		\item Si $\alpha$, $\gamma$ y $f_1$ son monomorfismos entonces $\beta$ es monomorfismo.
		\item Si $\alpha$, $\gamma$ y $g$ son epimorfismos entonces $\beta$ es epimorfismo.
		\item Si $\beta$ es monomorfismo y $\alpha$ y $g$ son epimorfismos entonces $\gamma$ es monomorfismo.
		\item Si $\beta$ es epimorfismo y $f_1$ y $\gamma$ son monomorfismos entonces $\alpha$ es epimorfismo.
	\end{enumerate}
%
%	Probemos primero (1). Si $b\in B$ entonces $\beta(b)=0$ y
%	\[
%		0=g'(\beta(b))=\gamma(g(b)).
%	\]
%	Como $\gamma$ es monomorfismo, $g(b)=0$ y
%	entonces $b\in\ker g=f(A)$. Luego existe $a\in A$ tal que $b=f(a)$. Tenemos entonces
%	\[
%		0=\beta(b)=\beta(f(a))=f'(\alpha(a))
%	\]
%	y luego, como $f'$ es monomorfismo, $\alpha(a)=0$. Como $\alpha$ es
%	monomorfismo, $a=0$ y en conclusión $b=0$.
\end{exercise}

\begin{proposition}
\label{pro:split}
	Si 
	\[  
		\xymatrix{
        0\ar[r]
        & M
        \ar[r]^f
        & N
        \ar[r]^g
        & T\ar[r]
        & 0	
        }
     \]
     es exacta, las siguientes afirmaciones son equivalentes:
     \begin{enumerate}
     \item $f$ es una sección.
     \item $g$ es una retracción.
     \item Existen un isomorfismo $\varphi$ de forma que el diagrama
   		\begin{equation}
   		\label{eq:diagrama}
        \xymatrix{
        0\ar[r] 
        & M
        \ar@{=}[d]
        \ar[r]
        & N
        \ar[r]
        %\ar@{->}[d]^\psi  
        & T\ar[r]
        \ar@{=}[d]
        & 0
        \\
        0\ar[r] 
        & M
        \ar[r]
        & M\oplus T        
        \ar[r]
        \ar@{->}[u]^\varphi
        & T\ar[r]
        & 0
        }
        \end{equation} 
		es conmutativo. 
     \end{enumerate}
\end{proposition}

\begin{proof}
Veamos que $(2)\implies(3)$. Como $g$ es una retracción, existe un morfismo $h\colon T\to N$ tal que
$g\circ h=\id_T$. Sea $\varphi\colon M\oplus T\to N$, $\varphi(m,t)=f(m)+h(t)$. 
Entonces $\varphi$ es morfismo y 
el diagrama~\eqref{eq:diagrama} es conmutativo pues
\[
(g\circ\varphi)(m,t)=g(f(m))+h(t))=t,\quad
\varphi(m,0)=f(m).
\]
Para ver que $\varphi$ es un isomorfismo, se utiliza el lema de los cinco. 

La demostración de la implicación $(1)\implies(3)$ es similar. Como
$f$ es una sección, existe un morfismo $h\colon N\to M$ tal que $h\circ f=\id_M$. 
Hay que usar entonces la función $\psi\colon N\to M\oplus T$, $n\mapsto (h(n),g(n))$, pues
\begin{align*}
	&\psi(f(m))=(h(f(m)),g(f(m)))=(m,0)=i_1(m),\\
	&p_2(\psi(n))=p_2(h(n),g(n))=g(n).
\end{align*}
Como $\psi$ es un isomorfismo gracias al lema de los cinco, $\varphi=\psi^{-1}$. 

Para ver que $(3)\implies (2)$, consideramos el diagrama conmutativo
\[
        \xymatrix{
        0\ar[r] 
        & M
        \ar@{=}[d]
        \ar[r]
        & N
        \ar[r]
        \ar@{->}[d]^\varphi
        & T\ar[r]
        \ar@{=}[d]
        & 0
        \\
        0\ar[r] 
        & M
        \ar[r]^{i_1}
        & M\oplus T        
        \ar[r]^{p_2}
        \ar@<1ex>[l]^{p_1}
        & T\ar[r]
        \ar@<1ex>[l]^{i_2}
        & 0
        }
\]
      donde $i_1(m)=(m,0)$, $p_1(m,t)=m$, $i_2(t)=(0,t)$ y $p_2(m,t)=t$. 
      Definimos entonces el morfismo $h\colon T\to N$, $t\mapsto \varphi(i_2(t))$. Tenemos
      \[
      g(h(t))=g(\varphi(0,t))=p_2(0,t)=t.
      \] 
      
      Para ver que $(3)\implies (1)$ 
      consideramos el mismo diagrama que en la implicación anterior 
      y definimos el morfismo $h\colon N\to M$, $n\mapsto p_1(\varphi(n))$. Entonces
      \[
      h(f(m))=p_1(\varphi(f(m)))=p_1(i_1(m))=p_1(m,0)=m.\qedhere
      \]
 \end{proof}

\index{Sucesión exacta!escindida}
\index{Sucesión exacta!que se parte}
Diremos que la sucesión exacta
	\[  
		\xymatrix{
        0\ar[r]
        & M
        \ar[r]^f
        & N
        \ar[r]^g
        & T\ar[r]
        & 0	
        }
     \]
es \textbf{escindida} (o que se parte) si cumple 
alguna de las condiciones de la proposición anterior.    


\begin{exercise}
\label{xca:exactas1}
	Sean 
	\[
	\xymatrix{
	0\ar[r] 
	& X
	\ar[r]^-{f}
	& M
	\ar[r]^-{g}
	& Y\ar[r]
	& 0
	}
	\]
	una sucesión exacta de módulos y $r\in R$. Pruebe que son equivalentes:
	\begin{enumerate}
		\item Para todo $x\in X$ tal que existe $m\in M$ con
			$f(x)=r\cdot m$, existe $x_1\in X$ con $x_1=r\cdot x$. 
		\item Para todo $y\in Y$ tal que $r\cdot y=0$ existe $m\in M$
			tal que $r\cdot m=0$ e $y=g(m)$. 
		\end{enumerate}
\end{exercise}

%		\begin{solution}
%			Supongamos que vale (1) y sea $m''\in M''$ tal que $am''=0$. Como $g$ es
%			epimorfismo, existe $m\in M$ tal que $g(m)=m''$. Luego $g(am)=am''=0$ y
%			$am\in\ker(g)=f(M')$. Existe entonces $m'\in M'$ tal que $am=f(m')$. Por
%			hipótesis, existe $m_1'\in M'$ tal que $am=f(am_1')$ y luego
%			$a(m-f(m_1'))=0$. El elemento de $M$ que buscamos es $m-f(m_1')$ pues
%			$g(m-f(m_1'))=g(m)=m''$.
%
%			Recíprocamente, supongamos que vale (2). Sea $m'\in M'$ tal que existe
%			$m\in M$ con $f(m')=am$. Si aplicamos $g$ obtenemos $0=gf(m')=ag(m)$. Si
%			usamos (2) con $g(m)\in M''$ entonces existe $m_1\in M$ tal que $am_1=0$
%			y $g(m)=g(m_1)$. Como $\ker(g)=f(M')$, existe $m_1'\in M'$ tal que
%			$m-m_1=f(m_1')$. Esto implica que $f(m')=am=am-am_1=af(m_1')=f(am_1')$.
%			Como $f$ es monomorfismo, $m'=am_1'$.
%		\end{solution}

\begin{exercise}
\label{xca:exactas2}
	Sea
	\[
	\xymatrix{
	 0\ar[r] 
	 & X
	 \ar[r]^-{f}
	 %\ar@<.7ex>[r]^-{f}
	 & M
	 %\ar@<.7ex>[r]^-{g}
	 %\ar@<.7ex>[l]^-{r}
	 \ar[r]^-{g}
	 & Y\ar[r]
	 %\ar@<.7ex>[l]^-{s}
	 & 0
	 }
	\]
	Demuestre que $g$ es una retracción si y sólo si 
	existen morfismos $s\colon Y\to M$ y $r\colon M\to X$  
	tales que $f\circ r+s\circ g=\id_M$.
%	
%	Probemos primero que $(1)\Rightarrow(3)$. Como $g$ es una retracción, sabemos
%	que existe un morfismo $s\colon M''\to M$ tal que $gs=\id_{M''}$.  Como $f$
%	es monomorfismo, 
%	\[
%		M'\simeq M'/\ker(f)\simeq f(M)=\ker(f).
%	\]
%	Además $g(\id_M-sg)=g-gsg=0$. Existe entonces un único morfismo $r\colon M\to
%	M'$ tal que $fr=\id_M-sg$. 
%
%	Probemos que $(3)\Rightarrow(1)$. Como la sucesión es exacta, 
%	$gsg=g(fr+sg)=g$. Como $g$ es epimorfismo, $gs=\id_{M''}$. 
\end{exercise}

Describiremos ahora el funtor $\Hom_R(M,-)$. 

Informalmente, un funtor $\Hom_R(M,-)$ 
es una regla que para cada módulo $N$ nos devolverá el grupo abeliano $\Hom_R(M,N)$ y para cada
morfismo $f\in\Hom_R(A,B)$ nos devoverá el morfismo $f_*\colon\Hom_R(M,A)\to\Hom_R(M,B)$, 
$f_*(\alpha)=f\circ \alpha$, de grupos abelianos.

\begin{proposition}
Si la sucesión
\begin{gather*}
	\label{eq:exacta}	
		\xymatrix{
        0\ar[r]
        & A
        \ar[r]^f
        & B
        \ar[r]^g
        & C\ar[r]
        & 0
        }
\shortintertext{es exacta, entonces}
	\label{eq:exacta}	
		\xymatrix{
        0\ar[r]
        & \Hom_R(M,A)
        \ar[r]^{f_*}
        & \Hom_R(M,B)
        \ar[r]^{g_*}
        & \Hom_R(M,C)
        }
\end{gather*}
es exacta. 
\end{proposition}

\begin{proof}
	Primero veamos que $f_*$ es monomorfismo. Si $f\circ\alpha=0$, entonces $\alpha=0$ pues $f$ es monomorfismo. 
	
	Veamos ahora que $\im(f_*)\subseteq \ker g_*$. Si $\beta=f_*\alpha=f\circ\alpha$, entonces 
	\[
	g\circ\beta=(g\circ f)\circ\alpha=0,
	\]
	pues
	$\im f\subseteq\ker g$, es decir $\beta\ker g_*$. 
	
	Veamos que vale también la inclusión $\im(f_*)\supseteq\ker g_*$. Si $g_*\beta=g\circ\beta=0$, entonces
	$\beta(m)\in\ker g=\im f$ para todo $m\in M$. Si $m\in M$, existe un único $a\in A$ tal que $\beta(m)=f(a)$. Si
	$\alpha\colon M\to A$, $m\mapsto a$, entonces $\alpha\in\Hom_R(M,A)$ es tal que $\beta=f\circ\alpha$. 
\end{proof}

\begin{example}
Obervemos que $g_*$ podría no ser un epimorfismo. Si $g\colon\Z\to\Z/n\Z$ es el morfismo canónico de $\Z$-módulos, 
entonces $\Hom_{\Z}(\Z/n\Z,\Z)=\{0\}$ aunque $\Hom_{\Z}(\Z/n\Z,\Z/n\Z)\ne\{0\}$. 	
\end{example}

Diremos que el funtor $\Hom_R(M,-)$ es \textbf{exacto} si para toda sucesión 
	\[
	\xymatrix{
	 0\ar[r] 
	 & A
	 \ar[r]^-{f}
	 & B
	 \ar[r]^-{g}
	 & C\ar[r]
	 & 0
	 }
	\]
exacta se tiene que la sucesión 
 	\[
	\xymatrix{
	 0\ar[r] 
	 & \hom_R(M,A)
	 \ar[r]^-{f_*}
	 & \hom_R(M,B)
	 \ar[r]^-{g_*}
	 & \hom_R(M,C)\ar[r]
	 & 0
	 }
	\]
	de grupos abelianos y morfismos
	es también exacta.


\begin{proposition}
	Si la sucesión exacta
		\[
	\xymatrix{
	 0\ar[r] 
	 & A
	 \ar[r]^-{f}
	 & B
	 \ar[r]^-{g}
	 & C\ar[r]
	 & 0
	 }
	\]
	se parte, entonces 
	 \[
	\xymatrix{
	 0\ar[r] 
	 & \Hom_R(M,A)
	 \ar[r]^-{f_*}
	 & \Hom_R(M,B)
	 \ar[r]^-{g_*}
	 & \Hom_R(M,C)\ar[r]
	 & 0
	 }
	\]
	también se parte.
\end{proposition}

\begin{proof}
Debemos demostrar que $g_*$ es sobreyectiva. Por hipótesis sabemos que existe $h\in\Hom_R(C,B)$ tal que
$g\circ h=\id_C$. 
Si $f\in\Hom_R(M,C)$, entonces
\[
(g_*\circ h_*)(f)=g_*(h_*(f))=g_*(h\circ f)=g\circ (h\circ f)=(g\circ h)\circ f=\id_B\circ f=f.
\]
Como $g_*\circ h_*=\id$, se concluye que $g_*$ es sobreyectiva y que la sucesión exacta además se parte. 
% Si $h\in\Hom_R(C,B)$ es tal que $g\circ h=\id_C$, entonces $g_*\circ h_*=\id$ pues
% \[
% (g_*\circ h_*)(\alpha)=g\circ (h\circ\alpha)=(g\circ h)\circ\alpha=\id_C\circ \alpha=\alpha
% \]
% para todo $\alpha\in\Hom_R(M,C)$. 
\end{proof}

De la misma forma se define el funtor $\Hom_R(-,M)$. 
Para cada 
módulo $N$ el funtor nos devolverá el módulo $\Hom_R(N,M)$ y para cada
morfismo $f\in\Hom_R(A,B)$ nos devoverá un morfismo $f^*\colon\Hom_R(A,M)\to\Hom_R(B,M)$, 
$f^*(\alpha)=\alpha\circ f$. Tal como demostramos la proposición anterior
puede verse que
si la sucesión exacta
		\[
	\xymatrix{
	 0\ar[r] 
	 & A
	 \ar[r]^-{f}
	 & B
	 \ar[r]^-{g}
	 & C\ar[r]
	 & 0
	 }
	\]
	se parte, entonces 
	 \[
	\xymatrix{
	 0\ar[r] 
	 & \hom_R(C,M)
	 \ar[r]^-{g^*}
	 & \hom_R(B,M)
	 \ar[r]^-{f^*}
	 & \hom_R(A,M)\ar[r]
	 & 0
	 }
	\]
también se parte. 

\begin{example}
Si usamos el resultado anterior
con la sucesión exacta
\[
	\xymatrix{
	 0\ar[r] 
	 & B
	 \ar[r]^-{i_B}
	 & A\oplus B
	 \ar[r]^-{p_A}
	 & A\ar[r]
	 & 0
	 }
	\]
donde $i_B(b)=(0,b)$ y $p_A(a,b)=a$,  
podemos demostrar que  
\begin{equation}
\label{eq:hom1}
\Hom_R(A\oplus B,M)\simeq \Hom_R(A,M)\times\Hom_R(B,M).
\end{equation}
Puede verificarse que el isomorfismo está dado por $f\mapsto (f\circ i_A,f\circ i_B)$, 
donde $i_A\colon A\to A\oplus B$, $i_A(a)=(a,0)$ y 
$i_B\colon B\to A\oplus B$, $i_B(b)=(0,b)$.
\end{example}

\begin{example}
Tal como se hizo en el ejemplo anterior, puede demostrarse que 
\begin{equation}
\label{eq:hom2}	
\Hom_R(M,A\times B)\simeq \Hom_R(M,A)\times\Hom_R(M,B).	
\end{equation}
En este caso el isomorfismo es $f\mapsto (p_A\circ f,p_B\circ f)$, 
donde $p_A\colon A\oplus B\to A$, $p_A(a,b)=a$ y  
$p_B\colon A\oplus B\to B$, $p_B(a,b)=b$. 
\end{example}

Para entender mejor por qué usamos sumas y productos en las fórmulas~\eqref{eq:hom1} y~\eqref{eq:hom2} 
mencionamos que pueden demostrarse las fórmulas  
\begin{align*}
&\Hom_R\left(\bigoplus_{i\in I}M_i,M\right)\simeq \prod_{i\in I}\Hom_R(M_i,M),\\
&\Hom_R\left(M,\prod_{i\in I}M_i\right)\simeq \prod_{i\in I}\Hom_R(M,M_i).	
\end{align*}




\chapter{Módulos finitamente generados}

\begin{definition}
\index{Submódulo!generado por un conjunto}
	Sea $M$ un módulo y $X$ un subconjunto de $M$. El submódulo $(X)$ de $M$ generado por $X$ se define
	como la intersección de todos los submódulos de $M$ que contienen al conjunto $X$. 
\end{definition}

El submódulo de $M$ generado por el conjunto $X$ es el menor submódulo de $M$ que 
contiene a $X$. 
Puede demostrarse además que 
\[
(X)=\left\{ \sum_{i=1}^n r_i\cdot x_i:n\in\N_0,\,r_i\in R,\,x_i\in X\right\}
\]

\begin{definition}
\index{Módulo!finitamente generado}
Diremos que un submódulo $S$ de $M$ es \textbf{finitamente generado} si $S=(X)$ para algún conjunto finito $X$. 
\end{definition}

\begin{examples}\
\begin{enumerate}
	\item $\{1\}$ y $\{2,3\}$ son conjuntos de generadores de $\Z$.
	\item $\{2\}$ no es un conjunto de generadores de $\Z$.
\end{enumerate}	
\end{examples}

\begin{example}
Sea $R=\{f\colon [0,1]\to\R\}$ el anillo de funciones $[0,1]\to\R$ con las operaciones
\[
(f+g)(x)=f(x)+g(x),\quad
(fg)(x)=f(x)g(x),\quad
x\in[0,1].
\]
Sea $M=\prescript{}{R}R$, es
decir el anillo $R$ con la estructura de módulo dada por la representación regular a izquierda. Como el conjunto 
\[
S=\{f\in R:f(x)\ne 0\text{ para finitos $x$}\}
\] 	
es un ideal a izquierda de $R$, es un submódulo de $M$. Como módulo, $M$ está generado por la función constantemente igual a uno. Sin embargo,
$S$ no es finitamente generado. En efecto, si $S=(f_1,\dots,f_k)$, sea 
\[
X=\{x\in[0,1]:f_i(x)\ne 0\text{ para algún $i$}\}.
\]
Como $X$ es finito, podemos suponer que $X=\{x_1,\dots,x_l\}$. Sea $x_0\in[0,1]\setminus X$ y sea $\varphi\colon [0,1]\to\R$ tal que 
$\varphi(x_0)=1$ y $\varphi(x)=0$ para todo $x\ne x_0$. Entonces $\varphi\in S$ pero $\varphi\not\in (f_1,\dots,f_k)$, pues
\[
\left(\sum_{i=1}^k r_i\cdot f_i\right)(x_0)
=\sum_{i=1}^k r_i(x_0)f_i(x_0)=0\ne 1=\varphi(x_0). 
\]
\end{example}

\begin{example}
Sea $K$ un cuerpo y sea 
$V$ un espacio vectorial. Si $T\colon V\to V$ es una transformación
lineal, entonces $V$ es un 
$K[X]$-módulo con
\[
\left(\sum_{i=0}^n a_iX^i\right)\cdot v=
\sum_{i=0}^n a_iT^i(v).
\]
Veamos que $V$ es de dimensión finita, entonces $V$ es finitamente generado como $K[X]$-módulo. En efecto,
si $\{v_1,\dots,v_n\}$ es una base del espacio vectorial $V$, entonces
$V=(v_1,\dots,v_n)$ como $K[X]$-módulo, pues para cada $v\in V$ existen constantes 
$\lambda_1,\dots,\lambda_k\in K\subseteq K[X]$ tales que 
$v=\sum_{i=1}^n\lambda_i v_i$.    
\end{example}

\begin{example}
Sea $G=\{g_1,\dots,g_k\}$ un grupo finito de orden $k$ y supongamos que $g_1=1$.  
Si $M$ es un $\C[G]$-módulo finitamente generado, entonces, 
en particular, $M$ es un espacio vectorial de dimensión finita. En efecto,
$M$ es un espacio vectorial con la acción 
\[
\lambda m=(\lambda g_1)\cdot m=(\lambda 1)\cdot m
\]
para todo $\lambda\in\C$ y $m\in M$. Supongamos ahora que $M=(m_1,\dots,m_l)$. Para cada $m\in M$ 
existen $\alpha_1,\dots,\alpha_l\in\C[G]$ tales que 
\[
m=\alpha_1\cdot m_1+\cdots+\alpha_l\cdot m_l.
\]
Además para cada $j\in\{1,\dots,l\}$ existen $\lambda_{ij}\in\C$ tales que
$\alpha_j=\sum_{i=1}^k\lambda_{ij}g_i$. En consecuencia, cada $m\in M$ puede escribirse 
como 
\[
m=\sum_{i=1}^k\sum_{j=1}^l\lambda_{ij}(g_j\cdot m_i)
\]
para ciertos $\lambda_{ij}\in\C$, donde $i\in\{1,\dots,k\}$ y $j\in\{1,\dots,l\}$. En particular, $M$ es de dimensión finita, pues $\dim M\leq kl$.
\end{example}

\begin{proposition}
	Si  
	\[  
		\xymatrix{
        0\ar[r]
        & M
        \ar[r]^f
        & N
        \ar[r]^g
        & T\ar[r]
        & 0	
        }
     \]
     es exacta, valen las siguientes afirmaciones.
     \begin{enumerate}
     \item Si $N$ es finitamente generado, entonces $T$ es finitamente generado.
     \item Si $M$ y $T$ son finitamente generados, entonces $N$ es finitamente generado.	
     \end{enumerate}
\end{proposition}

\begin{proof}
Comenzaremos con la demostración de la primera afirmación. Veremos que si $N=(n_1,\dots,n_k)$, entonces $T=(g(n_1),\dots,g(n_k))$. En efecto,
si $t\in T$, existe $n\in N$ tal que $g(n)=t$. Si escribimos
$n=\sum_{i=1}^k r_i\cdot n_i$, entonces 
\[
t=g(n)=\sum_{i=1}^k r_i\cdot g(n_i).
\] 

Demostremos ahora la segunda afirmación. 
Supongamos que $M=(m_1,\dots,m_k)$ y que $T=(t_1,\dots,t_l)$. Como $g$ es sobreyectiva,
para cada $i\in\{1,\dots,l\}$ existe $n_i\in N$ tal que $g(n_i)=t_i$. Vamos a demostrar que 
$N=(f(m_1),\dots,f(m_k),n_1,\dots,n_l)$. Sea $n\in N$. Como $g(n)\in T$, existen $r_1,\dots,r_l\in R$ tales que
\[
g(n)=\sum_{i=1}^l r_i\cdot t_i=g\left(\sum_{i=1}^l r_i\cdot n_i\right),
\]	
lo que implica que $n-\sum_{i=1}^l r_i\cdot n_i\in\ker g=f(M)$. En particular, existen $s_1,\dots,s_k\in R$ tales
que
\[
n-\sum_{i=1}^l r_i\cdot n_i=\sum_{j=1}^k s_j\cdot f(m_j),
\]
pues $f(M)=(f(m_1),\dots,f(m_k))$. 
\end{proof}


\begin{proposition}
Sea $M$ un módulo. Entonces $M$ es finitamente generado si y sólo $M$ 
es isomorfo a un cociente de $R^k$ para algún $k\in\N$. 	
\end{proposition}

\begin{proof}
Si $M=(m_1,\dots,m_k)$, entonces $\varphi\colon R^k\to M$, $(r_1,\dots,r_k)\mapsto \sum r_im_i$, es un epimorfismo
y luego $R^k/\ker\varphi\simeq M$. Recíprocamente, si $\varphi\colon R^k\to M$ es un epimorfismo, como
$R^k$ está generado por $\{e_i:1\leq i\leq k\}$, donde 
\[
(e_i)_j=\begin{cases}
1 & \text{si $i=j$},\\
0 & \text{si $i\ne j$}.	
\end{cases}
\]
el conjunto $\{\varphi(e_i):1\leq i\leq k\}$ genera $\varphi(R^k)=M$. 
\end{proof}

Tal como vimos en la teoría de anillos, tener objetos finitamente generados está relacionado con el concepto de noetherianidad. 

\begin{definition}
\index{Módulo!noetheriano}
Un módulo se dice \textbf{noetheriano} si toda sucesión $M_1\subseteq M_2\subseteq\cdots$ de submódulos de $M$ 
se estabiliza, es decir que existe $n$ tal que $M_k=M_{n+k}$ para todo $k\in\N$. 	
\end{definition}

\begin{proposition}
Sea $M$ un módulo. Las siguientes afirmaciones son e	quivalentes:
\begin{enumerate}
\item $M$ es noetheriano.
\item Todo submódulo de $M$ es finitamente generado.
\item Toda familia no vacía de submódulos de $M$ tiene un elemento maximal (con respecto a la inclusión).	
\end{enumerate}
\end{proposition}

\begin{proof}
	Demostremos que $(2)\implies(1)$. Si $S_1\subseteq S_2\subseteq\cdots$ es una sucesión de submódulos de $M$,
	puede demostrarse que $S=\cup_{i\geq 1}S_i$ es un submódulo de $M$. Como $S$ es finitamente generado,
	digamos $S=(x_1,\dots,x_n)$ para finitos elementos $x_1,\dots,x_n\in M$, entonces
	$x_1,\dots,x_n\in S_N$ para algún $N\in\N$. Luego $S\subseteq S_N$ y entonces $S_N=S_{N+k}$ para todo $k\in\N$. 
	
	Demostremos ahora que $(1)\implies(3)$. Si $F$ es una familia no vacía de submódulos de $M$ que no tiene
	elementos maximales, sea $S_1\in F$. Como $S_1$ no es maximal, existe entonces $S_2\in F$ tal que $S_1\subsetneq S_2$. 
	Si tenemos $S_1\subsetneq\dots\subsetneq S_k$, entonces la no maximalidad de $S_k$ nos dice que
	existe $S_{k+1}\in F$ tal que $S_k\subsetneq S_{k+1}$, de forma que la sucesión de los $S_j$ no se estabiliza. 
	
	Por último, demostramos que $(3)\implies(2)$. Sea $S$ un submódulo de $M$ y sea
	\[
	F=\{T\subseteq S:T\subseteq M\text{ submódulo finitamente generado}\}.
	\]
	Por hipótesis, $F$ tiene un elemento maximal, digamos $N$. Entonces $N$ es un submódulo de $M$ tal que $N\subseteq S$ y $N$
	es finitamente generado, digamos $N=(n_1,\dots,n_k)$. Si $N=S$, en particular $S$ es finitamente genreado. Si $N\ne S$, 
	sea  $x\in S\setminus N$. Entonces 
	$N\subseteq (n_1,\dots,n_k,x)\subseteq S$, lo que implica, por la maximalidad de $N$, que $N=(n_1,\dots,n_k,x)$, una contradicción
	pues $x\not\in N$. 
\end{proof}

\begin{exercise}
\label{xca:exacta_noetheriano}
	Si  
	\[  
		\xymatrix{
        0\ar[r]
        & M
        \ar[r]^f
        & N
        \ar[r]^g
        & T\ar[r]
        & 0	
        }
     \]
     es exacta, valen las siguientes afirmaciones:
     \begin{enumerate}
     	\item Si $N$ es noetheriano, entonces $M$ y $T$ son noetherianos.
     	\item Si $M$ y $T$ son noetherianos, entonces $N$ es noetheriano.
     \end{enumerate}	
\end{exercise}


\begin{exercise}
\label{xca:regular_noetheriano}
Un anillo $R$ es noetheriano si y sólo si el módulo $\prescript{}{R}R$ es noetheriano.	
\end{exercise}

\begin{exercise}
\label{xca:directa_noetheriano}
Si $M_1,\dots,M_n$ son noetherianos, entonces $M_1\oplus\cdots\oplus M_n$ es noetheriano. 	
\end{exercise}

Es interesante mencionar que el resultado del ejercicio anterior no vale para sumas infinitas de módulos. Puede demostrarse
por ejemplo que $\Z^{\N}$ no es noetheriano, pues no es finitamente generado. 
%$\Z^{(\N)}$ no es noetheriano, ya que no es finitamente generado.  

\begin{proposition}
Si $R$ es noetheriano y $M$ es un módulo finitamente generado, entonces $M$ es noetheriano.	
\end{proposition}

\begin{proof}
Como $M$ es finitamente generado, digamos $M=(m_1,\dots,m_k)$, existe un epimorfismo 
$R^k\to N$, $(r_1,\dots,r_k)\mapsto \sum_{i=1}^k r_i\cdot m_i$, donde
$R^k=\oplus_{i=1}^k R$. Como $R$ es noetheriano, $R^k$ es noetheriano y luego $M$ es también noetheriano.	
\end{proof}


\chapter{Módulos libres}

\begin{definition}
\index{Conjunto!linealmente independiente}
Sea $X$ un subconjunto de un módulo $M$. Diremos que $X$ es \textbf{linealmente independiente}
si para cada $k\in\N$, $r_1,\dots,r_k\in R$ y $m_1,\dots,m_k\in X$ tales que 
$r_1\cdot m_1+\cdots+r_k\cdot m_k=0$ se tiene que $r_1=\cdots=r_k=0$. 
Un conjunto que no es linealmente independiente se dirá \textbf{linealmente dependiente}.\end{definition}

La independencia lineal en módulos es levemente distinta a la que conocemos para
espacios vectoriales. En módulos la dependencia lineal no garantiza que uno de los elementos del conjunto
pueda escribirse como combinación lineal de los otros, ya que en el anillo $R$ no siempre podremos dividir.

\begin{example}
Consideramos $\Z$ como $\Z$-módulo. 
El conjunto $\{2,3\}\subseteq\Z$ es linealmente dependiente pues $(-3)\cdot 2+2\cdot 3=0$. Observemos
que $2$ no es un múltiplo entero de $3$, tampoco $3$ es un múltiplo entero de $2$.   	
\end{example}

\begin{example}
El conjunto $\{1\}$ del módulo $\prescript{}{R}R$ es linealmente independiente. Si $r$ es un 
divisor de cero de $R$, entonces $\{r\}$ es linealmente dependiente.	
\end{example}

En módulos, un conjunto minimal de generadores puede ser linealmente dependiente. 

\begin{example}
Sea $R=M_2(\R)$ y sea $M=\begin{pmatrix} 0 & \R\\ 0 & \R\end{pmatrix}$. Dejamos como ejercicio demostrar
que $M$ es un módulo sobre el anillo $R$ con la multiplicación usual de matrices. 
El conjunto $\left\{\begin{pmatrix} 0&1\\0&0\end{pmatrix}\right\}$ es un 
conjunto minimal de generadores y es linealmente dependiente. 	
\end{example}

\begin{example}
Sea $M=\Q$ como $\Z$-módulo. Si $x\in\Q\setminus\{0\}$, entonces $\{x\}$ es linealmente independiente. Si $x,y\in\Q$ son tales
que $x\ne y$, entonces $\{x,y\}$ es linealmente dependiente. 	
\end{example}

Si $V$ es un espacio vectorial y $v\in V\setminus\{0\}$, entonces $\{v\}$ es linealmente independiente, pues
si $\lambda\ne0$ y $v\ne 0$, entonces $\lambda v\ne 0$. 
En la teoría de módulos, las cosas pueden ser distintas. 

\begin{example}
Todo subconjunto del $\Z$-módulo $\Z/n$ es linealmente dependiente. 
\end{example}

\begin{exercise}
Sea $f\in\Hom_R(M.N)$ y sea $X$ un subconjunto de $M$. 
\begin{enumerate}
\item Si $X$ es linealmente dependiente, entonces $f(X)$ también.
\item Si $X$ es linealmente independiente y $f$ es monomorfismo, entonces $f(X)$ es linealmente independiente.
\item Si $M=(X)$ y $f$ es epimorfismo, entonces $N=(f(X))$. 	
\end{enumerate}
\end{exercise}

\begin{definition}
	\index{Base}
	Sea $M$ un módulo. Un subconjunto $B$ de $M$ es una \textbf{base} de $M$ si 
	es linealmente independiente y además $(B)=M$. Un módulo $M$ se dice \textbf{libre} si 
	admite una base.  
\end{definition}

\begin{examples}\
\begin{enumerate}
	\item Todo espacio vectorial es un módulo libre.
	\item $\prescript{}{R}R$ es libre con base $\{1\}$. 
	\item El $\Z$-módulo $\Q$ no es libre.  
	\item $R^n$ es libre como $R$-módulo.
\end{enumerate}	
\end{examples}

\begin{example}
Las únicas bases de $\Z$ como $\Z$-módulo son $\{1\}$ y $\{-1\}$.	
\end{example}

El ejemplo siguiente es bien conocido en el caso de espacios vectoriales. Los módulos
sobre anillos de división son muy similares a los espacios vectoriales 
y por esa razón se los llama 
\textbf{espacios vectoriales sobre anillos de división}.

\begin{example}
\index{Espacios vectoriales}
\index{Módulos!sobre anillos de división}
Sea $R$ un anillo de división y sea $M$ un $R$-módulo no nulo finitamente generado. 
Vamos a demostrar las siguientes propiedades: 
\begin{enumerate}
\item Todo conjunto finito de generadores contiene una base. En particular, $M$ es libre. 
\item Todo conjunto linealmente independiente puede extenderse a una base.
\item Dos bases cualesquiera tienen la misma cantidad de elementos. 
\end{enumerate}

Para demostrar la primera afirmación procederemos por inducción en la cantidad de generadores de $M$. Si $M=(m)$, entonces
$\{m\}$ es base pues $\{m\}$ es linealmente independiente: si $r\cdot m=0$ y $r\ne 0$, entonces
\[
m=1\cdot m=(r^{-1}r)\cdot m=r^{-1}\cdot (r\cdot m)=0.
\]
Si vale para $k-1$ generadores, sea $M=(m_1,\dots,m_k)$. Si $\{m_1,\dots,m_k\}$ no es linealmente
independiente, entonces existen $r_1,\dots,r_k\in R$ no todos cero tales
que
\[
r_1\cdot m_1+\cdots+r_k\cdot m_k=0.
\]
Sin perder generalidad podemos suponer que $r_k\ne 0$. Entonces
\[
v_k=\sum_{i=1}^{k-1} (r_k^{-1}r_i)\cdot m_i\in (m_1,\dots,m_{k-1}).
\]
Como entonces $M=(m_1,\dots,m_k)=(m_1,\dots,m_{k-1})$, la hipótesis inductiva implica que
$M$ es libre. 

Vamos a demostrar ahora que todo
conjunto $X$ linealmente independiente puede extenderse a una base. 	Sea $X=\{x_1,\dots,x_k\}$ tal que $M=(X)$. 
Como $M\ne\{0\}$, sin perder generalidad podemos suponer que $x_1\ne 0$. Como $R$ es de división,
el conjunto $\{x_1\}$ es linealmente independiente, pues si $r\ne 0$ y $r\cdot x_1=0$, entonces 
\[
x_1=1\cdot r=(r^{-1}r)\cdot x_1=r^{-1}\cdot (r\cdot x)=r^{-1}\cdot 0=0.
\]
Sea $Y=\{y_1,\dots,y_l\}$ un subconjunto de $X$ maximal tal que $Y$ es linealmente independiente. Veamos que $X\subseteq (Y)$. Sea $x\in X$. Si $x\not\in Y$,  
entonces, como $Y\subseteq Y\cup \{x\}$, la maximalidad de $Y$ implica que $\{x\}\cup Y$ es linealmente dependiente, es decir
que existen $r,r_1,\dots,r_k\in R$ no todos cero tales que
\[
r\cdot x+\sum_{i=1}^l r_i\cdot y_i=0.
\]
Si $r=0$, entonces $r_1=\cdots=r_l=0$ porque los $y_j$ son linealmente independientes, una contradicción. Luego $r\ne 0$ y entonces
\[
x=-\sum_{i=1}^l (-r^{-1}r_i)\cdot y_i\in (Y).
\]
Luego $X\subseteq (Y)$. En conclusión $Y$ es una base de $M$ pues $M=(Y)$ 
y además $Y$ es linealmente independiente. 

Demostremos que dos bases finitas cualesquiera tienen la misma cantidad de elementos. 
Para eso es suficiente demostrar
que si $X$ e $Y$ son conjuntos finitos linealmente independientes
tales que $(X)\subseteq (Y)$, entonces $|X|\leq |Y|$. Supongamos que $|X|=k$ e $|Y|=l$. 
Procederemos por inducción en $l$. Si $l=1$ y $k>1$, entonces
exiten $r_1,r_2\in R$ tales que $x_1=r_1\cdot y_1$ y $x_2=r_2\cdot y_1$. Luego
\[
x_2=r_2\cdot y_1=r_2\cdot (r_1^{-1}\cdot x_1)=(r_2r_1^{-1})\cdot x_1,
\]
una contradicción pues $\{x_1,x_2\}$ es linealmente independiente. 
Supongamos ahora que el resultado 
es verdadero para $l-1$ y sea $l=|Y|$. Para cada $j$ escribimos
\[
x_j = \sum_{i=1}^l r_{ji}\cdot y_i,
\]
donde $r_{ji}\in R$. Si $r_{j1}\ne 0$ para todo $j$, entonces 
$x_j=\sum_{i=2}^l r_{ji}\cdot y_i$ para todo $j$ y luego $(X)\subseteq (y_2,\dots,y_l)$, que implica que
$|X|\leq l-1<l=|Y|$. Si existe $j$ tal que $r_{j1}\ne 0$, sin perder generalidad podemos
suponer que $r_{11}\ne 0$. Para cada $j\in\{2,\dots,k\}$ sea
\[
z_j = x_j-(r_{j1}r_{11}^{-1})\cdot x_1.
\]
Como $z_j\in (y_2,\dots,y_l)$ para todo $j$ y los $z_j$ son linealmente independientes, 
la hipótesis inductiva impica que $k-1\leq l-1$, es decir $|X|\leq |Y|$.  
\end{example}

\index{Dimensión!de un módulo sobre un anillo de división}
\index{Dimensión!de un espacio vectorial}
El ejemplo anterior nos permite hablar de 
la \textbf{dimensión} de un módulo sobre un anillo de división. 

\begin{example}
El anillo $\R[X]$ es un $\R$-módulo libre con base $\{1,X,X^2,\dots\}$. También es un $\R[X]$-módulo libre con base $\{1\}$.  	
\end{example}

\begin{exercise}
Demuestre que el conjunto $\{(a,b),(c,d)\}$ es base del $\Z$-módulo $\Z\times\Z$ si y sólo si $ad-bc\in\{-1,1\}$. 
\end{exercise}

En particular, $\{(1,0),(0,1)\}$ es una base de $\Z\times\Z$ como $\Z$-módulo. 

\begin{example}
Si $u\in\mathcal{U}(R)$, entonces $\{u\}$ es una base de $\prescript{}{R}R$. Recíprocamente, si $R$ es un dominio íntegro y  
$\{z\}$ es una base de $\prescript{}{R}R$, entonces $z\in\mathcal{U}(R)$, pues, como
$1=yz$ para algún $y\in R$, también se tiene que $zy=1$ pues  
\[
(zy-1)z=z(yz)-z=z1-z=z-z=0.
\]	
\end{example}

\begin{example}
Sea $I$ un conjunto no vacío. El $R$-módulo $R^{(I)}$ es libre con base $\{e_i:i\in I\}$, donde 
\[
(e_i)_j=\begin{cases}
	1 & \text{si $i=j$},\\
	0 & \text{si $i\ne j$.}
	\end{cases}	
\]
\end{example}

\begin{example}
Si $R=M_2(\Z)$, entonces $M=\prescript{}{R}R$ es libre con base $\left\{\begin{pmatrix}	1&0\\0&1\end{pmatrix}\right\}$. El submódulo
$N=\begin{pmatrix}\Z&0\\\Z&0\end{pmatrix}$ no admite una base como $R$-módulo.	
\end{example}

A diferencia de lo que pasa con espacios vectoriales, el tamaño de una base
no es un invariante. 

\begin{example}
Sea $V$ el espacio vectorial (complejo) con base infinita $e_0,e_1,e_2,\dots$ y sea $R=\End(V)$ con la estructura
de anillo dada por 
\[
(f+g)(v)=f(v)+g(v),\quad
(fg)(v)=f(g(v))
\]
para $f,g\in R$ y $v\in V$. 

Sea $M=\prescript{}{R}R$. 
Sabemos que $\{\id\}$ es una base para $R$.  
Mostraremos que $M$ admite también una base que tiene dos elementos.  
Si $r,s\in R$ son tales que
\begin{align*}
&r(e_{2n})=e_n, && r(e_{2n+1})=0,\\
&s(e_{2n})=0,&& s(e_{2n+1})=e_{2n},
\end{align*}
entonces $\{r,s\}$ es base de $M$. 

Si $f\in R$, entonces
$f=\alpha r+\beta s$, donde $\alpha\colon V\to V$, $e_n\mapsto f(e_{2n})$ para todo $n\in\N$, y $\beta\in V\to V$, $e_n\mapsto f(e_{2n+1})$ para todo $n\in\N$. En efecto,
\begin{align*}
&(\alpha r+\beta s)(e_{2n})=\alpha(r(e_{2n}))+\beta(s(e_{2n}))=f(e_{2n}),\\
&(\alpha r+\beta s)(e_{2n+1})=\alpha(r(e_{2n+1}))+\beta(s(e_{2n+1}))=f(e_{2n+1}).
\end{align*}
Además $\{r,s\}$ es linealmente independiente, pues si $\alpha r+\beta s=0$ para $\alpha,\beta\in R$, entonces al evaluar 
en los $e_{2n}$ se obtiene que $\alpha=0$ y al evaluar en los $e_{2n+1}$ se obtiene que $\beta=0$.   
\end{example} 

\begin{example}
Si $M$ es un módulo libre con base $X$ y $N$ es un módulo libre con base $Y$, entonces
$M\oplus N$ es un módulo libre con base 
\[
\{(x,0):x\in X\}\cup \{(0,y):y\in Y\}.
\]	
\end{example}

\begin{exercise}
Sea $R$ es un anillo conmutativo. Si $M$ y $N$ son libres y finitamente generados, entonces
$\Hom_R(M,N)$ es libre y finitamente generado.	
\end{exercise}

% Como $R$ es conmutativo, $\Hom_R(M,N)$ es un $R$-módulo. Si $\{m_1,\dots,m_k\}$ es base de $M$ 
% y $\{n_1,\dots,n_l\}$ es base de $N$, entonces definimos para cada $i\in\{1,\dots,k\}$ y $j\in\{1,\dots,l\}$ 
% definimos $f_{ij}$ 
% \[
% f_{ij}(m_k)=\begin{cases}
% n_j & \text{si $k=i$},\\
% 0 & \text{si $k\ne i$}.
% \end{cases}
% \]
% Entonces $\{i_ij}$ es base de $\Hom_R(M,N)$. 
Veamos algunos resultados básicos que nos permiten entender qué significa tener un módulo libre. 

\begin{proposition}
Si $M$ es libre, entonces existe un subconjunto $\{m_i:i\in I\}$ de $M$ tal que 
para cada $m\in M$ existen únicos $r_i\in R$, $i\in I$, 
donde $r_i=0$ salvo finitos $i\in I$ 
tales que $m=\sum r_i\cdot m_i$. 
\end{proposition}

\begin{proof}
Como $M$ es libre, existe una base $\{m_i:i\in I\}$ de $M$. Si $m\in M$, entonces
podemos escribir $m=\sum r_i\cdot m_i$ (suma finita) para ciertos $r_i\in I$. Veamos que los $r_i$ son únicos. Si $m=\sum s_i\cdot m_i$, entonces
$\sum (r_i-s_i)\cdot m_i=0$. La independencia lineal del conjunto $\{m_i:i\in I\}$ implica entonces que $r_i=s_i$ para todo $i\in I$.  	
\end{proof}

\begin{proposition}
\label{pro:libre}
Sea $M$ libre con base $\{m_i:i\in I\}$ y sea $N$ un submódulo de $M$. Si $\{n_i:i\in I\}$ es un módulo, existe
un único $f\in\Hom_R(M,N)$ tal que $f(m_i)=n_i$ para todo $i\in I$.  
\end{proposition}

\begin{proof}[Bosquejo de la demostración]
Basta con observar que el único morfismo $f\colon M\to N$ debe definirse como $f(\sum r_im_i)=\sum r_i\cdot n_i$.  	
\end{proof}

Una aplicación sencilla de la proposición anterior:

\begin{example}
Veamos que no existe un epimorfismo $\Z\to\Z\times\Z$ (de $\Z$-módulos). En efecto, si $f\colon\Z\to\Z\times\Z$ es un epimorfismo, 
sea $\{u,v\}$ una base de $\Z\times\Z$. Entonces $f(k)=u$ y $f(l)=v$ para ciertos $k,l\in\Z$. La proposición~\ref{pro:libre} implica
que existe un morfismo $g\colon\Z\times\Z\to\Z$ tal que $g(u)=k$ y $g(v)=l$. En particular, $f\circ g=\id_{\Z\times\Z}$ y luego 
$g$ es un monomorfismo. 	Como 
\[
g(lu-kv)=lg(u)-kg(v)=lk-kl=0,
\]
se tiene entonces que $lu-kv=0$ y luego $k=l=0$, pues $\{u,v\}$ es linealmente independiente, una contradicción.  
\end{example}

Otra propiedad importante de los módulos libres:

\begin{proposition}
Si $M$ es libre, entonces $M\simeq R^{(I)}$ para algún conjunto $I$.
\end{proposition}

\begin{proof}
Supongamos que $M$ es libre con base $\{m_i:i\in I\}$. Vimos una proposición que nos dice que 
existe un único morfismo $f\in\Hom_R(M,R^{(I)})$ tal que 
$f(m_i)=e_i$ para todo $i\in I$, donde 
\[
(e_i)_j=\begin{cases}
	1 & \text{si $i=j$},\\
	0 & \text{si $i\ne j$.}
	\end{cases}	
\]	
Veamos que $f$ es un isomorfismo. 
Primero veamos que $f$ es epimorfismo: dado $(r_i)_{i\in I}\in R^{(I)}$, entonces
$f(\sum r_i\cdot m_i)=(r_i)_{i\in I}$. Para ver que es monomorfismo:
\[
0=f(\sum r_i\cdot m_i)=\sum r_i\cdot f(m_i)=\sum r_i\cdot e_i\implies r_i=0\text{ para todo $i\in I$}.\qedhere
\]
\end{proof}

\begin{corollary}
Todo módulo es (isomorfo a un) cociente de un módulo libre.
\end{corollary}

\begin{proof}
Si $M$ es un módulo, vamos a demostrar que existe un módulo libre $L$ y un epimorfismo $f\in\Hom_R(L,M)$, ya que
entonces $L/\ker f\simeq M$ por el primer teorema de isomorfismos. Sea 
$\{m_i:i\in I\}$ un conjunto de generadores de $M$, que siempre existe, ya que podríamos tomar por ejemplo el conjunto
$\{m:m\in M\}$ y sea $L=R^{(I)}$. Entonces $L$ es libre y $f\colon R^{(I)}\to M$, $e_i\mapsto m_i$, es un epimorfismo. 
\end{proof}

Otra propiedad importante de los módulos libres:

\begin{proposition}
	Si $M$ es un módulo libre, $f\in\Hom_r(N,T)$ es un epimorfismo y $h\in\Hom_R(M,T)$, entonces existe
	$\varphi\in\Hom_R(M,N)$ tal que $f\circ \varphi=h$. 
	% todo: diagrama
\end{proposition}

\begin{proof}
Si $\{m_i:i\in I\}$ es base de $M$, como $f$ es un epimorfismo, para cada $i\in I$ existe $n_i\in N$ tal que $f(n_i)=h(m_i)$. Como 
$M$ es libre, existe un único morfismo $\varphi\colon M\to N$ tal que $\varphi(m_i)=n_i$ para todo $i\in I$. Este morfismo
cumple $f\circ \varphi=h$. 
\end{proof}

%\index{Sumando directo!de un submódulo}
%Recordemos que un submódulo $N$ de $M$ es un \textbf{sumando directo} de $M$ si existe un submódulo $T$ de $M$ tal que
%$M=N\oplus T$. 

\index{Cuerpo!de fracciones}
Sea $R$ un dominio íntegro y sea $S=R\setminus\{0\}$. 

En el conjunto $R\times S$ definimos
la siguiente relación:
\[
(r,s)\sim (r_1,s_1)\Longleftrightarrow rs_1-r_1s=0.
\]

Dejamos como ejercicio verificar que $\sim$ es una relación de equivalencia. 

La clase
de equivalencia del par $(r,s)$ será denotada por $r/s$ o bien $\frac{r}{s}$. 

Puede demostrarse que 
el conjunto
$K(R)=(R\times S)/{\sim}$ de clases de equivalencia 
es un cuerpo con las operaciones
\begin{equation}
\label{eq:K(R)}
\frac{r}{s}+\frac{r_1}{s_1}=\frac{rs_1+r_1s}{ss_1},
\quad
\frac{r}{s}\frac{r_1}{s_1}=\frac{rr_1}{ss_1}.
\end{equation}
$K(R)$ se llama el \textbf{cuerpo de fracciones} de $R$. 
Por ejemplo, 
$K(\Z)=\Q$.

Para demostrar que $K(R)$ es un cuerpo primero
debemos ver que las operaciones~\ref{eq:K(R)} están bien definidas. Por ejemplo, si $r/s\sim r'/s'$ y $r_1/s_1\sim r_1'/s_1'$, entonces
$r/s+r_1/s_1\sim r'/s'+r_1'/s_1'$. En efecto,
como $r/s\sim r'/s'$, entonces $rs'-r's=0$. Similarmente, $r_1s_1'-r_1's_1=0$, pues $r_1/s_1\sim r_1'/s_1'$. Luego 
\begin{gather*}
\frac{rs_1+r_1s}{ss_1}=\frac{r's_1'+r_1's'}{s's_1'}
\shortintertext{pues}
(rs_1+r_1s)s's_1'=rs_1s's_1'+r_1ss's_1'
=r'ss_1s_1'+r_1's_1ss'
=(r's_1'+r_1's)ss_1.
\end{gather*}
De la misma forma se demuestra que el producto también está bien definido. Dejamos como ejercicio
verificar que con estas operaciones $K(R)$ es un cuerpo. 

\begin{theorem}
Sea $R$ un dominio íntegro. 
Si $M$ es un módulo libre y finitamente generado, entonces dos bases cualesquiera de $M$ 
tienen la misma cantidad de elementos.
\end{theorem}

\begin{proof}
Sea $K=K(R)$ el cuerpo de fracciones de $R$. Es sencillo verificar que el grupo abeliano  
$V=\Hom_R(M,K)$ es un $K$-espacio vectorial con la acción 
\[
(\lambda f)(m)=\lambda f(m),
\]
donde $\lambda\in K$, $f\in V$ y $m\in M$.

El espacio vectorial $V$ tiene bien definida su dimensión. 
Calculemos entonces $\dim V$. Para eso, sea $\{e_1,\dots,e_n\}$ una base de $M$. 
Para cada $i\in\{1,\dots,n\}$ sea
\[
f_i\colon M\to K,\quad
e_j\mapsto\begin{cases}
1 & \text{si $i=j$},\\
0 & \text{si $i\ne j$}.
\end{cases}
\]
Veamos que $\{f_1,\dots,f_n\}$ es base de $V$. Es un conjunto de generadores de $V$ pues 
si $f\in V$, entonces
\[
f=\sum_{i=1}^n f(e_i)f_i,
\]
pues estos morfismos coinciden en los elementos de una base de $M$, es decir 
$f(e_j)=(\sum_{i=1}^n f(e_i)f_i)(e_j)$ para todo $j\in\{1,\dots,n\}$.    
Además $\{f_1,\dots,f_n\}$ es linealmente independiente, pues si $0=\sum_{i=1}^n\lambda_if_i$, 
entonces, al evaluar en cada $e_j$, se tiene que 
\begin{align*}
0=\left(\sum_{i=1}^n \lambda_if_i\right)(e_j)=\lambda_j
\end{align*}
para todo $j\in\{1,\dots,n\}$. Luego $n=\dim V$. 
\end{proof}

\begin{definition}
\index{Rango!de un módulo}
Sea $R$ un dominio íntegro. Si $M$ es un módulo finitamente generado, se 
define el \textbf{rango} de $M$ como el cardinal de una base de $M$. Si $M$ no es finitamente
generado diremos que el rango de $M$ es infinito. El rango del módulo $M$ será denotado por $\rank(M)$.   
\end{definition}


El teorema anterior puede extenderse para módulos sobre anillos conmutativos. En consecuencia,
la noción de rango de un módulo queda bien definida para módulos sobre anillos conmutativos.
En efecto, sea $F$ un módulo libre con base en un conjunto $X$. 
Como $R$ es conmutativo, sabemos que $R$ admite un ideal 
maximal $I$. Como $I$ es maximal, el cociente $K=R/I$ es entonces un cuerpo. Puede verificarse
que el subconjunto
\[
I\cdot F=\left\{\sum_{i=1}^n r_i\cdot x_i:n\in\N,\,r_1,\dots,r_n\in I\right\}
\]
es un submódulo de $F$. El módulo cociente $V=F/(I\cdot F)$ es un $K$-espacio
vectorial con base $\{x+I\cdot F:x\in X\}$. Luego $|X|=\dim V$. 
\chapter{Módulos proyectivos}


\begin{definition}
\index{Módulo!proyectivo}
Un módulo $P$ se dice \textbf{proyectivo} si dados $f\in\Hom_R(P,N)$ y un epimorfismo $g\in\Hom_R(M,N)$ 
existe $h\in\Hom_R(P,M)$ tal que $g\circ h=f$, es decir que existe un morfismo $h$ que
hace conmutativo al diagrama
\[
\xymatrix{ & P\ar[d]^f\ar@{-->}[ld]_h\\ M\ar[r]^g & N\ar[r] & 0	}
\]
\end{definition}

\begin{example}
Todo módulo libre es proyectivo. En efecto, si $S$ es base de módulo libre $P$, dado $s\in S$ existe
$m\in M$ tal que $g(m)=f(s)$. Definimos entonces
$h\colon P\to M$, $h(s)=m$ y extendemos por linealidad.  
\end{example}

En particular, $\Z$ es libre como $\Z$-módulo. 

\begin{theorem}
\label{thm:proyectivo}
	Sea $P$ un módulo. Las siguientes afirmaciones son equivalentes:
	\begin{enumerate}
		\item $P$ es proyectivo.
		\item $\Hom_R(P,-)$ es exacto.
		\item Toda sucesión exacta
			\[
			\xymatrix{
			0\ar[r] 
			& N
			\ar[r]^-{f}
			& M
			\ar[r]^-{g}
			& P\ar[r]
			& 0
			}
			\]
 			se parte, es decir que existe $h\in\Hom_R(P,M)$ tal que $g\circ h=\id_P$. En particular, 
 			$M\simeq N\oplus P$. 
 		\item $P$ es sumando directo de un libre.
	\end{enumerate}
\end{theorem}

\begin{proof}
	Probemos primero que $(1)\Leftrightarrow(2)$. Sea 
	\[
	\xymatrix{
	 0\ar[r] 
	 & A
	 \ar[r]^-{f}
	 & B
	 \ar[r]^-{g}
	 & C\ar[r]
	 & 0
	 }
	\]
	una sucesión exacta. Tenemos que probar que si $P$ es proyectivo, entonces la
	sucesión
	\[
	\xymatrix{
	 0\ar[r] 
	 & \hom_A(P,A)
	 \ar[r]^-{f_*}
	 & \hom_A(P,B)
	 \ar[r]^-{g_*}
	 & \hom_A(P,C)\ar[r]
	 & 0
	 }
	\]
	es exacta. Dado un morfismo $\beta\colon P\to C$, como $P$ es proyectivo y
	$g$ es epimorfismo, existe $\alpha\in\Hom_R(P,B)$ tal que
	$g_*(\alpha)=g\circ \alpha=\beta$. Luego $g_*$ es epimorfismo. Recíprocamente, es
	claro que si $g_*$ es un epimorfismo entonces $P$ es proyectivo. 

Veamos ahora que $(1)\implies(3)$. Si $0\to N\to M\to P\to 0$ es una sucesión exacta, entonces se parte, pues como
$P$ es proyectivo 
existe $h\in\Hom_P(P,N)$ tal que el diagrama
	\[
	\xymatrix{ & P\ar@{=}[d]^{\id_P}\ar@{-->}[ld]_h\\ M\ar[r]^g & P\ar[r] & 0	}
	\]
es conmutativo, lo que significa que $g\circ h=\id_P$. Para ver
que $M\simeq N\oplus P$ basta usar la proposición~\ref{pro:split}.

Veamos ahora que $(3)\implies(4)$. Sabemos que $P$ es cociente de un módulo libre, es decir
que existe un conjunto $I$ y un epimorfismo $\varphi\colon R^{(I)}\to P$. Como 
la sucesión
	\[
	\xymatrix{
	 0\ar[r] 
	 & \ker\varphi 
	 \ar[r]
	 & R^{(I)}
	 \ar[r]^-{\varphi}
	 & P\ar[r]
	 & 0
	 }
	\]
de módulos y morfismos 
es exacta, se parte es decir que existe $h\in\Hom_R(P,R^{(I)})$ tal que
$\varphi\circ h=\id_P$. Luego $P$ es sumando directo de un libre, pues
$R^{(I)}\simeq\ker\varphi\oplus P$. 

	Finalmente probemos $(4)\Rightarrow(1)$. Si $f\colon M\to N$ es un epimorfismo y 
	$g\colon P\to N$, como $P$ es sumando directo del libre $R^{(I)}$, definimos
	$\beta\colon P\to M$, $\beta=\alpha\circ i$, donde
	$i\colon P\to R^{(I)}$ es la inclusión. El diagrama 
	\[
	\xymatrix{ & R^{(I)}\ar[d]^p\ar@{-->}[ddl]_{\alpha} \\
	& P\ar[d]^f\ar@{-->}[dl]^{\beta}\\ M\ar[r]_g & N\ar[r] & 0 }
	\]
	donde $p\colon R^{(I)}\to P$ es tal que $p\circ i=\id_P$, es entonces conmutativo, pues 
	\[
		g\circ \beta=(g\circ \alpha)\circ i=f\circ (p\circ i)=f\circ \id_P=f.\qedhere
	\]	
\end{proof}

No todo proyectivo es libre.

\begin{example}
Sea $R=\Z\times\Z$. Como $R$ es libre como $R$-módulo, 
$\Z\times\{0\}$ es proyectivo, pues es sumando directo de un módulo libre. Sin embargo, $\Z\times\{0\}$ no es libre como $R$-módulo. En efecto,
si lo fuera, sea $\{(x_i,0):i\in I\}$ una base de $\Z\times\{0\}$. Entonces
$(1,1)\cdot (x_1,0)=(x_1,0)=(1,0)\cdot (x_1,0)$, una contradicción. 
\end{example}

Cocientes de proyectivos pueden no ser proyectivos. 

\begin{example}
Veamos que el $\Z$-módulo $\Z/n\Z$ no es proyectivo. 
Si $\Z/n\Z$ es proyectivo, entonces, gracias al teorema anterior, 
la sucesión exacta
\[
	\xymatrix{
	 0\ar[r] 
	 & n\Z 
	 \ar[r]^i
	 & \Z
	 \ar[r]^-{\pi}
	 & \Z/n\Z\ar[r]
	 & 0,
	 }
\]
donde $i$ es la inclusión y $\pi$ es el epimorfismo canónico, 
se parte, es decir existe $h\in\Hom(\Z/n,\Z)$ tal que $\pi\circ h=\id_{\Z/n}$, una contradicción
pues sabemos que $\Hom(\Z/n,\Z)=\{0\}$.
% Si $f\in \Hom(\Z/n,\Z)$, entonces 
%\[
%0=f(0)=f(n)=nf(1)\in\Z
%\]
%y luego $f(1)=0$.  	
\end{example}

Submódulos de proyectivos pueden no ser proyectivos. 

\begin{example}
Sea $M=\Z/4$ como $\Z/4$-módulo. Como $M$ es libre, es proyectivo. 
Veamos que el submódulo $N=\{0,2\}$ no es proyectivo. Si $N$ fuera proyectivo, entonces
la sucesión exacta
\[
	\xymatrix{
	 0\ar[r] 
	 & \Z/2
	 \ar[r]^i
	 & \Z/4
	 \ar[r]^-{\pi}
	 & \Z/2
	 \ar[r]
	 & 0,
	 }
\]
debería partirse, lo que implicaría que tendríamos en particular un isomorfismo 
$\Z/4\simeq\Z/2\oplus\Z/2$ de grupos abelianos, una contradicción.
\end{example}

\begin{exercise}
\label{xca:cociente_libre}
Si $N$ es un submódulo de $M$ tal que $M/N$ es libre, entonces $N$ es sumando directo de $M$. 	
\end{exercise}

\index{Idempotente}
Si $R$ es un anillo, $e\in R$ es \textbf{idempotente} si $e^2=e$. Ejemplos triviales de idempotentes son 0 y 1.
Observemos que si $e$ es un idempotente no trivial, entonces 
$I_1=Re$ y $I_2=R(1-e)$ son ideales a izquierda tales que $I_1\cap I_2=\{0\}$ y además 
$R=I_1+I_2$. En efecto, para ver que $R=I_1+I_2$ basta observar que, como 
$1=e+(1-e)$, entonces $r=r1=r(e+(1-e))=re+r(1-e)$ para todo $r\in R$. Para ver que $I_1\cap I_2=\{0\}$ 
sea $x\in I_1\cap I_2$. Entonces $x=re=s(1-e)$ para ciertos $r,s\in R$ y luego
\[
0=s(1-e)e=xe=(re)e=re^2=re=x.   
\]
En particular, por el primer teorema de isomorfismos, 
\[
R\simeq R/(I_1\cap I_2)\simeq R/I_1\times R/I_2.
\]

\begin{example}
Si $e\in \Z/24$ es idempotente, entonces $e\in\{0,1,9,16\}$.  	
\end{example}

\begin{example}
Los idempotentes no triviales de $M_2(\R)$ son las matrices 
de la forma $\begin{pmatrix}a&b\\c&d\end{pmatrix}$ con $a+d=1$ y $ad=bc$.  	
\end{example}

Los idempotentes del anillo están en correspondencia biyectiva con los proyectores
de la representación regular $M=\prescript{}{R}R$. En efecto, si $p$ es un proyector, entonces $e=p(1)$ es un idempotente. Recíprocamente, 
si $e$ es un idempotente del anillo, entonces el morfismo $p\colon M\to M$, $m\mapsto e\cdot m$, es un proyector, pues
\[
p(p(m))=e\cdot (e\cdot m)=e^2\cdot m=e\cdot m=p(m)
\]
para todo $m\in M$.

\begin{example}
Como $0$ y $1$ son los únicos idempotentes del anillo $\Z$, la correspondencia biyectiva entre
idempotentes del anillo y proyectores del módulo 
nos da otra demostración de que los únicos sumandos directos de $\Z$ son $\{0\}$ y $\Z$.
\end{example}

\begin{example}
Si $e\in R$ es idempotente, el ideal a izquierda $Re$ es proyectivo como $R$-módulo. En efecto, la función
$\varphi\colon R\to Re$, $r\mapsto re$, es un epimorfismo de módulos. La sucesión
exacta
\[
	\label{eq:exacta}	
		\xymatrix{
        0\ar[r]
        & \ker\varphi
        \ar[r]
        & R
        \ar[r]^\varphi
        & Re\ar[r]
        & 0,
        }
\]
se parte, pues la inclusión $i\colon Re\to R$ es una sección: $\varphi(i(re))=\varphi(re)=re^2=re$. Luego
$R\simeq\ker\varphi\oplus Re$ y entonces $Re$ es proyectivo por ser sumando directo de un módulo libre. 
\end{example}

\begin{exercise}
\label{xca:I^2}
Sea $R$ un anillo conmutativo y sea $I$ un ideal de $R$. Demuestre que $I$ es un sumando directo de $R$ si y sólo si
existe $u\in R$ tal que $I=(u)=(u^2)$. 	
\end{exercise}

\begin{exercise}
\label{xca:ss_idempotente}
Demuestre que $\prescript{}{R}R$ es semisimple si y sólo si todo ideal a izquierda de $R$ está generado por un idempotente.
\end{exercise}

\begin{proposition}
Sea $M$ un módulo finitamente generado. Entonces $M$ es proyectivo si y sólo si existe $n\in\N$ y $p\in\Hom_R(R^n,R^n)$ tal que
$M\simeq p(R^n)$ y $p^2=p$. 	
\end{proposition}

\begin{proof}
	Supongamos primero que $M$ es proyectivo. Entonces 
	$M$ es sumando directo de $R^n$, digamos $R^n=M\oplus N$ para algún modulo $N$. 
	La función $p\colon R^n\to R^n$, $p(m,n)=m$, es un morfismo 
	de módulos
	tal que $p^2=p$ y $p(R^n)=M$. 
	
% 	todo módulo $M$ es cociente de un libre, 
% 	existen $n\in\N$ y un epimorfismo $\varphi\colon R^n\to M$. Como $M$ es proyectivo, existe una sección 
% 	$s\colon M\to R^n$ y luego $R^n=s(M)\oplus N$ para algún submódulo $N$ de $M$. Sea $p=s\circ\varphi\in\Hom_R(R^n,R^n)$. Entonces
% 	\[
% 	p(R^n)=s(\varphi(R^n))=s(M)\simeq M,
% 	\]
% 	pues $s$ es inyectiva. Además $p^2=s\circ(\varphi\circ s)\circ\varphi=s\circ\varphi=p$.   
	
	Recíprocamente, si existe $p\in\Hom_R(R^n,R^n)$ tal que $p^2=p$ y $p(R^n)\simeq M$, entonces $p\circ (\id-p)=0$. Esto nos
	permite descomponer al anillo $R^n$ en idempotentes ortogonales, es decir
	$R^n\simeq p(R^n)\oplus (\id-p)(R^n)$. Luego $M\simeq p(R^n)$ es proyectivo por ser sumando directo de un libre.  
\end{proof}

\begin{example}
Sea $R=\Z/6=\{0,1,\dots,5\}$ y sea $M=\prescript{}{R}R$. Sabemos que $M$ es libre con base $\{1\}$. Si $I=\{0,2,4\}$ y $J=\{0,3\}$, entonces
$I$ y $J$ son ideales de $R$, es decir que son submódulos de $M$. Si $m\in M$, entonces
$m=-2m+3m\in I+J$. Como además $m\in I\cap J=\{0\}$, se concluye que $M\simeq I\oplus J$. En particular, $I$ y $J$ son proyectivos
como $R$-módulos por ser sumandos directos del módulo libre $M$. Observemos que $I$ está generado por el idempotente
$e=4$ y que $J$ está generado por el idempotente $1-e=3$.
\end{example}


\begin{proposition}
El módulo $S\oplus T$ es proyectivo si y sólo si $S$ y $T$ son proyectivos. 	
\end{proposition}

\begin{proof}
Consideremos el diagrama
	\[
	\xymatrix{ & S\ar[d]^{i_1}\ar@{-->}[ddl]_{h_1} \\
	& S\oplus T\ar[d]^f\ar@{-->}[dl]^{h}\\ M\ar[r]_g & N\ar[r] & 0 }
	\quad
	\]

Veamos primero que si $S$ y $T$ son proyectivos, entonces $S\oplus T$ es proyectivo. Como $S$ y $T$ 
son proyectivos, existen $h_1\colon S\to M$ y $h_2\colon T\to M$ tales que
$g\circ h_1=f\circ i_1$ y $g\circ h_2=f\circ i_2$, donde
$i_1\colon S\to S\oplus T$, $i_1(s)=(s,0)$, $i_2\colon T\to S\oplus T$, $i_2(t)=(0,t)$.  
Si definimos el morfismo 
$h\colon S\oplus T\to M$,
\[
h(s,t)=h_1(s)+h_2(t),
\] 	
entonces tenemos que 
\begin{align*}
g(h(s,t))&=g(h_1(s)+h_2(t))=g(h_1(s))+g(h_2(t))\\
&=f(i_1(s))+f(i_2(t))=f(s,0)+f(0,t)=f(s,t).
\end{align*}
	
Demostremos ahora que si $S\oplus T$ es proyectivo, entonces $S$ es proyectivo. 
Consideramos el diagrama
	\[
	\xymatrix{ & S\oplus T\ar[d]^{p_1}\ar@{-->}[ddl]_{h} \\
	& S\ar[d]^f\ar@{-->}[dl]^{h_1}\\ M\ar[r]_g & N\ar[r] & 0 }
	\quad
	\]

Sea $i_1\colon S\to S\oplus T$, $i_1(s)=(s,0)$.  
Como por hipótesis $S\oplus T$ es proyectivo, existe un morfismo 
$h\colon S\oplus T\to M$ tal que $g\circ h=f\circ p_1$. 
Definimos entonces $h_1\colon S\to M$, $h_1(s)=h(s,0)$ 
y vemos que
\[
g(h_1(s))=g(h(s,0))=f(s).
\]
Similarmente se demuestra que $T$ es también proyectivo.
\end{proof}

Dejamos como ejercicio demostrar 
que $\oplus_{i\in I}M_i$ es proyectivo si y sólo si $M_i$ es proyectivo para todo $i\in\I$. 	

\begin{example}
	Sea $R=\Z/6$. 
	Veamos que todo $R$-módulo es proyectivo. Si $M$ es un $R$-módulo, afirmamos que $M=2M\oplus 3M$, donde
	$2M=\{2m:m\in M\}\subseteq M$ y $3M=\{3m:m\in M\}\subseteq M$, ambos submódulos de $M$. En efecto, $M=2M+3M$, pues $m=-2m+3m\in 2M+3M$. Además 
	$2M\cap 3M=\{0\}$ pues si $m\in 2M\cap 3M$, entonces, como $m=2x=3y$ para ciertos $x,y\in M$, se tiene que
	\[
	0=6x=3(2x)=3m=3(3y)=9y=3y=m.
	\]
	
	Veamos que $2M$ es proyectivo como $R$-módulo. 
	Como $2M$ es un $R$-módulo, tenemos un morfismo de anillos
	$\rho\colon R\to\End(2M)$ con núcleo $\ker\rho=\{0,3\}$. 
	Como $(\Z/6)/\ker\rho\simeq\Z/3$, la propiedad universal del cociente 
	implica  
	que existe un único morfismo $\varphi$ 
	tal que el diagrama 
	\[
        \xymatrix{
        \Z/6
        \ar[d]_g
        \ar[r]^\rho
        & \End(2M)
        \\
        \Z/3
        \ar[d]
        \ar@{-->}[ur]_{\varphi}
        \\
        0
        }
	\]
	donde $g$ es un epimorfismo, 
	conmuta, es decir
	\[
	\rho(k)(2m)=\varphi(g(k))(2m).
	\] 
	Luego $2M$ es un $R$-módulo si y sólo si $2M$ es un $\Z/3$-módulo. 
	Como $\Z/3$ es un cuerpo, $2M$ es
	libre como $\Z/3$-módulo, 
	por ser un espacio vectorial sobre el cuerpo $\Z/3$. En particular, 
	$2M$ es proyectivo como $\Z/3$-módulo. En conclusión, $2M$ es proyectivo
	como $R$-módulo. Análogamente se demuestra que $3M$ es proyectivo. Luego
	$M$ es proyectivo por ser suma directa de proyectivos.  
\end{example}

\begin{example}
Sea $R=\begin{pmatrix}
\R & \R\\
0 & \R
\end{pmatrix}$. Como $R$ es libre como $R$-módulo y además 
\[
R=\begin{pmatrix}
\R & 0\\
0 & 0
\end{pmatrix}
\oplus\begin{pmatrix}
0 & \R\\
0 & \R
\end{pmatrix}
\]
los submódulos $\begin{pmatrix}
\R & 0\\
0 & 0
\end{pmatrix}$ y 
$\begin{pmatrix}
0 & \R\\
0 & \R
\end{pmatrix}$ son proyectivos. 
%El submódulo $\begin{pmatrix}
%\R & \R\\
%0 & 0
%\end{pmatrix}$ no es proyectivo porque
%no admite un complemento en $R$. 
\end{example}

Veamos otro ejemplo de un módulo proyectivo que no es libre. Recordemos
que si $R$ es un anillo conmutativo e $I$ y $J$ son ideales de $R$, entonces
	\[
		I+J=\{x+y\mid x\in I,\;y\in J\}
	\]
es el menor ideal que contiene a $I\cup J$. Además $IJ$ es el ideal formado
por las sumas finitas $\sum_i x_iy_i$, donde $x_i\in I$, $y_i\in J$. 

	\begin{lemma}
		Sea $R$ un dominio íntegro y sean $I$ y $J$ ideales tales que
		$I+J=R$. Consideremos el morfismo de $R$-módulos 
		$g\colon I\times J\to R$, $(x,y)\mapsto x+y$. 
		Entonces:
		\begin{enumerate}
			\item $g$ es sobreyectiva.
			\item $\ker(g)=\{(x,-x)\mid x\in I\cap J\}$.
			\item $I\times J\simeq (I\cap J)\oplus R$ como $R$-módulos.
			\item Si $I\cap J$ es principal entonces $I$ y $J$ son $R$-módulos
				proyectivos.
		\end{enumerate}
	\end{lemma}

	\begin{proof}
		Las primeras dos afirmaciones son evidentes. 
		
		Para demostrar (3) consideremos la sucesión exacta
		\[
		\xymatrix{
		0\ar[r] 
		& I\cap J
		\ar[r]^-{f}
		& I\times J
		\ar[r]^-{g}
		& R\ar[r]
		& 0
		}
		\]
		donde $f(x)=(x,-x)$. Como $R$ es libre como $R$-módulo, $R$ es proyectivo 
		y entonces la sucesión se parte, es decir $I\oplus J\simeq 
		I\times J\simeq (I\cap J)\oplus R$
		como $R$-módulos. 
		
		Para demostrar (4), supongamos que $I\cap
		J=(x)$. Si $x=0$, como $I\cap J=\{0\}$, el 
		ítem nos dice que $I\oplus J\simeq I\times J\simeq R$ y 
		luego $I$ y $J$ son proyectivos por ser sumandos directos de un libre.  
		Si $x\ne 0$, entonces, como $R$ es un dominio íntegro, $R\to I\cap J$, $r\mapsto rx$, es un isomorfismo de
		$R$-módulos. Luego 
		$I\oplus J\simeq I\times J\simeq (I\cap J)\oplus R\simeq R\oplus R$ y entonces 
		$I$ y $J$ son proyectivos por ser sumandos directos de
		un libre.
	\end{proof}

Veamos una aplicación concreta del lema anterior.  

\begin{example}
	Sea $R=\Z[\sqrt{-5}]$ y consideremos los ideales 
	\[
		I=(3,1+\sqrt{-5}),\quad 
		J=(3,1-\sqrt{-5}).
	\]
	Vamos a demostrar lo siguiente:
	\begin{enumerate}
		\item $I$ no es principal.
		\item $I$ no es libre como $R$-módulo.
		\item $I$ es proyectivo.
		\item $R/I\simeq\Z_3$ y luego $I$ es maximal.
	\end{enumerate}
	Valen además las afirmaciones análogas para $J$.

	Demostremos la primera afirmación. Si $I$ es principal, digamos $I=(a+b\sqrt{-5})$ entonces,
	como $3\in I$, existen $c,d\in\Z$ tales que
	$3=(a+b\sqrt{-5})(c+\sqrt{-5}d)$.  Si aplicamos la función multiplicativa 
	$N(a+b\sqrt{-5})=a^2+5b^2$,
	\[
	9=(a^2+5b^2)(c^2+5d^2)
	\]
	y entonces $a^2+5b^2\in\{1,3,9\}$. Como $I\ne R$,  
	$a^2+5b^2\geq5$ y luego $a^2+5b^2=9$. En particular,
	\[
	(a,b)\in\{(-3,0),(3,0),(2,1),(2,-1),(-2,-1),(-2,1)\}
	\]
	Vimos que los casos $(a,b)\in\{(-3,0),(3,0)\}$ no son posibles. Para el resto de los posibles valores de $(a,b)$ 
	se tiene que 
	$9=9(c^2+5d^2)$ y luego $c^2+5d^2=1$, lo que implica que $c\in\{-1,1\}$ y entonces 
	$3=a+b\sqrt{-5}$ o bien $3=-a-b\sqrt{-5}$, una contradicción. 

	Demostremos la segunda afirmación. Supongamos que $I$ es libre como $R$-módulo. Como 
	$R$ es numberable, sea $\{x_i\mid i\in\N\}$ es una base de $I$. Tenemos entonces
	que $|I|=1$, pues de lo contrario, si $|I|\geq2$, como $x_1(-x_2)+x_2x_1=0$, se
	tendría que el conjunto $\{x_i\mid i\in I\}$ es linealmente
	dependiente. Tenemos entonces que $|I|=1$, es decir $I$ es principal, una contradicción. 
	
	Demostremos ahora la tercera afirmación. Primero observemos que $I+J=R$ pues $I+J$ es un ideal
	y además $1 = (3-(1+\sqrt{-5})) - (1-\sqrt{-5})\in I+J$. Probemos ahora $I\cap J$ es
	principal, más precisamente $I\cap J=(3)$. Para esto primero observemos que,
	en general, vale la fórmula
	\[
		(I+J)(I\cap J)\subseteq IJ
	\]
	y entonces, como $I+J=R$, se tiene que $I\cap J=IJ$. Todo elemento del ideal $IJ$ 
	es una suma finita $\sum_i x_iy_i$, donde $x_i\in I$, $y_i\in J$. Como
	\begin{align*}
		\sum (3u+(1&+\sqrt{-5})v(3u'+(1-\sqrt{-5})v'\\
				&=3\sum (3uu'+uv'(1-\sqrt{-5})+u'v(1+\sqrt{-5})+2vv'),
	\end{align*}
	es claro que $IJ\subseteq (3)$. La otra inclusión es evidente. El lema
	anterior implica que entonces $I$ y $J$ son ambos proyectivos.

	Probemos ahora la última afirmación. Consideremos la inclusión $i\colon
	\Z\to\Z[\sqrt{-5}]$ y el epimorfismo canónico $p\colon
	\Z[\sqrt{-5}]\to\Z[\sqrt{-5}]/I$. Entonces $f=p\circ i$ es un morfismo sobreyectivo 
	tal que $\ker(p\circ i)=(3)$ pues si $m\in\ker(p\circ i)$ entonces 
	\begin{align*}
		m&=3(a+b\sqrt{-5})+(1+\sqrt{-5})(c+d\sqrt{-5})\\
		&=(3a+c-5d)+(3b+d+c)\sqrt{-5}.
	\end{align*}
	Luego $m=3a-3b-6d\in3\Z$. La inclusión $3\Z\subseteq\ker(p\circ i)$ es trivial.
	Para ver que $f$ es sobreyectivo basta observar que $f(a-b)=a+b\sqrt{-5}$, pues
    \[
    f(a-b)=p(a-b)=(a-b)+I=(a+b\sqrt{-5})+I
    \]
    ya que $(a-b)-(a+b\sqrt{-5})=-b(1+\sqrt{-5})\in I$. 
	Por el teorema de isomorfismos, $\Z/3\Z\simeq p(i(\Z))$ y
	luego $I$ es un ideal maximal. 
	\end{example}

\begin{exercise}
\label{xca:proyectivo1}
	Sean $P$ y $P_1$ dos módulos proyectivos. Consideremos el diagrama
	\begin{align*}
		&\xymatrix{
		0\ar[r] 
		& K
		\ar[r]^-{f}
		& P
		\ar[r]^-{g}
		& M\ar[r]\ar@{=}[d]
		& 0\\
		0\ar[r] 
		& K_1
		\ar[r]^-{f_1}
		& P_1
		\ar[r]^-{g_1}
		& M\ar[r]
		& 0
		}
	\end{align*}
	con filas exactas. Demuestre que $P\oplus K_1\simeq P_1\oplus K$.
\end{exercise}


\chapter{El teorema de estructura}

En este capítulo $R$ será un dominio de ideales principales. Esta hipótesis hace
que los módulos tengan buenas propiedades, casi como pasa en el caso de espacios vectoriales.

\begin{theorem}
\label{thm:rango}
Sea $R$ un dominio de ideales principales. Si $F$ es un módulo libre finitamente generado 
y $N$ es un submódulo de $F$, entonces $N$ es también libre y además 
$\rank(N)\leq\rank(F)$. 
\end{theorem}

\begin{proof}
	Procederemos por inducción en $n=\rank(F)$. Si $n=1$, entonces
	$F\simeq\prescript{}{R}R$ 
	y los submódulos de $F$ son exactamente los ideales a izquierda de $R$. En particular,
	$N=(r)$ para algún $r\in R$. Si $r=0$, entonces $N=\{0\}$ y el resultado es verdadero. Si $r\ne 0$, entonces $\{r\}$ es base de $N$ y también el resultado es cierto. 
	
	Supongamos ahora que el resultado es válido para todo módulo libre de rango $<n$. 
	Sea $\{f_1,\dots,f_n\}$ una
	base de $F$ y sea $F_n=(f_1,\dots,f_{n-1})$. Por hipótesis inductiva, 
	el submódulo $U=N\cap F_n$ es libre de rango $\leq n-1$. Sea
	$\{n_1,\dots,n_k\}$ una base de $U$ (por convención, si $U=\{0\}$, entonces $k=0$). Si $f\in F$, existen únicos $r_1,\dots,r_n\in R$ tales que 
	\[
	f=\sum_{i=1}^n r_i\cdot f_i.
	\]
	Queda definido 
	entonces un morfismo 
	\[
	\varphi\colon F\to R,
	\quad
	\sum_{i=1}^nr_i\cdot f_i\mapsto r_n.
	\] 
	Si $\varphi(N)=\{0\}$, entonces $N\subseteq (f_1,\dots,f_{n-1})$ y luego $N=U$. 
	Si $\varphi(N)\ne\{0\}$, entonces $\varphi(N)$ es un ideal
	de $R$, digamos $\varphi(N)=(x)$ para algún $x\in R\setminus\{0\}$. Sea $n_{k+1}\in N$ 
	tal que $\varphi(n_{k+1})=x$. 
	Veamos que $\{n_1,\dots,n_k,n_{k+1}\}$ es base de $N$. 
	Si $n\in N$, entonces $\varphi(n)=rx$ 
	para algún $r\in R$. Entonces $n-r\cdot n_{k+1}\in N\cap\ker\varphi=N\cap F_n=U$ pues 
	$\varphi(n-r\cdot n_{k+1})=0$. En particular, 
	\[
	n-r\cdot n_{k+1}\in (n_1,\dots,n_k)\implies  
	n\in (n_1,\dots,n_k,n_{k+1}).
	\]
	Veamos ahora que
	$\{n_1,\dots,n_k,n_{k+1}\}$ es linealmente independiente. Si 
	\[
	0=\sum_{i=1}^{k+1}r_i\cdot n_i,
	\]
	para $r_1,\dots,r_{k+1}\in R$, entonces, como $\varphi(n_i)=0$ para todo $i\in\{1,\dots,k\}$, se tiene que 
	\[
	0=\varphi(r_{k+1}\cdot n_{k+1})=r_{k+1}x,
	\]
	que implica $r_{k+1}=0$. Queda entonces que $\sum_{i=1}^kr_i\cdot n_i=0$. Como $\{n_1,\dots,n_k\}$ es base de $U$, se concluye que
	$r_1=\dots=r_k=0$. 
\end{proof}

El teorema anterior también vale en el caso de bases infinitas. La demostración, sin embargo, depende del lema de Zorn.
% todo: referencia al libro de lang
	
\begin{corollary}
Sea $R$ un dominio de ideales principales. Si $M$ es proyectivo y finitamente generado, 
entonces $M$ es libre.
\end{corollary}

\begin{proof}
Supongamos que $M=(m_1,\dots,m_k)$. Sabemos que $M$ es sumando directo de un libre $F$. Fijemos una base de $F$ y sea 
$X=\{f_1,f_2,\dots\}$ un subconjunto finito de esa base de $F$ tal que
\[
m_j=\sum_{i=1}^{n_j}r_{ij}\cdot f_i
\]
para ciertos $r_{ij}\in R$ y ciertos $n_1\dots,n_k\in\N$. 
Por construcción, $X$ es linealmente indepdendiente y $M=(X)$. 
\end{proof}

El corolario anterior vale también para módulos arbitrarios. La demostración puede
consultarse por ejemplo en~\cite[I, Theorem 5.1]{MR1438546}.

\begin{corollary}
Sea $R$ un dominio de ideales principales. Si $M$ es finitamente generado y $N$ 
es un submódulo de $M$, entonces $N$ es también finitamente generado.  
\end{corollary}

\begin{proof}
Sabemos que existe un módulo libre $F$ de rango finito y un epimorfismo 
$\varphi\colon F\to M$. Como $N_1=\varphi^{-1}(N)$ es un submódulo de $F$, el teorema anterior
implica que 
$\rank(N_1)\leq\rank F$. Si $\{x_1,\dots,x_k\}$ es base de $N_1$, entonces, como $\varphi$ es un epimorfismo, $\{\varphi(x_1),\dots,\varphi(x_k)\}$ es un conjunto de generadores de 
$\varphi(N_1)=\varphi(\varphi^{-1}(N))=N$. En particular, $\rank(N)\leq k=\rank(N_1)\leq\rank(F)$. 
\end{proof}

Algunos ejercicios útiles:

\begin{exercise}
\label{xca:rank}
    Sea $R$ un dominio de ideales principales  
	y sea $M$ un módulo libre. Si 
	$S$ es un submódulo de $M$ tal que $M/S$ es libre, entonces $M\simeq S\oplus (M/S)$. Si además 
	$S$ es libre, entonces 
	\[
	\rank(M)=\rank(S)+\rank(M/S).
	\] 
\end{exercise}

\begin{exercise}
\label{xca:n_elements}
Sea $R$ un dominio de ideales principales. 
Si $M$ es un módulo libre de rango $n$, entonces todo conjunto linealmente independiente de $M$ tiene
a lo sumo $n$ elementos. 
\end{exercise}

\begin{exercise}
Sea $R$ un dominio de ideales principales 
y sean $M$ y $N$ dos módulos libres. Demuestre que $M\simeq N$ si y sólo si $\rank(M)=\rank(N)$. 	
\end{exercise}

\begin{exercise}
\label{xca:base}
Sea $R$ un dominio de ideales principales. 
Si $M$ es un módulo libre tal que $n=\rank(M)$ y $\{s_1,\dots,s_n\}$ es un conjunto de generadores, entonces $\{s_1,\dots,s_n\}$ es una base de $M$.
\end{exercise}

\index{Anulador!de un módulo}
\index{Anulador!de un elemento del anillo}
Si $M$ es un módulo, el \textbf{anulador} de $M$ se define como
\[
\Ann(M)=\{r\in R:r\cdot m=0\text{ para todo $m\in M$}\}.
\] 
Si $m\in M$ se define 
\[
\Ann(m)=\{r\in R:r\cdot m=0\}.
\]  
Observar que $\Ann(M)=\cap_{m\in M}\Ann(m)$. 
Dejamos como ejercicio demostrar que $\Ann(M)$ y $\Ann(m)$ son ideales de $R$. 
Si $r\in R$, el anulador de $r$ en $M$ se define como
\[
\Ann_M(r)=\{m\in M:r\cdot m=0\}.
\] 
Dejamos como ejercicio demostrar que $\Ann_M(r)$ es un submódulo de $M$. 

\begin{exercise}
Sean $M$ un módulo, $m\in M$ y $r\in R$ tal que $\Ann(m)=(r)$. Sea 
$p\in R$ irreducible. 
\begin{enumerate}
\item Si $p$ divide a $r$, entonces $(m)/p\cdot (m)\simeq R/(p)$.
\item Si $p$ no divide a $r$, entonces $p\cdot (m)=(m)$.
\end{enumerate}
\end{exercise}

% Demostremos la primera afirmación. Sea $\varphi\colon R\to (m)$, $s\mapsto s\cdot m$, y 
% sea $f=\pi\circ \varphi\colon R\to (m)/p\cdot (m)$, donde $\pi\colon (m)\to (m)/p\cdot (m)$ es 
% el epimorfismo canónico. Veamos que $\ker f=(p)$. Trivialmente vale 
% que $\ker f\supseteq (p)$. Por otro lado, si $s\in R$ es tal que
% s\cdot m\in p\cdot (m)$, entonces, como $r=pt$ para algún $t\in R$, $s\cdot m=t\cdot (p\cdot m)$.
% Como $f$ es epimorfismo por ser composición
% de epimorfismos, se tiene que $R/(p)\simeq (m)/p\cdot (m)$.  

% Demostremos ahora la segunda afirmación. Como $p$ y $r$ son coprimos, existen 
% $a,b\in R$ tales que $ap+br=1$. Como $r\cdot \m=0$ pues $\Ann(m)=(r)$, 
% $m=1\cdot m=(ap+br)\cdot m=p\cdot (a\cdot m)\in p\cdot (m)$.  

\index{Torsión!de un módulo}
\index{Módulo!sin torsión}
\index{Módulo!de torsión}
La \textbf{torsión} de un módulo $M$ se define como
el subconjunto 
\[
T(M)=\{m\in M:r\cdot m=0\text{ para algún $r\in R$}\}.
\]
Queda como ejercicio demostrar que $T(M)$ es un submódulo de $M$. 
Se dice que un módulo $M$ 
\textbf{no tiene torsión} si $T(M)=\{0\}$ y que es \textbf{de torsión} si $T(M)=M$.   

\begin{exercise}
Si $M\simeq N$, entonces $T(M)\simeq T(N)$.
\end{exercise}

\begin{exercise}
Demuestre que $T(\oplus_{i\in I}M_i)\simeq \oplus_{i\in I}T(M_i)$.
\end{exercise}

\begin{exercise}
\label{xca:free}
    Sean $R$ un dominio principal y $M$ un $R$-módulo. Pruebe que si $M$ es
    finitamente generado y $S\subseteq M$ es un submódulo libre tal que $M/S$ no
    tiene torsión, entonces $M$ es libre
\end{exercise}

\begin{examples}\
\begin{enumerate}
\item $T(\prescript{}{R}R)=\{r\in R:rs=0\text{ para algún $s\in R$}\}$.
\item $T(\Z/n)=\Z/n$.
\item $T(\Q)=\{0\}$.
\item $T(\Z^{\N})=\{0\}$. 
\end{enumerate}
\end{examples}

\begin{example}
Si $V$ es un espacio vectorial de dimensión finita y $T\colon V\to V$ es una
transformación lineal, sabemos que $V$ es un $K[X]$-módulo con la acción
\[
\left(\sum_{i=0}^n a_iX^i\right)\cdot v=\sum_{i=0}^n a_iT^i(v).
\]

Veamos que
$V$ es de torsión, es decir $V=T(V)$. Sea $n=\dim V$. Si $v\in V$, entonces 
$\{v,T(v),\dots,T^n(v)\}$ es linealmente dependiente, pues
tiene $n+1$ elementos. En particular, existen $a_0,\dots,a_n\in K$ no todos cero tales
que 
\[
0=\sum_{i=0}^n a_iT^i(v)=\left(\sum_{i=0}^n a_iX^i\right)\cdot v.
\]
Luego $v\in T(V)$. 
\end{example}

\begin{theorem}
Sea $R$ un dominio de ideales principales  
y sea $M$ finitamente generado. Si $T(M)=\{0\}$, entonces $M$ es libre. 
\end{theorem}

\begin{proof}
Sin perder generalidad podemos suponer que $M$ es no nulo. Supongamos además que $M=(X)$, donde $X$ es un conjunto 
finito de generadores. Si $x\in X$, entonces $r\cdot x=0\Longleftrightarrow r=0$, pues $T(M)=\{0\}$. Sea $S=\{x_1,\dots,x_k\}\subseteq X$ 
maximal con respecto a la siguiente propiedad:
\[
r_1\cdot x_1+\cdots+r_k\cdot x_k=0\text{ para $r_1,\dots,r_k\in R$}\implies r_1=\cdots=r_k=0.
\]	
Sea $F=(S)$ el módulo libre con base en el conjunto $S$. Si $X=S$, no hay nada para demostrar. 
Si $y\in X\setminus S$, entonces
existen $r_y,r_1,\dots,r_k\in R$ no todos cero tales que
\[
r_y\cdot y+\sum_{i=1}^k r_i\cdot x_i=0.
\]
Como entonces $r_y\cdot y=-\sum_{i=1}^k r_i\cdot x_i\in F$, se concluye que $r_y\ne 0$, pues si 
$r_y=0$ entonces $r_1=\cdots=r_k=0$. Como $X$ es finito, 
\[
r=\prod_{y\in X\setminus S}r_y
\]
está bien definido pues $R$ es conmutativo y es tal que $r\cdot X\subseteq F$. Si $f\colon M\to M$, $x\mapsto r\cdot x$, entonces
$f$ es un morfismo tal que $f(M)=r\cdot M$. Como $T(M)=\{0\}$, entonces $\ker f=\{0\}$. Luego
\[
r\cdot M=f(M)\simeq M
\]
y en consecuencia $M$ es libre. 
\end{proof}

%\begin{theorem}
%Si $M$ es libre y finitamente generado y $N$ es un submódulo de $M$, entonces $N$ es también libre.
%\end{theorem}

%\begin{proof}
%Sea $\{m_1,\dots,m_k\}$ una base de $M$. Procederemos por inducción en $k$. Para cada $j\in\{1,\dots,k\}$ 
%sea $M_j=M\cap (m_1,\dots,m_j)$. El caso $k=1$ es fácil: 
%como $M_1=M\cap (m_1)\subseteq (m_1)$, existe $r_1\in R$ tal que $M_1=(r_1\cdot m_1)$. Luego $M_1=\{0\}$ o bien
%$M_1$ es libre de rango uno. 
%%Supongamos ahora que $M_j$ es libre de rango $\leq j$. Sea 
%%\[
%%I=\{r\in R:\text{ existe $m\in M$ tal que $m=\sum_{i=1}^j s_j\cdot m_j+r\cdot m_{j+1}$ para $s_1,\dots,s_j\in R$}\}.
%%\]   
%%Como $I$ es un ideal de $R$, podemos escribir $I=(r_{j+1})$, pues $R$ es un dominio de ideales principales. Si $r_{j+1}=0$, entonces
%%$M_{j+1}=M_j$ y el teorema queda demostrado. Si $r_{j+1}\ne 0$, sea $n\in M_{j+1}$ un elemento de la forma
%%\[
%%n=r_{j+1}\cdot m_{j+1}+\cdots, 
%%\]
%%donde usamos la definición del ideal $I$. Si $m\in M_{j+1}$, entonces $m=r\cdot x_{j+1}+\cdots$, donde
%%$r_{j+1}$ divide a $r$, pues como $m\in M$, entonces $r\in I=(a_{j+1})$ y luego $r=sr_{j+1}$ para algún $s\in S$. Como entonces
%%$m-s\cdot n\in M_j=M\cap (m_1,\dots,m_j)$, se concluye que
%%$M_{j+1}=M_j+(n)$. Además $M_j\cap (n)=\{0\}$, pues si $x=r\cdot y=\sum_{i=1}^j r_i\cdot m_i$, entonces
%%\[
%%(rr_{j+1})\cdot m_{j+1}+\sum_{i=1} s_i\cdot m_i=0
%%\]
%%implica que $rr_{j+1}=s_1=\cdots s_j=0$ y luego $r=0$ pues $a_{j+1}\ne 0$. 
%%\end{proof}
%%
%%El teorema anterior también vale en el caso de bases infinitas. La demostración, sin embargo, depende del lema de Zorn.
% todo: referencia al libro de lang

\begin{theorem}
Sea $R$ un dominio de ideales principales. 
Si $M$ es un módulo finitamente generado, entonces $M=T(M)\oplus F$ con $F\simeq M/T(M)$ libre y finitamente generado. La
parte de torsión es única y la parte libre es única salvo isomorfismos.
\end{theorem}

\begin{proof}
	Veamos que $T(M/T(M))\simeq\{0\}$. Si $x+T(M)\in T(M/T(M))$, sea $r\in R\setminus\{0\}$ tal que
	$r\cdot (x+T(M))=T(M)$. Entonces $r\cdot x\in T(M)$, es decir que existe $s\in R\setminus\{0\}$ tal que $s\cdot (r\cdot x)=(sr)\cdot x=0$. 
	Como $sr\ne 0$, se concluye que $x\in T(M)$. 
	
	Como $M$ es finitamente generado, el cociente $M/T(M)$ es también finitamente generado. Como además $M/T(M)$ no tiene torsión, 
	se concluye que $M/T(M)$ es libre. 
	Como entonces $M/T(M)$ es proyectivo, la sucesión exacta 
	\[
	\xymatrix{
	 0\ar[r] 
	 & T(M)
	 \ar[r]^i
	 & M
	 \ar[r]^{\pi}
	 & M/T(M)
	 \ar[r]
	 & 0
	 }
	 \]
	 donde $i$ es la inclusión y $\pi$ es el morfismo canónico, 	el teorema~\ref{thm:proyectivo} 
	 implica que $M\simeq T(M)\oplus M/T(M)$. 	
	 
	 Para demostrar la unicidad, supongamos que $M=T\oplus L$, donde $T$ es de torsión y $L$ es libre. Primero
	 demostraremos que $T=T(M)$. Claramente 
	 $T\subseteq T(M)$. Por otro lado, si $m\in T(M)$, entonces $m=t+l$ para ciertos $t\in T$ y $l\in L$. En particular, 
	 $r\cdot m=0$ y $s\cdot t=0$ para ciertos $r,s\in R$. Como $R$ es conmutativo, 
	 \[
	 0=(rs)\cdot m=(rs)\cdot (t+l)=(rs)\cdot t+(rs)\cdot l=(rs)\cdot l.
	 \]
	 y entonces $l\in T(L)$. Pero 
	 como $L$ es libre, $T(L)=\{0\}$ (pues $T(L)$ es también libre y entonces cualquier elemento $x$  
	 de una base de $T(L)$ sería tal que $\{x\}$ es linealmente independiente, una contradicción). Luego 
	 $l=0$ y en consecuencia $m=t\in T$. La unicidad salvo isomorfismo de la parte libre
	 se obtiene inmediatamente al observar que 
	 ese submódulo es isomorfo a $M/T(M)$. 
\end{proof}


\index{compontente $p$-primaria!de un módulo}
Si $p\in R$ es un elemento irreducible, se define la componente $p$-primaria de $M$ como
el subconjunto 
\[
M_p=\{x\in M:p^l\cdot x=0\text{ para algún $l\in\N$}\}.
\]
Dejamos como ejercicio demostrar que $M_p$ es un submódulo de $M$. 

\begin{lemma}
	\label{lem:Mp}
	Sean $p,t\in R$ tales que $p$ y $t$ son coprimos. Si $p\in R$ es irreducible y además   
	$\Ann(M)=(p^kt)$, entonces $\Ann_M(p^k)=M_p$. 
\end{lemma}

\begin{proof}
Demostremos que $\Ann_M(p^k)\supseteq M_p$, ya que la otra inclusión es trivial. 
Si $x\in M_p$, entonces $p^k\cdot x=0$ para algún $l\in\N$. Observemos que 
$(p^kt)\cdot x=0$, pues $x\in M$. Como $p^l$ y $t$ son coprimos, existen 
$r,s\in R$ tales que $1=rp^l+st$ y luego
\[
p^k\cdot x=1\cdot (p^k\cdot x)=(rp^l+st)\cdot (p^k\cdot x)=rp^{l+k}\cdot x+(st)\cdot (p^k\cdot x)=0.\qedhere	
\]
\end{proof}

\begin{lemma} 
\label{lem:Ann(ab)}
Si $\Ann(M)=(ab)$ y $a$ y $b$ son coprimos, entonces 
\[
M=\Ann_M(a)\oplus\Ann_M(b).
\] 	
\end{lemma}

\begin{proof}
	Como $a$ y $b$ son coprimos, existen $r,s\in R$ tales que $1=ra+sb$. Si $m\in M$, entonces
	\[
	m=1\cdot m=(ra+sb)\cdot m=(ra)\cdot m+(sb)\cdot m\in\Ann_M(b)+\Ann_M(a),
	\]
	pues $(ab)\cdot M=0$. Además, si $m\in\Ann_M(a)\cap\Ann_M(b)$, entonces 
	\[
	m=1\cdot m=(ra+sb)\cdot m=0.\qedhere
	\]
\end{proof}

Los dos lemas que vimos nos permiten demostrar el siguiente resultado:

\begin{theorem}[de la descomposición primaria]
\index{Teorema!de la descomposición primaria}
Sea $R$ un dominio de ideales principales y 
sean $p_1,\dots,p_k\in R$ irreducibles distintos y no asociados tales que $\Ann(M)=(p_1^{\alpha_1}\cdots p_k^{\alpha_k})$. 
Si $M$ es finitamente generado y $T(M)=M$, entonces 
\[
M=\bigoplus_{i=1}^kM_{p_i}.
\]   
\end{theorem}

\begin{proof}
	Como $T(M)=M$, $\Ann(M)\ne\{0\}$. En efecto, si $M=(m_1,\dots,m_l)$, entonces cada $m_i\in T(M)$ y 
	entonces existe $r_i\in R\setminus\{0\}$ tal que $r_i\cdot m_i=0$ para todo $i$. Como $R$ es un dominio, $r_1\cdots r_l\ne 0$ 
	y cumple $(r_1\cdots r_l)\cdot m_i=0$ para todo $i\in\{1,\dots,l\}$, es decir $r_1\cdots r_l\in\Ann M$. Los dos lemas
	que vimos anteriormente nos permiten escribir entonces
	\[
	M=\bigoplus_{i=1}^k\Ann_M(p_i^{\alpha_i})=\bigoplus_{i=1}^k M_{p_i}.\qedhere
	\] 
\end{proof}
 
Puede demostrarse que la descomposición mencionada en el teorema anterior es única salvo en el orden de los sumandos. 
Más precisamente, si \[
M=N_{q_1}\oplus\cdots\oplus N_{q_l},
\]
donde los $N_{q_i}$ son primarios tales que 
$\Ann(N_{q_i})=(q_i^{\beta_i})$ 
y los $q_1,\dots,q_l$ son irreducibles distintos no asociados, entonces $k=l$ y existe $\sigma\in\Sym_k$ tal que
$M_{p_i}=N_{q_{\sigma(i)}}$ para todo $i$, los $q_{\sigma(i)}$ y los $p_i$ son asociados para todo $i$ y además 
\[
q_1^{\beta_1}\cdots q_k^{\beta_k}=p_1^{\alpha_1}\cdots p_k^{\alpha_k}.
\]
Para la demostración ver por ejemplo~\cite[Theorem 6.9]{MR2344656}.

% \medskip
% \index{Orden!de un submódulo}
% Para el teorema siguiente necesitamos algunas nociones básicas sobre órdenes de submódulos. Un módulo 
% $M$ se dirá de \textbf{orden} $r\in R$ si $\Ann(M)=(r)$. 

% \begin{lemma}
% Sean $R$ un dominio de ideales principales y $M$ un módulo.
% \begin{enumerate}
%     \item Si $m\in M$ es tal que $\Ann(m)=(r)$, entonces $(m)\simeq R/(r)$.
%     \item Los submódulos de módulos cíclicos son cíclicos.
%     \item Si $m\in M$ es tal que $\Ann(m)=(r)$, entonces $\Ann(s\cdot m)=r/\gcd(r,s)$. 
%     En particular, si $r$ y $s$ son coprimos, entonces $\Ann(s\cdot m)=(s)$. 
%     \item Si $m_1,\dots,m_k\in M\setminus\{0\}$ son tales que $\Ann(m_i)=(r_i)$ y los $r_i$ son coprimos dos a dos, 
%     entonces $\Ann(m_1+\cdots+m_k)=(r_1\cdots r_k)$. 
% \end{enumerate}
% \end{lemma}

% \begin{proof}
%     Para el primer ítem hay que usar el primer teorema de isomorfismos con el epimorfismo $R\to (m)$, $r\mapsto r\cdot m$, 
%     que tiene núcleo $\Ann(M)=(r)$. 
    
%     El segundo ítem es consecuencia del teorema \ref{thm:rango} que afirma que un submódulo de un cíclico estará generado 
%     por un elemento. 
    
%     Demostremos el tercer ítem. 
% \end{proof}


% El lema anterior implica que si $M=S_1+\cdots + S_k$ y cada $\Ann(S_i)=(s_i)$ donde los $s_i$ son coprimos dos a dos, entonces
% $M=S_1\oplus\cdots\oplus S_k$. 

%\begin{exercise}
%	Si en el teorema anterior, $M$ admite además una descomposición de la forma $M=N_{q_1}\oplus\cdots\oplus N_{q_l}$,
%	donde $\Ann(N_{q_i})=(q_i^{\beta_i})$ para todo $i\in\{1,\dots,l\}$ y los $q_i$ son irreducibles distintos no asociados, entonces
%	$k=l$; más aún, después de reordenar,
%	tenemos que para todo $i\in\{1,\dots,k\}$ vale que 
%$N_{q_i}=M_{q_i}$, $\alpha_i=\beta_i$ y que $p_i$ y $q_i$ son asociados.
%\end{exercise} 
%

\begin{theorem}[de la descomposición cíclica]
\index{Teorema!de la descomposición cíclica}
Sea $R$ un dominio de ideales principales. 
Sea $M$ un módulo finitamente generado tal que $T(M)=M$. Si $\Ann(M)=(p^{\alpha})$ para algún $p\in R$ irreducible, entonces
\[
M=(m_1)\oplus\cdots\oplus (m_k),
%\bigoplus_{i=1}^k R/(p^{\alpha_i}),
\]
donde cada $m_j\in M$ es tal que $\Ann(m_j)=(p^{\alpha_j})$ y 
$\alpha=\alpha_1\geq\alpha_2\geq\cdots\geq\alpha_k$.
\end{theorem}

\begin{proof}
  Sea $m\in M$ tal que $p^{\alpha}\cdot m=0$ y $p^{\alpha-1}\cdot m\ne 0$. Si $M=(m)$, no hay nada para demostrar. En caso contrario,
   vamos a demostrar que existe un submódulo $S$ de $M$ tal que 
	$M=(m)\oplus S$. En este caso, como $S$ también cumple las hipótesis del teorema,
	podemos repetir el procedimiento y obtener
	\[
	M=(m)\oplus (m_1)\oplus\cdots\oplus (m_k)\oplus S_k,
	\]
	donde $\Ann(m_i)=(p^{\alpha_i})$. Como $M$ es noetheriano, la sucesión 
	\[
	(m)\subseteq (m,m_1)\subseteq (m,m_1,m_2)\subseteq\cdots
	\]
	debe estabilizarse, lo que se traduce en la existencia de algún $k\in\N$ tal que $S_k=\{0\}$. 
	
	Sea $M_1=(m)$. Si
	$M_1\ne M$, sea $x\in M\setminus M_1$. Necesitamos encontrar $r\in R$ tal que $S_2=(x-r\cdot m)$ esté en suma directa con $(m)$, es decir 
	$(m)\cap S_2=\{0\}$. Si $y\in (m)\cap S_2$, 
	\begin{gather*}
	y=a\cdot m=b\cdot (x-r\cdot m)\implies 
	b\cdot x=(a+br)\cdot m\in (m)
	\end{gather*}
	para ciertos $a,b\in R$. 
	Un cálculo sencillo nos muestra que el conjunto  
	\[
	I=\{\lambda\in R:\lambda\cdot x\in (m)\}.
	\] 
	es un ideal de $R$ tal que $p^{\alpha}\in I$ y $b\in I$. 
	Como $R$ es principal y $p$ es irreducible,
	entonces $I=(p^\beta)$ para algún $\beta\leq \alpha$. Si $\beta=0$, entonces $I=R$ y luego $x=1\cdot x\in (m)$, una contradicción. Luego $\beta\ne0$. Como
	$p^\beta\in I$, entonces $p^\beta\cdot x=c\cdot m$ para algún $c\in R$. En particular,
	\[
	0=p^{\alpha}\cdot x=p^{\alpha-\beta}\cdot (p^\beta\cdot x)
	=(p^{\alpha-\beta}c)\cdot m,
	\]
	lo que implica que $p^{\alpha-\beta} c\in\Ann(m)=(p^{\alpha})$ y entonces  
	$c=dp^{\beta}$ para algún $d\in R$. En particular, 
	\[
	p^\beta\cdot x=c\cdot m=(dp^{\beta})\cdot m.
	\] 
	Por definición $b\in I$ y entonces $b=ep^{\beta}$ para algún $e\in R$. Luego 
	\[
	b\cdot x=(ep^\beta)\cdot x=(edp^{\beta})\cdot m,
	\]
	lo que implica que si $r=d\in R$, entonces 
	\[
	y=b\cdot (x-d\cdot m)=b\cdot x-b\cdot (d\cdot m)=0.
	\] 
	
	La contrucción general es similar. Si $M_k=(m)\oplus S_k$ es tal que $M_k\ne M$, afirmamos que existe un submódulo $S_{k+1}$ de $M$ tal que
	$S_k\subsetneq S_{k+1}$ y además $(m)\cap S_{k+1}=\{0\}$. En efecto, como $M\ne M_k$, existe $x\in M\setminus M_k$. 
	Queremos encontrar $r\in R$ tal que  $(m)\cap (S_k,x-r\cdot m)=\{0\}$. Sea 
	$S_{k+1}=(S_k,x-r\cdot m)$. Si $y\in (m)\cap S_{k+1}$, entonces existen $s\in S_k$ y $a,b\in R$ tales que   
	$y=a\cdot m=s+b\cdot (x-r\cdot m)$, es decir 
	\[
	b\cdot x=(a+br)\cdot m-s\in (m)\oplus S_k.
	\]
	El conjunto 
	\[
	I=\{\lambda\in R:\lambda\cdot x\in (m)\oplus S_k\}
	\]
	es un ideal de $R$ tal que $p^{\alpha}\in I$. Como
	$I$ es principal, $I=(p^\beta)$ para algún $\beta\leq\alpha$. Además $\beta\ne0$ pues $x=1\cdot x\not\in (m)\oplus S_k$. Como $p^\beta\in I$, entonces
	$p^\beta\cdot x=c\cdot m+t$ para algún $c\in R$ y $t\in S_k$. En particular, 
	\[
	0=p^{\alpha}\cdot x=p^{\alpha-\beta}\cdot (p^\beta \cdot x)=(p^{\alpha-\beta}c)\cdot m+p^{\alpha-\beta}\cdot t.
	\]
	Luego $(p^{\alpha-\beta} c)\cdot m=0$, pues 
	$(m)\cap S_k=\{0\}$, y entonces podemos escribir 
	$c=dp^\beta$ para algún $d\in R$, ya que  
	$p^{\alpha-\beta}c\in\Ann(m)=(p^{\alpha})$. Tenemos entonces que
	\[
	p^\beta\cdot x=(dp^\beta)\cdot m+t.
	\]
	Por otro lado, como $b\in I$, existe $e\in R$ tal que $b=ep^\beta$ y entonces
	\[
	b\cdot x=e\cdot (p^\beta\cdot x)=e\cdot((dp^\beta)\cdot m+t)=(edp^\beta)\cdot m+e\cdot t.
	\]
	$y=b\cdot (x-r\cdot m)=(edp^\beta-ep^\beta r)\cdot m$, se concluye que $y=0$ si elegimos 
	$r=d$. En efecto, si $y\in (m)\cap S_{k+1}$, entonces $y=s+e\cdot t\in S_k$. Luego 
	$y\in (m)\cap S_k=\{0\}$.
\end{proof}

Puede demostrarse que la descomposición del teorema anterior es única. Más precisamente, 
si $M=(n_1)\oplus\cdots\oplus (n_l)$, donde $\Ann(n_i)=(q^{\beta_i})$ para todo $i$ y algún irreducible $q$ de $R$ 
y $\beta_1\geq\beta_2\geq\cdots\geq\beta_l$, entonces $\Ann(n_i)=\Ann(m_i)$ para todo $i$. En particular, $k=l$, 
$p$ y $q$ son asociados y $\alpha_i=\beta_i$ para todo $i\in\{1,\dots,k\}$. 
Para la demostración ver 
por ejemplo~\cite[Theorem 6.11]{MR2344656}.

%\index{Módulo!cíclico}
%Un submódulo $S$ de $M$ se dice \textbf{cíclico} si $S=(m)$ para algún $m\in M$. 
%
\begin{corollary}
\index{Divisores elementales!de un módulo}
Sea $R$ un dominio de ideales principales. Si $M$ es un módulo finitamente generado, entonces
$M=F\oplus T$, donde $F$ es libre y $T$ es un submódulo de torsión. Si $\Ann(T)=(p_1^{\alpha_1}\cdots p_k^{\alpha_k})$, donde
los $p_i$ son irreducibles distintos y no asociados, entonces 
$M$ puede descomponerse como
\[
M=F\oplus \underbrace{\left( (m_{1,1})\oplus\cdots\oplus (m_{1,l_1})\right)}_{M_{p_1}}\oplus\cdots\oplus 
\underbrace{\left( (m_{k,1})\oplus\cdots\oplus (m_{k,l_k})\right)}_{M_{p_k}}
\]
donde $\Ann(m_{i,j})=(p_i^{\alpha_{i,j}})$ y $\alpha_i=\alpha_{i,1}\geq\alpha_{i,2}\geq\cdots\geq \alpha_{i,l_i}$. Los $p_i^{\alpha_{i,j}}$ son 
los \textbf{divisores elementales} de $M$. 
\end{corollary}

\begin{proof}
	Sabemos que $M$ puede descomponerse como $M=F\oplus T(M)$, donde $F$ es un submódulo libre y $T(M)$ es el submódulo de torsión de $M$. 
	Por el teorema de la descomposición primaria, podemos descomponer al submódulo $T(M)$ 
	como $T(M)=M_{p_1}\oplus\cdots\oplus M_{p_k}$ para ciertos irreducibles distintos y no asociados
	$p_1,\dots,p_k\in R$. A su vez, por el teorema de la descomposición cíclica, cada 
	$M_{p_j}$ puede descomponerse como suma directa de cíclicos, digamos
	\[
	M_{p_j}=(m_{j,1})\oplus\cdots\oplus (m_{j,l_j}),
	\] 
	donde $\Ann(m_{i,j})=(p_i^{\alpha_{i,j}})$ y $\alpha_i=\alpha_{i,1}\geq\alpha_{i,2}\geq\cdots\geq \alpha_{i,l_i}$. 
\end{proof}

También puede demostrarse la unicidad de la descomposición del corolario anterior. Más precisamente, 
el rango de la parte libre, los submódulos primarios y los anuladores quedarán unívocamente determinados
por $M$. Una demostración puede consultarse en~\cite[Theorem 6.12]{MR2344656}.

\begin{corollary}[descomposicion en factores invariantes]
	\index{Teorema!de la descomposición en factores invariantes}
	\index{Factores invariantes}
	Sea $R$ un dominio de ideales principales. Si $M$ es un módulo finitamente generado, entonces
	\[
	M=F\oplus D_1\oplus\cdots\oplus D_k,
	\]
	donde $F$ es un submódulo libre de $M$ y los $D_i$ son submódulos
	cíclicos de $M$ tales que $\Ann(D_i)=(d_i)$, donde 
	$d_i\mid d_{i-1}$ para todo $i\in\{2,\dots,k\}$. Los $d_i$ son los \textbf{factores invariantes} de la descomposicion.
\end{corollary}

\begin{proof}
	El corolario anterior nos permite descomponer a $M$ como 
	\[
M=F\oplus \underbrace{\left( (m_{1,1})\oplus\cdots\oplus (m_{1,l_1})\right)}_{M_{p_1}}\oplus\cdots\oplus 
\underbrace{\left( (m_{k,1})\oplus\cdots\oplus (m_{k,l_k})\right)}_{M_{p_k}},
\]
donde $\alpha_i=\alpha_{i,1}\geq\alpha_{i,2}\geq\cdots\geq\alpha_{i,l_i}$. Si agrupamos ordenadamente 
los sumandos cíclicos
de orden coprimo, digamos
\begin{align*}
D_1 &= (m_{1,1})\oplus (m_{2,1})\oplus\cdots\\
D_2 &= (m_{1,2})\oplus (m_{2,2})\oplus\cdots\\
D_3 &= (m_{1,3})\oplus (m_{2,3})\oplus\cdots\\
&\vdots
\end{align*}
obtenemos la descomposición que buscamos. 
\end{proof}

Puede demostrarse que los factores invariantes quedan únivocamente determinados, salvo multiplicación por unidades, por el módulo $M$. 
Ver por ejemplo~\cite[Theorem 6.13]{MR2344656}.
  
\begin{example}
Para entender mejor la descomposición del teorema anterior
hagamos un ejemplo. Si un módulo $M$ admite una descomposición cíclica de la forma 
\[
M=\left((m_{1,1}\oplus (m_{1,2})\right)\oplus (m_{2,1})\oplus \left((m_{3,1})\oplus (m_{3,2})\oplus (m_{3,3})\right),
\]
donde $\Ann(m_{1,1})=p_1^3$, $\Ann(m_{1,2})=p_1^2$, $\Ann(m_{2,1})=(p_2)$, $\Ann(m_{3,1})=(p_3^3)$, $\Ann(m_{3,2})=(p_3^3)$ y $\Ann(m_{3,3})=(p_3)$ 
para irreducibles distintos no asociados $p_1$, $p_2$ y $p_3$, entonces 
entonces $M=D_1\oplus D_2\oplus D_3$, donde 
\[
D_1=(m_{1,1})\oplus (m_{2,1})\oplus (m_{3,1}),
\quad
D_2=(m_{1,2})\oplus (m_{3,2}),\quad
D_3=(m_{3,3}).
\]
En este caso, $d_1=p_1^3p_2p_3^3$, $d_2=p_1^2p_3^3$ y $d_3=p_3$. 
\end{example}


%\begin{proof}
%Observemos
%que la función $R\to (m)$, $r\mapsto r\cdot m$, es un epimorfismo de módulos con núcleo
%$\Ann(m)$. Luego $R/\Ann(m)\simeq (m)$ por el primer teorema de isomorfismos. En particular,
%si $\Ann(m)=(p^\alpha)$ para algún irreducible $p\in R$, entonces 
%$(m)\simeq R/(p^\alpha)$.  
%El teorema anterior nos permite descomponer un módulo $p$-primario 
%como suma directa de módulos cíclicos. Observemos que...
%\end{proof}

\index{Submódulo!de relaciones de un módulo}
Para terminar el capítulo veremos un algoritmo que nos permite calcular
efectivamente los factores invariantes de un módulo finitamente generado. Aunque lo que haremos puede hacerse sobre dominios
de ideales principalers, nos concentraremos únicamente en el caso
en que $R$ sea un dominio euclidiano, ya que solamente en este contexto el algoritmo es constructivo.
 
Si $M$ es un módulo finitamente generado y $\{m_1,\dots,m_k\}$ es
un conjunto de generadores,  
existe un epimorfismo $\varphi\colon R^k\to M$, $(r_1,\dots,r_k)\mapsto \sum_{i=1}^k r_i\cdot m_i$. Por el primer teorema de isomorfismos, 
$M\simeq R^k/\ker\varphi$. 
El submódulo $\ker\varphi$ es el \textbf{submódulo de relaciones} de $M$. 
Como $R$ es en particular un dominio de ideales
principales y $M$ es finitamente generado, $\ker\varphi$ es también finitamente generado. Sea 
$\{e_1,\dots,e_l\}$ un conjunto de generadores de $\ker\varphi$, digamos 
\begin{align*}
e_1&=(a_{11},a_{12},\dots,a_{1k}),\\
e_2&=(a_{21},a_{22},\dots,a_{2k}),\\
&\vdots\\
e_l&=(a_{l1},a_{l2},\dots,a_{lk}).	
\end{align*}

La matriz $A=(a_{ij})_{1\leq i\leq l,1\leq j\leq k}$ es la 
\textbf{matriz de relaciones} de $M$ con respecto a los conjuntos $\{m_1,\dots,m_k\}$ y $\{e_1,\dots,e_l\}$. 
Veamos algunas propiedades de la matriz de relaciones.
\begin{enumerate}
\item Si $P\in R^{l\times l}$ es inversible, entonces las filas $\{f_1,\dots,f_l\}$ de la matriz $PA$ generan $\ker\varphi$. Además $PA$ es la matriz de relaciones
con respecto a $\{m_1,\dots,m_k\}$ y $\{f_1,\dots,f_l\}$. 
\item Si $Q\in R^{k\times k}$ es inversible con 
$Q^{-1}=(q_{ij})$ y para cada
$j\in\{1,\dots,k\}$ se define  
$n_j=\sum_{i=1}^k q_{ij}\cdot m_i$, el conjunto $\{n_1,\dots,n_k\}$ genera a $M$ y las 
filas de la matriz $AQ$ generan $\ker\varphi$. Además $AQ$ es la matriz de relaciones respecto de $\{n_1\dots,n_k\}$.  
\end{enumerate}

Para demostrar la primera afirmación supongamos que $P=(p_{ij})$. Las filas de la matriz $PA$ son entonces
$f_1,\dots,f_l$, donde 
\begin{align*}
f_1 &= p_{11}e_1+\cdots+p_{1l}e_l,\\
f_2 &= p_{21}e_1+\cdots+p_{2l}e_l,\\
&\phantom{=}\vdots\\
f_l &= p_{l1}e_1+\cdots+p_{ll}e_l.	
\end{align*}
Además $f_j\in\ker\varphi$ para todo $j\in\{1,\dots,l\}$. 
Como $P$ es inversible, $\{f_1,\dots,f_l\}$ es un conjunto de generadores de $\ker\varphi$, pues
cada $e_j$ es combinación lineal de los $f_i$,  
\[
\begin{pmatrix}
e_1\\
e_2\\
\vdots\\
e_l	
\end{pmatrix}
=P^{-1}\begin{pmatrix}
f_1\\
f_2\\
\vdots\\
f_l
\end{pmatrix}.
\]
En particular, $PA$ es la matriz de relaciones respecto de $\{m_1,\dots,m_k\}$ y $\{f_1,\dots,f_l\}$.  

Demostremos ahora la segunda afirmación. Como
\[
\begin{pmatrix}
n_1\\
n_2\\
\vdots\\
n_k	
\end{pmatrix}
=Q^{-1}\begin{pmatrix}
m_1\\
m_2\\
\vdots\\
m_k
\end{pmatrix},
\]
entonces
\begin{align*}
\begin{pmatrix}
0\\
0\\
\vdots\\
0	
\end{pmatrix}
=A\begin{pmatrix}
m_1\\
m_2\\
\vdots\\
m_k	
\end{pmatrix}
&=
(AQ)\begin{pmatrix}
	n_1\\
	n_2\\
	\vdots\\
	n_k
\end{pmatrix},
%=(AQ)Q^{-1}\begin{pmatrix}
%	m_1\\
%	m_2\\
%	\vdots\\
%	m_k
%\end{pmatrix}
%=\begin{pmatrix}
%0\\
%0\\
%\vdots\\
%0	
%\end{pmatrix}.
\end{align*}
lo que nos dice que las filas de la matriz $AQ$ son relaciones respecto del conjunto de generadores $\{n_1,\dots,n_k\}$. 
Sea $\psi\colon R^k\to M$, $(r_1,\dots,r_k)\mapsto \sum_{i=1}^k r_i\cdot n_i$.  
Veamos que las filas de $AQ$ generan el módulo de relaciones $\ker\psi$ respecto de $\{n_1,\dots,n_k\}$. 
Si $(r_1,\dots,r_k)\in \ker\psi$, entonces
$\sum_{i=1}^k r_i\cdot n_i=0$. Si escribimos esta fórmula matricialmente,  
\[
\begin{pmatrix}
0\\
0\\
\vdots\\
0	
\end{pmatrix}
=(r_1\cdots r_k)Q^{-1}\begin{pmatrix}
m_1\\
m_2\\
\vdots\\
m_k	
\end{pmatrix},
\]
donde conviene recordar que cada $e_j$ es un elemento de $R^k$, entonces 
\[
(r_1\cdots r_k)Q^{-1} = \left(\sum_{i=1}^k r_i\cdot q_{i1},\sum_{i=1}^k r_i\cdot q_{2i},\dots,\sum_{i=1}^k r_i\cdot q_{ki}\right)\in \ker\varphi.
\]
Como $\ker\varphi$ está generado por el conjunto $\{e_1,\dots,e_l\}$, existen  
$s_1,\dots,s_l\in R$ tales que $(r_1\cdots r_k)Q^{-1}=\sum_{i=1}^l s_i\cdot e_i$, es decir  
\[
(r_1\dots r_k)Q^{-1}=(s_1\cdots s_l)\begin{pmatrix}e_1\\\vdots\\ e_l\end{pmatrix}
=(s_1\cdots s_l)\begin{pmatrix}
a_{11} & a_{12} & \cdots & a_{1k}\\
\vdots & \vdots & & \vdots\\
a_{l1} & a_{l2} & \cdots & a_{lk}	
\end{pmatrix}
.
\]
Si reescribimos esta fórmula como
\[
(r_1\cdots r_k)=(s_1\cdots s_l)AQ,
\]
obtenemos entonces que $(r_1,\dots,r_k)$ es combinación lineal de las filas de $AQ$, pues
\begin{align*}
(r_1,\dots,r_k)&=\left(\sum_{i=1}^l s_i\cdot x_{i1},\dots,\sum_{i=1}^l s_i\cdot x_{ik}\right)
=\sum_{i=1}^l s_i\cdot (x_{11},\dots,x_{1k}).
\end{align*}
En conclusión, $\{n_1,\dots,n_k\}$ es un conjunto de generadores de $M$ y las filas de $AQ$ generan el correspondiente
submódulo de relaciones y $AQ$ es la matriz de relaciones respecto de $\{n_1,\dots,n_k\}$ y $\{e_1,\dots,e_l\}$. 

\begin{proposition}
Sea $A$ la matriz de relaciones de un módulo $M$. 
Si existen matrices inversibles $P\in R^{l\times l}$ y $Q\in R^{k\times k}$ tales que
\[
PAQ=
\begin{pmatrix}
a_1 & 0 & \cdots & \cdot & \cdot & \cdots & 0\\
0 & a_2 & \cdots & \cdot & \cdot & \cdots & 0\\
\vdots && \ddots &  & & & \vdots\\	
0 & \cdot & \cdots & a_r & \cdot & \cdots & 0\\	
0 & \cdot & \cdots & \cdot & 0 & \cdots & 0\\	
\vdots &&&&&\ddots &\vdots\\
0 & \cdot & \cdots & \cdot & \cdot & \cdots & 0
\end{pmatrix}
\]
donde $a_i\ne0$ para todo $i\in\{1,\dots,r\}$ y $a_i\mid a_{i+1}$ para todo $i\in\{1,\dots,r-1\}$,  
entonces 
\[
M\simeq R/(a_1)\oplus\cdots\oplus R/(a_r)\oplus R^{n-r}.
\]
\end{proposition}

\begin{proof}
	Sabemos que $PAQ$ es la matriz de relaciones con respecto a un cierto conjunto de generadores $\{m_1,\dots,m_k\}$ de $M$
	y respecto al submódulo de relaciones generado por las filas de $PAQ$. 
	Si $\varphi\colon R^k\to M$, $(r_1,\dots,r_k)\mapsto \sum_{i=1}^k r_i\cdot m_i$, 
	entonces $R^k/\ker\varphi\simeq M$, pues $\varphi$ es un epimorfismo. Para $j\in\{r+1,\dots,k\}$ sea $a_j=0$.
	Sea $\psi\colon R^k\to R/(a_1)\oplus\cdots\oplus R/(a_k), (s_1,\dots,s_k)\mapsto (s_1+(a_1),\dots,s_k+(a_k))$. 
	 
	Un cálculo sencillo muestra que 
	\[
	\ker\psi=(a_1)\oplus\cdots\oplus (a_k)
	\]
	y luego 
	$R^k/\ker\psi\simeq \oplus_{i=1}^k R/(a_i)$. 
	
	Dejamos como ejercicio demostrar que $\ker\varphi=\ker\psi$. 
	
% 	Veamos que $\ker\psi\subseteq\ker\varphi$. 
% 	Si $(s_1\cdot a_1,\dots,s_k\cdot a_k)\in (a_1)\oplus\cdots\oplus (a_k)$, entonces
% 	\[
% 	\varphi(s_1\cdot a_1,\dots,s_k\cdot a_k)=\sum_{i=1}^k s_i\cdot (a_i\cdot m_i)=0,
% 	\]
% 	pues $PAQ$ es la matriz de relaciones de $M$ con respecto a $\{m_1,\dots,m_k\}$. 
% 	Recíprocamente, si $(s_1,\dots,s_k)\in\ker\varphi, entonces
% 	\[
% 	\psi(s_1,\dots,s_k)=(r_1+(a_1),\dots,r_k+(a_k))=(0,\dots,0)
% 	\]
% 	pues $PAQ$ es la matriz de relaciones de $M$ con respecto a $\{m_1,\dots,m_k\}$. 

    En conclusión, 
 	tenemos que $M\simeq R/(a_1)\oplus\cdots\oplus R/(a_k)$. Para completar la demostración hay que 
 	observar que $R/(a_i)\simeq R$ para todo $i\in\{r+1,\dots,k\}$. 
\end{proof}

Veamos cómo encontrar las matrices $P$ y $Q$. La descomposición se conoce
como la \textbf{forma normal de Smith}. 
Para simplificar la presentación supondremos que $R$ es un domino euclidiano con norma $\varphi$. 
Transformaremos nuestra matriz mediante las siguientes operaciones:
\begin{enumerate}
	\item Intercambiar las filas $i$-ésima y la $j$-ésima, es decir $F_i\leftrightarrow F_j$.
	\item Reemplazar la fila $F_i$ por $F_i+\lambda F_j$ para algún $\lambda\in R$ y $j\ne i$.
	\item Intercambiar las columnas $i$-ésima y la $j$-ésima, es decir $C_i\leftrightarrow C_j$.
	\item Reemplazar la columna $C_i$ por $C_i+\lambda C_j$ para algún $\lambda\in R$ y $j\ne i$. 
\end{enumerate}
Todas estas operaciones son inversibles. Por ejemplo, la operación (1) corresponde a multiplicar
a izquierda a la matriz $A$ por una matriz de permutación. La operación (2) corresponde 
a multiplicar a izquierda a la matriz $A$ por $I+\lambda E_{ij}$, donde 
\[
(E_{ij})_{kl}=\begin{cases}
1 & \text{si $i=k$ y $j=l$},\\
0 & \text{en otro caso}.	
\end{cases}
\]
Similarmente, las operaciones de columnas se corresponden con multiplicar a derecha por matriz de 
permutación o matrices de la forma $I+\lambda E_{ij}$. 
  
\begin{theorem}[forma normal de Smith]
\index{Forma normal!de Smith}
Sea $(R,\varphi)$ un dominio euclidiano. Si $A\in R^{l\times k}$,  
existen matrices inversibles $P\in R^{l\times l}$ y $Q\in R^{k\times k}$ tales que
\[
PAQ=\begin{pmatrix}
a_1 & 0 & \cdots & \cdot & \cdot & \cdots & 0\\
0 & a_2 & \cdots & \cdot & \cdot & \cdots & 0\\
\vdots && \ddots &  & & & \vdots\\	
0 & \cdot & \cdots & a_r & \cdot & \cdots & 0\\	
0 & \cdot & \cdots & \cdot & 0 & \cdots & 0\\	
\vdots &&&&&\ddots &\vdots\\
0 & \cdot & \cdots & \cdot & \cdot & \cdots & 0
\end{pmatrix}
\]
con $a_1\cdots a_k\ne 0$ y $a_i\mid a_{i+1}$ para todo $i\in\{1,\dots,r-1\}$. 
\end{theorem}

\begin{proof}[Bosquejo de la demostración]
Vamos a demostrar que podemos transformar a la matriz $A$ en una matriz de la forma
\begin{equation*}
B=\begin{pmatrix}
	b_{11} & 0 & \cdots & 0\\
	0 & b_{22} & \cdots & b_{2m}\\
	\vdots & \vdots &&  \vdots\\
	0 & b_{n2} & \cdots & b_{nm}
\end{pmatrix}
\end{equation*}
tal que cada $b_{ij}$ es divisible por $b_{11}$. Repetimos el mismo procedimiento en la submatriz 
\[
\begin{pmatrix}
	\frac{b_{22}}{b_{11}} & \cdots & \frac{b_{2m}}{b_{11}}\\
	\vdots & &\vdots \\
	\frac{b_{n2}}{b_{11}} & \cdots & \frac{b_{nm}}{b_{11}}
\end{pmatrix}
\]
e iteramos el algoritmo hasta que no podamos continuar. 

Veamos cómo conseguir la matriz $B$. 
Al aplicar operaciones de fila y columna podemos suponer que el elemento de la matriz $A$ 
de menor norma está en la posición
$(1,1)$.  
Si algún $a_{i1}$ no es divisible por $a_{11}$, escribimos  
$a_{i1}=a_{11}u+r$ para $u\in R$ y $r\in R$ tal que $\varphi(r)<\varphi(a_{11})$. Aplicamos entonces la transformación
$F_i\leftarrow F_i-uF_1$ y nos queda una matriz
que en el lugar $(1,i)$ tendrá al elemento $r$. Similarmente, si algún $a_{1j}$
no es divisible por $a_{11}$, entonces $a_{1j}=va_{11}+s$ con $\varphi(s)<\varphi(a_{11})$ y 
aplicamos la transformación $C_j\leftarrow C_j-vC_1$ para quedarnos con una matriz que en el lugar 
$(j,1)$ tiene al elemento $s$.  
Si todos los $a_{i1}$ son divisibles por $a_{11}$, digamos $a_{i1}=a_{11}\lambda_i$, entonces
aplicamos la transformación $F_i\leftarrow \lambda_i F_1-F_i$. Similarmente, si todos los
$a_{1j}$ son divisibles por $a_{11}$, digamos $a_{1j}=a_{11}\mu_j$, entonces aplicamos
la transformación $C_j\leftarrow \mu_j C_1-C_j$. Esto nos permite quedarnos con una matriz
de la forma
\[
\begin{pmatrix}
	a_{11} & 0\\
	0 & A_1
\end{pmatrix}
=\begin{pmatrix}
	a_{11} & 0 & \cdots & 0\\
	0 & * & \cdots & *\\
	\vdots & \vdots & \ddots & \vdots \\
	0 & * & \cdots & *
\end{pmatrix}.
\]
Si alguna de las entradas de la submatriz $A_1$ no fuera divisible por $a_{11}$, hacemos la operación
$F_1\leftarrow F_1+F_i$ o bien la operación $C_1\leftarrow C_1+C_j$ y repetimos
el procedimiento.
\end{proof}

Veamos algunos ejemplos. 

\begin{example}
Sea 
\[
A=\begin{pmatrix}
	2 & 5 & 3\\
	8 & 6 & 4\\
	3 & 1 & 0
\end{pmatrix}\in\Z^{3\times3}.
\]	
Vamos a encontrar la forma normal de Smith de la matriz $A$. Como el elemento de menor norma se encuentra en la posición $(3,2)$,
hacemos las operaciones $F_1\leftrightarrow F_3$ y $C_1\leftrightarrow C_2$ y nos queda la matriz
\[
\begin{pmatrix}
	1 & 3 & 0\\
	6 & 8 & 4\\
	5 & 2 & 3
\end{pmatrix}.
\]
Para obtener ceros en las posiciones $(1,2)$, $(1,3)$, $(2,1)$ y $(2,3)$ hacemos las operaciones
$F_2\leftrightarrow 6F_1-F_2$, $F_3\leftrightarrow 5F_1-F_3$ y luego 
$C_1\leftrightarrow 3C_1-C_2$. Nos queda entonces la matriz  
\[
\begin{pmatrix}
	1 & 3 & 0\\
	0 & -10 & -4\\
	0 & -13 & -3
\end{pmatrix}.
\]
Para sacarnos de encima los números negativos multiplicamos la segunda y la tercera fila por $-1$, que es una unidad de $\Z$:
\[
\begin{pmatrix}
	1 & 3 & 0\\
	0 & 10 & 4\\
	0 & 13 & 3
\end{pmatrix}.
\]
Ahora hacemos lo mismo en la submatriz $\begin{pmatrix}10&4\\13&3\end{pmatrix}$. Para que en el lugar $(2,2)$ nos
quede el elemento de la submatriz que tiene menor norma, aplicamos las transformaciones $F_2\leftrightarrow F_3$ y
$C_3\leftrightarrow C_2$. Nos queda entonces
\[ 
\begin{pmatrix}
3 & 13\\
4 & 10
\end{pmatrix}.
\]
Escribimos $13=3\cdot 4+1$ y aplicamos
la transformación $C_2\leftarrow C_2-4C_1$ para obtener $\begin{pmatrix}3&1\\4&-6\end{pmatrix}$. Intercambiamos 
la primera y la segunda columna y nos queda la matriz
$\begin{pmatrix}1&3\\-6&4\end{pmatrix}$. Aplicamos ahora la transformación
$F_2\leftarrow 6F_1+F_2$ y al resultado le aplicamos la transformación $C_2\leftarrow 3C_1-C_2$:
\[
\begin{pmatrix}
    1 & 0\\
    0 & 22
\end{pmatrix}.
\]
Luego
\[
\begin{pmatrix}
	1 & 0 & 0\\
	0 & 1 & 0\\
	0 & 0 & -22	
\end{pmatrix}
\]
es la forma normal de Smith de la matriz $A$.
\end{example}

Veamos ahora algunas aplicaciones.

\begin{example}
Sea $M$ el grupo abeliano con generadores $m_1,m_2,m_3$ y relaciones
$8m_1+4m_2+8m_3=0$, $4m_1+8m_2+4m_3=0$. La matriz de relaciones es entonces
\[
A=\begin{pmatrix}
8 & 4 & 8\\
4 & 8 & 4
\end{pmatrix}.
\]	
Veamos que $M\simeq\Z/4\times\Z/{12}$. Si hacemos la operación $F_1\leftarrow 2F_2-F_1$, nos queda la matriz
\[
\begin{pmatrix}
0 & 12 & 0\\
4 & 8 & 4	
\end{pmatrix},
\]
que corresponde a los generadores $\{m_1,m_2,m_3\}$ con las relaciones $12m_2=0$ y $4m_1+8m_2+4m_3=0$. Si 
hacemos simultáneamente las operaciones $C_2\leftarrow C_2-2C_1$ y $C_3\leftarrow C_3-C_1$, nos queda 
la matrix
\[
\begin{pmatrix}
0 & 12 & 0\\
4 & 0 & 0	
\end{pmatrix},
\]
que corresponde al conjunto de generadores $\{m_1+2m_2+m_3,m_2\}$ con las relaciones $12m_2=0$ y $4(m_1+2m_2+m_3)=0$. 
Por último, al intercambiar
la primera y la segunda columna, nos queda la matriz 
\[
\begin{pmatrix}
4 & 0 & 0\\
0 & 12 & 0	
\end{pmatrix},
\]
que corresponde al conjunto de generadores $\{m_2,m_1+2m_2+m_3\}$ con las relaciones $4(m_1+2m_2+m_3)=0$ y $12m_2=0$. 
En conclusión, $M\simeq\Z/4\times\Z/{12}$. 
\end{example}


\begin{example}
Sea $M$ el grupo abeliano generdo por $\{m_1,\dots,m_4\}$ y sea $K$ el subgrupo
generado por $\{e_1,e_2,e_3\}$, donde
\[
e_1=22m_3,\quad
e_2=-2m_1+2m_2-6m_3-4m_4,\quad
e_3=2m_1+2m_2+6m_3+8m_4.
\]
Queremos calcular $M/K$. 
La matriz de relaciones es 
\[
A=\begin{pmatrix}
	0 & 0 & 22 & 0\\
	-2 & 2 & -6 & -4\\
	2 & 2 & 6 & 8
\end{pmatrix}.
\]
Si aplicamos la operación $F_1\leftrightarrow F_3$ y después la operación $F_2\leftarrow F_1+F_2$ nos da la matriz
\[
\begin{pmatrix}
	2 & 2 & 6 & 8\\
	0 & 4 & 0 & 4\\
	0 & 0 & 22 & 0
\end{pmatrix}.
\]
Ahora aplicamos las operaciones $C_2\leftarrow C_2-C_1$, $C_3\leftarrow C_3-3C_1$ y $C_4\leftarrow C_4-4C_1$ y obtenemos
\[
\begin{pmatrix}
	2 & 0 & 0 & 0\\
	0 & 4 & 0 & 4\\
	0 & 0 & 22 & 0
\end{pmatrix}.
\]
Hacemos ahora $C_4\leftarrow C_4-C_2$ y nos queda la matriz
\[
\begin{pmatrix}
	2 & 0 & 0 & 0\\
	0 & 4 & 0 & 0\\
	0 & 0 & 22 & 0
\end{pmatrix}
\]
pero ahora nos encontramos con que $4\nmid 22$. Utilizamos las operaciones $F_2\leftarrow F_2+F_3$, $C_3\leftarrow C_3-5C_2$, $C_3\leftarrow C_2$, $F_3\leftarrow F_3-11F_2$ y 
$C_3\leftarrow C_3-C_2$ y nos queda
\[
\begin{pmatrix}
	2 & 0 & 0 & 0\\
	0 & 2 & 0 & 0\\
	0 & 0 & -44 & 0
\end{pmatrix}.
\]
Luego el grupo abeliano $M/K$ puede presentarse con la base $\{n_1,n_2,n_3,n_4\}$ y las relaciones
$2n_1=0$, $2n_2=0$ y $44n_3=0$. En conclusión, $M/K\simeq\Z\times (\Z/2)^2\times (\Z/44)$. 
\end{example}


%\chapter{El teorema de Wedderburn}

Daremos una demostración sencilla de los teoremas de Wedderburn y Artin--Wedderburn 
en el caso de álgebras de dimensión finita. 
Seguiremos la demostración del teorema de Wedderburn de~\cite{MR184969} y 
la demostración del teorema de Artin--Wedderburn de~\cite{MR1244013}.  

\begin{definition}
\index{Álgebra!semiprima}
Diremos que un álgebra $A$ es \textbf{semiprima} si $I^2\ne\{0\}$ para todo ideal no nulo $I$ de $A$.	
\end{definition}

\index{Ideal!minimal}
\index{Ideal a izquierda!minimal}
Un ideal (a izquierda) $I$ de $A$ se dirá \textbf{minimal} si no contiene ideales (a izquierda) 
propios no nulos de $A$, es decir si $J\subseteq I$ es un ideal (a izquierda) no nulo de $A$, entonces
$J=I$. 

\begin{lemma}[Brauer]
\label{lem:Brauer1}
	Sea $A$ un álgebra de dimensión finita. Si $K$ es un ideal a izquierda no nulo minimal tal que
	$K^2\ne\{0\}$, entonces $K=Ae$ para algún idempotente $e\in A$ y $eAe$ es un álgebra de división.
\end{lemma}

\begin{proof}
Como $K^2\ne\{0\}$, existe $u\in K$ tal que $Ku\ne\{0\}$. La minimalidad de $K$ implica que $Ku=K$ y luego
$eu=u$ para algún $e\in K$. 
Sea 
\[
L=\{x\in K:xu=0\}\subsetneq K.
\] 
Si $x\in K$, entonces $xe-x\in L$, pues $(xe-x)u=x(eu)-xu=0$. 
Como $L$ es un ideal a izquierda de $A$, la minimalidad de $K$ implica que
$L=\{0\}$. Luego $e^2-e=0$. 
Nuevamente la minimalidad de $K$ implica que $K=Ae$. 

Dejamos como ejercicio
verificar que $eAe$ es un álgebra con identidad $e$. 

Si $x\in eAe\setminus\{0\}$, entonces $\{0\}\ne Ax\subseteq Ae=K$ y luego
$Ax=Ae$ por la minimalidad de $K$. Sea $y\in A$ tal que $e=yx$. 
Como $x\in eAe$, digamos $e=eae$, entonces $ex=x$. Similarmente $xe=e$.
Luego
\[
(eye)x=(ey)(ex)=eyx=e^2=e.
\]
Como $0\ne eye\in eAe$, existe $z\in A$ tal que $(eze)(eye)=e$. Luego $eze=x$. 
\end{proof}

\begin{lemma}
\label{lem:Brauer2}
Sea $A$ un álgebra semiprima de dimensión finita. 
Todo ideal a izquierda no nulo de $A$ contiene 
un idempotente no nulo.	
\end{lemma}

\begin{proof}
Como $A$ es de dimensión finita, existe un ideal a izquierda 
no nulo $I$ de $A$ de la menor dimensión posible. 
Como $A$ es semiprima, $(IA)^2\ne\{0\}$ 
y luego $I^2\ne\{0\}$, pues  
$(IA)^2=IAIA\subseteq I^2A$. 	
Por el lema de Brauer, existe un idempotente $e\in I$ tal que $I=Ae$.
\end{proof}

Observemos que si $e$ y $g$ son idempotentes, entonces 
\[
eAe\subseteq gAg\Longleftrightarrow eg=g=ge.
\]
Si $eg=e=ge$, entonces $eAe=(ge)A(ge)\subseteq gAg$. Recíprocamente, si $eAe\subseteq gAg$, entonces
$e=e^2\subseteq gAg$, digamos $e=gag$ para algún $a\in A$. Esto implica que 
\[
eg=(gag)g=gag^2=gag=e,\quad
ge=g(gag)=g^2ag=gag=e.
\]  

\begin{theorem}[Wedderburn]
\index{Teorema!de Wedderburn}
	Si $A$ es un álgebra simple 
	de dimensión finita, 
	entonces $A\simeq M_n(D)$ para algún $n\in\N$ y alguna álgebra de división $D$.  
\end{theorem}

\begin{proof}
	Sea $K$ un ideal a izquierda minimal. Como $KA$ es un ideal de $A$, la simplicidad de $A$ implica que
	$KA=A$. Además 
	\[
	A=A^2=(KA)^2=KAKA\subseteq K^2A
	\]
	y luego $K^2\ne\{0\}$. El lema de Brauer implica entonces que existe un idempotente $e\in K$ tal que
	$K=Ae$ y además $D=eAe$ es un álgebra de división. Vemos que $K$ es un $D$-módulo a derecha 
	con la multiplicación a derecha y que 
	para cada $a\in A$ 
	la función $\varphi_a\colon K\to K$, $x\mapsto ax$, es un morfismo de $D$-módulos a derecha, pues
	\[
	\varphi_a(xd)=a(xd)=(ax)d=\varphi_a(x)d.
	\]  
	La función $\varphi\colon A\to\End_D(K)$, $a\mapsto\varphi_a$, es un morfismo de álgebras. 
	
	Veamos que
	$\varphi$ es inyectiva. Si $a\in A$ es tal que $\varphi_a=0$, entonces 
	\[
	0=\varphi_a(K)=aK=aAe
	\]
	Como $A$ es simple, $AeA=A$. Se concluye así que $a=0$, pues 
	$0=aAe=aAeAe=aA$.   
	
	Veamos ahora que $\varphi$ es sobreyectiva. Como $A=AeA$, escribamos
	\[
	1=a_1eb_1+\cdots+a_neb_n
	\]
	para $a_1,\dots,a_n,b_1,\dots,b_n\in A$. 
	Si $\alpha\in\End_D(K)$, veamos que existe $a\in D$ tal que $\alpha=\varphi_a$. 
	Si $a=\sum_{i=1}^n\alpha(a_ie)eb_i$ y $x\in A$, entonces,
	como $\alpha$ es un morfismo de $D$-módulos y cada $e(b_ix)e\in D$, 
	\[
	\alpha(xe)=\alpha\left(\sum_{i=1}^na_ie^2b_ixe\right)=\sum_{i=1}^n\alpha(a_ie)eb_i(xe)=\varphi_a(xe).
	\] 
	Luego $A\simeq\End_D(K)$. Veamos ahora que $K$ es de dimensión finita sobre $D$. Si $\dim_D K=\infty$, entonces
	el conjunto
	\[
	\{\alpha\in\End_D(K):\dim\alpha(K)<\infty\}
	\]
	es un ideal no nulo propio de $\End_D(K)\simeq A$, 
	una contradicción pues $A$ es simple. En conclusión, 
	$A\simeq\End_D(K)\simeq M_n(D)$ para algún $n\in\N$.  
\end{proof}

\begin{theorem}[Artin--Wedderburn]
\index{Teorema!de Artin--Wedderburn}
	Sea $A$ un álgebra de dimensión finita. Si $A$ es semiprima, entonces
	$A$ es isomorfa a una 
	suma directa de finitas álgebras de matrices sobre álgebras de división.
\end{theorem}

\begin{proof}
	Sea $K$ un ideal a izquierda minimal de $A$. 
	
	Afirmamos que $S=KA$ es un ideal de $A$ minimal. Es claro que $S$ es ideal a izquierda de $A$. Por otro lado, 
	si $a\in S$, entonces $aS=a(KA)\subseteq KA=S$ pues $K$ es ideal a izquierda de $A$.  
	Si $I$ es un ideal no nulo de $A$ tal que 
	$I\subseteq S$. entonces $I\cap K\ne\{0\}$, pues si $I\cap K=\{0\}$,  
	\[
	I^2\subseteq I(KA)\subseteq (I\cap K)A=\{0\},
	\]
	una contradicción a la semiprimalidad de $A$. Como entonces $I\cap K\subseteq K$, la minimalidad de $K$ implica que
	$I\cap K=K\subseteq I$. Luego
	$S=KA\subseteq IA\subseteq I$.     	
	
	Por el lema~\ref{lem:Brauer2}, existe un idempotente no nulo $e\in S$ tal que $S=Ae$. 
	Como $A$ es de dimensión finita, podemos elegir 
	$e\in S$ de forma tal que la dimensión de $eAe$ sea lo más grande posible. 

	Afirmamos que $M=\{a\in A:Sa=\{0\}\}$ es un ideal de $A$. Es claro que es un ideal a izquierda de $A$. Si $a\in A$, 
	y $m\in M$, entonces $S(ma)=(Sm)=a\{0\}$. 
	
	Veamos que $A=S+M$. Sabemos que existe un idempotente $e\in K$ tal que $K=Ae$. Veamos que $1-e\in M$. Si $1-e\not\in M$, entonces
	$S(1-e)\ne 0$. Como $S(1-e)$ es ideal a izquierda y $A$ es un álgebra semiprima, 
	existe $f\in S(1-e)$ idempotente no nulo, digamos $f=s(1-e)$ para algún $s\in S$. Luego 
	\[
	fe=s(1-e)e=s(e^2-e)=0.
	\]
	y entonces $g=e+f-ef\in S$ es un idempotente de $S$. Como
	\begin{align*}
	&eg=e(e+f-ef)=e^2+ef-e^2f=e+ef-ef=e,\\
	&ge=(e+f-ef)e=e^2+fe-efe=e,
	\end{align*}
	se tiene que $eAe\subseteq gAg$. La maximalidad de la dimensión de $eAe$ implica entonces
	que $eAe=gAg$, es decir $e=g$. Esto implica que $f=ef$ y luego  
	\[
	f=f^2=(ef)^2=e(fe)f=0,
	\]
	una contradicción. 
	Luego $1-e\in M$ y entonces $A=S+M$. 
	
	Para ver que $A=S\oplus M$ falta ver que $S\cap M=\{0\}$. Como 
	\[
	(S\cap M)^2=(S\cap M)(S\cap M)\subseteq SM=\{0\},
	\]
	la semiprimalidad
	de $A$ implica que $S\cap M=\{0\}$.
	
	Ahora procedemos por inducción en la dimensión de $A$. 
	Si $M=\{0\}$, el resultado se obtiene del teorema de Wedderburn. Si $M\ne\{0\}$, entonces
	$S$ y $M$ son álgebras. Como $S$ es simple, el teorema de Wedderburn implica que $S$ es isomorfa a un álgebra
	de matrices sobre un álgebra de división. Como además $\dim M<\dim A$, la hipótesis inductiva
	implica que $M$ es isomorfa a una suma directa de álgebras de matrices sobre álgebras de división.  
\end{proof}



\backmatter
%\include{glossary}
\chapter{Algunas soluciones}

\section*{Morfismos}

\begin{sol}{xca:size9}
Si existe $x\in G$ tal que $|x|=9$, entonces $G\simeq\Z/9$. Supongamos entonces que no hay elementos de orden nueve. Por el teorema
de Lagrange, todo elemento no trivial tiene orden 3. Sea $x\in G$ tal que $|x|=3$ y sea $y\in G\setminus\langle x\rangle$. 
Entonces 
\[
G=\langle x,y\rangle=\{1,x,y,x^2,y^2,x^2y,xy^2,x^2y^2,xy\}.
\]
Si $yx=xy$, entonces $G\simeq\Z/3\times\Z/3$. 
Si $yx=x^2y^2$, entonces $(xy)^2=1$, una contradicción. 
Si $yx=x^2y$, entonces, como $yx^2$ tiene orden tres, tenemos
\[
1=(yx^2)^3=yx^2yx^2yx^2=y(yx)x^2yx^2=x^2,
\]
una contradicción. De la misma forma vemos que $yx\ne xy^2$. 
\end{sol}

\begin{sol}{xca:para_3er}
	Observemos que la conmutatividad del diagrama es $\pi_V\circ f=g\circ \pi_U$. 
	Por el ejercicio~\ref{xca:cocientes} sabemos 
	que existe $g$ si y sólo si 
	\[
	U\subseteq \ker(\pi_V\circ f)\Longleftrightarrow f(U)\subseteq\ker(\pi_V)=V.
	\] 
	
	Queremos demostrar (1). Si $h\in H$ y $hV\in H/V$, queremos ver que $g(xU)=hV$ para algún $x\in G$. 
	Como $f$ es sobreyectiva, $f(x)=h$ para algún $x\in G$. Luego
	\[
	g(xU)=g(\pi_U(x))=\pi_V(f(x))=\pi_V(h)=hV.
	\]
	
	Veamos ahora (2). Sea $x\in G$ tal que $g(xU)=V$. Como
	\[
	\pi_V(f(x))=g(\pi_U(x))=g(xU)=V,
	\]
	se tiene que $f(x)\in\ker(\pi_V)=V$, es decir $x\in f^{-1}(V)=U$. Luego $\ker g$ es trivial y entonces $g$ es inyectiva. 
\end{sol}

\begin{sol}{xca:3er}
    Sean $H=G/S$, $U=T$, $V=T/S$ y $f=\pi_S\colon G\to G/S$ el morfismo canónico. 
    El ejercicio~\ref{xca:para_3er} nos dice que 
	la existencia de un morfismo  
	\[
	g\colon G/T\to\frac{G/S}{T/S}
	\]
	tal que $g\circ\pi_T=\pi_{T/S}\circ f$  
	es equivalente a pedir que $\pi_S(T)\subseteq T/S$, algo trivial. Como $\pi_S$ es sobreyectiva, $g$ es también sobreyectiva. Además $g$ es inyectiva pues
	$\pi_S^{-1}(T/S)=T$. 
\end{sol}

\section*{Grupos de automorfismos}

\begin{sol}{xca:autgeq2}
Como la identidad es siempre un automorfismo, es necesario encontrar otro automorfismo de $G$. 
Supongamos primero que $G$ es no abeliano. Sea $g\in G\setminus Z(G)$ y sea $\gamma_g$ la conjugación
por $g$, es decir $x\mapsto gxg^{-1}$. Como $g$ no es central, entonces 
$\gamma_g$ es un automorfismo no trivial de $G$ y luego $|\Aut(G)|\geq2$. 

Supongamos ahora que $G$ es abeliano. Si existe $g\in G$ tal que $g^2\ne 1$, entonces
la función $x\mapsto x^{-1}$ es un automorfismo no trivial de $G$ y luego 
$|\Aut(G)|\geq2$. Supongamos entonces que $g^2=1$ para todo $g\in G$. En este caso, 
$G$ es un espacio vectorial sobre $\Z/2$. Como $|G|>2$, entonces
$\dim_{\Z/p} G>1$. Sea $\{x_i:i\in I\}$ una base de $G$. Como esta base tiene al menos dos elementos,
sean $i,j\in I$ elementos distintos. Sea $f\colon G\to G$ la transformación lineal dada por 
\[
f(x_k)=\begin{cases}
x_j & \text{si $k=i$},\\
x_i & \text{si $k=j$},\\
x_k & \text{en otro caso}.	
\end{cases}
\]
Como $f$ es una transformación lineal inversible,  
$f$ es un automorfismo de $G$ no trivial. Luego $|\Aut(G)|\geq2$.   
\end{sol}

\section*{Ideales}

\begin{sol}{xca:PcapQ} 
	Si $J\not\subseteq P$ y $J\not\subseteq Q$ entonces sean
	$x\in J\setminus P$, $y\in J\setminus Q$. Como $J\subseteq P\cup Q$
	entonces $x\in Q$ y además $y\in P$. Como $J$ es un ideal, $x+y\in J$. Como
	$y\in P$ y $x\not\in P$ entonces $x+y\not\in P$. Similarmente $x+y\not\in
	Q$.  Luego $x+y\in J\setminus P\cup Q$.
\end{sol}

\begin{sol}{xca:Zsqrtd}
Si $f\in\Hom(\Z[\sqrt{d}],R)$, definimos $\varphi(f)=r$ donde $r$ es tal que 
$r^2=f(d)$. La función $\varphi$ está bien definida pues, como 
\[
f(a+b\sqrt{d})=f(a)+f(b)f(\sqrt{d})=a1_R+b1_Rf(\sqrt{d}),
\]
$f$ queda unívocamente determinado por $r=f(\sqrt{d})\in R$
que cumple $r^2=f(d)$. La función $\varphi$ es inyectiva. Además $\varphi$ es sobreyectiva, pues
si $r\in R$ es tal que $r^2=d1_R$, entonces 
$f(a+b\sqrt{d})=a1_R+b1_Rr$ es un morfismo de anillos $\Z[\sqrt{d}]\to R$. 
Luego $\Hom(\Z[\sqrt{d},R)$ está en biyección con 
el conjunto $\{r\in R:r^2=d1_R\}$. 
\end{sol}

\begin{sol}{xca:sqrt2sqrt3}
 Si $f\colon\Q(\sqrt{2})\to\Q(\sqrt{3})$ es un isomorfismo de anillos,
 \[
 f(2)=f(1+1)=f(1)+f(1)=1+1=2.
 \] 
 Si $\alpha=f(\sqrt{2})$, entonces $2=f(2)=f(\sqrt{2})^2=\alpha^2$. Veamos
 que $\sqrt{2}\not\in\Q(\sqrt{3})$. En efecto, si $\sqrt{2}=(a/b)+(c/d)\sqrt{2}$, entonces
 $\sqrt{6}\in\Q$, una contradicción.
\end{sol}

\begin{sol}{xca:Z6Z15}
 Si $f\colon\Z/6\to\Z/15$, entonces $f(1)=1$. Por otro lado,
 \[
 0=f(0)=f(6)=f(1+5)=f(1)+f(5)=f(1)+5f(1)=6f(1).
 \]
 Luego $f(1)\in\{0,5,10\}$, una contradicción.
\end{sol}

\section*{El lema de Zorn}

\begin{sol}{xca:Jacobson}
    Sea $x\in J(R)$ y supongamos que $1-xy$ no
    es una unidad de $R$. Entonces $1-xy$ pertenece a algún ideal maximal $M$ y
    luego $1\in M$, una contradicción. Recíprocamente, si existe
    un ideal maximal $M$ tal que $x\not\in M$ entonces $R=(x,M)$ pues $M$ es
    maximal. Luego $1=xy+m$ para algún $y\in R$ y algún $m\in M$. Esto implica que
    $1-xy=m\in M$ y por lo tanto $1-xy\not\in\mathcal{U}(R)$. 
\end{sol}

\begin{sol}{xca:maxZn}
Sea $f\colon\Z\to \Z/n$, $k\mapsto k\bmod n$. Por el teorema de la correspondencia, los ideales
maximales de $\Z/n$ están en biyección con los ideales maximales de $\Z$ que contienen a $\ker f=n\Z$. 
Los ideales maximales de $\Z$ son de la forma $p\Z$ para algún primo $p$. 
Como $n\Z\subseteq p\Z$ si y sólo si $p$ divide a $n$, se concluye que 
$(p)$ es un ideal maximal de $\Z/n$ si y sólo si $p$ es un primo que divide a $n$.  
\end{sol}

\section*{Módulos}

\begin{sol}{xca:mod_iso_max}
    Supongamos primero que $f\colon R/M_1\to R/M_2$ es un isomorfismo de
    módulos. Entonces existe $r\in R\setminus M_2$ tal que $f(1+M_1)=r+M_2$.
    Probemos que $rM_1\subseteq M_2$. En efecto, si $m_1\in M_1$ entonces
    \[
    M_2=f(M_1)=f(m_1+M_1)=m_1\cdot f(1+M_1)=m_1\cdot (r+M_2)=rm_1+M_2. 
    \]
    
    Supongamos ahora que existe $r\in R\setminus M_2$ tal que
    $rM_1\subseteq M_2$.  Como $M_2$ es maximal, $R/M_2$ es un cuerpo. Sea $\pi\colon R\to R/M_2$ el morfismo canónico. Vamos
    a demostrar que $M_1=M_2$. Si $x\in M_1$, entonces, como $rx\in rM_1\subseteq M_2$, en el cuerpo $R/M_2$ se tiene  
    \[
    0=\pi(rx)=\pi(r)\pi(x)
    \]
    Hay entonces dos posibiblidades: $\pi(r)=0$ o bien $\pi(x)=0$. Como $r\not\in M_2=\ker\pi$, 
    entonces $\pi(x)=0$, es decir $x\in\ker \pi=M_1$. Luego $M_1\subseteq M_2$, que por la maximalidad
    del ideal $M_1$ implica $M_1=M_2$. 
\end{sol}


\section*{Sucesiones exactas}

\begin{sol}{xca:exactas1}
			Supongamos que vale (1) y sea $m''\in M''$ tal que $am''=0$. Como $g$ es
			epimorfismo, existe $m\in M$ tal que $g(m)=m''$. Luego $g(am)=am''=0$ y
			$am\in\ker(g)=f(M')$. Existe entonces $m'\in M'$ tal que $am=f(m')$. Por
			hipótesis, existe $m_1'\in M'$ tal que $am=f(am_1')$ y luego
			$a(m-f(m_1'))=0$. El elemento de $M$ que buscamos es $m-f(m_1')$ pues
			$g(m-f(m_1'))=g(m)=m''$.

			Recíprocamente, supongamos que vale (2). Sea $m'\in M'$ tal que existe
			$m\in M$ con $f(m')=am$. Si aplicamos $g$ obtenemos $0=gf(m')=ag(m)$. Si
			usamos (2) con $g(m)\in M''$ entonces existe $m_1\in M$ tal que $am_1=0$
			y $g(m)=g(m_1)$. Como $\ker(g)=f(M')$, existe $m_1'\in M'$ tal que
			$m-m_1=f(m_1')$. Esto implica que $f(m')=am=am-am_1=af(m_1')=f(am_1')$.
			Como $f$ es monomorfismo, $m'=am_1'$.
\end{sol}

\begin{sol}{xca:exactas2}	
	Si $m\in M$, $m-s(g(m))\in\ker g=\im f$ pues
	$g(m-s(g(m)))=0$. Como $f$ es inyectiva, dado $m\in M$ se tiene que 
	$m-s(g(m))=f(x)$ para un único $x\in X$. Definimos entonces
	$r\colon M\to X$, $m\mapsto x$, y entonces
	$m-s(g(m))=f(r(m))$ para todo $m\in M$.

	Supongamos que existen $r$ y $s$ tales que $f\circ r+s\circ g=\id_M$. Si $y\in Y$, entonces $y=g(m)$ 
	para algún $m\in M$ pues $g$ es epimorfismo. Como $m=f(r(m))+s(y)$, entonces
	\[
	g(s(y))=g(m-f(r(m)))=g(m)-g(f(r(m)))=y
	\]
	pues $\ker g=f(X)$. 
	9 
\end{sol}

\section*{Módulos finitamente generados}

\begin{sol}{xca:exacta_noetheriano}
	Probemos la primera afirmación. Si $M_1\subseteq M$ es un submódulo, entonces
	$M_1$ es finitamente generado porque $M_1\simeq f(M_1)\subseteq M$ y $M$ es
	noetheriano. Si $T_1\subseteq T$ es un submódulo, entonces $T_1$ es
	finitamente generado por ser isomorfo a un submódulo de $M$ que
	contiene a $\ker(g)$ y $M$ es noetheriano.

	Probemos la segunda afirmación. Si $K\subseteq M$ es un submódulo,
	consideremos la sucesión exacta
	\[
		\xymatrix{
		0\ar[r] 
		& f^{-1}(K)
		\ar[r]^-{f}
		& K
		\ar[r]^-{g}
		& g(K)\ar[r]
		& 0
		}
	\]
	Como $S$ y $T$ son noetherianos, $f^{-1}(K)$ y $g(K)$ son finitamente
	generados. Luego $K$ también es finitamente
	generado.
\end{sol}

\section*{Módulos libres}

\section*{Módulos proyectivos}

\begin{sol}{xca:cociente_libre}
Como $M/N$ es un módulo libre, el módulo $M/S$ es proyectivo. Existe entonces un morfismo $s\colon M/N\to N$ tal 
que $\pi\circ s=\id_{M/N}$. En particular, $M\simeq N\oplus M/N$. 	
\end{sol}

\begin{sol}{xca:ss_idempotente}
Si $\prescript{}{R}R$ es semisimple e $I$ es un ideal de $R$, sabemos que existe un submódulo $J$ (es decir, $J$ es ideal a izquierda de $R$) tal que $R=I\oplus J$. En particular, 
existen $e\in I$ y $f\in J$ tales que $1=e+f$. Como $ef\in I\cap J=\{0\}$, entonces
\[
e=e1=e^2+ef=e^2.
\]
Veamos que $I=Re$. Si $x\in I$, entonces $x=x1=xe+xf=xe\in Re$ pues $xf=x-xe\in I\cap J=\{0\}$. 

Si $I=Re$ para algún idempotente $e\in R$, entonces $J=R(1-e)$ es tal que $R=I\oplus J$. En efecto, $R=I+J$, pues 
$r=re+r(1-e)$. Además $I\cap J=\{0\}$ pues si $r=xe=y(1-e)$, entonces 
\[
r=xe=xe^2=y(1-e)e=0.
\]
\end{sol}

\begin{sol}{xca:I^2}
Si existe un ideal $J$ de $R$ tal que $R=I\oplus J$, entonces $1=u+v$ para ciertos $u\in I$ y $v\in V$. Veamos que $I=(u)$. Si $x\in I$,
entonces $x=x1=xu+xv$. Como $xv\in I\cap J=\{0\}$, se concluye que $x=xu\in (u)$. Además 
\[
1=1\cdot 1=(u+v)^2=u^2+2uv+v^2=u^2+v^2,
\]
pues $uv\in I\cap J=\{0\}$. Luego $(u)=(u^2)$, pues $u=u1=u(u^2+v^2)=u^3\in (u^2)$.  

Supongamos ahora que existe $u\in R$ tal que $I=(u)=(u^2)$. Entonces $u=ru^2$ para algún $r\in R$. En particular, 
$ru=r^2u^2$. Si  
$e=ru$, entonces $e$ es idempotente, pues
\[
e^2=(ru)^2=r^2u^2=ru=e.
\]
Veamos que $I=(u)=(e)$. Basta con demostrar que $(u)\subseteq (e)$. 
Si $\lambda\in R$, entonces 
\[
\lambda u=\lambda(ru^2)=(\lambda u)(ru)=(\lambda u)e\in (e).
\]
Si $J=(1-e)$, entonces $R=I\oplus J$, pues ya vimos que
$I\cap J=\{0\}$ y $R=I+J$.  	
\end{sol}

\begin{sol}{xca:proyectivo1}
	Como $P$ es proyectivo existe un morfismo $\beta\colon P\to P'$ tal que $g'\circ \beta=g$.
	Luego $g'\circ \beta\circ F=g\circ f=0$ y entonces existe un morfismo $\alpha\colon K\to K'$
	tal que $\beta\circ f=f'\circ \alpha$. 

	Sea $\phi\colon K\to K'\oplus P$ el morfismo dado por
	$\phi(k)=(\alpha(k),f(k))$ y $\psi\colon K'\oplus P\to P'$ el morfismo dado
	por $\psi(k',p)=(f'(k'),-\beta(p))$. 
	La sucesión 
	\[
	\xymatrix{
	0\ar[r] 
	& K
	\ar[r]^-{\phi}
	& K'\oplus P
	\ar[r]^-{\psi}
	& P'\ar[r]
	& 0
	}
	\]
	es exacta y se parte porque $P'$ es proyectivo.
\end{sol}

\section*{El teorema de estructura}

\begin{sol}{xca:rank}
Como $M$ es libre, al usar el morfismo canónico $\varphi\colon M\to M/S$, tenemos que $M\simeq S\oplus (M/S)$.   
\end{sol}

\begin{sol}{xca:n_elements}
Sea $K=K(R)$ el cuerpo de fracciones de $R$. Como $M$ es libre de rango $n$, entonces $M\simeq R^n$. Como 
$R^n$ es un subgrupo de $K^n$, vemos que $\{v_1,\dots,v_n\}$ es linealmente independiente sobre $K$ si y sólo si $\{v_1,\dots,v_n\}$ 
es linealmente independiente sobre $R$.   
\end{sol}

\begin{sol}{xca:base}
Sea $\{m_1,\dots,m_n\}$ una base de $M$. Como $M$ es libre, existe un morfismo $\varphi\colon M\to M$ tal que $m_j\mapsto s_j$ para todo
$j\in\{1,\dots,n\}$. Como $M$ es libre, $M$ es proyectivo y entonces $M\simeq \ker\varphi\oplus M$. Como $R$ es principal, el submódulo
$\ker\varphi$ es libre de rango $\leq n$. Como además $\rank(M)=\rank(\ker\varphi)+\rank(M)$, se concluye que $\rank(\ker\varphi)=0$ y luego
$\ker\phi=\{0\}$, es decir que $\varphi$ es un isomorfismo. En particular, $\{s_1,\dots,s_n\}$ es una base de $M$.
\end{sol}

\begin{sol}{xca:free}
    Como $M/S$ es finitamente generado y sin torsión, es libre y por tanto
    proyectivo. La sucesión exacta $0\to S\to M\to M/S\to0$ se
    parte y entonces $M\simeq S\oplus M/S$. Luego $M$ es proyectivo por
    ser suma de proyectivos. Como $R$ es un dominio de ideales principales, se concluye que
    $M$ es libre.
\end{sol}

\bibliographystyle{abbrv}
\bibliography{refs}

\printindex



\end{document}





Cocient
